%Executive Summary

\section{Executive Summary}
\textit{Note:Working Paper}


Update assessments were conducted for the twenty stocks in the Northeast Multispecies Fishery Management Plan in 2015 (Table \ref{stock_abbrv_tab}). The updates replicated the methods recommended in the most recent benchmark decisions, as modified by any subsequent operational assessments or updates (Table \ref{stock_info_tab}), with the intention of simply adding years of data (Table \ref{data_used_tab}). However, minor flexibility was allowed to address emerging issues (Table \ref{Assess_type_tab}).

Stock status did not change for 15 of the 20 stocks, worsened for two stocks, improved for one stock, and became more uncertain for two stocks (Table \ref{sos_tab}).

The number of stocks with retrospective adjustments applied increased from the last assessment from 2 to 7 (Table \ref{RhoAdjust_tab}). The previous Georges Bank cod assessment did apply a retrospective adjustment, however, the assessment model was not approved at the 2015 Updates so it has been excluded from these counts.

While the number of overfished stocks and stocks experiencing overfishing has generally decreased since 2007 (Figure \ref{stock_status}), the magnitude of overfishing or depletion for several stocks has worsened considerably (Figures \ref{propFmsy} and \ref{propBmsy}); Gulf of Maine cod, Southern New England/Mid-Atlantic yellowtail flounder, witch flounder and Cape Cod/Gulf of Maine yellowtail flounder). Of those Northeast groundfish stocks for which stock status can be determined, the majority remain below their biomass targets ($69\%$; Figures \ref{stock_status} and \ref{propBmsy}).

Recent NEFSC survey biomass indices for both the spring and fall surveys are below the long term means. For the majority of stocks the average of the most recent five years are below the time series means (Figures \ref{nefscSpringResiduals} and \ref{nefscFallResiduals})

Estimates of overall (aggregate) groundfish minimum swept area biomass are at, or near, all-time highs (Figures \ref{nefscSpringMinSweptAreaBiomass} and \ref{nefscFallMinSweptAreaBiomass}).  However, the current stock diversity of the overall groundfish biomass is less than that seen in the 1960s and 1970s. Current groundfish biomass is dominated by only a few stocks: For example the combined biomass of the Georges Bank haddock, Gulf of Maine haddock, and redfish stocks currently make up more than $80\%$ of the overall groundfish biomass (Figure \ref{propBiomassHaddockRedfish}). 

Information supplemental to the assessment report for each stock can found on the Stock Assessment Support Information (\href{http://www.nefsc.noaa.gov/saw/sasi/sasi_report_options.php}{SASINF}{}) website.

The appendix to this document contains: The letter from the Northeast Regional Coordinating Council providing guidance on the operational assessment procedure (Section \ref{NRCCletter}), a summary of the meeting with the Assessment Oversight Panel during which assessment plans were developed (Section \ref{AOPsum}), a summary of NEFSC outreach on 2015 groundfish operational assessments (Section \ref{OutreachSum}) and statements from fishing industry members (Section \ref{IndLetter}).  

%%%%%%%%%%%%%%%%%%%%%%%%%%%%%%%%%%%%%%%%%%%%%%%%%%%%%%%%%%%%%%%%%%%%%%%%%%%%%%%%%%%%%%%%%%%%%%%%%%%%%%%
%Tables
\clearpage
\begin{table}
	\centering
	
	\caption{ List of stocks included in the groundfish operational assessment and the abbreviations used for each in this document.}
	\label{stock_abbrv_tab}
	\begin{tabular}{ll}
\hline
Stock Abbrev & Stock Name \\
\hline 
CODGM & Gulf of Maine cod \\
CODGB & Georges Bank cod \\
HADGM & Gulf of Maine haddock \\
HADGB & Georges Bank haddock \\
YELCCGM & Cape Cod/Gulf of Maine yellowtail flounder \\
YELSNEMA & Southern New England/Mid-Atlantic yellowtail flounder \\
FLWGB & Georges Bank winter flounder \\
FLWSNEMA & Southern New England/Mid-Atlantic winter flounder \\
REDUNIT & Acadian redfish \\
PLAUNIT & American plaice \\
WITUNIT & Witch flounder \\
HKWUNIT & White hake \\
POLUNIT & Pollock \\
CATUNIT & Wolffish \\
HALUNIT & Atlantic halibut \\
FLDGMGB & Gulf of Maine/Georges Bank windowpane flounder \\
FLDSNEMA & Southern New England/Mid-Atlantic windowpane flounder \\
OPTUNIT & Ocean pout \\
FLWGM & Gulf of Maine winter flounder \\
YELGB & Georges Bank yellowtail flounder \\
\hline
	\end{tabular}
\end{table}
\clearpage
\newcolumntype{L}{>{\centering}m{2cm}} 
\newcolumntype{O}{>{\centering}m{3cm}}
\newcolumntype{P}{>{\centering}m{1.8cm}} 
\begin{sidewaystable}[ht]  
	\centering
	\captionsetup{width=\textwidth}
	\caption{Lead scientist for each stock (current$/$previous if different), information about last assessment, including: the forum for review of the last assessment (Forum), the type of assessment done (Type), publication year (Pub.), the terminal year of the catch data included (Term. yr.), overfished/overfishing status, rebuilding status, and reference. \textit{Note: Op. Assess $=$ Operational Assessment}}
	\label{stock_info_tab}
	\small{	
	\begin{tabular}{
	m{2cm}@{\hspace{.1cm}}
	O@{\hspace{.1cm}}
	m{2cm}@{\hspace{.1cm}}
	m{1.75cm}@{\hspace{.1cm}}
	m{1.cm}@{\hspace{.1cm}}
	m{1cm}@{\hspace{.1cm}}
	P@{\hspace{.1cm}}
	P@{\hspace{.1cm}}
	m{1.5cm}@{\hspace{.1cm}}
	m{1.5cm}@{\hspace{.1cm}}
	}
	\hline

Stock & 
Lead & 
Forum & 
Type & 
Pub. & 
Term. yr. & 
Overfished? & 
Overfishing? & 
Rebuild status & 
Reference \\
	
	\hline
CODGM & Palmer & Op. Assess & Update & 2014 & 2013 & Yes & Yes & By 2024 & \href{http://www.nefsc.noaa.gov/publications/crd/crd1414/}{CRD14-14} \\
CODGB & O'Brien & SARC 55 & Benchmark & 2012 & 2011 & Yes & Yes & By 2026 & \href{http://nefsc.noaa.gov/publications/crd/crd1311/}{CRD13-11} \\
HADGM & Palmer & SARC 59 & Benchmark & 2014 & 2013 & No & No & Rebuilt & \href{http://nefsc.noaa.gov/publications/crd/crd1409/}{CRD14-09} \\
HADGB & Brooks & GARM2012 & Update & 2012 & 2010 & No & No & Rebuilt & \href{http://www.nefsc.noaa.gov/publications/crd/crd1206/}{CRD12-06} \\
YELCCGM & Alade$/$Legault & GARM2012 & Update & 2012 & 2010 & Yes & Yes & By 2023 & \href{http://www.nefsc.noaa.gov/publications/crd/crd1206/}{CRD12-06} \\
YELSNEMA & Alade & SARC 54 & Benchmark & 2012 & 2011 & No & No & Rebuilt & \href{http://www.nefsc.noaa.gov/publications/crd/crd1218/}{CRD12-18} \\
FLWGB & Hendrickson & Op. Assess & Update & 2015 & 2013 & No & No & By 2017 & \href{http://www.nefsc.noaa.gov/publications/crd/crd1501/}{CRD15-01} \\
FLWSNEMA & Wood$/$Terciero & SARC 52 & Benchmark & 2011 & 2010 & Yes & No & By 2023 &  \href{http://www.nefsc.noaa.gov/saw/saw52/crd1117.pdf}{SARC52} \\
REDUNIT & Linton$/$Miller & GARM2012 & Update & 2012 & 2010 & No & No & Rebuilt & \href{http://www.nefsc.noaa.gov/publications/crd/crd1206/}{CRD12-06} \\
PLAUNIT & O'Brien & GARM2012 & Update & 2012 & 2010 & No & No & By 2024 & \href{http://www.nefsc.noaa.gov/publications/crd/crd1206/}{CRD12-06} \\
WITUNIT & Wigley & GARM2012 & Update & 2012 & 2010 & Yes & Yes & By 2017 & \href{http://www.nefsc.noaa.gov/publications/crd/crd1206/}{CRD12-06} \\
HKWUNIT & Sosebee & SARC 56 & Benchmark & 2013 & 2011 & No & No & By 2014 & \href{http://www.nefsc.noaa.gov/publications/crd/crd1310/}{CRD13-10} \\
POLUNIT & Linton & Op. Assess & Update & 2015 & 2013 & No & No & Rebuilt & \href{http://www.nefsc.noaa.gov/publications/crd/crd1501/}{CRD15-01} \\
CATUNIT & Adams$/$Keith & GARM2012 & Update & 2012 & 2010 & Yes & No & Unknown & \href{http://www.nefsc.noaa.gov/publications/crd/crd1206/}{CRD12-06} \\
HALUNIT & Hennen$/$Blaylock & GARM2012 & Update & 2012 & 2010 & Yes & No & By 2055 & \href{http://www.nefsc.noaa.gov/publications/crd/crd1206/}{CRD12-06} \\
FLDGMGB & Chute$/$Hendrickson & GARM2012 & Update & 2012 & 2010 & Yes & Yes & By 2017 & \href{http://www.nefsc.noaa.gov/publications/crd/crd1206/}{CRD12-06} \\
FLDSNEMA & Chute$/$Hendrickson & GARM2012 & Update & 2012 & 2010 & No & No & Rebuilt & \href{http://www.nefsc.noaa.gov/publications/crd/crd1206/}{CRD12-06} \\
OPTUNIT & Wigley & GARM2012 & Update & 2012 & 2010 & Yes & No & By 2014 & \href{http://www.nefsc.noaa.gov/publications/crd/crd1206/}{CRD12-06} \\
FLWGM & Nitschke & Op. Assess & Update & 2015 & 2013 & Unknown & No & Unknown & \href{http://www.nefsc.noaa.gov/publications/crd/crd1501/}{CRD15-01} \\
YELGB & Legault & TRAC 2015 & Update & 2015 & 2014 & Unknown & Unknown & By 2032 & \href{http://www.nefsc.noaa.gov/saw/trac/TSR_2015_GBYellowTailFlounder.pdf}{TRAC2015}\\
	\hline
	\end{tabular}
}
\end{sidewaystable}



\clearpage
\newcommand{\colspc}{.2cm}
\newcolumntype{Q}{c@{\hspace{\colspc}}}

\begin{sidewaystable}[ht]
%\begin{table}
\captionsetup{width=\textwidth}
\centering
\caption{Data used in each assessment. The column heads are US commercial landings (US c-lnd), US commercial discards (US c-dis), US recreational landings (US r-lnd), US recreational discards (US r-dis), Canadian catch (CA cat), Northeast Fisheries Science Center spring, fall and winter surveys (NE S, NE F and NE W), Massachusetts spring and fall surveys (MA S and MA F), Maine/New Hampshire spring and fall surveys (ME/NH S and ME/NH F) and Canadian Department of Fisheries and Oceans February survey (DFO S).} 
\label{data_used_tab}
{\small
\begin{tabular}{l@{\hspace{\colspc}}
Q 
Q 
Q 
Q 
Q
m{1mm} 
Q 
Q 
Q 
Q 
Q 
Q 
Q 
Q 
}

\hline

& \multicolumn{5}{c}{\textbf{Catch}} && \multicolumn{8}{c}{\textbf{Surveys}} \\  
\cline{2-6} \cline{8-15}
Stock & US c-lnd & US c-dis & US r-lnd & US r-dis & CA Cat && NE S & NE F & NE W & MA S & MA F & ME/NH S & ME/NH F & DFO S \\

\hline
CODGM & Yes & Yes & Yes & Yes & No && Yes & Yes & No & Yes & No & No & No & No \\
CODGB & Yes & Yes & Yes & Yes & Yes && Yes & Yes & No & No & No & No & No & Yes \\
HADGM & Yes & Yes & Yes & Yes & No && Yes & Yes & No & No & No & No & No & No \\
HADGB & Yes & Yes & No & No & Yes && Yes & Yes & No & No & No & No & No & Yes \\
YELCCGM & Yes & Yes & No & No & No && Yes & Yes & No & Yes & Yes & Yes & Yes & No \\
YELSNEMA & Yes & Yes & No & No & No && Yes & Yes & Yes & No & No & No & No & No \\
FLWGB & Yes & Yes & No & No & Yes && Yes & Yes & No & No & No & No & No & Yes \\
FLWSNEMA & Yes & Yes & Yes & Yes & No && Yes & Yes & Yes & Yes & No & No & No & No \\
REDUNIT & Yes & Yes & No & No & No && Yes & Yes & No & No & No & No & No & No \\
PLAUNIT & Yes & Yes & No & No & Yes && Yes & Yes & No & Yes & Yes & No & No & No \\
WITUNIT & Yes & Yes & No & No & No && Yes & Yes & No & No & No & No & No & No \\
HKWUNIT & Yes & Yes & No & No & Yes && Yes & Yes & No & No & No & No & No & No \\
POLUNIT & Yes & Yes & Yes & Yes & No && Yes & Yes & No & No & No & No & No & No \\
CATUNIT & Yes & Yes & Yes & No & No && Yes & Yes & No & Yes & No & No & No & No \\
HALUNIT & Yes & Yes & No & No & Yes &&  No & Yes & No & No & No & No & No & No \\
FLDGMGB & Yes & Yes & No & No & No && No & Yes & No & No & No & No & No & No \\
FLDSNEMA & Yes & Yes & No & No & No &&  No & Yes & No & No & No & No & No & No \\
OPTUNIT & Yes & Yes & No & No & No && Yes & No & No & No & No & No & No & No \\
FLDWGM & Yes & Yes & Yes & Yes & No && Yes & Yes & No & Yes & Yes & Yes & Yes & No \\
YELGB & Yes & Yes & No & No & Yes &&  Yes & Yes & No & No & No & No & No & Yes \\
   \hline
\end{tabular}
}

\end{sidewaystable}
%\end{table}
\clearpage
\newcolumntype{L}{>{\centering}m{2cm}}
\newcolumntype{M}{>{\centering}m{2.25cm}}
\newcolumntype{N}{>{\centering}m{1cm}}


\begin{sidewaystable}[ht]  
	\centering
	\captionsetup{width=\textwidth}
	\caption{Assessment type and reference points from previous assessment. Biomass and yield values are in metric tons. \textit{Note: sp=stochastic projection and surv. B = survey biomass.}}
	\label{Assess_type_tab}
	\small{	
	\begin{tabular}{
	m{2.5cm}@{\hspace{.1cm}}
	m{1.25cm}@{\hspace{.1cm}}
	M@{\hspace{.1cm}}   
	L@{\hspace{.2cm}}
	N@{\hspace{.2cm}}   
	L@{\hspace{.1cm}}
	m{0.75cm}@{\hspace{.1cm}}
	L@{\hspace{.2cm}}
	m{1.5cm}@{\hspace{.1cm}}
	L@{\hspace{.2cm}}
	m{1.5cm}@{\hspace{0cm}}
	}

	\hline

Stock & Assess. & Type & F def. & B def. & $F_{MSY}$ type & $F_{MSY}$ value & $B_{MSY}$ type & $B_{MSY}$ value & MSY type & MSY value \\
\hline
CODGM(M=.2) & ASAP & age-based & $F_{Full}$ & SSB & $F_{40\%SPR}$ & 0.18 & sp & 47,184 & sp & 7,753 \\
CODGM($M_{ramp}$) & ASAP & age-based & $F_{Full}$ & SSB & $F_{40\%SPR}$ & 0.18 & sp & 69,621 & sp & 11,388 \\
CODGB & ASAP & age-based & $F_{Full}$ & SSB & $F_{40\%SPR}$ & 0.18 & sp & 186,535 & sp & 30,622 \\
HADGM & ASAP & age-based & $F_{Full}$ & SSB & $F_{40\%SPR}$ & 0.46 & sp & 4,108 & sp & 955 \\
HADGB & VPA & age-based & avg F ages 5-7 & SSB & $F_{40\%SPR}$ & 0.39 & sp & 124,900 & sp & 28,000 \\
YELCCGOM & VPA & age-based & avg F ages 4-6 & SSB & $F_{40\%SPR}$ & 0.26 & sp & 7,080 & sp & 1,600 \\
YELSNEMA & ASAP & age-based & avg F ages 4-5 & SSB & $F_{40\%SPR}$ & 0.32 & sp & 2,995 & sp & 773 \\
FLWGB & VPA & age-based & avg F ages 4-6 & SSB & Fmsy & 0.44 & sp & 8,100 & sp & 3,200 \\
FLWSNEMA & ASAP & age-based & avg F ages 4-5 & SSB & Fmsy & 0.29 & sp & 43,661 & sp & 11,728 \\
REDUNIT & ASAP & age-based & $F_{Full}$ & SSB & $F_{50\%SPR}$ & 0.04 & sp & 238,480 & sp & 8,891 \\
PLAUNIT & VPA & age-based & avg F ages 6-9 & SSB & $F_{40\%SPR}$ & 0.18 & sp & 18,398 & sp & 3,385 \\
WITUNIT & VPA & age-based & avg F ages 8-11 & SSB & $F_{40\%SPR}$ & 0.27 & sp & 10,051 & sp & 2,075 \\
HKWUNIT & ASAP & age-based & $F_{Full}$ & SSB & $F_{40\%SPR}$ & 0.20 & sp & 32,400 & sp & 5,630 \\
POLUNIT & ASAP & age-based & avg F ages 5-7 & SSB & $F_{40\%SPR}$ & 0.27 & sp & 76,879 & sp & 14,791 \\
CATUNIT & SCALE & length-based & $F_{Full}$ & SSB & $F_{40\%SPR}$ & 0.33 & sp & 1,756 & sp & 261 \\
HALUNIT & RYM & \centering{surplus production} & \centering{biomass wted F} & B & F0.1 & 0.07 & deterministic & 48,509 & \center{deterministic} & 3,546 \\
FLDGMGB & AIM & index & $\frac{catch}{surv. B}$ & surv. B & replacement ratio & 0.44 & $\frac{MSY{}\textit{proxy}}{F_{MSY proxy}}$ & 1.60 & median catch 1995-2001 & 700 \\
FLDSNEMA & AIM & index & $\frac{catch}{surv. B}$ & surv. B & replacement ratio & 2.09 & $\frac{MSY{}\textit{proxy}}{F_{MSY proxy}}$ & 0.24 & median catch 1995-2001 & 500 \\
OPTUNIT & index & index & $\frac{catch}{surv. B}$ & surv. B & med. $F_{1977-1985}$ & 0.76 & med. surv. $B_{1977-1985}$ & 4.94 & \centering{$F_{MSY}$ * $B_{MSY}$} & 3,754 \\
FLWGM & empirical & \centering{survey expansion} & $\frac{catch}{B_{30+cm}}$ & surv. B &  \centering{$F_{40\%}$ from YPR} & 0.23 & NA & NA & NA & NA \\
YELGB & empirical & survey expansion & NA & surv. B & NA & NA & NA & NA & NA & NA \\


	\hline
	\end{tabular}
}
\end{sidewaystable}


%\centering{7,753 ($M=0.2$) or 11,388 (Mramp)}
%\centering{47,184 (M=0.2) or 69,621 (Mramp)}
\clearpage
%sos_tab
\begin{table}
	\centering	
	\caption{ Synopsis of status by stock.}
	\label{sos_tab}
	\begin{tabular}{lcccc}
	\hline


Stock & Last Assessment & Status Change? & Overfishing? & Overfished? \tabularnewline
\hline
CODGM & 2014 & \cellcolor{yellow} Same & \cellcolor{red} Yes & \cellcolor{red} Yes \tabularnewline
CODGB & 2012 & \cellcolor{orange} More uncertain & \cellcolor{yellow} Unknown & \cellcolor{red} Yes \tabularnewline
HADGM & 2012 & \cellcolor{yellow} Same & \cellcolor{green} No & \cellcolor{green} No \tabularnewline
HADGB & 2014 & \cellcolor{yellow} Same & \cellcolor{green} No & \cellcolor{green} No \tabularnewline
YELCCGM & 2012 & \cellcolor{yellow} Same & \cellcolor{red} Yes & \cellcolor{red} Yes \tabularnewline
YELSNEMA & 2012 & \cellcolor{red} Worse & \cellcolor{red} Yes & \cellcolor{red} Yes \tabularnewline
FLWGB & 2014 & \cellcolor{red} Worse & \cellcolor{red} Yes & \cellcolor{red} Yes \tabularnewline
FLWSNEMA & 2011 & \cellcolor{yellow} Same & \cellcolor{green} No & \cellcolor{red} Yes \tabularnewline
REDUNIT & 2012 & \cellcolor{yellow} Same & \cellcolor{green} No & \cellcolor{green} No \tabularnewline
PLAUNIT & 2012 & \cellcolor{yellow} Same & \cellcolor{green} No & \cellcolor{green} No \tabularnewline
WITUNIT & 2012 & \cellcolor{yellow} Same & \cellcolor{red} Yes & \cellcolor{red} Yes \tabularnewline
HKWUNIT & 2013 & \cellcolor{yellow} Same & \cellcolor{green} No & \cellcolor{green} No \tabularnewline
POLUNIT & 2014 & \cellcolor{yellow} Same & \cellcolor{green} No & \cellcolor{green} No \tabularnewline
CATUNIT & 2012 & \cellcolor{yellow} Same & \cellcolor{green} No & \cellcolor{red} Yes \tabularnewline
HALUNIT & 2012 & \cellcolor{orange} More uncertain & \cellcolor{yellow} Unknown & \cellcolor{red} Yes \tabularnewline
FLDGMGB & 2012 & \cellcolor{green} Better & \cellcolor{green} No & \cellcolor{red} Yes \tabularnewline
FLDSNEMA & 2012 & \cellcolor{yellow} Same & \cellcolor{green} No & \cellcolor{green} No \tabularnewline
OPTUNIT & 2012 & \cellcolor{yellow} Same & \cellcolor{green} No & \cellcolor{red} Yes \tabularnewline
FLWGM & 2014 & \cellcolor{yellow} Same & \cellcolor{green} No & \cellcolor{yellow} Unknown \tabularnewline
YELGB & 2014 & \cellcolor{yellow} Same & \cellcolor{yellow} Unknown & \cellcolor{yellow} Unknown \tabularnewline

\hline
	\end{tabular}
\end{table}


%\cellcolor{blue} foo & \cellcolor{red}
\clearpage
\begin{sidewaystable}[ht]
%\begin{table}
\captionsetup{width=\textwidth}

\centering
%\caption{}
\caption{Comparison of biomass ($B$) and fishing mortality ($F$) rate Mohn's rho values ($\rho$) by stock between the previous assessment and the 2015 updates. The biomass and fishing mortality rate point estimates and $\rho$ adjusted values (Adj.) are provided for the 2015 operational assessments. The total number of stocks using $\rho$ adjusted values in the last assessment and the 2015 assessments ($\rho$ adj. vs. pt. est. for those stocks that did not use the $\rho$ adjustment), along with the type of $\rho$ adjustment used in the 2015 assessment (NAA$=$numbers at age, SSB$=$spawning stock biomass applied to all ages), are also provided. Only age-based and length-based stocks that could exhibit retrospective patterns are included in this table. \textit{Note: Because the Georges Bank cod assessment was rejected at the 2015 OA it has been excluded from this table.} }
\label{RhoAdjust_tab}
{\small
\begin{tabular}{
l@{\hspace{.1cm}}
c@{\hspace{.1cm}}
c@{\hspace{.2cm}}
c@{\hspace{.2cm}}
c@{\hspace{.2cm}}
c@{\hspace{.1cm}}
m{1mm} 
c@{\hspace{.2cm}}
c@{\hspace{.2cm}}
c@{\hspace{.2cm}}
c@{\hspace{.1cm}}
m{1mm} 
c@{\hspace{.2cm}} 
c@{\hspace{.2cm}}
c@{\hspace{.2cm}} 
}
\hline
 & & &&&&&&&&&& \\[1pt] %blank row
 & & \multicolumn{4}{c}{\textbf{Biomass}} && \multicolumn{4}{c}{\textbf{Fishing Mortality Rate}} && \multicolumn{3}{c}{\textbf{Used}} \\  
\cline{3-6} \cline{8-11} \cline{13-15}
Stock & Model &$\rho_{last}$ & $\rho_{2014}$ & $B_{2014}$ & Adj. && $\rho_{last}$ & $\rho_{2014}$ & 
$F_{2014}$ & Adj. && Last assess. & 2014 & Proj. adj.\\
  \hline
CODGM & ASAP(M=0.2) & 0.53 & 0.54 & 2225 & 1445 && -0.33 & -0.31 & 0.956 & 1.386 && pt. est. & pt. est. & none \\
CODGM & ASAP(M-ramp) & 0.17 & 0.2 & 2536 & 2113 && -0.05 & -0.08 & 0.932 & 1.013 && pt. est. & pt. est. & none \\
HADGM & ASAP & -0.15 & -0.04 & 10325 & 10755 && 0.3 & 0.03 & 0.257 & 0.25 && pt. est. & pt. est. & none \\
HADGB & VPA & 0.2 & 0.5 & 225080 & 150053 && -0.15 & -0.34 & 0.159 & 0.241 && pt. est. & $\rho$ adj. & SSB \\
YELCCGM & VPA & 0.68 & 0.98 & 1695 & 857 && -0.19 & -0.45 & 0.35 & 0.64 && $\rho$ adj. & $\rho$ adj. & NAA \\
YELSNEMA & ASAP & 0.14 & 1.06 & 502 & 243 && -0.16 & -0.53 & 1.64 & 3.53 && pt. est. & pt. est. & none \\
FLWGB & VPA & 0.26 & 0.83 & 5275 & 2883 && -0.16 & -0.51 & 0.379 & 0.778 && pt. est. & $\rho$ adj. & SSB \\
FLWSNEMA & ASAP & 0.35 & 0.21 & 6151 & 5105 && -0.31 & -0.25 & 0.16 & 0.214 && pt. est. & pt. est. & none \\
REDUNIT & ASAP & 0.04 & 0.26 & 414544 & 330004 && -0.04 & -0.19 & 0.012 & 0.015 && pt. est. & $\rho$ adj. & NAA \\
PLAUNIT & VPA & 0.62 & 0.32 & 14439 & 10915 && -0.35 & -0.32 & 0.08 & 0.12 && $\rho$ adj. & $\rho$ adj. & NAA \\
WITUNIT & VPA & 0.61 & 0.51 & 3129 & 2077 && -0.33 & -0.38 & 0.428 & 0.687 && pt. est. & $\rho$ adj. & SSB \\
HKWUNIT & ASAP & 0.15 & 0.18 & 28553 & 24197 && -0.13 & -0.12 & 0.076 & 0.086 && pt. est. & pt. est. & none \\
POLUNIT & ASAP & 0.29 & 0.28 & 198847 & 154865 && -0.25 & -0.28 & 0.051 & 0.07 && pt. est. & $\rho$ adj. & NAA \\
CATUNIT & SCALE & 0.96 & 0.83 & 592 & 324 && -0.55 & -0.36 & 0.003 & 0.005 & pt. est. && pt. est. & none \\
   \hline
\end{tabular}
}

\end{sidewaystable}
%\end{table}

\clearpage
% latex table generated in R 3.2.1 by xtable 1.7-4 package
% Fri Nov  6 14:46:57 2015
\begin{table*}[ht]
\centering
\caption{The biomass ($B$) and exploitation rate ($F$) values used for status determination were  adjusted to account for a retrospective pattern in some stocks.   In general, when the $B$ or $F$ values adjusted for restrospective pattern ($B_{\rho}$ and $F_{\rho}$)  were outside of the approximate $90\%$ confidence interval (Conf. limits), the  $\rho$ adjusted values were used to determine stock status (Adj. $=$ Yes).  There were exceptions however, such as YELSNEMA and CODGM(M=0.2) and details regarding each decision can be found in the  report and reviewer comments sections for each stock.  Only stocks that had both an estimable 7-year Mohn's $\rho$ for $B$ and $F$ and estimable approximate  90\% confidence limits on  terminal year $B$ and $F$ values are included. } 
\label{RhoDecision_tab}
\begin{tabular}{c@{\hspace{.2cm}}c@{\hspace{.2cm}}c@{\hspace{.2cm}}c@{\hspace{.2cm}}c@{\hspace{.2cm}}c@{\hspace{.2cm}}c@{\hspace{.2cm}}c@{\hspace{.2cm}}}
  \hline
Stock & $B_{2014}$ & $B_{\rho}$ & Conf. limits & $F_{2014}$ & $F_{\rho}$ & Conf. limits & Adj? \\ 
  \hline
CODGM(M=0.2) & 2,225 & 1,443 & 1,942 - 2,892 & 0.956 &  1.39 & 0.654 - 1.387 & No \\ 
  CODGM(M ramp) & 2,536 & 2,106 & 1,921 - 3,298 & 0.932 &  1.01 & 0.662 - 1.304 & No \\ 
  HADGB & 225,080 & 150,053 & 171,911 - 301,282 & 0.159 & 0.241 & 0.13 - 0.203 & Yes \\ 
  HADGM & 10,325 & 10,712 & 7,229 - 14,453 & 0.257 &  0.25 & 0.164 - 0.373 & No \\ 
  YELSNEMA & 502 & 243 & 355 - 739 &  1.64 &  3.53 & 1.053 - 2.348 & No \\ 
  YELCCGM & 1,695 & 857 & 1,375 - 2,111 & 0.355 &  0.64 & 0.25 - 0.52 & Yes \\ 
  FLWSNEMA & 6,151 & 5,105 & 5,045 - 7,500 &  0.16 &  0.21 & 0.12 - 0.213 & No \\ 
  FLWGB & 5,275 & 2,883 & 3,783 - 6,767 & 0.379 & 0.778 & 0.254 - 0.504 & Yes \\ 
  PLAUNIT & 14,543 & 10,977 & 12,742 - 16,439 &  0.08 & 0.116 & 0.069 - 0.093 & Yes \\ 
  WITUNIT & 3,129 & 2,077 & 2,643 - 3,864 & 0.428 & 0.687 & 0.321 - 0.603 & Yes \\ 
  HWKUNIT & 28,553 & 24,197 & 24,351 - 33,480 & 0.076 & 0.086 & 0.063 - 0.092 & No \\ 
  POLUNIT & 198,847 & 154,919 & 37,243 - 255,097 & 0.051 &  0.07 & 0.084 - 0.066 & Yes \\ 
  REDUNIT & 414,544 & 330,004 & 368,906 - 465,828 & 0.012 & 0.015 & 0.011 - 0.014 & Yes \\ 
   \hline
\end{tabular}
\end{table*}





% Figures
\clearpage
%\IfFileExists{\ExSumPath/figures/stock_status.png}{
	\begin{sidewaysfigure}
		\centering	
		\adjustimage{max size={.95\textwidth}{.8\textheight}}{\ExSumPath/figures/StockStatus.png}
		%\adjustimage{max size={.95\textwidth}{.8\textheight}}{../../ExSum/figures/StockStatus.png}		
		\captionsetup{width=\textwidth}
		\caption[.]{Status of the Northeast Multispecies Fishery Management Plan (groundfish) stocks in 2007 (GARM III)  and 2014 (OA 2015) with respect to the $F_{MSY}$ and $B_{MSY}$ proxies. The 'Intermediate assessment' represents the last stock assessment conducted prior to the OA 2015 assessment (year varies by stock). Stocks on which overfishing is occurring are those where the $\frac{F_{terminal}}{F_{MSY proxy}}$ ratio is greater than 1 and overfished stocks are those where the $\frac{B_{terminal}}{B_{MSY{}proxy}}$ ratio is less than 0.5. \textit{Notes: (1) the GARM III assessments did not include wolfish; (2) for the intermediate assessments stock status could not be determined for Gulf of Maine winter flounder (OA 2014) or Georges Bank yellowtail (TRAC 2015); and, (3) based on the OA 2015 assessments stock status could not be determined for Atlantic halibut, Gulf of Maine winter flounder and Georges Bank yellowtail flounder. In the OA 2015 assessment, the stock status for Georges Bank cod remained overfished and overfishing is occurring; however, since the assessment was rejected, ratios of terminal conditions to reference points cannot be determined. Species codes: COD-Atlantic cod, HAD-haddock, POL-pollock, RED-redfish, WHK-white hake, OPT-ocean pout, CAT-wolffish, PLA-American plaice, FLW-winter flounder, YEL-yellowtail flounder, WIT-witch flounder, FLD-windowpane flounder, HAL-Atlantic halibut.}}		
		\label{stock_status}
	\end{sidewaysfigure}
	\clearpage
%}{}  %otherwise do nothing




%\IfFileExists{\ExSumPath/figures/stock_status.png}{
	\begin{figure}
		\centering	
		\adjustimage{max size={.95\textwidth}{.7\textheight}}{\ExSumPath/figures/propFmsy.png}
		%\adjustimage{max size={.95\textwidth}{.8\textheight}}{../../ExSum/figures/StockStatus.png}		
		\captionsetup{singlelinecheck=off}
		\caption[.]{Changes in the ratio of fishing mortality to FMSY proxy from 2007 (GARM III) to 2014 (OA 2015) for the twenty Northeast Multispecies Fishery Management Plan (groundfish) stocks. The results from the assessment prior to the OA 2015 assessment are shown for each stock to provide an 'Intermediate' value. Stocks on which overfishing is occurring are those where the $\frac{F_{terminal}}{F_{MSY{}proxy}}$ ratio is greater than 1. \textit{Notes: (1) the GARM III assessments did not include wolfish; (2) stock status in the 'Intermediate' assessment could not be determined for Gulf of Maine winter flounder or Georges Bank yellowtail flounder; and, (3) based on the OA 2015 assessments stock status could not be determined for Atlantic halibut, Gulf of Maine winter flounder and Georges Bank yellowtail flounder. In the OA 2015 assessment, the stock status for Georges Bank cod remained overfished and overfishing is occurring; however, since the assessment was rejected, ratios of terminal conditions to reference points cannot be determined.}}		
		\label{propFmsy}
	\end{figure}
	\clearpage
%}{}  %otherwise do nothing



%\IfFileExists{\ExSumPath/figures/stock_status.png}{
	\begin{figure}
		\centering	
		\adjustimage{max size={.95\textwidth}{.7\textheight}}{\ExSumPath/figures/propBmsy.png}
		%\adjustimage{max size={.95\textwidth}{.8\textheight}}{../../ExSum/figures/StockStatus.png}		
		\captionsetup{singlelinecheck=off}
		\caption[.]{Changes in the ratio of stock biomass to BMSY proxy from 2007 (GARM III) to 2014 (OA 2015) for the twenty Northeast Multispecies Fishery Management Plan (groundfish) stocks. The results from the assessment prior to the OA 2015 assessment are shown for each stock to provide an 'Intermediate' value. Stocks that are overfished stocks are those where the $\frac{B_{terminal}}{B_{MSY{}proxy}}$ ratio is less than 0.5. \textit{Notes: (1) the GARM III assessments did not include wolfish; (2) stock status in the 'Intermediate' assessment could not be determined for Gulf of Maine winter flounder or Georges Bank yellowtail flounder; and, (3) based on the OA 2015 assessments stock status could not be determined for Atlantic halibut, Gulf of Maine winter flounder and Georges Bank yellowtail flounder. In the OA 2015 assessment, the stock status for Georges Bank cod remained overfished and overfishing is occurring; however, since the assessment was rejected, ratios of terminal conditions to reference points cannot be determined.}}		
		\label{propBmsy}
	\end{figure}
	\clearpage
%}{}  %otherwise do nothing


%\IfFileExists{\ExSumPath/figures/stock_status.png}{
	\begin{figure}
		\centering	
		\adjustimage{max size={.95\textwidth}{.8\textheight}}{\ExSumPath/figures/nefscSpringResiduals.png}
		%\adjustimage{max size={.95\textwidth}{.8\textheight}}{../../ExSum/figures/StockStatus.png}		
		\captionsetup{singlelinecheck=off}
		\caption[.]{NEFSC spring bottom trawl survey index standardized anomalies (Z-score) for the Northeast Multispecies Fishery Management Plan (groundfish) stocks from 1968 to 2015. \textit{Note that both the Georges Bank$/$Gulf of Maine and Southern New England$/$Mid-Atlantic windowpane flounder stocks are not included since the spring survey is uninformative as an index of abundance and not used in the stock assessment.}	}	
		\label{nefscSpringResiduals}
	\end{figure}
	\clearpage
%}{}  %otherwise do nothing



%\IfFileExists{\ExSumPath/figures/stock_status.png}{
	\begin{figure}
		\centering	
		\adjustimage{max size={.95\textwidth}{.8\textheight}}{\ExSumPath/figures/nefscFallResiduals.png}
		%\adjustimage{max size={.95\textwidth}{.8\textheight}}{../../ExSum/figures/StockStatus.png}		
		\captionsetup{singlelinecheck=off}
		\caption[.]{NEFSC fall bottom trawl survey index standardized anomalies (Z-score) for the Northeast Multispecies Fishery Management Plan (groundfish) stocks from 1963 to 2014. \textit{Note that ocean pout is not included since the fall survey is uninformative as an index of abundance and not used in the stock assessment.}	}	
		\label{nefscFallResiduals}
	\end{figure}
	\clearpage
%}{}  %otherwise do nothing




%\IfFileExists{\ExSumPath/figures/stock_status.png}{
	\begin{figure}
		\centering	
		\adjustimage{max size={.95\textwidth}{.8\textheight}}{\ExSumPath/figures/nefscSpringMinSweptAreaBiomass.png}
		%\adjustimage{max size={.95\textwidth}{.8\textheight}}{../../ExSum/figures/StockStatus.png}		
		\captionsetup{singlelinecheck=off}
		\caption[.]{NEFSC spring bottom trawl survey minimum swept area biomass (mt) for the Northeast Multispecies Fishery Management Plan (groundfish) stocks from 1968 to 2015, by stock. Minimum swept area estimates assume a trawl swept area of 0.0112 $nm^{2}$) (0.0384 $km^{2}$) based on the wing spread of the trawl net. \textit{Note that both the Georges Bank$/$Gulf of Maine and Southern New England$/$Mid-Atlantic windowpane flounder stocks are not included since the spring survey is uninformative as an index of abundance and not used in the stock assessment.}	}	
		\label{nefscSpringMinSweptAreaBiomass}
	\end{figure}
	\clearpage
%}{}  %otherwise do nothing


%\IfFileExists{\ExSumPath/figures/stock_status.png}{
	\begin{figure}
		\centering	
		\adjustimage{max size={.95\textwidth}{.8\textheight}}{\ExSumPath/figures/nefscFallMinSweptAreaBiomass.png}
		%\adjustimage{max size={.95\textwidth}{.8\textheight}}{../../ExSum/figures/StockStatus.png}		
		\captionsetup{singlelinecheck=off}
		\caption[.]{NEFSC fall bottom trawl survey minimum swept area biomass (mt) for for the Northeast Multispecies Fishery Management Plan (groundfish) stocks from 1963 to 2014, by stock. Minimum swept area estimates assume a trawl swept area of 0.0112 $nm^{2}$ (0.0384 $km^{2}$) based on the wing spread of the trawl net. \textit{Note that ocean pout is not included since the fall survey is uninformative as an index of abundance and not used in the stock assessment.}	}
		\label{nefscFallMinSweptAreaBiomass}
	\end{figure}
	\clearpage
%}{}  %otherwise do nothing


%\IfFileExists{\ExSumPath/figures/stock_status.png}{
	\begin{figure}
		\centering	
		\adjustimage{max size={.95\textwidth}{.8\textheight}}{\ExSumPath/figures/propBiomassHaddockRedfish.png}
		%\adjustimage{max size={.95\textwidth}{.8\textheight}}{../../ExSum/figures/StockStatus.png}		
		\captionsetup{singlelinecheck=off}
		\caption[.]{Proportion of the total groundfish swept minimum swept area biomass contributed by Georges Bank and Gulf of Maine haddock and Redfish based on the NEFSC spring and fall bottom trawl surveys.}
		\label{propBiomassHaddockRedfish}
	\end{figure}
	\clearpage
%}{}  %otherwise do nothing


\clearpage
