
\section{Executive Summary}
\subsection{Process}
 
Assessments for all 20 groundfish stocks (Table \ref{stock_abbrv_tab}) in the New England Fishery Management Council’s (NEFMC) Multispecies Groundfish Fisheries Management Plan were updated and reviewed during September 14-18, 2015 at the Northeast Fisheries Science Center (NEFSC), Woods Hole, MA. This represents the fourth comprehensive assessment of the status of  all the  groundfish stocks since 2001. The first three comprehensive assessments were produced through the Groundfish Assessment Review Meeting (GARM) process (NEFSC 2002, 2005, 2008).  Thirteen of the groundfish stocks were updated through the Operational Assessment process (NEFSC 2012).  Operational assessments, first described by the Northeast Regional Coordinating Council (NRCC) in 2011, rely on decisions of previous benchmarks for model formulation and definition of biological reference points (BRPs).    The terms of reference for the operational assessments are provided in Section \ref{TOR}.  The efficiency of the Operational Assessment process increases the frequency of assessments, but reduces the ability to modify model structure either in response to new data or external inputs.  Major modifications of the assessment models are restricted to benchmark assessments that can incorporate a much greater range of information but for far fewer stocks.   In this context, the scope of admissible changes in the assessment was summarized in a letter from the NRCC (\ref{NRCCletter}).  Of particular note, newly available research resulted in modifications of discard mortality rates applied in several assessments.  

On July 22, 2015 the NEFSC held 5 port-based outreach meetings for fishermen and other stakeholders. These occurred in Maine (Portland), New Hampshire (Hampton), and Massachusetts (Gloucester, Woods Hole, New Bedford).  Assessment analysts met with attendees at each location to learn more about recent observations from the fleet and ports that might help focus future research to improve assessments and interpret patterns in the current assessments. Each meeting started with a brief introduction on the timeline for the assessments, what new information would be considered, and how the results would be reviewed before use in the fishery management process. 

This was not the first time  outreach meetings have been held for industry ahead of an assessment, but this is the first time that summaries of the outreach meetings are included in the assessment report and were provided to peer reviewers ahead of their review meeting. The summaries were prepared from notes taken by NEFSC communications staff, and then provided to meeting attendees for comment before they were finalized for publication.  See Section \ref{OutreachSum} for details.  A formal statement from a fishing industry member was made available at the meeting and is provided in Section \ref{IndLetter}. 

The NRCC guidance was taken into consideration by the Assessment Oversight Panel, which reviewed the plans for each assessment prepared by the individual analysts.  See Section \ref{AOPsum} for a summary of the July 27, 2015 meeting.  Given the relatively new process associated with these operational assessments, the NEFSC made an extra effort to promote understanding of the mechanism ahead of the peer review meeting.  These efforts included a webinar/seminar for in-house outreach staff, sector managers, and NEFMC groundfish and recreational fishing advisors on July 20, and a data-rich dedicated \href{http://www.nefsc.noaa.gov/groundfish/operational-assessments-2015/}{website}{}.

The Peer Review Panel (i.e., Panel) consisted of the following individuals:
\begin{itemize}
\item Steven X. Cadrin,  (Chair) School for  Marine Science and Technology, University of Massachusetts, N. Dartmouth, MA, NEFMC Scientific and Statistical Committee
\item Jean-Jacques Maguire, NEFMC Scientific and Statistical Committee
\item Gary Nelson, Massachusetts Division of Marine Fisheries, Gloucester, MA
\item Jim Berkson, NMFS Office of Science and Technology, Silver Spring,  MD.
\end{itemize}

The Panel was responsible for reviewing each of the stock assessments.  Primary and supporting documents for each assessment were available prior to the meeting. Each lead assessment scientist (Table \ref{stock_info_tab}) prepared a short presentation to describe the assessment results and address key sources of uncertainty (See \href{http://www.nefsc.noaa.gov/groundfish/operational-assessments-2015/agenda.html}{agenda}). Following the presentation, the Panel was responsible for addressing four topics:

\begin{itemize}
\item Accept/ Not Accept assessment as a basis for setting Overfishing Limit (OFL)
\item If assessment not accepted, then recommend alternative basis for setting OFL
\item Identify key sources of uncertainty
\item Identify important research needs
\end{itemize}

If an assessment was not considered suitable for estimation of OFL the Panel was responsible for approving an alternative basis, such as some function of recent average catch.  

The individual assessment sections within this report are standardized and designed to capture the most relevant information for reviewers and fishery managers. The report structure was developed with and approved by a subcommittee of the NRCC, followed by NRCC review of the report structure.  Each assessment is  supported by an online set of companion tables, figures and maps which provide primary users of the assessment information (e.g., Plan Development Teams, Science and Statistical Committee) with necessary  details.  The online data portal (\href{http://www.nefsc.noaa.gov/saw/sasi/sasi_report_options.php}{SASINF}{}) also contains model inputs and outputs that can be used directly in \href{http://nft.nefsc.noaa.gov/}{NOAA Fisheries Toolbox}{} applications.  

The meeting was broadcast as a webinar using Adobe Connect and all sessions were open to the public. The meeting agenda included a daily public comment period.   Members of the audience and individuals on the phone were included in the discussions of the panel.  However, the tight timeline for completing the assessments required a strong adherence to the terms of reference and guidance from the NRCC.  Onsite participants in Woods Hole are listed in Section \ref{Participants}.

\subsection{Data}
The groundfish updates used the following standard procedures for updating data from landings, discards and surveys (Table \ref{data_used_tab}). The US commercial landings are estimated by market category from the area allocation (``AA") tables, which combine dealer and vessel trip reports to determine where fish were caught. The US commercial discards are estimated by gear types using the Standardized Bycatch Reporting Methodology (SBRM), which combines observer data (including at-sea monitors) and dealer landings. The US recreational landings and discards come from the Marine Recreational Information Program (MRIP), including recent revisions to historical data. Both commercial and recreational discards have species-specific discard mortality rates applied to the discarded fish. Catch-at-age is estimated using age-length keys applied to expanded length frequency distributions. For white hake, which is landed headed, the age-length key is applied to predicted lengths based on dorsal fin to caudal fin length.  Additional sources of catch for some species come from Canadian or other foreign fishing. 

The NEFSC spring and fall bottom trawl surveys are the most common source of information for population trends (Table \ref{data_used_tab}). These surveys are calibrated to ``Albatross units" in most cases to allow for the longest time series possible. NOAA ship \textit{Henry B. Bigelow} replaced the \textit{Albatross IV} as the primary bottom trawl survey vessel in spring 2009.  In some instances the calibration coefficient varies by length but in others a simple scalar adjustment is applied to all length classes. Other surveys used include the Massachusetts Division of Marine Fisheries spring and fall bottom trawl surveys, the Maine-New Hampshire spring and fall bottom trawl surveys, the Canadian Department of Fisheries and Oceans February survey, and some additional state surveys. Catch per unit effort is not typically used as a source of population trends due to the many regulatory changes that have occurred over time in the Northeast.  All updated assessments used a consistent quality assurance criterion  (known as TOGA; Politis et al. 2014) for surveys conducted by the NOAA ship \textit{Henry B. Bigelow}.

\subsection{Models}
Based on previous benchmark assessments (Table \ref{Assess_type_tab}), there are 13 stocks assessed with an age-based approach.  Eight use the statistical catch-at-age model ASAP while the other 5 use virtual population analysis (VPA). The stock assessments using ASAP were all configured to not include the likelihood constants due to a potential bias associated with the `Use Likelihood Constants' option, as agreed by both the AOP and Review Panel. For the 5 VPA stocks, the 2015 spring survey information was included in the model. The remaining 7 stocks are assessed with a range of model types including surplus production, length-based (SCALE), index (AIM), and direct survey expansion. The reference points for the age- and length-based assessments were derived from stochastic projections of the $F_{MSY}$ (or $F_{MSY}$ proxy) for many years (typically 100) while the other assessment types use stock-specific rules for deriving the reference points.

\subsection{Results}

Operational Assessments were conducted in 2015 for the 20 stocks in the Northeast Multispecies Fishery Management Plan (Table \ref{stock_abbrv_tab}). The updates replicated the methods recommended in the most recent benchmark decisions, as modified by any subsequent operational assessments or updates (Table \ref{stock_info_tab}). Information supplemental to the assessment report for each stock can found on the Stock Assessment Support Information (\href{http://www.nefsc.noaa.gov/saw/sasi/sasi_report_options.php}{SASINF}{}) website. The Review Panel accepted 18 of the 20 assessments as a scientific basis for management and provided estimates of overfishing limits (OFLs) for all 20 stocks.  The 2 stock assessments which were not accepted as a basis for setting OFLs were Georges Bank cod and Atlantic halibut.  Stock status did not change for 15 of the 20 stocks, worsened for 2 stocks, improved for 1 stock, and became more uncertain for 2 stocks (Table \ref{sos_tab}).

Each of the 20 species chapters contains the assessment results provided to the Panel for peer review followed by a section entitled ``Reviewer Comments", which describes final Panel decisions at the conclusion of the peer review.  For most of the stocks, the assessment results and the ``Reviewer Comments" are consistent with each other.  However, for those stock assessments that were not accepted by the Panel (e.g., Georges Bank cod and Atlantic halibut), the ``Reviewer Comments" pertaining to stock status differ from those in the ``State of Stock" and ``Special Comments" sections of those chapters which were written prior to the peer review. Although the Panel agreed to include these two assessments in this report as one interpretation of the available information, it is important to note that in these cases the Panel drew different conclusions about stock status and about the appropriate basis for catch advice. In the Executive Summary, tables and figures related to stock status reflect the Panel decisions. Specifically, for both Georges Bank cod and Atlantic halibut overfishing is described as unknown and both stocks are described as overfished (Table \ref{sos_tab}). Furthermore, for these two stocks, estimates from the updated assessment models are not provided (Table \ref{BFEstimates}; Figures \ref{propFmsy} and \ref{propBMSY}).

The number of stocks with retrospective adjustments (also called rho adjustments)  applied increased from the last assessment from 2 to 7 (Table \ref{RhoAdjust_tab}). The previous Georges Bank cod assessment did apply a retrospective adjustment; however, the assessment model was not accepted  at the 2015 Operational Assessment so it has been excluded from these counts.  Decisions to apply a retrospective adjustment to estimates of terminal year biomass and fishing mortality rates were based on several considerations.   A primary consideration was whether the rho adjusted value was outside the joint confidence region for the model estimates.  This principle was applied to adjust biomass estimates for Georges Bank haddock, Cape Cod/Gulf of Maine yellowtail flounder, Georges Bank winter flounder, American plaice and redfish (Table \ref{RhoDecision_tab}).  This principle was not applied to Gulf of Maine cod because of earlier guidance from the SARC 55 review panel.  Despite the presence of a significant retrospective pattern at that meeting no adjustments were made; the review panel for the Operational Assessments followed that precedent.   Biomass and F estimates for 2014 for Southern New England/Mid-Atlantic  yellowtail flounder also fell outside the joint confidence region. The Review Panel did not suggest application of the rho adjustment in this case because the majority  of the rho-adjusted biomass estimates were insufficient to support the projected catches in 2015.  This reduced the reliability of those biomass estimates as a basis for estimating OFL in 2016. Finally, the 2014 biomass estimate  for pollock was inside the confidence interval, but the fishing mortality estimate exceeded the upper bound of the model based confidence interval. A retrospective adjustment was therefore applied. 

Stock status for the 20 groundfish stocks is summarized in Tables \ref{sos_tab} and \ref{BFEstimates}.  While the number of overfished stocks and stocks experiencing overfishing has generally decreased since 2007 (Figure \ref{stock_status}), the magnitude of overfishing or depletion for several stocks has worsened considerably (Figures \ref{propFmsy} and \ref{propBMSY}); Gulf of Maine cod, Southern New England/Mid-Atlantic yellowtail flounder, witch flounder and Cape Cod/Gulf of Maine yellowtail flounder. Of those Northeast groundfish stocks for which stock status can be determined, the majority remain below their biomass targets ($69\%$; Figures \ref{propBMSY} and \ref{stock_status}).

Simultaneous assessments of all 20 groundfish stocks allowed a comprehensive examination of trends in spring and fall survey indices (Figures \ref{nefscSpringResiduals} and \ref{nefscFallResiduals}, respectively).  For the majority of stocks the average of the most recent 5 years is below the time series mean for that stock.

Estimates of overall (aggregate) groundfish minimum swept area biomass are at, or near, an all-time high (Figures \ref{nefscSpringMinSweptAreaBiomass} and \ref{nefscFallMinSweptAreaBiomass}). However, the current stock diversity of the overall groundfish biomass is less than that seen in the 1960s and 1970s. Current groundfish biomass is dominated by only a few stocks. For example, the combined biomass of the Georges Bank haddock, Gulf of Maine haddock, and redfish stocks currently constitute more than $80\%$ of the overall groundfish biomass observed in the surveys (Figure \ref{ModelB}).  It is important to note that the minimum swept area biomass estimates assume a common capture efficiency of 1.0 across all years.  Actual biomasses, as derived from models, are adjusted for catchability and selectivity estimates and are higher than the swept area estimates. Unfortunately model-based estimates are not available for all stocks over the entire time period of the surveys (ie. since 1963); the primary limitation is the availability of age information from the commercial catches that would be needed to support full age-based assessments.

For 13 stocks, model-based biomass estimates can be computed for 1985 onward.  The striking increase in abundance since 1985 is driven primarily by redfish, Georges Bank haddock, and pollock (Figure \ref{ModelB}). Pollock biomass from the stock assessment is much higher than the swept area estimates because of a dome-shaped selectivity pattern in both the survey and catch data. This suggests that a large fraction of the stock biomass is unavailable to either the fishery or survey gear (note however, that traditional stock assessment methods cannot confirm or deny this assertion so caution was suggested by the Review Panel). The increase in model based estimates of overall biomass, with or without pollock, is consistent with the trends revealed in the swept area estimates (Figures \ref{nefscSpringMinSweptAreaBiomass}, \ref{nefscFallMinSweptAreaBiomass} and \ref{ModelB}).   

The rapid increase in haddock, redfish, pollock and white hake contrasts sharply with the decline of cod and the flatfish species (Figure \ref{ContrastBTrends}). Total biomass of haddock, redfish, pollock and white hake have increased from less than 200 kt in 1994 to nearly 900 kt in 2014. Cod and the flatfish stocks have declined from about 140 kt to about 40kt  over the same period. Underlying causes for the decline are not known, but fishing mortality, poor recruitment, and ecosystem changes are possible causes. The widely differing responses of haddock and cod, species with similar habitats and patterns of co-occurrence are especially worthy of study.  One important contrast is that haddock age composition has gradually rebuilt following the imposition of management restrictions in 1994 and a series of strong to very strong year classes have led to rapid increases in spawning stock biomass.  In contrast, cod, which exhibits less extreme variations in recruitment, did not have a rapid increase in spawning stock biomass nor has it increased following strong year classes. 

An advantage of conducting multiple assessments simultaneously is that measures of productivity can be compared over time. Reductions in average weight-at-age, declines in recruitment and shifts in age-at-maturity all influence the estimated biomass at maximum sustainable yield and total $MSY$.  As such, the single species stock assessments provide valuable measures of ecosystem productivity, irrespective of the underlying environmental or ecological causes.   Reductions in average weights-at-age have occurred for stocks at  high  abundance, such as Georges Bank haddock, but also for stocks at low abundance, such as witch flounder.  Hence, density dependence alone is  insufficient to explain this across all stocks.  Reductions in recruitment are often associated with declines in stock size but inter-annual variation often masks trends.  Aggregate estimates of total $B_{MSY}$ are available for 10 stocks over the past decade (Figure \ref{SumBMSY}).  Total $B_{MSY}$ for these stocks declined by $12\%$ between 2005 and  2008 from 760 kt to 668 kt.  Estimates further declined by $21\%$ between 2008 and 2015 to 525 kt (Figure \ref{SumBMSY}). 

\subsection*{References}
Politis PJ, Galbraith JK, Kostovick P, Brown RW. 2014. Northeast Fisheries Science Center bottom trawl survey protocols for the NOAA Ship Henry B. Bigelow. US Dept Commer, Northeast Fish Sci Cent Ref Doc. 14-06; 138 p. Available from: National Marine Fisheries Service, 166 Water Street, Woods Hole, MA 02543-1026. \href{http://nefsc.noaa.gov/publications/}{CRD14-06}


%%%%%%%%%%%%%%%%%%%%%%%%%%%%%%%%%%%%%%%%%%%%%%%%%%%%%%%%%%%%%%%%%%%%%%%%%%%%%%%%%%%%%%%%%%%%%%%%%%%%%%%
%Tables
\clearpage
%\subsection{Tables}
\begin{table}
	\centering
	
	\caption{ List of stocks included in the groundfish operational assessment and the abbreviations used for each in this document.}
	\label{stock_abbrv_tab}
	\begin{tabular}{ll}
\hline
Stock Abbrev & Stock Name \\
\hline 
CODGM & Gulf of Maine cod \\
CODGB & Georges Bank cod \\
HADGM & Gulf of Maine haddock \\
HADGB & Georges Bank haddock \\
YELCCGM & Cape Cod/Gulf of Maine yellowtail flounder \\
YELSNEMA & Southern New England/Mid-Atlantic yellowtail flounder \\
FLWGB & Georges Bank winter flounder \\
FLWSNEMA & Southern New England/Mid-Atlantic winter flounder \\
REDUNIT & Acadian redfish \\
PLAUNIT & American plaice \\
WITUNIT & Witch flounder \\
HKWUNIT & White hake \\
POLUNIT & Pollock \\
CATUNIT & Wolffish \\
HALUNIT & Atlantic halibut \\
FLDGMGB & Gulf of Maine/Georges Bank windowpane flounder \\
FLDSNEMA & Southern New England/Mid-Atlantic windowpane flounder \\
OPTUNIT & Ocean pout \\
FLWGM & Gulf of Maine winter flounder \\
YELGB & Georges Bank yellowtail flounder \\
\hline
	\end{tabular}
\end{table}
\clearpage
\newcolumntype{L}{>{\centering}m{2cm}} 
\newcolumntype{O}{>{\centering}m{3cm}}
\newcolumntype{P}{>{\centering}m{1.8cm}} 
\begin{sidewaystable}[ht]  
	\centering
	\captionsetup{width=\textwidth}
	\caption{Lead scientist for each stock (current$/$previous if different), information about last assessment, including: the forum for review of the last assessment (Forum), the type of assessment done (Type), publication year (Pub.), the terminal year of the catch data included (Term. yr.), overfished/overfishing status, rebuilding status, and reference. \textit{Note: Op. Assess $=$ Operational Assessment}}
	\label{stock_info_tab}
	\small{	
	\begin{tabular}{
	m{2cm}@{\hspace{.1cm}}
	O@{\hspace{.1cm}}
	m{2cm}@{\hspace{.1cm}}
	m{1.75cm}@{\hspace{.1cm}}
	m{1.cm}@{\hspace{.1cm}}
	m{1cm}@{\hspace{.1cm}}
	P@{\hspace{.1cm}}
	P@{\hspace{.1cm}}
	m{1.5cm}@{\hspace{.1cm}}
	m{1.5cm}@{\hspace{.1cm}}
	}
	\hline

Stock & 
Lead & 
Forum & 
Type & 
Pub. & 
Term. yr. & 
Overfished? & 
Overfishing? & 
Rebuild status & 
Reference \\
	
	\hline
CODGM & Palmer & Op. Assess & Update & 2014 & 2013 & Yes & Yes & By 2024 & \href{http://www.nefsc.noaa.gov/publications/crd/crd1414/}{CRD14-14} \\
CODGB & O'Brien & SARC 55 & Benchmark & 2012 & 2011 & Yes & Yes & By 2026 & \href{http://nefsc.noaa.gov/publications/crd/crd1311/}{CRD13-11} \\
HADGM & Palmer & SARC 59 & Benchmark & 2014 & 2013 & No & No & Rebuilt & \href{http://nefsc.noaa.gov/publications/crd/crd1409/}{CRD14-09} \\
HADGB & Brooks & GARM2012 & Update & 2012 & 2010 & No & No & Rebuilt & \href{http://www.nefsc.noaa.gov/publications/crd/crd1206/}{CRD12-06} \\
YELCCGM & Alade$/$Legault & GARM2012 & Update & 2012 & 2010 & Yes & Yes & By 2023 & \href{http://www.nefsc.noaa.gov/publications/crd/crd1206/}{CRD12-06} \\
YELSNEMA & Alade & SARC 54 & Benchmark & 2012 & 2011 & No & No & Rebuilt & \href{http://www.nefsc.noaa.gov/publications/crd/crd1218/}{CRD12-18} \\
FLWGB & Hendrickson & Op. Assess & Update & 2015 & 2013 & No & No & By 2017 & \href{http://www.nefsc.noaa.gov/publications/crd/crd1501/}{CRD15-01} \\
FLWSNEMA & Wood$/$Terciero & SARC 52 & Benchmark & 2011 & 2010 & Yes & No & By 2023 &  \href{http://www.nefsc.noaa.gov/saw/saw52/crd1117.pdf}{SARC52} \\
REDUNIT & Linton$/$Miller & GARM2012 & Update & 2012 & 2010 & No & No & Rebuilt & \href{http://www.nefsc.noaa.gov/publications/crd/crd1206/}{CRD12-06} \\
PLAUNIT & O'Brien & GARM2012 & Update & 2012 & 2010 & No & No & By 2024 & \href{http://www.nefsc.noaa.gov/publications/crd/crd1206/}{CRD12-06} \\
WITUNIT & Wigley & GARM2012 & Update & 2012 & 2010 & Yes & Yes & By 2017 & \href{http://www.nefsc.noaa.gov/publications/crd/crd1206/}{CRD12-06} \\
HKWUNIT & Sosebee & SARC 56 & Benchmark & 2013 & 2011 & No & No & By 2014 & \href{http://www.nefsc.noaa.gov/publications/crd/crd1310/}{CRD13-10} \\
POLUNIT & Linton & Op. Assess & Update & 2015 & 2013 & No & No & Rebuilt & \href{http://www.nefsc.noaa.gov/publications/crd/crd1501/}{CRD15-01} \\
CATUNIT & Adams$/$Keith & GARM2012 & Update & 2012 & 2010 & Yes & No & Unknown & \href{http://www.nefsc.noaa.gov/publications/crd/crd1206/}{CRD12-06} \\
HALUNIT & Hennen$/$Blaylock & GARM2012 & Update & 2012 & 2010 & Yes & No & By 2055 & \href{http://www.nefsc.noaa.gov/publications/crd/crd1206/}{CRD12-06} \\
FLDGMGB & Chute$/$Hendrickson & GARM2012 & Update & 2012 & 2010 & Yes & Yes & By 2017 & \href{http://www.nefsc.noaa.gov/publications/crd/crd1206/}{CRD12-06} \\
FLDSNEMA & Chute$/$Hendrickson & GARM2012 & Update & 2012 & 2010 & No & No & Rebuilt & \href{http://www.nefsc.noaa.gov/publications/crd/crd1206/}{CRD12-06} \\
OPTUNIT & Wigley & GARM2012 & Update & 2012 & 2010 & Yes & No & By 2014 & \href{http://www.nefsc.noaa.gov/publications/crd/crd1206/}{CRD12-06} \\
FLWGM & Nitschke & Op. Assess & Update & 2015 & 2013 & Unknown & No & Unknown & \href{http://www.nefsc.noaa.gov/publications/crd/crd1501/}{CRD15-01} \\
YELGB & Legault & TRAC 2015 & Update & 2015 & 2014 & Unknown & Unknown & By 2032 & \href{http://www.nefsc.noaa.gov/saw/trac/TSR_2015_GBYellowTailFlounder.pdf}{TRAC2015}\\
	\hline
	\end{tabular}
}
\end{sidewaystable}



\clearpage
\newcommand{\colspc}{.2cm}
\newcolumntype{Q}{c@{\hspace{\colspc}}}

\begin{sidewaystable}[ht]
%\begin{table}
\captionsetup{width=\textwidth}
\centering
\caption{Data used in each assessment. The column heads are US commercial landings (US c-lnd), US commercial discards (US c-dis), US recreational landings (US r-lnd), US recreational discards (US r-dis), Canadian catch (CA cat), Northeast Fisheries Science Center spring, fall and winter surveys (NE S, NE F and NE W), Massachusetts spring and fall surveys (MA S and MA F), Maine/New Hampshire spring and fall surveys (ME/NH S and ME/NH F) and Canadian Department of Fisheries and Oceans February survey (DFO S).} 
\label{data_used_tab}
{\small
\begin{tabular}{l@{\hspace{\colspc}}
Q 
Q 
Q 
Q 
Q
m{1mm} 
Q 
Q 
Q 
Q 
Q 
Q 
Q 
Q 
}

\hline

& \multicolumn{5}{c}{\textbf{Catch}} && \multicolumn{8}{c}{\textbf{Surveys}} \\  
\cline{2-6} \cline{8-15}
Stock & US c-lnd & US c-dis & US r-lnd & US r-dis & CA Cat && NE S & NE F & NE W & MA S & MA F & ME/NH S & ME/NH F & DFO S \\

\hline
CODGM & Yes & Yes & Yes & Yes & No && Yes & Yes & No & Yes & No & No & No & No \\
CODGB & Yes & Yes & Yes & Yes & Yes && Yes & Yes & No & No & No & No & No & Yes \\
HADGM & Yes & Yes & Yes & Yes & No && Yes & Yes & No & No & No & No & No & No \\
HADGB & Yes & Yes & No & No & Yes && Yes & Yes & No & No & No & No & No & Yes \\
YELCCGM & Yes & Yes & No & No & No && Yes & Yes & No & Yes & Yes & Yes & Yes & No \\
YELSNEMA & Yes & Yes & No & No & No && Yes & Yes & Yes & No & No & No & No & No \\
FLWGB & Yes & Yes & No & No & Yes && Yes & Yes & No & No & No & No & No & Yes \\
FLWSNEMA & Yes & Yes & Yes & Yes & No && Yes & Yes & Yes & Yes & No & No & No & No \\
REDUNIT & Yes & Yes & No & No & No && Yes & Yes & No & No & No & No & No & No \\
PLAUNIT & Yes & Yes & No & No & Yes && Yes & Yes & No & Yes & Yes & No & No & No \\
WITUNIT & Yes & Yes & No & No & No && Yes & Yes & No & No & No & No & No & No \\
HKWUNIT & Yes & Yes & No & No & Yes && Yes & Yes & No & No & No & No & No & No \\
POLUNIT & Yes & Yes & Yes & Yes & No && Yes & Yes & No & No & No & No & No & No \\
CATUNIT & Yes & Yes & Yes & No & No && Yes & Yes & No & Yes & No & No & No & No \\
HALUNIT & Yes & Yes & No & No & Yes &&  No & Yes & No & No & No & No & No & No \\
FLDGMGB & Yes & Yes & No & No & No && No & Yes & No & No & No & No & No & No \\
FLDSNEMA & Yes & Yes & No & No & No &&  No & Yes & No & No & No & No & No & No \\
OPTUNIT & Yes & Yes & No & No & No && Yes & No & No & No & No & No & No & No \\
FLDWGM & Yes & Yes & Yes & Yes & No && Yes & Yes & No & Yes & Yes & Yes & Yes & No \\
YELGB & Yes & Yes & No & No & Yes &&  Yes & Yes & No & No & No & No & No & Yes \\
   \hline
\end{tabular}
}

\end{sidewaystable}
%\end{table}
\clearpage
\newcolumntype{L}{>{\centering}m{2cm}}
\newcolumntype{M}{>{\centering}m{2.25cm}}
\newcolumntype{N}{>{\centering}m{1cm}}


\begin{sidewaystable}[ht]  
	\centering
	\captionsetup{width=\textwidth}
	\caption{Assessment type and reference points from previous assessment. Biomass and yield values are in metric tons. \textit{Note: sp=stochastic projection and surv. B = survey biomass.}}
	\label{Assess_type_tab}
	\small{	
	\begin{tabular}{
	m{2.5cm}@{\hspace{.1cm}}
	m{1.25cm}@{\hspace{.1cm}}
	M@{\hspace{.1cm}}   
	L@{\hspace{.2cm}}
	N@{\hspace{.2cm}}   
	L@{\hspace{.1cm}}
	m{0.75cm}@{\hspace{.1cm}}
	L@{\hspace{.2cm}}
	m{1.5cm}@{\hspace{.1cm}}
	L@{\hspace{.2cm}}
	m{1.5cm}@{\hspace{0cm}}
	}

	\hline

Stock & Assess. & Type & F def. & B def. & $F_{MSY}$ type & $F_{MSY}$ value & $B_{MSY}$ type & $B_{MSY}$ value & MSY type & MSY value \\
\hline
CODGM(M=.2) & ASAP & age-based & $F_{Full}$ & SSB & $F_{40\%SPR}$ & 0.18 & sp & 47,184 & sp & 7,753 \\
CODGM($M_{ramp}$) & ASAP & age-based & $F_{Full}$ & SSB & $F_{40\%SPR}$ & 0.18 & sp & 69,621 & sp & 11,388 \\
CODGB & ASAP & age-based & $F_{Full}$ & SSB & $F_{40\%SPR}$ & 0.18 & sp & 186,535 & sp & 30,622 \\
HADGM & ASAP & age-based & $F_{Full}$ & SSB & $F_{40\%SPR}$ & 0.46 & sp & 4,108 & sp & 955 \\
HADGB & VPA & age-based & avg F ages 5-7 & SSB & $F_{40\%SPR}$ & 0.39 & sp & 124,900 & sp & 28,000 \\
YELCCGOM & VPA & age-based & avg F ages 4-6 & SSB & $F_{40\%SPR}$ & 0.26 & sp & 7,080 & sp & 1,600 \\
YELSNEMA & ASAP & age-based & avg F ages 4-5 & SSB & $F_{40\%SPR}$ & 0.32 & sp & 2,995 & sp & 773 \\
FLWGB & VPA & age-based & avg F ages 4-6 & SSB & Fmsy & 0.44 & sp & 8,100 & sp & 3,200 \\
FLWSNEMA & ASAP & age-based & avg F ages 4-5 & SSB & Fmsy & 0.29 & sp & 43,661 & sp & 11,728 \\
REDUNIT & ASAP & age-based & $F_{Full}$ & SSB & $F_{50\%SPR}$ & 0.04 & sp & 238,480 & sp & 8,891 \\
PLAUNIT & VPA & age-based & avg F ages 6-9 & SSB & $F_{40\%SPR}$ & 0.18 & sp & 18,398 & sp & 3,385 \\
WITUNIT & VPA & age-based & avg F ages 8-11 & SSB & $F_{40\%SPR}$ & 0.27 & sp & 10,051 & sp & 2,075 \\
HKWUNIT & ASAP & age-based & $F_{Full}$ & SSB & $F_{40\%SPR}$ & 0.20 & sp & 32,400 & sp & 5,630 \\
POLUNIT & ASAP & age-based & avg F ages 5-7 & SSB & $F_{40\%SPR}$ & 0.27 & sp & 76,879 & sp & 14,791 \\
CATUNIT & SCALE & length-based & $F_{Full}$ & SSB & $F_{40\%SPR}$ & 0.33 & sp & 1,756 & sp & 261 \\
HALUNIT & RYM & \centering{surplus production} & \centering{biomass wted F} & B & F0.1 & 0.07 & deterministic & 48,509 & \center{deterministic} & 3,546 \\
FLDGMGB & AIM & index & $\frac{catch}{surv. B}$ & surv. B & replacement ratio & 0.44 & $\frac{MSY{}\textit{proxy}}{F_{MSY proxy}}$ & 1.60 & median catch 1995-2001 & 700 \\
FLDSNEMA & AIM & index & $\frac{catch}{surv. B}$ & surv. B & replacement ratio & 2.09 & $\frac{MSY{}\textit{proxy}}{F_{MSY proxy}}$ & 0.24 & median catch 1995-2001 & 500 \\
OPTUNIT & index & index & $\frac{catch}{surv. B}$ & surv. B & med. $F_{1977-1985}$ & 0.76 & med. surv. $B_{1977-1985}$ & 4.94 & \centering{$F_{MSY}$ * $B_{MSY}$} & 3,754 \\
FLWGM & empirical & \centering{survey expansion} & $\frac{catch}{B_{30+cm}}$ & surv. B &  \centering{$F_{40\%}$ from YPR} & 0.23 & NA & NA & NA & NA \\
YELGB & empirical & survey expansion & NA & surv. B & NA & NA & NA & NA & NA & NA \\


	\hline
	\end{tabular}
}
\end{sidewaystable}


%\centering{7,753 ($M=0.2$) or 11,388 (Mramp)}
%\centering{47,184 (M=0.2) or 69,621 (Mramp)}
\clearpage
%sos_tab
\begin{table}
	\centering	
	\caption{ Synopsis of status by stock.}
	\label{sos_tab}
	\begin{tabular}{lcccc}
	\hline


Stock & Last Assessment & Status Change? & Overfishing? & Overfished? \tabularnewline
\hline
CODGM & 2014 & \cellcolor{yellow} Same & \cellcolor{red} Yes & \cellcolor{red} Yes \tabularnewline
CODGB & 2012 & \cellcolor{orange} More uncertain & \cellcolor{yellow} Unknown & \cellcolor{red} Yes \tabularnewline
HADGM & 2012 & \cellcolor{yellow} Same & \cellcolor{green} No & \cellcolor{green} No \tabularnewline
HADGB & 2014 & \cellcolor{yellow} Same & \cellcolor{green} No & \cellcolor{green} No \tabularnewline
YELCCGM & 2012 & \cellcolor{yellow} Same & \cellcolor{red} Yes & \cellcolor{red} Yes \tabularnewline
YELSNEMA & 2012 & \cellcolor{red} Worse & \cellcolor{red} Yes & \cellcolor{red} Yes \tabularnewline
FLWGB & 2014 & \cellcolor{red} Worse & \cellcolor{red} Yes & \cellcolor{red} Yes \tabularnewline
FLWSNEMA & 2011 & \cellcolor{yellow} Same & \cellcolor{green} No & \cellcolor{red} Yes \tabularnewline
REDUNIT & 2012 & \cellcolor{yellow} Same & \cellcolor{green} No & \cellcolor{green} No \tabularnewline
PLAUNIT & 2012 & \cellcolor{yellow} Same & \cellcolor{green} No & \cellcolor{green} No \tabularnewline
WITUNIT & 2012 & \cellcolor{yellow} Same & \cellcolor{red} Yes & \cellcolor{red} Yes \tabularnewline
HKWUNIT & 2013 & \cellcolor{yellow} Same & \cellcolor{green} No & \cellcolor{green} No \tabularnewline
POLUNIT & 2014 & \cellcolor{yellow} Same & \cellcolor{green} No & \cellcolor{green} No \tabularnewline
CATUNIT & 2012 & \cellcolor{yellow} Same & \cellcolor{green} No & \cellcolor{red} Yes \tabularnewline
HALUNIT & 2012 & \cellcolor{orange} More uncertain & \cellcolor{yellow} Unknown & \cellcolor{red} Yes \tabularnewline
FLDGMGB & 2012 & \cellcolor{green} Better & \cellcolor{green} No & \cellcolor{red} Yes \tabularnewline
FLDSNEMA & 2012 & \cellcolor{yellow} Same & \cellcolor{green} No & \cellcolor{green} No \tabularnewline
OPTUNIT & 2012 & \cellcolor{yellow} Same & \cellcolor{green} No & \cellcolor{red} Yes \tabularnewline
FLWGM & 2014 & \cellcolor{yellow} Same & \cellcolor{green} No & \cellcolor{yellow} Unknown \tabularnewline
YELGB & 2014 & \cellcolor{yellow} Same & \cellcolor{yellow} Unknown & \cellcolor{yellow} Unknown \tabularnewline

\hline
	\end{tabular}
\end{table}


%\cellcolor{blue} foo & \cellcolor{red}
\clearpage
%Table from Paul Rago
%new columns defined earlier
%\newcolumntype{L}{>{\centering}m{2cm}}
%\newcolumntype{M}{>{\centering}m{2.25cm}}
%\newcolumntype{N}{>{\centering}m{1cm}}

\begin{sidewaystable}[ht]
%\begin{table}
\captionsetup{width=\textwidth}
\centering
%\caption{}
\caption{Summary of Operational Assessment estimates of biomasses and fishing mortality rates in 2014 and biological reference points for 20 groundfish stocks. \textit{Note: different units for SSB and F for windowpane flounder stocks and ocean pout.}}
\label{BFEstimates}
{\small	
	\begin{tabular}{
	m{2cm}@{\hspace{.1cm}}
	M@{\hspace{.1cm}}
	N@{\hspace{.3cm}}   
	N@{\hspace{.2cm}}
	N@{\hspace{.2cm}}   
	N@{\hspace{.1cm}}
	m{0.75cm}@{\hspace{.1cm}}
	N@{\hspace{.2cm}}
	m{0.75cm}@{\hspace{.1cm}}
	N@{\hspace{.2cm}}
	m{6cm}@{\hspace{0cm}}
	}


\hline

Stock & Model type & $B_{2014}$ (mt) & $B_{MSY}$ (mt) & $\frac{B_{2014}}{B_{MSY}}$ & $F_{2014}$ & $F_{MSY}$ & $\frac{F_{2014}}{F_{MSY}}$ & $MSY$ (mt) & $\rho$ adj? & Comments \\

\hline

CODGM & ASAP (M=0.2) &  2,225  &  40,187  & 0.06 & 0.956 & 0.185 & 5.17 &  6,797  & No &  \\
CODGM & ASAP (M-ramp) &  2,536  &  59,045  & 0.04 & 0.932 & 0.187 & 4.98 &  10,043  & No &  \\
CODGB & ASAP (not accepted) &  NA  &  NA  & NA & NA & NA & NA &  NA  & NA & Model not accepted.  Catch recommendations based on recent average catch and survey trends. \\
HADGB & VPA &  150,053  &  108,300  & 1.39 & 0.241 & 0.39 & 0.62 &  24,900  & Yes &  \\
HADGM & ASAP &  10,325  &  4,623  & 2.23 & 0.257 & 0.468 & 0.55 &  1,083  & No &  \\
YELCCGM & VPA &  857  &  5,259  & 0.16 & 0.640 & 0.279 & 2.29 &  1,285  & Yes & \\
YELSNEMA & ASAP &  502  &  1,959  & 0.26 & 1.640 & 0.349 & 4.70 &  541  & No & \\
FLWGB & VPA &  2,883  &  6,700  & 0.43 & 0.778 & 0.536 & 1.45 &  2,840  & Yes & \\
FLWSNEMA & ASAP &  6,151  &  26,928  & 0.23 & 0.160 & 0.325 & 0.49 &  7,831  & No & \\ 
PLAUNIT & VPA &  10,977  &  13,107  & 0.84 & 0.116 & 0.196 & 0.59 &  2,675  & Yes & \\
WITUNIT & VPA &  2,077  &  9,473  & 0.22 & 0.687 & 0.279 & 2.46 &  1,957  & Yes & \\
REDUNIT & ASAP &  330,004  &  281,112  & 1.17 & 0.015 & 0.038 & 0.39 &  10,466  & Yes & \\ 
HWKUNIT & ASAP &  28,553  &  32,550  & 0.88 & 0.076 & 0.188 & 0.40 &  5,422  & No & \\
POLUNIT & ASAP (base) &  154,919  &  105,226  & 1.47 & 0.070 & 0.277 & 0.25 &  19,678  & Yes & Flat-top selectivity model was  done as a sensitivity analysis. \\
POLUNIT & ASAP (flat top sel. sensitivity) &  32,040  &  54,900  & 0.58 & 0.233 & 0.252 & 0.92 &  10,995  & Yes & see above \\
CATUNIT & SCALE &  638  &  1,663  & 0.38 & 0.003 & 0.243 & 0.01 &  244  & No & Projections not recommended by earlier Peer Review Panel \\
HALUNIT & Replacement Yield (not accepted) & NA & NA & NA & NA & NA & NA &  NA  & NA & Model rejected (only 2016 OFL set - 198 mt). PDT to recommend based on status quo adjustment \\
FLDGBGM & AIM & 0.535 & 1.554 & 0.34 & 0.393 & 0.450 & 0.87 &  700  & NA & Biomass in terms of kg/tow, F values reflect exploitation rate \\
FLDSNEMA & AIM & 0.413 & 0.247 & 1.67 & 1.308 & 2.027 & 0.65 &  500  & NA & Biomass in terms of kg/tow, F values reflect exploitation rate \\
OPTUNIT & Index-based & 0.290 & 4.940 & 0.06 & 0.269 & 0.76 & 0.35 &  3,754  & NA & Biomass in terms of kg/tow, F values reflect exploitation rate. Adjusted exploitation rate reflects anticipated data update to CFDBS \\
FLWGM & None (Area-swept) & 4655 & NA & NA & 0.060 & 0.23 & 0.26 &  NA  & NA & 30+ cm biomass, exploitation ratio \\
YELGB & None (Area-swept) & 2240 & NA & NA & 0.071 & NA & NA &  NA  & NA & Average survey biomass (2014), exploitation ratio (ABC set to 354 mt based on status quo, OFL set as unknown) \\
   \hline

\end{tabular}
}

\end{sidewaystable}


\clearpage
\begin{sidewaystable}[ht]
%\begin{table}
\captionsetup{width=\textwidth}

\centering
%\caption{}
\caption{Comparison of biomass ($B$) and fishing mortality ($F$) rate Mohn's rho values ($\rho$) by stock between the previous assessment and the 2015 updates. The biomass and fishing mortality rate point estimates and $\rho$ adjusted values (Adj.) are provided for the 2015 operational assessments. The total number of stocks using $\rho$ adjusted values in the last assessment and the 2015 assessments ($\rho$ adj. vs. pt. est. for those stocks that did not use the $\rho$ adjustment), along with the type of $\rho$ adjustment used in the 2015 assessment (NAA$=$numbers at age, SSB$=$spawning stock biomass applied to all ages), are also provided. Only age-based and length-based stocks that could exhibit retrospective patterns are included in this table. \textit{Note: Because the Georges Bank cod assessment was rejected at the 2015 OA it has been excluded from this table.} }
\label{RhoAdjust_tab}
{\small
\begin{tabular}{
l@{\hspace{.1cm}}
c@{\hspace{.1cm}}
c@{\hspace{.2cm}}
c@{\hspace{.2cm}}
c@{\hspace{.2cm}}
c@{\hspace{.1cm}}
m{1mm} 
c@{\hspace{.2cm}}
c@{\hspace{.2cm}}
c@{\hspace{.2cm}}
c@{\hspace{.1cm}}
m{1mm} 
c@{\hspace{.2cm}} 
c@{\hspace{.2cm}}
c@{\hspace{.2cm}} 
}
\hline
 & & &&&&&&&&&& \\[1pt] %blank row
 & & \multicolumn{4}{c}{\textbf{Biomass}} && \multicolumn{4}{c}{\textbf{Fishing Mortality Rate}} && \multicolumn{3}{c}{\textbf{Used}} \\  
\cline{3-6} \cline{8-11} \cline{13-15}
Stock & Model &$\rho_{last}$ & $\rho_{2014}$ & $B_{2014}$ & Adj. && $\rho_{last}$ & $\rho_{2014}$ & 
$F_{2014}$ & Adj. && Last assess. & 2014 & Proj. adj.\\
  \hline
CODGM & ASAP(M=0.2) & 0.53 & 0.54 & 2225 & 1445 && -0.33 & -0.31 & 0.956 & 1.386 && pt. est. & pt. est. & none \\
CODGM & ASAP(M-ramp) & 0.17 & 0.2 & 2536 & 2113 && -0.05 & -0.08 & 0.932 & 1.013 && pt. est. & pt. est. & none \\
HADGM & ASAP & -0.15 & -0.04 & 10325 & 10755 && 0.3 & 0.03 & 0.257 & 0.25 && pt. est. & pt. est. & none \\
HADGB & VPA & 0.2 & 0.5 & 225080 & 150053 && -0.15 & -0.34 & 0.159 & 0.241 && pt. est. & $\rho$ adj. & SSB \\
YELCCGM & VPA & 0.68 & 0.98 & 1695 & 857 && -0.19 & -0.45 & 0.35 & 0.64 && $\rho$ adj. & $\rho$ adj. & NAA \\
YELSNEMA & ASAP & 0.14 & 1.06 & 502 & 243 && -0.16 & -0.53 & 1.64 & 3.53 && pt. est. & pt. est. & none \\
FLWGB & VPA & 0.26 & 0.83 & 5275 & 2883 && -0.16 & -0.51 & 0.379 & 0.778 && pt. est. & $\rho$ adj. & SSB \\
FLWSNEMA & ASAP & 0.35 & 0.21 & 6151 & 5105 && -0.31 & -0.25 & 0.16 & 0.214 && pt. est. & pt. est. & none \\
REDUNIT & ASAP & 0.04 & 0.26 & 414544 & 330004 && -0.04 & -0.19 & 0.012 & 0.015 && pt. est. & $\rho$ adj. & NAA \\
PLAUNIT & VPA & 0.62 & 0.32 & 14439 & 10915 && -0.35 & -0.32 & 0.08 & 0.12 && $\rho$ adj. & $\rho$ adj. & NAA \\
WITUNIT & VPA & 0.61 & 0.51 & 3129 & 2077 && -0.33 & -0.38 & 0.428 & 0.687 && pt. est. & $\rho$ adj. & SSB \\
HKWUNIT & ASAP & 0.15 & 0.18 & 28553 & 24197 && -0.13 & -0.12 & 0.076 & 0.086 && pt. est. & pt. est. & none \\
POLUNIT & ASAP & 0.29 & 0.28 & 198847 & 154865 && -0.25 & -0.28 & 0.051 & 0.07 && pt. est. & $\rho$ adj. & NAA \\
CATUNIT & SCALE & 0.96 & 0.83 & 592 & 324 && -0.55 & -0.36 & 0.003 & 0.005 & pt. est. && pt. est. & none \\
   \hline
\end{tabular}
}

\end{sidewaystable}
%\end{table}

\clearpage
% latex table generated in R 3.2.1 by xtable 1.7-4 package
% Fri Nov  6 14:46:57 2015
\begin{table*}[ht]
\centering
\caption{The biomass ($B$) and exploitation rate ($F$) values used for status determination were  adjusted to account for a retrospective pattern in some stocks.   In general, when the $B$ or $F$ values adjusted for restrospective pattern ($B_{\rho}$ and $F_{\rho}$)  were outside of the approximate $90\%$ confidence interval (Conf. limits), the  $\rho$ adjusted values were used to determine stock status (Adj. $=$ Yes).  There were exceptions however, such as YELSNEMA and CODGM(M=0.2) and details regarding each decision can be found in the  report and reviewer comments sections for each stock.  Only stocks that had both an estimable 7-year Mohn's $\rho$ for $B$ and $F$ and estimable approximate  90\% confidence limits on  terminal year $B$ and $F$ values are included. } 
\label{RhoDecision_tab}
\begin{tabular}{c@{\hspace{.2cm}}c@{\hspace{.2cm}}c@{\hspace{.2cm}}c@{\hspace{.2cm}}c@{\hspace{.2cm}}c@{\hspace{.2cm}}c@{\hspace{.2cm}}c@{\hspace{.2cm}}}
  \hline
Stock & $B_{2014}$ & $B_{\rho}$ & Conf. limits & $F_{2014}$ & $F_{\rho}$ & Conf. limits & Adj? \\ 
  \hline
CODGM(M=0.2) & 2,225 & 1,443 & 1,942 - 2,892 & 0.956 &  1.39 & 0.654 - 1.387 & No \\ 
  CODGM(M ramp) & 2,536 & 2,106 & 1,921 - 3,298 & 0.932 &  1.01 & 0.662 - 1.304 & No \\ 
  HADGB & 225,080 & 150,053 & 171,911 - 301,282 & 0.159 & 0.241 & 0.13 - 0.203 & Yes \\ 
  HADGM & 10,325 & 10,712 & 7,229 - 14,453 & 0.257 &  0.25 & 0.164 - 0.373 & No \\ 
  YELSNEMA & 502 & 243 & 355 - 739 &  1.64 &  3.53 & 1.053 - 2.348 & No \\ 
  YELCCGM & 1,695 & 857 & 1,375 - 2,111 & 0.355 &  0.64 & 0.25 - 0.52 & Yes \\ 
  FLWSNEMA & 6,151 & 5,105 & 5,045 - 7,500 &  0.16 &  0.21 & 0.12 - 0.213 & No \\ 
  FLWGB & 5,275 & 2,883 & 3,783 - 6,767 & 0.379 & 0.778 & 0.254 - 0.504 & Yes \\ 
  PLAUNIT & 14,543 & 10,977 & 12,742 - 16,439 &  0.08 & 0.116 & 0.069 - 0.093 & Yes \\ 
  WITUNIT & 3,129 & 2,077 & 2,643 - 3,864 & 0.428 & 0.687 & 0.321 - 0.603 & Yes \\ 
  HWKUNIT & 28,553 & 24,197 & 24,351 - 33,480 & 0.076 & 0.086 & 0.063 - 0.092 & No \\ 
  POLUNIT & 198,847 & 154,919 & 37,243 - 255,097 & 0.051 &  0.07 & 0.084 - 0.066 & Yes \\ 
  REDUNIT & 414,544 & 330,004 & 368,906 - 465,828 & 0.012 & 0.015 & 0.011 - 0.014 & Yes \\ 
   \hline
\end{tabular}
\end{table*}





% Figures
\clearpage
%\subsection{Figures}
%\IfFileExists{\ExSumPath/figures/stock_status.png}{
	\begin{figure}
		\centering	
		\adjustimage{max size={.95\textwidth}{.7\textheight}}{\ExSumPath/figures/propFmsy.png}
		\captionsetup{singlelinecheck=off}
		\caption[.]{Changes in the ratio of fishing mortality to FMSY proxy from 2007 (GARM III) to 2014 (OA 2015) for the twenty Northeast Multispecies Fishery Management Plan groundfish stocks. The results from the assessment prior to the OA 2015 assessment are shown for each stock to provide an 'Intermediate' value. Stocks on which overfishing is occurring are those where the $\frac{F_{terminal}}{F_{MSY{}proxy}}$ ratio is greater than 1. \textit{Notes: (1) the GARM III assessments did not include wolffish; (2) stock status in the 'Intermediate' assessment could not be determined for Gulf of Maine winter flounder or Georges Bank yellowtail flounder; and, (3) based on the OA 2015 assessments stock status could not be determined for Atlantic halibut, Gulf of Maine winter flounder and Georges Bank yellowtail flounder. In the OA 2015 assessment, the stock status for Georges Bank cod remained overfished and overfishing is occurring; however, since the assessment was not accepted, ratios of terminal conditions to reference points cannot be determined.}}		
		\label{propFmsy}
	\end{figure}
	\clearpage
%}{}  %otherwise do nothing



%\IfFileExists{\ExSumPath/figures/stock_status.png}{
	\begin{figure}
		\centering	
		\adjustimage{max size={.95\textwidth}{.7\textheight}}{\ExSumPath/figures/propBmsy.png}
		\captionsetup{singlelinecheck=off}
		\caption[.]{Changes in the ratio of stock biomass to $B_{MSY}$ proxy from 2007 (GARM III) to 2014 (OA 2015) for the twenty Northeast Multispecies Fishery Management Plan groundfish stocks. The results from the assessment prior to the OA 2015 assessment are shown for each stock to provide an 'Intermediate' value. Stocks that are overfished stocks are those where the $\frac{B_{terminal}}{B_{MSY{}proxy}}$ ratio is less than 0.5. \textit{Notes: (1) the GARM III assessments did not include wolffish; (2) stock status in the 'Intermediate' assessment could not be determined for Gulf of Maine winter flounder or Georges Bank yellowtail flounder; and, (3) based on the OA 2015 assessments stock status could not be determined for Atlantic halibut, Gulf of Maine winter flounder and Georges Bank yellowtail flounder. In the OA 2015 assessment, the stock status for Georges Bank cod remained overfished and overfishing is occurring; however, since the assessment was not accepted, ratios of terminal conditions to reference points cannot be determined.}}		
		\label{propBMSY}
	\end{figure}
	\clearpage
%}{}  %otherwise do nothing

\clearpage
%\IfFileExists{\ExSumPath/figures/stock_status.png}{
	\begin{sidewaysfigure}
		\centering	
		\adjustimage{max size={.95\textwidth}{.8\textheight}}{\ExSumPath/figures/StockStatus.png}
		\captionsetup{width=\textwidth}
		\caption[.]{Status of the Northeast Multispecies Fishery Management Plan groundfish stocks in 2007 (GARM III)  and 2014 (OA 2015) with respect to the $F_{MSY}$ and $B_{MSY}$ proxies. The 'Intermediate assessment' represents the last stock assessment conducted prior to the OA 2015 assessment (year varies by stock). Stocks on which overfishing is occurring are those where the $\frac{F_{terminal}}{F_{MSY proxy}}$ ratio is greater than 1 and overfished stocks are those where the $\frac{B_{terminal}}{B_{MSY{}proxy}}$ ratio is less than 0.5. \textit{Notes: (1) the GARM III assessments did not include wolffish; (2) for the intermediate assessments stock status could not be determined for Gulf of Maine winter flounder (OA 2014) or Georges Bank yellowtail (TRAC 2015); and, (3) based on the OA 2015 assessments stock status could not be determined for Atlantic halibut, Gulf of Maine winter flounder and Georges Bank yellowtail flounder. In the OA 2015 assessment, the stock status for Georges Bank cod remained overfished and overfishing is occurring; however, since the assessment was not accepted, ratios of terminal conditions to reference points cannot be determined. Species codes: COD-Atlantic cod, HAD-haddock, POL-pollock, RED-redfish, WHK-white hake, OPT-ocean pout, CAT-wolffish, PLA-American plaice, FLW-winter flounder, YEL-yellowtail flounder, WIT-witch flounder, FLD-windowpane flounder, HAL-Atlantic halibut.}}		
		\label{stock_status}
	\end{sidewaysfigure}
	\clearpage
%}{}  %otherwise do nothing

%\IfFileExists{\ExSumPath/figures/stock_status.png}{
	\begin{figure}
		\centering	
		\adjustimage{max size={.95\textwidth}{.8\textheight}}{\ExSumPath/figures/nefscSpringResiduals.png}
		\captionsetup{singlelinecheck=off}
		\caption[.]{NEFSC spring bottom trawl survey index standardized anomalies (Z-score) for the Northeast Multispecies Fishery Management Plan groundfish stocks from 1968 to 2015. \textit{Note that both the Georges Bank$/$Gulf of Maine and Southern New England$/$Mid-Atlantic windowpane flounder stocks are not included since the spring survey is uninformative as an index of abundance and not used in the stock assessment.}	}	
		\label{nefscSpringResiduals}
	\end{figure}
	\clearpage
%}{}  %otherwise do nothing



%\IfFileExists{\ExSumPath/figures/stock_status.png}{
	\begin{figure}
		\centering	
		\adjustimage{max size={.95\textwidth}{.8\textheight}}{\ExSumPath/figures/nefscFallResiduals.png}
		\captionsetup{singlelinecheck=off}
		\caption[.]{NEFSC fall bottom trawl survey index standardized anomalies (Z-score) for the Northeast Multispecies Fishery Management Plan groundfish stocks from 1963 to 2014. \textit{Note that ocean pout is not included since the fall survey is uninformative as an index of abundance and not used in the stock assessment.}	}	
		\label{nefscFallResiduals}
	\end{figure}
	\clearpage
%}{}  %otherwise do nothing




%\IfFileExists{\ExSumPath/figures/stock_status.png}{
	\begin{figure}
		\centering	
		\adjustimage{max size={.95\textwidth}{.8\textheight}}{\ExSumPath/figures/nefscSpringMinSweptAreaBiomass.png}
		\captionsetup{singlelinecheck=off}
		\caption[.]{NEFSC spring bottom trawl survey minimum swept area biomass (mt) for the Northeast Multispecies Fishery Management Plan groundfish stocks from 1968 to 2015, by stock. Minimum swept area estimates assume a trawl swept area of 0.0112 $nm^{2}$) (0.0384 $km^{2}$) based on the wing spread of the trawl net. \textit{Note that both the Georges Bank$/$Gulf of Maine and Southern New England$/$Mid-Atlantic windowpane flounder stocks are not included since the spring survey is uninformative as an index of abundance and not used in the stock assessment.}	}	
		\label{nefscSpringMinSweptAreaBiomass}
	\end{figure}
	\clearpage
%}{}  %otherwise do nothing


%\IfFileExists{\ExSumPath/figures/stock_status.png}{
	\begin{figure}
		\centering	
		\adjustimage{max size={.95\textwidth}{.8\textheight}}{\ExSumPath/figures/nefscFallMinSweptAreaBiomass.png}
		\captionsetup{singlelinecheck=off}
		\caption[.]{NEFSC fall bottom trawl survey minimum swept area biomass (mt) for for the Northeast Multispecies Fishery Management Plan groundfish stocks from 1963 to 2014, by stock. Minimum swept area estimates assume a trawl swept area of 0.0112 $nm^{2}$ (0.0384 $km^{2}$) based on the wing spread of the trawl net. \textit{Note that ocean pout is not included since the fall survey is uninformative as an index of abundance and not used in the stock assessment.}	}
		\label{nefscFallMinSweptAreaBiomass}
	\end{figure}
	\clearpage
%}{}  %otherwise do nothing


%\IfFileExists{\ExSumPath/figures/stock_status.png}{
%	\begin{figure}
%		\centering	
%		\adjustimage{max size={.95\textwidth}{.8\textheight}}{\ExSumPath/figures/propBiomassHaddockRedfish.png}
%		%\adjustimage{max size={.95\textwidth}{.8\textheight}}{../../ExSum/figures/StockStatus.png}		
%		\captionsetup{singlelinecheck=off}
%		\caption[.]{Proportion of the total groundfish swept minimum swept area biomass contributed by Georges Bank and %Gulf of Maine haddock and Redfish based on the NEFSC spring and fall bottom trawl surveys.}
%		\label{propBiomassHaddockRedfish}
%	\end{figure}
%	\clearpage
%}{}  %otherwise do nothing



	\begin{figure}
		\centering	
		\adjustimage{max size={.95\textwidth}{.8\textheight}}{\ExSumPath/figures/ModelBiomass.png}
		\captionsetup{singlelinecheck=off}
		\caption[.]{Model-based spawning stock biomass estimates for 13 groundfish stocks, 1985-2014 based on the Operational Assessments in 2015.  The Georges Bank cod model estimates were not used for management advice due to a strong retrospective pattern in recent years.}
		\label{ModelB}
	\end{figure}
	\clearpage


	\begin{figure}
		\centering	
		\adjustimage{max size={.95\textwidth}{.8\textheight}}{\ExSumPath/figures/ContrastTrends2.png}
		\captionsetup{singlelinecheck=off}
		\caption[.]{Contrasting biomass trends for  Georges Bank and Gulf of Maine haddock, redfish, pollock, and white hake (thick solid line, left axis) versus Georges Bank and Gulf of Maine cod, Georges Bank and Southern New England winter flounder, Southern New England  and Cape Cod/Gulf of Maine yellowtail flounder,  witch flounder, and American plaice (solid line, open circles, right axis). The Georges Bank cod model estimates were not used for management advice due to a strong retrospective pattern in recent years.}
		\label{ContrastBTrends}
	\end{figure}
	\clearpage


	\begin{figure}
		\centering	
		\adjustimage{max size={.95\textwidth}{.8\textheight}}{\ExSumPath/figures/SumBiomass.png}	
		\captionsetup{singlelinecheck=off}
		\caption[.]{Sum of $B_{MSY}$ estimates for ten stocks which had $B_{MSY}$ estimates in 2005 (759,950 mt), 2008 (667,713 mt) and 2015 (525,496 mt) assessments. The following stocks were excluded:  Gulf of Maine haddock are excluded because $B_{MSY}$  estimates were not derived until GARM III.  Pollock is not included since biomass targets not established until 2010 at SARC 50.  $B_{MSY}$ estimates for Gulf of Maine winter flounder and Georges Bank yellowtail flounder are not available as both stock assessments are based on swept area expansions.  The assessment model for Georges Bank cod was not accepted for catch advice in 2015.}
		\label{SumBMSY}
	\end{figure}
	\clearpage

