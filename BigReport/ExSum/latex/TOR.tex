
%\section{Generic terms of reference for operational assessments}\label{TOR}
%\footnote{Source: NRCC. 2011. A new process for assessment of managed fishery resources off the Northeastern United States. Internal Report.}

test test test



\begin{enumerate}
%\item Update all fishery-dependent data (landings, discards, catch-at-age, etc.) and all fishery-independent data (research survey information) used as inputs in the baseline model or in the last operational assessment.
\item Estimate fishing mortality and stock size for the current year, and update estimates of these parameters in previous years, if these have been revised.
%\item Identify and quantify data and model uncertainty that can be considered for setting Acceptable Biological Catch limits. 
%\item If appropriate, update the values of biological reference points (BRP).
%\item Evaluate stock status with respect to updated status determination criteria.
%\item Perform short-term projections; compare results to rebuilding schedules.
%\item Comment on whether assessment diagnostics, or the availability of new types of assessment input data, indicate that a new assessment approach is warranted (i.e., referral to the research track).
%\item Should the baseline model fail when applied in the operational assessment, provide guidance on how stock status might be evaluated. Should an alternative assessment approach not be readily available, provide guidance on the type of scientific and management advice that can be.
\end{enumerate}
 
