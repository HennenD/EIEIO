 \def\REDUNITMyPathTab{/home/dhennen/EIEIO/BigReport/RED_UNIT/tables} \def\REDUNITMyPathFig{/home/dhennen/EIEIO/BigReport/RED_UNIT/figures} \def\REDUNITfigFishCap{Total catch of Acadian redfish between 1913 and 2014 by fleet (commercial and other)  and disposition (landings and discards).} \def\REDUNITfigSSBCap{Trends in spawning stock biomass of Acadian redfish between 1913 and 2014 from the current  (solid line)  and previous (dashed line)  assessment and the corresponding  $SSB_{Threshold}${} (0.5 * $SSB_{MSY}${} \textit{proxy}{}; horizontal dashed line)  as well as  $SSB_{Target}${} ($SSB_{MSY}${} \textit{proxy}{}; horizontal dotted line)  based on the 2015 assessment. Biomass was adjusted for a retrospective pattern and the adjustment is shown in red. The approximate 90\percent{} lognormal confidence intervals are shown.} \def\REDUNITfigFCap{Trends in the fully selected fishing mortality ($F_{Full}${})  of Acadian redfish between 1913 and 2014 from the current (solid line)  and previous (dashed line)  assessment and the corresponding  $F_{Threshold}${} ($F_{MSY}${} \textit{proxy}{}=0.038; horizontal dashed line)  based on the 2015 assessment.  $F_{Full}${} was adjusted for a retrospective pattern and the adjustment is shown in red. The approximate 90\percent{} lognormal confidence intervals are shown.} \def\REDUNITfigRecrCap{Trends in Recruits (age 1)  (000s)  of Acadian redfish between 1913 and 2014 from the current (solid line)  and previous (dashed line)  assessment. The approximate 90\percent{} lognormal confidence intervals are shown.} \def\REDUNITfigSurvCap{Indices of abundance for Acadian redfish from the Northeast Fisheries Science Center (NEFSC)  spring (1963 to 2015)  and fall (1963 to 2014)  bottom trawl surveys. The approximate 90\percent{} lognormal confidence intervals are shown.} \def\REDUNITPreAmb{This assessment of the Acadian redfish (\textit{Sebastes fasciatus})  stock is an operational assessment of the existing 2012 operational assessment (NEFSC 2012). This assessment updates commercial fishery catch data, research survey indices of abundance, the ASAP analytical model, and biological reference points through 2014. Additionally, stock projections have been updated through 2018. The most recent benchmark assessment of the Acadian redfish stock was in 2008 as part of the 3$^{rd}$ Groundfish Assessment Review Meeting (GARM III; NEFSC 2008), which includes a full description of the model formulations.} \def\REDUNITSoS{ \textbf{State of Stock: }{}Based on this updated assessment, the Acadian redfish (\textit{Sebastes fasciatus})  stock is not overfished and overfishing is not occurring (Figures \ref{REDUNITSSB_plot1}-\ref{REDUNITF_plot1}){}. Retrospective adjustments were made to the model results. Retrospective adjusted spawning stock biomass (SSB)  in 2014 was estimated to be 330,004 (mt)  which is 117\percent{} of the biomass target ($SSB_{MSY}${} \textit{proxy}{} of SSB at  $F_{50\percent{}}${} = 281,112;  Figure \ref{REDUNITSSB_plot1}{}).  The retrospective adjusted 2014 fully selected fishing mortality (F)  was estimated to be 0.015 which is 39\percent{} of the overfishing threshold ($F_{MSY}${} \textit{proxy}{} of  $F_{50\percent{}}${} = 0.038;  Figure \ref{REDUNITF_plot1}{}).} \def\REDUNITProj{ \textbf{Projections: }{}Short term projections of median total fishery yield and spawning stock biomass for Acadian redfish were conducted based on a harvest scenario of fishing at the  $F_{MSY}${} \textit{proxy}{} between 2016 and 2018. Catch in 2015 has been estimated at 5,204 (mt). Recruitments were sampled from a cumulative distribution function derived from ASAP estimated age 1 recruitment between 1969 and 2014. The annual fishery selectivity, natural mortality, maturity ogive, and mean weights used  in projections are the same as those used in the assessment model. Retrospective adjusted SSB and fully selected F in 2014 fell outside the 90\percent{} confidence intervals of the unadjusted 2014 values. Therefore, retrospective adjustments were applied in the projections. } \def\REDUNITSpecCmt{ \textbf{Special Comments: } \begin{itemize}{} \item{}What are the most important sources of uncertainty in this stock assessment?  Explain, and describe qualitatively how they affect the assessment results (such as estimates of biomass, F, recruitment, and population projections).  \linebreak{} \hspace*{0.5cm} \textit{The largest source of uncertainty in the Acadian redfish assessment is the lack of age data, particularly from the commercial fishery. Age measurements from landings were not collected after 1985 due to relatively low landings. Current landings have increased to levels seen in the mid-1980s. If landings continue to increase, then age data from the fishery will become increasingly important. Dimorphic growth is another source of uncertainty in this assessment, with females growing faster than males. The use of female weights at age in the stock projections may lead to overestimation of stock productivity, as well as having an unknown effect on biological reference points.}  \item{} Does this assessment model have a retrospective pattern? If so, is the pattern minor, or major? (A major retrospective pattern occurs when the adjusted SSB or  $F_{Full}${} lies outside of the approximate  joint confidence region for SSB and  $F_{Full}${}; see  Table \ref{RhoDecision_tab}{}). \linebreak{} \hspace*{0.5cm} \textit{ The 7-year Mohn's  \textrho{}, relative to SSB, was 0.036 in the 2012 assessment and was 0.256 in 2014. The 7-year Mohn's  \textrho{}, relative to F, was -0.035 in the 2012 assessment and was -0.190 in 2014. There was a major retrospective pattern for this assessment because the  \textrho{} adjusted estimates of 2014 SSB ($SSB_{\rho}${}=330,004)  and 2014 F ($F_{\rho}${}=0.015)  were outside the approximate 90\percent{} confidence region around SSB (368,906 - 465,828)  and F (0.011 - 0.014).  A retrospective  adjustment was made for both the determination of stock status and for projections of catch in 2016. The retrospective adjustment changed the 2014 SSB from 414,544 to 330,004 and the 2014  $F_{Full}${} from 0.012 to 0.015.}  \item{}Based on this stock assessment, are population projections well determined or uncertain? \linebreak{} \hspace*{0.5cm} \textit{Population projections for Acadian redfish appear to be reasonably well determined. }  \item{}Describe any changes that were made to the current stock assessment, beyond incorporating additional years of data  and the effect these changes had on the assessment and stock status. \linebreak{} \hspace*{0.5cm} \textit{Only one major change was made to the Acadian redfish assessment as part of this update. Likelihood constants were excluded from likelihood calculations to avoid potential bias caused by one of the recruitment likelihood constants, which is the sum of the log-scale predicted recruitments, and therefore not a constant. Inclusion of this likelihood constant allows the assessment model to minimize the negative log likelihood by estimating lower recruitments. Exclusion of the likelihood constants led to slightly higher estimates of SSB in recent years. }  \item{}If the stock status has changed a lot since the previous assessment, explain why this occurred.  \linebreak{} \hspace*{0.5cm} \textit{There has been no change in the stock status of Acadian redfish since the previous assessment.}  \item{}Indicate what data or studies are currently lacking and which would be needed most to improve this stock assessment in the future.  \linebreak{} \hspace*{0.5cm} \textit{The Acadian redfish assessment could be improved by 1)  including additional age data, particularly from the commercial fishery, and 2)  investigating the sensitivity of biological reference points and stock projections to the mean weights at age. }  \item{}Are there other important issues? \linebreak{} \hspace*{0.5cm} \textit{Northeast Fisheries Science Center (NEFSC)  fall bottom trawl index values for 2013 and 2014 are lower than in previous years (Figure \ref{REDUNITSurv_plot1}{}), but the current assessment model continues to predict an increase in SSB for the last two years (Figure \ref{REDUNITSSB_plot1}{}). If future index values remain low (i.e., if the index is responding to a change in abundance, rather than interannual variability), then the predicted trend in SSB may change abruptly in a future assessment. Such an abrupt change may lead to an increase in the retrospective pattern.} \end{itemize}{}} \def\REDUNITRefr{ \textbf{References: }{} \linebreak{}Northeast Fisheries Science Center. 2008. Assessment of 19 Northeast Groundfish Stocks through 2007: Report of the 3$^{rd}$ Groundfish Assessment Review Meeting (GARM III), Northeast Fisheries Science Center, Woods  Hole, Massachusetts, August 4-8, 2008. US Dept Commer, Northeast Fish Sci Cent Ref Doc. 08-15; 884 p + xvii. Available from: National Marine Fisheries Service, 166 Water Street, Woods Hole, MA 02543-1026. \href{http://www.nefsc.noaa.gov/publications/crd/crd0815/}{CRD08-15} \linebreak{} \linebreak{}Northeast Fisheries Science Center. 2012. Assessment or Data Updates of 13 Northeast Groundfish Stocks through 2010. US Dept Commer, Northeast Fish Sci Cent Ref Doc. 12-06; 789 p. Available from: National Marine Fisheries Service, 166 Water Street, Woods Hole, MA 02543-1026. \href{http://www.nefsc.noaa.gov/publications/crd/crd1206/}{CRD12-06}} \def\REDUNITDraft{} \def\REDUNITSPPname{Acadian redfish} \def\REDUNITSPPnameT{Acadian redfish} \def\REDUNITRptYr{2015} \def\REDUNITAuthor{Brian Linton} \def\REDUNITReviewerComments{/home/dhennen/EIEIO/BigReport/RED_UNIT/latex}