 \def\CODGMMyPathTab{/home/dhennen/EIEIO/BigReport/COD_GM/tables} \def\CODGMMyPathFig{/home/dhennen/EIEIO/BigReport/COD_GM/figures} \def\CODGMfigFishCap{Total catch of Gulf of Maine Atlantic cod between 1982 and 2014 by fleet (commercial and recreational)  and disposition (landings and discards).} \def\CODGMfigSSBCap{Estimated trends in the spawning stock biomass (SSB)  of Gulf of Maine Atlantic cod between 1982 and 2014 from the current  (solid line)  and previous (dashed line)  assessment and the corresponding  $SSB_{Threshold}${} ($\dfrac{1}{2}${} $SSB_{MSY}${}; horizontal dashed line)  as well as  $SSB_{Target}${} $SSB_{MSY}${}; horizontal dotted line)   based on the 2015 M=0.2 (A)  and M-ramp (B)  assessment models. The 90\percent{} lognormal confidence intervals are shown. The red dot indicates the rho-adjusted SSB values that would have resulted had a retrospective adjusment been made to either model (see Special Comments section).} \def\CODGMfigFCap{Estimated trends in the fully selected fishing mortality (F)  of Gulf of Maine Atlantic cod between 1982 and 2014 from the current  (solid line)  and previous (dashed line)  assessment and the corresponding  $F_{Threshold}${} (0.185 (M=0.2), 0.187 (M-ramp); dashed line)  based on the 2015 M=0.2 (A)  and M-ramp (B)  assessment models. The 90\percent{} lognormal confidence intervals are shown. The red dot indicates the rho-adjusted F values that would have resulted had a retrospective adjusment been made to either model (see Special Comments section).} \def\CODGMfigRecrCap{Estimated trends in age-1 recruitment  (000s)  of Gulf of Maine Atlantic cod between 1982 and 2014 from the current (solid line)  and previous (dashed line)  M=0.2 (A)  and M-ramp (B)  assessment models. The 90\percent{} lognormal confidence intervals are shown.} \def\CODGMfigSurvCap{Indices of biomass for the Gulf of Maine Atlantic cod between 1963 and 2015 for the Northeast Fisheries Science Center (NEFSC)  spring and fall bottom trawl surveys and Massachusetts Division of Marine Fisheries (MADMF)  spring bottom trawl survey.  The 90\percent{} lognormal confidence intervals are shown.} \def\CODGMPreAmb{This assessment of the Gulf of Maine Atlantic cod (\textit{Gadus morhua})  stock is an operational assessment of the existing 2014 assessment (Palmer 2014). This assessment updates commercial and recreational fishery catch data, research survey indices of abundance, and the analytical ASAP assessment models through 2014. Additionally, stock projections have been updated through 2018. In what follows, there are two population assessment models brought forward from the most recent benchmark assessment (2012), the M=0.2 (natural mortality = 0.2)  and the M-ramp (M ramps from 0.2 to 0.4)  assessment models (see NEFSC 2013 for a full description of the model formulations).} \def\CODGMSoS{ \textbf{State of Stock: }{}Based on this updated assessment, the Gulf of Maine Atlantic cod (\textit{Gadus morhua})  stock is overfished and overfishing is occurring (Figures \ref{CODGMSSB_plot1}-\ref{CODGMF_plot1}){}. Retrospective adjustments were not made to the model results (see Special Comments section of this report). Spawning stock biomass (SSB)  in 2014 was estimated to be 2,225 (mt)  under the M=0.2 model and 2,536 (mt)  under the M-ramp model scenario (Table \ref{CODGMCatch_Status_Table}{})  which is 6\percent{} and 4\percent{} (respectively)  of the biomass target,  $SSB_{MSY}${} \textit{proxy}{} (40,187 (mt)  and 59,045 (mt);  Figure \ref{CODGMSSB_plot1}{}).  The 2014 fully selected fishing mortality was estimated to be 0.956 and 0.932 which is 517\percent{} and 498\percent{} of the  $F_{MSY}${} \textit{proxy}{}($F_{40\percent{}}${}; 0.185 and 0.187;  Figure \ref{CODGMF_plot1}{}).} \def\CODGMProj{ \textbf{Projections: }{} Short term projections of median total fishery yield and spawning stock biomass for Gulf of Maine Atlantic cod were conducted based on a harvest scenario of fishing at the FMSY proxy between 2016 and 2018. Catch in 2015 was estimated at 279 mt. Recruitment was sampled from a cumulative distribution function derived from ASAP estimated age-1 recruitment between 1982 and 2012.  The projection recruitment model declines linearly to zero when SSB is below 6.3 kmt under the M=0.2 model and 7.9 kmt under the M-ramp model. The 2015 age-1 recruitment was estimated from the geometric mean of the 2010-2014 ASAP recruitment estimates. No retrospective adjustments were applied in the projections as the retrospective patterns are similar to the 2014 update for which no retrospective adjustments were made; however, the 2015 assessment review panel recommended that that M=0.2 projections with retrospective adjustments be brought forward to the SSC for consideration in the evaluation of uncertainty when setting catch advice (provided in the Supplemental Information Report, \href{http://www.nefsc.noaa.gov/saw/sasi/sasi_report_options.php}{SASINF}{}). Assumed weights are based on an average of the most recent three years. For the M-ramp model, projections are shown under two assumptions of short-term natural mortality: M=0.2 and M=0.4.} \def\CODGMSpecCmt{ \textbf{Special Comments: } \begin{itemize}{} \item{}What are the most important sources of uncertainty in this stock assessment?  Explain, and describe qualitatively how they affect the assessment results (such as estimates of biomass, F, recruitment, and population projections).  \linebreak{} \hspace*{0.5cm} \textit{The largest source of uncertainty is the estimate of natural mortality. Past investigations into changes in natural mortality over time have been inconclusive (NEFSC 2013). Different assumptions about natural mortality affect the scale of the biomass, recruitment, and fishing mortality estimates. Other areas of uncertainty include the retrospective error in the M=0.2 model, residual patterns in the model fits to some of the survey series (e.g., aggregate MADMF spring survey)  and stock structure.}  \item{} Does this assessment model have a retrospective pattern? If so, is the pattern minor, or major? (A major retrospective pattern occurs when the adjusted SSB or  $F_{Full}${} lies outside of the approximate  joint confidence region for SSB and  $F_{Full}${}; see  Table \ref{RhoDecision_tab}{}). \linebreak{} \hspace*{0.5cm} \textit{The M=0.2 model has a major retrospective pattern (7-year Mohn's rho SSB=0.54, F=-0.31)  and the M-ramp model has a minor retrospective pattern (7-year Mohn's rho SSB=0.20, F=-0.08). The 7-year Mohn's rho values from the current assessment are similar to those from the 2014 assessment (M=0.2: SSB=0.53, F=-0.33; M-ramp: SSB=0.17, F=-0.05)  where the M=0.2 model had a major retrospective pattern and the M-ramp model had a minor pattern. No retrospective adjustments have been applied to the terminal model results or in the base catch projections following the recommendations of the SARC 55 and 2014 assessment review panels. The 2015 assessment review panel supported this decision, noting that the most recent retrospective 'peel' suggested that an adjustment using the 7-year average may not be appropriate. However, the 2015 review panel highlighted the retrospective error in the M=0.2 model as a source of uncertainty - it should be noted that the retrospective error of the most recent peel is larger for the M-ramp model. Should the retrospective patterns continue then the models may have overestimated spawning stock size and underestimated fishing mortality.}  \item{}Based on this stock assessment, are population projections well determined or uncertain? \linebreak{} \hspace*{0.5cm} \textit{Population projections for Gulf of Maine Atlantic cod are reasonably well determined and projected biomass from the last assessment  was within the confidence bounds of the biomass estimated in the current assessment. }  \item{}Describe any changes that were made to the current stock assessment, beyond incorporating additional years of data  and the effect these changes had on the assessment and stock status. \linebreak{} \hspace*{0.5cm} \textit{ This update included several minor changes to model input data including: (1)  re-estimation of recreational catch from 2004-2014 to account for recent updates to the MRIP data; (2)  a revised assumption on recreational discard mortality from 30\percent{} to 15\percent{} following a Capizzano et al. 2015 study (unpublished); and (3)  re-estimation of 2009-2014 NEFSC spring and fall survey time series using the TOGA station acceptance criterion. Additionally, the ASAP assessment model was run with the likelihood constants option turned off. All of these changes had minimal impacts on model results - summaries of the impacts of these changes are provided in the Supplemental Information Report (\href{http://www.nefsc.noaa.gov/saw/sasi/sasi_report_options.php}{SASINF}{}).}  \item{}If the stock status has changed a lot since the previous assessment, explain why this occurred.  \linebreak{} \hspace*{0.5cm} \textit{There has been no change in stock status since the 2014 udpate assessment.}  \item{}Indicate what data or studies are currently lacking and which would be needed most to improve this stock assessment in the future.  \linebreak{} \hspace*{0.5cm} \textit{The Gulf of Maine Atlantic cod assessment could be improved with additional studies on natural mortality and stock structure. Additionally, future assessments should consider possible changes in recent fishery selectivity patterns and explore alternative methods for estimating recruitment. Potential causes of low stock productivity (i.e., low recruitment)  should also be investigated.}  \item{}Are there other important issues? \linebreak{} \hspace*{0.5cm} \textit{ When setting catch advice, careful attention should be given to the retrospective error present in both models, particularly given the poor performance of previous stock projections. Additionally, it is unclear which level of natural mortality (M=0.2 or 0.4)  to assume for the short-term projections under the M-ramp model.} \end{itemize}{}} \def\CODGMRefr{ \textbf{References: }{} \linebreak{}Northeast Fisheries Science Center. 2013. 55$^{th}$ Northeast Regional Stock Assessment Workshop (55$^{th}$ SAW)  Assessment Summary Report. US Dept Commer, Northeast Fish Sci Cent Ref Doc. 13-11; 41 p. Available from: National Marine Fisheries Service, 166 Water Street, Woods Hole, MA 02543-1026. \href{http://www.nefsc.noaa.gov/publications/crd/crd1311/}{CRD13-11} \linebreak{} \linebreak{}Palmer MC. 2014. 2014 Assessment update report of the Gulf of Maine Atlantic cod stock. US Dept Commer, Northeast Fish Sci Cent Ref Doc. 14-14; 119 p. Available from: National Marine Fisheries Service,166 Water Street, Woods Hole, MA 02543-1026. \href{http://www.nefsc.noaa.gov/publications/crd/crd1414/}{CRD14-14}} \def\CODGMDraft{} \def\CODGMSPPname{Gulf of Maine Atlantic cod} \def\CODGMSPPnameT{Gulf of Maine Atlantic cod} \def\CODGMRptYr{2015} \def\CODGMAuthor{Michael Palmer} \def\CODGMReviewerComments{/home/dhennen/EIEIO/BigReport/COD_GM/latex}