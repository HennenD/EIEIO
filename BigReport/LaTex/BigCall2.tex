\def\FLWGMMyPathTab{/home/dhennen/EIEIO/BigReport/FLW_GM/tables} \def\FLWGMMyPathFig{/home/dhennen/EIEIO/BigReport/FLW_GM/figures} \def\FLWGMfigFishCap{Total catch of Gulf of Maine Winter Flounder between 2009 and 2014 by fleet \(commercial and recreational\)  and disposition \(landings and discards\). A 15\% mortality rate is assumed on recreational discards and a 50\% mortality rate on commercial discards.} \def\FLWGMfigSSBCap{Trends in 30+ cm area\-swept biomass of Gulf of Maine Winter Flounder between 2009 and 2014 from the current assessment based on the fall \(MENH, MDMF, NEFSC\)  surveys.  The approximate 90\% lognormal confidence intervals are shown.} \def\FLWGMfigFCap{Trends in the exploitation rates \(\$E\_{Full}\${}\)  of Gulf of Maine Winter Flounder between 2009 and 2014 from the current assessment and the corresponding  \$F\_{Threshold}\${} \(\$E\_{MSY}\${} \textit{proxy}{}=0.23\; horizontal dashed line\).  The approximate 90\% lognormal confidence intervals are shown.} \def\FLWGMfigRecrCap{} \def\FLWGMfigSurvCap{Indices of biomass for the Gulf of Maine Winter Flounder between 1978 and 2015 for the Northeast Fisheries Science Center \(NEFSC\), Massachusetts Division of Marine Fisheries \(MDMF\), and the Maine New Hampshire \(MENH\)  spring and fall bottom trawl surveys. NEFSC indices are calculated with gear and vessel conversion factors where appropriate.  The approximate 90\% lognormal confidence intervals are shown.} \def\FLWGMPreAmb{This assessment of the  Gulf of Maine Winter Flounder  \(\textit{Pseudopleuronectes americanus}\)   stock is an operational update of the  existing  2014  operational update area\-swept assessment \(NEFSC 2014\).  Based on the previous assessment the biomass status is unknown but overfishing was not occurring.  This assessment  updates commercial and recreational fishery catch data, research survey indices  of abundance, and the area\-swept estimates of 30+ cm biomass based on the fall NEFSC, MDMF, and MENH surveys.} \def\FLWGMSoS{ \textbf{State of Stock: }{}Based on this updated assessment, the Gulf of Maine Winter Flounder \(\textit{Pseudopleuronectes americanus}\)  stock biomass status is unknown and overfishing is not occurring \(Figures \ref{FLWGMSSB\_plot1}\-\ref{FLWGMF\_plot1}\){}. Retrospective adjustments were not made to the model results.  Biomass  \(30+ cm mt\)  in 2014 was estimated to be 4,655 mt \(Figure \ref{FLWGMSSB\_plot1}{}\). The 2014 30+ cm exploitation rate was estimated to be 0.06 which is 26\% of the overfishing exploitation threshold proxy \(\$E\_{MSY}\${} \textit{proxy}{} = 0.23\;  Figure \ref{FLWGMF\_plot1}{}\).} \def\FLWGMProj{ \textbf{Projections: }{}Projections are not possible with area\-swept based assessments. Catch advice was based on 75\% of  \$E\_{40\%}\${}\(75\% \$E\_{MSY}\${} \textit{proxy}{}\)  using the fall area\-swept estimate assuming q=0.6 on the wing spread. Updated 2014 fall 30+ cm area\-swept biomass \(4,655 mt\)  implies an OFL of 1,080 mt based on the  \$E\_{MSY}\${} \textit{proxy}{} and a catch of 810 mt for 75\% of the  \$E\_{MSY}\${} \textit{proxy}{}.} \def\FLWGMSpecCmt{ \textbf{Special Comments: } \begin{itemize}{} \item{}What are the most important sources of uncertainty in this stock assessment?  Explain, and describe qualitatively how they affect the assessment results \(such as estimates of biomass, F, recruitment, and population projections\).  \linebreak{} \hspace\*{0.5cm} \textit{The largest source of uncertainty with the direct estimates of stock biomass from survey area\-swept estimates originate from the assumption of survey gear catchability \(q\). Biomass and exploitation rate estimates are sensitive to the survey q assumption \(0.6 on wing spread\). The 2014 empirical benchmark assessement of Georges bank yellowtail flounder based the area\-swept q assumption on an average value taken from the literature for west coast flatfish \(0.37 on door spread\). The yellowtail q assumption corresponds to a value close to 1 on the wing spread which would result in a lower estimate of biomass \(2,995 mt\). Another major source of uncertainty with this method is that biomass based reference points cannot be determined and overfished status is unknown. }  \item{} Does this assessment model have a retrospective pattern? If so, is the pattern minor, or major? \(A major retrospective pattern occurs when the adjusted SSB or  \$F\_{Full}\${} lies outside of the approximate  joint confidence region for SSB and  \$F\_{Full}\${}\; see  Figure \ref{RhoDecision\_tab}{}\). \linebreak{} \hspace\*{0.5cm} \textit{ The model used to determine status of this stock does not allow estimation of a retrospective pattern.  An analytical stock assessment model does not exist for Gulf of Maine Winter Flounder.  An analytical model was no longer used for stock status determination at SARC 52 \(2011\)  due to concerns with a strong retrospective pattern.  Models have difficulty with the apparent lack of a relationship between a large decrease in the catch with little change in the indices and age and\/or size structure over time. }  \item{}Based on this stock assessment, are population projections well determined or uncertain? \linebreak{} \hspace\*{0.5cm} \textit{Population projections for Gulf of Maine Winter Flounder, do not exist for area\-swept assessments. Catch advice from area\-swept estimates tend to vary with interannual variability in the surveys.}  \item{}Describe any changes that were made to the current stock assessment, beyond incorporating additional years of data  and the affect these changes had on the assessment and stock status. \linebreak{} \hspace\*{0.5cm} \textit{ No changes, other than the incorporation of new data were made to the Gulf of Maine Winter Flounder assessment for this update. However, stabilizing the catch advice may be desired and could be obtained through the averaging of the area\-swept fall and spring survey estimates.}  \item{}If the stock status has changed a lot since the previous assessment, explain why this occurred.  \linebreak{} \hspace\*{0.5cm} \textit{The overfishing status of Gulf of Maine Winter Flounder has not changed. }  \item{}Indicate what data or studies are currently lacking and which would be needed most to improve this stock assessment in the future.  \linebreak{} \hspace\*{0.5cm} \textit{Direct area\-swept assessment could be improved with additional studies on survey gear efficiency.  Quantifying the degree of herding between the doors and escapement under the footrope and\/or above the headrope for each survey is needed since area\-swept biomass estimates and catch advice are sensitive to the assumed catchability.}  \item{}Are there other important issues? \linebreak{} \hspace\*{0.5cm} \textit{The general lack of a response in survey indices and age\/size structure is the primary source of concern with catches remaining far below the overfishing level. } \end{itemize}{}} \def\FLWGMRefr{ \textbf{References: }{} \linebreak{}Hendrickson L, Nitschke P, Linton B. 2015. 2014 Operational Stock Assessments for Georges Bank winter flounder, Gulf of Maine winter flounder, and pollock. US Dept Commer, Northeast Fish Sci Cent Ref Doc. 15\-01\; 228 p. Available from: National Marine Fisheries Service, 166 Water Street, Woods Hole, MA 02543\-1026, or online at http:\/\/nefsc.noaa.gov\/publications\/ \linebreak{} \linebreak{}Northeast Fisheries Science Center. 2011. 52$^{nd}$ Northeast Regional Stock AssessmentWorkshop \(52$^{nd}$ SAW\)  Assessment Report. US Dept Commer, Northeast Fish SciCent Ref Doc. 11\-17\; 962 p. Available from: National Marine Fisheries Service, 166 Water Street, Woods Hole, MA 02543\-1026, or online at http:\/\/www.nefsc.noaa.gov\/nefsc\/publications\/ \linebreak{} \linebreak{}} \def\FLWGMDraft{} \def\FLWGMSPPname{Gulf of Maine Winter Flounder} \def\FLWGMSPPnameT{Gulf of Maine Winter Flounder} \def\FLWGMRptYr{2015} \def\FLWGMAuthor{Paul Nitschke} \def\FLWGMReviewerComments{/home/dhennen/EIEIO/BigReport/FLW_GM/latex}  \def\FLWSNEMAMyPathTab{/home/dhennen/EIEIO/BigReport/FLW_SNEMA/tables} \def\FLWSNEMAMyPathFig{/home/dhennen/EIEIO/BigReport/FLW_SNEMA/figures} \def\FLWSNEMAfigFishCap{Total catch of Southern New England Mid\-Atlantic Winter Flounder between 1981 and 2014 by fleet \(commercial, recreational\)  and disposition \(landings and discards\).} \def\FLWSNEMAfigSSBCap{Trends in spawning stock biomass of Southern New England Mid\-Atlantic Winter Flounder between 1981 and 2014 from the current  \(solid line\)  and previous \(dashed line\)  assessment and the corresponding  \$SSB\_{Threshold}\${} \(\$\dfrac{1}{2}\${} \$SSB\_{MSY}\${} \textit{proxy}{}\; horizontal dashed line\)  as well as  \$SSB\_{Target}\${} \(\$SSB\_{MSY}\${} \textit{proxy}{}\; horizontal dotted line\)   based on the 2015 assessment. The approximate 90\% lognormal confidence intervals are shown.} \def\FLWSNEMAfigFCap{Trends in the fully selected fishing mortality \(\$F\_{Full}\${}\)  of Southern New England Mid\-Atlantic Winter Flounder between 1981 and 2014 from the current  \(solid line\)  and previous \(dashed line\)  assessment and the corresponding  \$F\_{Threshold}\${} \(\$F\_{MSY}\${}=0.325\; horizontal dashed line\)   based on the 2015 assessment. The approximate 90\% lognormal confidence intervals are shown.} \def\FLWSNEMAfigRecrCap{Trends in Recruits \(age 1\)  \(000s\)  of Southern New England Mid\-Atlantic Winter Flounder between 1981 and 2014 from the current \(solid line\)  and previous \(dashed line\)  assessment. The approximate 90\% lognormal confidence intervals are shown.} \def\FLWSNEMAfigSurvCap{Indices of biomass for the Southern New England Mid\-Atlantic Winter Flounder between 1963 and 2014 for the Northeast Fisheries Science Center \(NEFSC\)  spring and fall bottom trawl surveys, the MADMF spring survey, and the CT LISTS survey  The approximate 90\% lognormal confidence intervals are shown.} \def\FLWSNEMAPreAmb{This assessment of the Southern New England Mid\-Atlantic Winter Flounder \(\textit{Pseudopleuronectes americanus}\)  stock is an operational update of the existing 2011 benchmark ASAP assessment \(NEFSC 2011\). Based on the previous assessment the stock was overfished, but overfishing was not ocurring. This assessment updates commercial fishery catch data, recreational fishery catch data, and research survey indices of abundance, and the analytical ASAP assessment models and reference points through 2014. Additionally, stock projections have been updated through 2018} \def\FLWSNEMASoS{ \textbf{State of Stock: }{}Based on this updated assessment, the Southern New England Mid\-Atlantic Winter Flounder \(\textit{Pseudopleuronectes americanus}\)  stock is overfished but overfishing is not occurring \(Figures \ref{FLWSNEMASSB\_plot1}\-\ref{FLWSNEMAF\_plot1}\){}. Spawning stock biomass \(SSB\)  in 2014 was estimated to be 6,151 \(mt\)  which is 23\% of the biomass target \(26,928 mt\), and 23\% of the biomass threshold for an overfished stock \(\$SSB\_{Threshold}\${} = 13464 \(mt\)\;  Figure \ref{FLWSNEMASSB\_plot1}{}\).  The 2014 fully selected fishing mortality was estimated to be 0.16 which is 49\% of the overfishing threshold \(\$F\_{MSY}\${} = 0.325\;  Figure \ref{FLWSNEMAF\_plot1}{}\). Retrospective adjustments were not made to the model results. } \def\FLWSNEMAProj{ \textbf{Projections: }{}Short term projections of biomass were derived by sampling from a cumulative  distribution  function of recruitment estimates assuming a Beverton\-Holt stock recruitment relationship. The annual fishery selectivity, maturity ogive, and mean weights at age used  in projection  are the most recent 5 year averages\;  The model exhibited minor retrospective pattern in F and SSB so no retrospective adjustments were applied in the projections.} \def\FLWSNEMASpecCmt{ \textbf{Special Comments: } \begin{itemize}{} \item{}What are the most important sources of uncertainty in this stock assessment?  Explain, and describe qualitatively how they affect the assessment results \(such as estimates of biomass, F, recruitment, and population projections\).  \linebreak{} \hspace\*{0.5cm} \textit{A large source of uncertainty is the estimate of natural mortality based on longevity, which is not well studied in Southern New England Mid\-Atlantic Winter Flounder, and assumed constant over time.  Natural mortality affects the scale of the biomass and fishing mortality estimates.  Natural mortality was adjusted upwards from 0.2 to 0.3 during the last benchmark assessment assuming a max age of 16. However, there is still uncertainty in the true max age of the population and the resulting natural mortality estimate. Other sources of uncertainty include length distribution of the recreational discards.  The recreational discards, are a small component of the total catch, but the assessment suffers from very little length information used to characterize the recreational discards \(1 to 2 lengths in recent years\).}  \item{} Does this assessment model have a retrospective pattern? If so, is the pattern minor, or major? \(A major retrospective pattern occurs when the adjusted SSB or  \$F\_{Full}\${} lies outside of the approximate  joint confidence region for SSB and  \$F\_{Full}\${}\; see  Figure \ref{RhoDecision\_tab}{}\). \linebreak{} \hspace\*{0.5cm} \textit{ No retrospective adjustment of spawning stock biomass or fishing mortality in 2014 was required. }  \item{}Based on this stock assessment, are population projections well determined or uncertain? \linebreak{} \hspace\*{0.5cm} \textit{Population projections for Southern New England Mid\-Atlantic Winter Flounder are reasonably well determined. There is uncertainty in the estimates of M. In addition, while the retrospective pattern is considered minor \(within the 90\% CI of both F and SSB\)  the rho adjusted terminal value is very close to falling out of the bounds, becoming a major retrospective pattern. This would lead to retrospective adjustments being needed for the projections.}  \item{}Describe any changes that were made to the current stock assessment, beyond incorporating additional years of data  and the affect these changes had on the assessment and stock status. \linebreak{} \hspace\*{0.5cm} \textit{ No changes, other than the incorporation of new data were made to the Southern New England Mid\-Atlantic Winter Flounder assessment for this update.}  \item{}If the stock status has changed a lot since the previous assessment, explain why this occurred.  \linebreak{} \hspace\*{0.5cm} \textit{The stock status of Southern New England Mid\-Atlantic Winter Flounder has not changed since the previous benchmark in 2011.}  \item{}Indicate what data or studies are currently lacking and which would be needed most to improve this stock assessment in the future.  \linebreak{} \hspace\*{0.5cm} \textit{The Southern New England Mid\-Atlantic Winter Flounder assessment could be improved with additional studies on maximum age, as well additional information  of recreational discard lengths.  In addition, further investigation into the localized struture\/genetics of the stock is warranted. Also, a future shift to ASAP version 4 will provide the ability to model envirionmental factors that may influence both survey catchability and the modeled S\-R relationship}  \item{}Are there other important issues? \linebreak{} \hspace\*{0.5cm} \textit{None. } \end{itemize}{}} \def\FLWSNEMARefr{ \textbf{References: }{} \linebreak{}Smith, A. and S. Jones.  2008.  In.  Northeast Fisheries Science Center. 2008. Assessment of 19 Northeast Groundfish Stocks through 2007: Report of the 3$^{rd}$ Groundfish Assessment Review Meeting \(GARM III\), Northeast Fisheries Science Center, Woods Hole, Massachusetts, August 4\-8, 2008. US Dep Commer, NOAA Fisheries, Northeast Fish Sci Cent Ref Doc. 08\-15\; 884 p + xvii. http:\/\/www.nefsc.noaa.gov\/publications\/crd\/crd0815\/ \linebreak{} \linebreak{}Northeast Fisheries Science Center. 2011. 52$^{nd}$ Northeast Regional Stock AssessmentWorkshop \(52$^{nd}$ SAW\)  Assessment Report. US Dept Commer, Northeast Fish SciCent Ref Doc. 11\-17\; 962 p. Available from: National Marine Fisheries Service, 166Water Street, Woods Hole, MA 02543\-1026, or online at http:\/\/www.nefsc.noaa.gov\/nefsc\/publications\/ \linebreak{} \linebreak{}} \def\FLWSNEMADraft{} \def\FLWSNEMASPPname{Southern New England Mid-Atlantic Winter Flounder} \def\FLWSNEMASPPnameT{Southern New England Mid-Atlantic Winter Flounder} \def\FLWSNEMARptYr{2015} \def\FLWSNEMAAuthor{Anthony Wood} \def\FLWSNEMAReviewerComments{/home/dhennen/EIEIO/BigReport/FLW_SNEMA/latex}  \def\FLWGBMyPathTab{/home/dhennen/EIEIO/BigReport/FLW_GB/tables} \def\FLWGBMyPathFig{/home/dhennen/EIEIO/BigReport/FLW_GB/figures} \def\FLWGBfigFishCap{Total catches \(mt\)  of Georges Bank Winter Flounder between 1982 and 2015 by country and disposition \(landings and discards\).} \def\FLWGBfigSSBCap{Trends in spawning stock biomass \(mt\)  of Georges Bank Winter Flounder between 1982 and 2014 from the current  \(solid line\)  and previous \(dashed line\)  assessments and the corresponding  \$SSB\_{Threshold}\${} \(\$\dfrac{1}{2}\${} \$SSB\_{MSY}\${}\; horizontal dashed line\)  as well as  \$SSB\_{Target}\${} \(\$SSB\_{MSY}\${}\; horizontal dotted line\)   based on the 2015 assessment.  Biomass was adjusted for a retrospective pattern  and the adjustment is shown in red.  The approximate 90\% normal confidence intervals are shown.} \def\FLWGBfigFCap{Trends in fully selected fishing mortality \(\$F\_{Full}\${}\)  of Georges Bank Winter Flounder between 1982 and 2014 from the current  \(solid line\)  and previous \(dashed line\)  assessments and the corresponding  \$F\_{Threshold}\${} \(\$F\_{MSY}\${}=0.536\; horizontal dashed line\)  as well as \(\$F\_{Target}\${}= 75\% of FMSY\;  horizontal dotted line\). \$F\_{Full}\${} was adjusted for a retrospective pattern  and the adjustment is shown in red.  The approximate 90\% normal confidence intervals are also shown.} \def\FLWGBfigRecrCap{Trends in Recruits \(age 1\)  \(000s\)  of Georges Bank Winter Flounder between 1982 and 2014 from the current \(solid line\)  and previous \(dashed line\)  assessments. The approximate 90\% normal confidence intervals are shown.} \def\FLWGBfigSurvCap{Indices of biomass for the Georges Bank Winter Flounder for the Northeast Fisheries Science Center \(NEFSC\)  spring \(1968\-2015\)  and fall \(1963\-2014\)   bottom trawl surveys and the Canadian DFO spring survey \(1987\-2015\).  The approximate 90\% normal confidence intervals are shown.} \def\FLWGBPreAmb{This assessment of the Georges Bank Winter Flounder \(\textit{Pseudopleuronectes americanus}\)  stock is an operational update of the existing 2014 operational VPA assessment which included data for 1982\-2013 \(Hendrickson et al. 2015\). Based on the previous assessment the stock was not overfished and overfishing was not ocurring. This assessment updates commercial fishery catch data, research survey biomass indices, and the analytical VPA assessment model and reference points through 2014. Additionally, stock projections have been updated through 2018.} \def\FLWGBSoS{ \textbf{State of Stock: }{}Based on this updated assessment, the Georges Bank Winter Flounder \(\textit{Pseudopleuronectes americanus}\)  stock is overfished and overfishing is occurring \(Figures \ref{FLWGBSSB\_plot1}\-\ref{FLWGBF\_plot1}\){}. Retrospective adjustments were made to the model results.  Spawning stock biomass \(SSB\)  in 2014 was estimated to be 2,883 \(mt\)  which is 43\% of the biomass target for an overfished stock \(\$SSB\_{MSY}\${} = 6,700 with a threshold of 50\% of SSBMSY\;  Figure \ref{FLWGBSSB\_plot1}{}\).  The 2014 fully selected fishing mortality \(F\)  was estimated to be 0.778 which is 145\% of the overfishing threshold \(\$F\_{MSY}\${} = 0.536\;  Figure \ref{FLWGBF\_plot1}{}\). However, the 2014 point estimate of SSB and F, when adjusted for retrospective error \(83\% for SSB and \-51\% for F\), is outside the 90\% confidence interval of the unadjusted 2014 point estimate. Therefore, the 2014 F and SSB values used in the stock status determination were the retrospective\-adjusted values of 0.778 and 2,883 mt, respectively.} \def\FLWGBProj{ \textbf{Projections: }{}Short\-term projections of biomass were derived by sampling from a cumulative  distribution  function of recruitment estimates \(1982\-2013 YC\)  from the final run of the ADAPT VPA model. The annual fishery selectivity, maturity ogive, and mean weights\-at\-age used in the projection  are the most recent 5 year averages \(2010\-2014\). An SSB retrospective adjustment factor of 0.546 was applied in the projections.} \def\FLWGBSpecCmt{ \textbf{Special Comments: } \begin{itemize}{} \item{}What are the most important sources of uncertainty in this stock assessment?  Explain, and describe qualitatively how they affect the assessment results \(such as estimates of biomass, F, recruitment, and population projections\).  \linebreak{} \hspace\*{0.5cm} \textit{The largest source of uncertainty is the estimate of natural mortality based on longevity \(max. age = 20 for this stock\), which is not well studied in Georges Bank Winter Flounder, and assumed constant over time.  Natural mortality affects the scale of the biomass and fishing mortality estimates. Other sources of uncertainty include the underestimation of catches. Discards from the Canadian bottom trawl fleet were not provided by the CA DFO and the precision of the Canadian scallop dredge discard estimates, with only 1\-2 trips per month, are uncertain.The lack of age data for the Canadian spring survey catches requires the use of the US spring survey A\/L keys despite selectivity differences. In addition, there are no length or age composition data from the Canadian landings or discards GB winter flounder.}  \item{} Does this assessment model have a retrospective pattern? If so, is the pattern minor, or major? \(A major retrospective pattern occurs when the adjusted SSB or  \$F\_{Full}\${} lies outside of the approximate  joint confidence region for SSB and  \$F\_{Full}\${}\; see  Figure \ref{RhoDecision\_tab}{}\). \linebreak{} \hspace\*{0.5cm} \textit{ The 7\-year Mohn\'s  \textrho{}, relative to SSB, was 0.26 in the 2014 assessment and was 0.83 in 2014. The 7\-year Mohn\'s  \textrho{}, relative to F, was \-0.16 in the 2014 assessment and was \-0.51 in 2014. There was a major retrospective pattern for this assessment because the  \textrho{} adjusted estimates of 2014 SSB \(\$SSB\_{\rho}\${}=2,883\)  and 2014 F \(\$F\_{\rho}\${}=0.778\)  were outside the approximate 90\% confidence region around SSB \(3,783 \- 6,767\)  and F \(0.254 \- 0.504\).  A retrospective  adjustment was made for both the determination of stock status and for projections of catch in 2016. The retrospective adjustment changed the 2014 SSB from 5,275 to 2,883 and the 2014  \$F\_{Full}\${} from 0.379 to 0.778.}  \item{}Based on this stock assessment, are population projections well determined or uncertain? \linebreak{} \hspace\*{0.5cm} \textit{Population projections for Georges Bank Winter Flounder are reasonably well determined.}  \item{}Describe any changes that were made to the current stock assessment, beyond incorporating additional years of data  and the affect these changes had on the assessment and stock status. \linebreak{} \hspace\*{0.5cm} \textit{ The only change made to the Georges Bank Winter Flounder assessment, other than the incorporation of an additional  year of data, involved fishery selectivity.  During the 2014 assessment update, stock size estimates of age 1 and age 2 fish were not estimable  in the VPA during year t + 1 \(CVs near 1.0\). When age 2 stock size is not estimated in year t + 1,  the VPA model calculates the stock size of age 1 fish \(i.e., recruitment\)  in the terminal year by  using the age 1 partial recruitment \(PR\)  value to derive the F at age 1 in the terminal year. The  age 1 PR value used in the 2014 assessment update was 0.001. However, when this same age 1 PR value  was used in a VPA run for the current assessment update, the low PR value combined with the low age  1 catch in 2014 resulted in an unlikely high stock size estimate for age 1 recruitment in 2014 \(i.e.,  41,587,000 fish\)  when compared to survey observations of the same cohort \(i.e., age 1 in 2014 and age  2 in 2015\). In order to obtain a more realistic estimate of age 1 recruitment in 2014, I allowed the  VPA model to estimate age 2 stock size in 2015 \(i.e., and thereby avoided the use of an age 1 PR  value in the age 1 stock size calculation for 2014\)  and used the back\-calculated PR values from this  VPA run to derive a new PR\-at\-age vector which was used in the final 2015 VPA run. Similar to the  2014 assessment update, the final 2015 VPA run did not include the estimation of age 2 stock size  and the new PR\-at\-age vector was computed using the same methods as in the 2014 assessment.   Full selectivity occurs at age 4. For the 2015 assessment update, fishery selectivity for ages  1\-3 was changed from the 2014 assessment values of 0.001, 0.10 and 0.43, respectively, to 0.01,  0.08 and 0.55, respectively. Differences between estimates  of F, SSB and R values from the final  2015 VPA run, with the new PR vector, and a 2015 VPA run that utilized the PR vector from the 2014  assessment are shown in Table G30.}  \item{}If the stock status has changed a lot since the previous assessment, explain why this occurred.  \linebreak{} \hspace\*{0.5cm} \textit{The overfished and overfishing status of Georges Bank Winter Flounder has changed in the current assessment update due to a worsening of the retrospective error associated with fishing mortality and SSB.}  \item{}Indicate what data or studies are currently lacking and which would be needed most to improve this stock assessment in the future.  \linebreak{} \hspace\*{0.5cm} \textit{The Georges Bank Winter Flounder assessment could be improved with discard estimates from the Canadian bottom trawl fleet and age data from the Canadian spring bottom trawl surveys.}  \item{}Are there other important issues? \linebreak{} \hspace\*{0.5cm} \textit{None. } \end{itemize}{}} \def\FLWGBRefr{ \textbf{References: }{} \linebreak{} Hendrickson L, Nitschke P, Linton B. 2015. 2014 Operational Stock Assessments for Georges Bank winter flounder, Gulf of Maine winter flounder, and pollock. US Dept Commer, Northeast Fish Sci Cent Ref Doc. 15\-01\; 228 p. \linebreak{} \linebreak{}} \def\FLWGBDraft{} \def\FLWGBSPPname{Georges Bank Winter Flounder} \def\FLWGBSPPnameT{Georges Bank Winter Flounder} \def\FLWGBRptYr{2015} \def\FLWGBAuthor{Lisa Hendrickson} \def\FLWGBReviewerComments{/home/dhennen/EIEIO/BigReport/FLW_GB/latex}  \def\FLDGMGBMyPathTab{/home/dhennen/EIEIO/BigReport/FLD_GMGB/tables} \def\FLDGMGBMyPathFig{/home/dhennen/EIEIO/BigReport/FLD_GMGB/figures} \def\FLDGMGBfigFishCap{Total catch of northern windowpane flounder between 1975 and 2014 by disposition \(landings and discards\).} \def\FLDGMGBfigSSBCap{Trends in the biomass index \(a 3\-year moving average of the NEFSC fall bottom trawl survey index\)  of northern windowpane flounder between 1975 and 2014 from the current  assessment, and the corresponding  \$B\_{Threshold}\${} =  \$\dfrac{1}{2}\${} \$B\_{MSY}\${} \textit{proxy}{} = 0.777 kg\/tow \(horizontal dashed line\). } \def\FLDGMGBfigFCap{Trends in relative fishing mortality  of northern windowpane flounder between 1975 and 2014 from the current  assessment, and the corresponding  \$F\_{MSY}\${} \textit{proxy}{}=0.45 \(horizontal dashed line\). } \def\FLDGMGBfigRecrCap{} \def\FLDGMGBfigSurvCap{NEFSC fall bottom trawl survey indices in kg\/tow for northern windowpane flounder between 1975 and 2014  The approximate 90\% lognormal confidence intervals are shown.} \def\FLDGMGBPreAmb{This assessment of the northern windowpane flounder \(\textit{Scophthalmus aquosus}\)  stock is an operational update of the 2012 assessment which included updates through 2010 \(NEFSC 2012\). Based on the 2012 assessment the stock was overfished, and overfishing was ocurring. This assessment updates commercial fishery catch data, survey indices of abundance, AIM model results,  and reference points through 2014.} \def\FLDGMGBSoS{ \textbf{State of Stock: }{}Based on this updated assessment, the northern windowpane flounder \(\textit{Scophthalmus aquosus}\)  stock is overfished but overfishing is not occurring \(Figures \ref{FLDGMGBSSB\_plot1}\-\ref{FLDGMGBF\_plot1}\){}. Retrospective adjustments were not made to the model results. The mean NEFSC fall bottom trawl survey index from years 2012, 2013 and 2014 \(a 3\-year moving average is used as a biomass index\)  was 0.535 kg\/tow which is lower than the \$B\_{Threshold}\${} of 0.777 kg\/tow. The 2014 relative fishing mortality was estimated to be 0.393 kt per kg\/tow which is lower than the  \$F\_{MSY}\${} \textit{proxy}{} of 0.450 kt per kg\/tow.} \def\FLDGMGBProj{} \def\FLDGMGBSpecCmt{ \textbf{Special Comments: } \begin{itemize}{} \item{}What are the most important sources of uncertainty in this stock assessment?  Explain, and describe qualitatively how they affect the assessment results \(such as estimates of biomass, F, recruitment, and population projections\).  \linebreak{} \hspace\*{0.5cm} \textit{The main source of uncertainty in this assessment is the lack of windowpane discard estimates from Canadian fisheries to add to the catch component of model input. Discard estimates were from the U.S. only. There is overlap between the survey area and Canadian fishing grounds \(Van Eeckhaute et al. 2010\), which means catch from within the stock area was likely underestimated. }  \item{} Does this assessment model have a retrospective pattern? If so, is the pattern minor, or major? \(A major retrospective pattern occurs when the adjusted SSB or  \$F\_{Full}\${} lies outside of the approximate  joint confidence region for SSB and  \$F\_{Full}\${}\; see  Figure \ref{RhoDecision\_tab}{}\). \linebreak{} \hspace\*{0.5cm} \textit{ The model used to estimate status of this stock does not allow estimation of a retrospective pattern. }  \item{}Based on this stock assessment, are population projections well determined or uncertain? \linebreak{} \hspace\*{0.5cm} \textit{N\/A }  \item{}Describe any changes that were made to the current stock assessment, beyond incorporating additional years of data  and the affect these changes had on the assessment and stock status. \linebreak{} \hspace\*{0.5cm} \textit{No changes were made to the northern windowpane flounder assessment for this update  other than the incorporation of four years of new NEFSC fall bottom trawl survey data and  four years of new U.S. commercial landings and discard data \(2011 \- 2014\). }  \item{}If the stock status has changed a lot since the previous assessment, explain why this occurred.  \linebreak{} \hspace\*{0.5cm} \textit{The stock status of northern windowpane flounder changed from \'overfished and overfishing is occurring\' to \'overfished and overfishing is not occurring\' due to stable\-to\-decreasing catch since 2008, and an increasing trend in the survey index since 2010. }  \item{}Indicate what data or studies are currently lacking and which would be needed most to improve this stock assessment in the future.  \linebreak{} \hspace\*{0.5cm} \textit{The northern windowpane flounder assessment could be improved by estimating the Canadian windowpane removals and, although to a lesser degree, the \'general category\' scallop dredge fleet discards from within the stock area and using them as additional catch input to the AIM model.  While the model fit now is reasonable \(the relationship between ln\(relative F\)  and ln\(replacement ratio\), a measure of the relationship between catch and survey index values, has a p\-value of 0.079\)  there are probably removals unaccounted for in the model and the fit can likely be improved. }  \item{}Are there other important issues? \linebreak{} \hspace\*{0.5cm} \textit{None. } \end{itemize}{}} \def\FLDGMGBRefr{ \textbf{References: }{} \linebreak{} Most recent assessment update:  \linebreak{} Northeast Fisheries Science Center. 2012. Assessment or Data Updates of 13 Northeast Groundfish Stocks through 2010.  US Dept Commer, Northeast Fish Sci Cent Ref Doc. 12\-06\; 789 p. Available online at http:\/\/nefsc.noaa.gov\/publications\/  \linebreak{} \linebreak{} Most recent benchmark assessment:  \linebreak{} Northeast Fisheries Science Center. 2008. Assessment of 19 Northeast Groundfish Stocks through 2007:  Report of the 3$^{rd}$ Groundfish Assessment Review Meeting \(GARM III\), Northeast Fisheries Science Center,  Woods Hole, Massachusetts, August 4\-8, 2008. US Dep Commer, NOAA FIsheries, Northeast Fish Sci Cent Ref Doc. 08\-15\; 884 p + xvii.  \linebreak{} \linebreak{} Van Eeckhaute, L., Sameoto, J., and A. Glass. 2010. Discards of Atlantic cod, haddock and yellowtail flounder  from the 2009 Canadian scallop fishery on Georges Bank. TRAC Ref. Doc. 2010\/10. 7p.  \linebreak{} \linebreak{}} \def\FLDGMGBDraft{} \def\FLDGMGBSPPname{northern windowpane flounder} \def\FLDGMGBSPPnameT{Northern windowpane flounder} \def\FLDGMGBRptYr{2015} \def\FLDGMGBAuthor{Toni Chute} \def\FLDGMGBReviewerComments{/home/dhennen/EIEIO/BigReport/FLD_GMGB/latex}  \def\FLDSNEMAMyPathTab{/home/dhennen/EIEIO/BigReport/FLD_SNEMA/tables} \def\FLDSNEMAMyPathFig{/home/dhennen/EIEIO/BigReport/FLD_SNEMA/figures} \def\FLDSNEMAfigFishCap{Total catch of southern windowpane flounder between 1975 and 2014 by disposition \(landings and discards\).} \def\FLDSNEMAfigSSBCap{Trends in the biomass index \(a 3\-year moving average of the NEFSC fall bottom trawl survey index\)  of southern windowpane flounder between 1975 and 2014 from the current  assessment, and the corresponding  \$B\_{Threshold}\${} =  \$\dfrac{1}{2}\${} \$B\_{MSY}\${} \textit{proxy}{} = 0.123 kg\/tow\(horizontal dashed line\). } \def\FLDSNEMAfigFCap{Trends in relative fishing mortality  of southern windowpane flounder between 1975 and 2014 from the current  assessment, and the corresponding  \$F\_{MSY}\${} \textit{proxy}{}=2.027 \(horizontal dashed line\). } \def\FLDSNEMAfigRecrCap{} \def\FLDSNEMAfigSurvCap{NEFSC fall bottom trawl survey indices in kg\/tow for southern windowpane flounder between 1975 and 2014. The approximate 90\% lognormal confidence intervals are shown.} \def\FLDSNEMAPreAmb{This assessment of the southern windowpane flounder \(\textit{Scophthalmus aquosus}\)  stock is an operational update of the 2012 assessment which included updates through 2010 \(NEFSC 2012\). Based on the 2012 assessment the stock was not overfished, and overfishing was not ocurring. This assessment updates commercial fishery catch data, survey indices of abundance, AIM model results, and reference points through 2014. } \def\FLDSNEMASoS{ \textbf{State of Stock: }{}Based on this updated assessment, the southern windowpane flounder \(\textit{Scophthalmus aquosus}\)  stock is not overfished and overfishing is not occurring \(Figures \ref{FLDSNEMASSB\_plot1}\-\ref{FLDSNEMAF\_plot1}\){}. Retrospective adjustments were not made to the model results. The mean NEFSC fall bottom trawl survey index from years 2012, 2013, and 2014 \(a 3\-year moving average is used as a biomass index\)  was  0.413 \(kg\/tow\)  which is higher than the \$B\_{Threshold}\${}of 0.123 \(kg\/tow\). The 2014 relative fishing mortality was estimated to be  1.308 \(kt per kg\/tow\)  which is lower than the  \$F\_{MSY}\${} \textit{proxy}{} of 2.027 \(kt per kg\/tow\). } \def\FLDSNEMAProj{} \def\FLDSNEMASpecCmt{ \textbf{Special Comments: } \begin{itemize}{} \item{}What are the most important sources of uncertainty in this stock assessment?  Explain, and describe qualitatively how they affect the assessment results \(such as estimates of biomass, F, recruitment, and population projections\).  \linebreak{} \hspace\*{0.5cm} \textit{A source of uncertainty for this assessment is missing commercial discard estimates from the general category scallop dredge fleet that should be added to the catch time series for model input. }  \item{} Does this assessment model have a retrospective pattern? If so, is the pattern minor, or major? \(A major retrospective pattern occurs when the adjusted SSB or  \$F\_{Full}\${} lies outside of the approximate  joint confidence region for SSB and  \$F\_{Full}\${}\; see  Figure \ref{RhoDecision\_tab}{}\). \linebreak{} \hspace\*{0.5cm} \textit{ The model used to estimate status of this stock does not allow estimation of a retrospective pattern. }  \item{}Based on this stock assessment, are population projections well determined or uncertain? \linebreak{} \hspace\*{0.5cm} \textit{N\/A}  \item{}Describe any changes that were made to the current stock assessment, beyond incorporating additional years of data  and the affect these changes had on the assessment and stock status. \linebreak{} \hspace\*{0.5cm} \textit{ No changes were made to the southern windowpane flounder assessment for this update  other than the incorporation of four years of new NEFSC fall bottom trawl survey data and  four years of new U.S. commercial landings and discard data \(2011 \- 2014\). }  \item{}If the stock status has changed a lot since the previous assessment, explain why this occurred.  \linebreak{} \hspace\*{0.5cm} \textit{The stock status of southern windowpane flounder has not changed since the previous assessment. }  \item{}Indicate what data or studies are currently lacking and which would be needed most to improve this stock assessment in the future.  \linebreak{} \hspace\*{0.5cm} \textit{Estimates of discards from the general category scallop dredge fleet should be added to the catch time series for model input. However, the model fit is presently good with a randomization test indicating the correlation between ln\(relative F\)  and ln\(replacement ratio\), a measure of the relationship between catch and survey index values, is significant \(p = 0.002.\)  }  \item{}Are there other important issues? \linebreak{} \hspace\*{0.5cm} \textit{None. } \end{itemize}{}} \def\FLDSNEMARefr{ \textbf{References: }{} \linebreak{} Most recent assessment update:  \linebreak{} Northeast Fisheries Science Center. 2012. Assessment or Data Updates of 13 Northeast Groundfish Stocks through 2010.  US Dept Commer, Northeast Fish Sci Cent Ref Doc. 12\-06\; 789 p. Available online at http:\/\/nefsc.noaa.gov\/publications\/  \linebreak{} \linebreak{} Most recent benchmark assessment:  \linebreak{} Northeast Fisheries Science Center. 2008. Assessment of 19 Northeast Groundfish Stocks through 2007:  Report of the 3$^{rd}$ Groundfish Assessment Review Meeting \(GARM III\), Northeast Fisheries Science Center,  Woods Hole, MA, August 4\-8, 2008. US Dep Commer, NOAA Fisheries, Northeast Fish Sci Cent Ref Doc. 08\-15\; 884 p + xvii. \linebreak{} \linebreak{}} \def\FLDSNEMADraft{} \def\FLDSNEMASPPname{southern windowpane flounder} \def\FLDSNEMASPPnameT{Southern windowpane flounder} \def\FLDSNEMARptYr{2015} \def\FLDSNEMAAuthor{Toni Chute}