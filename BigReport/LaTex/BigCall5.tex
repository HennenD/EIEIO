\def\YELCCGMMyPathTab{/home/dhennen/EIEIO/BigReport/YEL_CCGM/tables} \def\YELCCGMMyPathFig{/home/dhennen/EIEIO/BigReport/YEL_CCGM/figures} \def\YELCCGMfigFishCap{Total catch of Cape Cod\-Gulf of Maine Yellowtail flounder between 1985 and 2014 by disposition \(landings and discards\).} \def\YELCCGMfigSSBCap{Trends in spawning stock biomass of Cape Cod\-Gulf of Maine Yellowtail flounder between 1985 and 2014 from the current  \(solid line\)  and previous \(dashed line\)  assessment and the corresponding  \$SSB\_{Threshold}\${} \(\$\dfrac{1}{2}\${} \$SSB\_{MSY}\${} \textit{proxy}{}\; horizontal dashed line\)  as well as  \$SSB\_{Target}\${} \(\$SSB\_{MSY}\${} \textit{proxy}{}\; horizontal dotted line\)   based on the 2015 assessment.  Biomass was adjusted for a retrospective pattern  and the adjustment is shown in red.   The 90\% bootstrap probability intervals are shown.} \def\YELCCGMfigFCap{Trends in the fully selected fishing mortality \(\$F\_{Full}\${}\)  of Cape Cod\-Gulf of Maine Yellowtail flounder between 1985 and 2014 from the current  \(solid line\)  and previous \(dashed line\)  assessment and the corresponding  \$F\_{Threshold}\${} \(\$F\_{MSY}\${} \textit{proxy}{}=0.279\; horizontal dashed line\).  \$F\_{Full}\${} was adjusted for a retrospective pattern  and the adjustment is shown in red  based on the 2015 assessment.  The 90\% bootstrap probability intervals are shown.} \def\YELCCGMfigRecrCap{Trends in Recruits \(age 1\)  \(000s\)  of Cape Cod\-Gulf of Maine Yellowtail flounder between 1985 and 2014 from the current \(solid line\)  and previous \(dashed line\)  assessment.  The 90\% bootstrap probability intervals are shown.} \def\YELCCGMfigSurvCap{Indices of biomass for the Cape Cod\-Gulf of Maine Yellowtail flounder between 1985 and 2015 for the Northeast Fisheries Science Center \(NEFSC\)  spring and fall bottom trawl surveys,  Massachusetts Department of Marine Fisheries \(MADMF\)  inshore state spring and fall bottom trawl surveys,and the Maine New Hampshire inshore state spring and fall state surveys  The 90\% bootstrap probability intervals are shown.} \def\YELCCGMPreAmb{This assessment of the Cape Cod\-Gulf of Maine Yellowtail flounder \(\textit{Limanda ferruginea}\)  stock is an operational update of the existing 2012 VPA assessment \(Legault et al., 2012\). The last benchmark for this stock was in 2008 \(Legault et al., 2008\). Based on the previous assessment the stock was overfished, and overfishing was ocurring. This assessment updates commercial fishery catch data, research survey indices of abundance, weights at age, and the analytical VPA assessment model and reference points through 2014. Additionally, stock projections have been updated through 2018} \def\YELCCGMSoS{ \textbf{State of Stock: }{}Based on this updated assessment, Cape Cod\-Gulf of Maine Yellowtail flounder \(\textit{Limanda ferruginea}\)  stock is overfished and overfishing is occurring \(Figures \ref{YELCCGMSSB\_plot1}\-\ref{YELCCGMF\_plot1}\){}.  Retrospective adjustments were made to the model results.  Spawning stock biomass \(SSB\)  in 2014 was estimated to be 857 \(mt\)  which is 16\% of the biomass target \(\$SSB\_{MSY}\${} \textit{proxy}{} = 5,259\;  Figure \ref{YELCCGMSSB\_plot1}{}\).  The 2014 fully selected fishing mortality was estimated to be 0.64 which is 229\% of the overfishing threshold proxy \(\$F\_{MSY}\${} \textit{proxy}{} = 0.279\;  Figure \ref{YELCCGMF\_plot1}{}\).} \def\YELCCGMProj{ \textbf{Projections: }{}Short term projections of biomass were derived by sampling from a cumulative  distribution function of recruitment estimates from ADAPT VPA. Recruitment estimates were hindcasted based on a simple linear regression between the NEFSC Fall survey abundance at age 1 and the VPA estimate at age 1.  The most recent two years \(2013 and 2014\)  were not included in the series of values due to high uncertainty in these estimates. This resulted in a total of 36 recruitment values: 8 from the hindcast predictions \(years 1977\-1984\)  and 28 from the VPA \(years 1985\-2012\). The annual fishery selectivity, maturity ogive, and mean weights at age used  in projection  are the most recent 5 year averages\;  retrospective adjustments were applied in the projections.} \def\YELCCGMSpecCmt{ \textbf{Special Comments: } \begin{itemize}{} \item{}What are the most important sources of uncertainty in this stock assessment?  Explain, and describe qualitatively how they affect the assessment results \(such as estimates of biomass, F, recruitment, and population projections\).  \linebreak{} \hspace\*{0.5cm} \textit{The largest source of uncertainty is the source of the retrospective pattern.This pattern has persisted for a number of years causing SSB estimates to decrease and F estimates to increaseas more years of data are added.}  \item{} Does this assessment model have a retrospective pattern? If so, is the pattern minor, or major? \(A major retrospective pattern occurs when the adjusted SSB or  \$F\_{Full}\${} lies outside of the approximate  joint confidence region for SSB and  \$F\_{Full}\${}\; see RhoDecisionTab.ref\). \linebreak{} \hspace\*{0.5cm} \textit{ The 7\-year Mohn\'s  \textrho{}, relative to SSB, was 0.68 in the 2012 assessment and was 0.98 in 2014. The 7\-year Mohn\'s  \textrho{}, relative to F, was \-0.19 in the 2012 assessment and was \-0.45 in 2014. There was a major retrospective pattern for this assessment because the  \textrho{} adjusted estimates of 2014 SSB \(\$SSB\_{\rho}\${}=857\)  and 2014 F \(\$F\_{\rho}\${}=0.64\)  were outside the approximate 90\% confidence regions around SSB \(1,375 \- 2,111\)  and F \(0.25 \- 0.52\).  A retrospective  adjustment was made for both the determination of stock status and for projections of catch in 2016. The retrospective adjustment changed the 2014 SSB from 1,695 to 857 and the 2014  \$F\_{Full}\${} from 0.355 to 0.64.}  \item{}Based on this stock assessment, are population projections well determined or uncertain? \linebreak{} \hspace\*{0.5cm} \textit{Population projections for Cape Cod\-Gulf of Maine Yellowtail flounder, are uncertain with projected biomass from the last assessmentabove the confidence bounds of the biomass estimated in the current assessment.}  \item{}Describe any changes that were made to the current stock assessment, beyond incorporating additional years of data  and the affect these changes had on the assessment and stock status. \linebreak{} \hspace\*{0.5cm} \textit{ No changes, other than the incorporation of new data were made to the Cape Cod\-Gulf of Maine Yellowtail flounder assessment for this update.}  \item{}If the stock status has changed a lot since the previous assessment, explain why this occurred.  \linebreak{} \hspace\*{0.5cm} \textit{The stock status has not changed since the previous assessment.}  \item{}Indicate what data or studies are currently lacking and which would be needed most to improve this stock assessment in the future.  \linebreak{} \hspace\*{0.5cm} \textit{Extensive studies have examined the causes of the retrospective patterns with no definitive conclusions other than a change in model does not resolve the issue.}  \item{}Are there other important issues? \linebreak{} \hspace\*{0.5cm} \textit{No. } \end{itemize}{}} \def\YELCCGMRefr{ \textbf{References: }{} \linebreak{}Legault, C,  L. Alade, S.Cadrin, J. King, and S. Sherman.  2008.  In.  Northeast Fisheries Science Center. 2008. Assessment of 19 Northeast Groundfish Stocks through 2007: Report of the 3$^{rd}$ Groundfish Assessment Review Meeting \(GARM III\), Northeast Fisheries Science Center, Woods Hole, Massachusetts, August 4\-8, 2008. US Dep Commer, NOAA Fisheries, Northeast Fish Sci Cent Ref Doc. 08\-15\; 884 p + xvii. http:\/\/www.nefsc.noaa.gov\/publications\/crd\/crd0815\/ \linebreak{} \linebreak{} Legault, C,  L. Alade, S.Emery, J. King, and S. Sherman.  2012.  In.  Northeast Fisheries Science Center. 2012. Assessment or Data Updates of 13 Northeast Groundfish Stocks through 2010. US Dept Commer, NOAA Fisheries, Northeast Fish Sci Cent Ref Doc. 12\-06.\; 789 p. http:\/\/nefsc.noaa.gov\/publications\/crd\/crd1206\/ \linebreak{} \linebreak{}} \def\YELCCGMDraft{} \def\YELCCGMSPPname{Cape Cod-Gulf of Maine Yellowtail flounder} \def\YELCCGMSPPnameT{Cape Cod-Gulf of Maine Yellowtail flounder} \def\YELCCGMRptYr{2015} \def\YELCCGMAuthor{Larry Alade} 