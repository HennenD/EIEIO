 \def\YELGBMyPathTab{/home/dhennen/EIEIO/BigReport/YEL_GB/tables} \def\YELGBMyPathFig{/home/dhennen/EIEIO/BigReport/YEL_GB/figures} \def\YELGBfigFishCap{Total catch of Georges Bank yellowtail flounder between 1935 and 2014 by fleet (US, Canadian, or Other)  and disposition (landings or discards).} \def\YELGBfigSSBCap{Trends in average survey biomass (mt)  of Georges Bank yellowtail flounder between 2010 and 2015 from the current assessment.} \def\YELGBfigFCap{Trends in the exploitation rate (catch/average survey biomass)  of Georges Bank yellowtail flounder between 2010 and 2014 from the current assessment.} \def\YELGBfigRecrCap{} \def\YELGBfigSurvCap{Indices of biomass for the Georges Bank yellowtail flounder between 1963 and 2015 for the Canadian DFO and Northeast Fisheries Science Center (NEFSC)  spring and fall bottom trawl surveys.  The approximate 90\percent{} lognormal confidence intervals are shown.} \def\YELGBPreAmb{This assessment of the Georges Bank yellowtail flounder (\textit{Limanda ferruginea})  stock was reviewed during the July 2015 TRAC meeting (Legault et al. 2015). It is an operational assessment of the existing 2014 update assessment (Legault et al. 2014). Based on the previous assessment the stock status was unknown, but stock condition was poor. This assessment updates commercial fishery catch data through 2014 (Table \ref{YELGBCatch_Status_Table}{},  Figure \ref{YELGBFish_plot1}{}), and updates research survey indices of abundance and the empirical approach assessment through 2015 (Figure \ref{YELGBSurv_plot1}{}). No stock projections can be computed using the empirical approach.} \def\YELGBSoS{ \textbf{State of Stock: }{}Based on this updated assessment, Georges Bank yellowtail flounder (\textit{Limanda ferruginea})  stock status is unknown due to a lack of biological reference points associated with the empirical approach, but stock condition is poor.  Retrospective adjustments were not made to the model results. The average survey biomass in 2015 (the arithmetic average of the 2015 DFO, 2015 NEFSC spring, and 2014 NEFSC fall surveys)  was estimated to be 2,240 (mt)  (Figure \ref{YELGBSSB_plot1}{}).  The 2014 exploitation rate (2014 catch divided by 2014 average survey biomass)  was estimated to be 0.071 (Figure \ref{YELGBF_plot1}{}).} \def\YELGBProj{ \textbf{Projections: }{}Short term projections cannot be computed using the empirical approach. Application of an exploitation rate of 2\percent{} to 16\percent{} to the 2015 average survey biomass (2,240 mt)  results in catch advice for 2016 of 45 mt to 359 mt.} \def\YELGBSpecCmt{ \textbf{Special Comments: } \begin{itemize}{} \item{}What are the most important sources of uncertainty in this stock assessment?  Explain, and describe qualitatively how they affect the assessment results (such as estimates of biomass, F, recruitment, and population projections).  \linebreak{} \hspace*{0.5cm} \textit{The largest source of uncertainty is the estimate of survey catchability, which currently relies on literature values for other species in other regions of the world using different gear. The survey catchability affects the expansion of the stratified mean catch per tow for each survey and is inversely related to the catch advice. Other sources of uncertainty include the appropriate exploitation rate to apply to this stock, which has seen continued decrease in survey biomass despite low exploitation rates. }  \item{} Does this assessment model have a retrospective pattern? If so, is the pattern minor, or major? (A major retrospective pattern occurs when the adjusted SSB or  $F_{Full}${} lies outside of the approximate  joint confidence region for SSB and  $F_{Full}${}; see RhoDecisionTab.ref). \linebreak{} \hspace*{0.5cm} \textit{ The model used to estimate status of this stock does not allow estimation of a retrospective pattern. }  \item{}Based on this stock assessment, are population projections well determined or uncertain? \linebreak{} \hspace*{0.5cm} \textit{Population projections for Georges Bank yellowtail flounder are not computed. Catch advice is derived from applying an exploitation rate to the current estimate of survey biomass. }  \item{}Describe any changes that were made to the current stock assessment, beyond incorporating additional years of data  and the effect these changes had on the assessment and stock status. \linebreak{} \hspace*{0.5cm} \textit{The 2014 NMFS spring survey value was changed from 2,684 mt to 2,763 mt due to using preliminary data during the 2014 TRAC meeting. However, this has no impact on the 2015 stock status or 2016 catch advice in this update assessment.}  \item{}If the stock status has changed a lot since the previous assessment, explain why this occurred.  \linebreak{} \hspace*{0.5cm} \textit{The stock status of Georges Bank yellowtail flounder remains unknown and stock condition continues to be poor.}  \item{}Indicate what data or studies are currently lacking and which would be needed most to improve this stock assessment in the future.  \linebreak{} \hspace*{0.5cm} \textit{The Georges Bank yellowtail flounder assessment could be improved with studies on NMFS and DFO survey catchability for flatfish.}  \item{}Are there other important issues? \linebreak{} \hspace*{0.5cm} \textit{None. } \end{itemize}{}} \def\YELGBRefr{ \textbf{References: }{} \linebreak{}Legault, C.M., L. Alade, W.E. Gross, and H.H. Stone. 2014. Stock Assessment of Georges Bank Yellowtail Flounder for 2014. TRAC Ref. Doc. 2014/01. 214 p. \href{http://www.nefsc.noaa.gov/saw/trac/TRAC_GBYT_2014_WP.pdf}{TRAC2014} \linebreak{} \linebreak{}Legault, C.M., L. Alade, D. Busawon, and H.H. Stone. 2015. Stock Assessment of Georges Bank Yellowtail Flounder for 2015. TRAC Ref. Doc. 2015/01. 66 p. \href{http://www.nefsc.noaa.gov/saw/trac/TSR_2015_GBYellowTailFlounder.pdf}{TRAC2015} \linebreak{}} \def\YELGBDraft{} \def\YELGBSPPname{Georges Bank yellowtail flounder} \def\YELGBSPPnameT{Georges Bank yellowtail flounder} \def\YELGBRptYr{2015} \def\YELGBAuthor{Chris Legault} \def\YELGBReviewerComments{/home/dhennen/EIEIO/BigReport/YEL_GB/latex}