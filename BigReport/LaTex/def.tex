\\def\\HALUNITMyPathTab{/home/dhennen/EIEIO/BigReport/HAL_UNIT/tables} \\def\\HALUNITMyPathFig{/home/dhennen/EIEIO/BigReport/HAL_UNIT/figures} \\def\\HALUNITfigFishCap{Total catch of Atlantic halibut between 1963 and 2014 by disposition \(landings and discards\).} \\def\\HALUNITfigSSBCap{Estimated trends in the biomass of Atlantic halibut between 1963 and 2014 from the current  \(solid line\)  and previous \(dashed line\)  assessment and the corresponding  \$B\_{Threshold}\${}= \$\\dfrac{1}{2}\${} \$B\_{MSY}\${} \\textit{proxy}{}\(horizontal dashed line\)  as well as  \$B\_{Target}\${} \(\$B\_{MSY}\${} \\textit{proxy}{}\; horizontal dotted line\)   based on the 2015 assessment.} \\def\\HALUNITfigFCap{Estimated trends in the fully selected fishing mortality \(\$F\_{Full}\${}\)  of Atlantic halibut between 1963 and 2014 from the current  \(solid line\)  and previous \(dashed line\)  assessment and the corresponding  \$F\_{Threshold}\${} \(0.073\; horizontal dashed line\)  as well as  \$F\_{Target}\${} \(0.8 \* \$F\_{MSY}\${} \\textit{proxy}{}\; dotted line\)   based on the 2015 assessment. } \\def\\HALUNITfigRecrCap{} \\def\\HALUNITfigSurvCap{Indices of biomass for the Atlantic halibut between 1963 and 2014 for the Northeast Fisheries Science Center \(NEFSC\)  fall bottom trawl survey.  The 90\\% lognormal confidence intervals are shown.} \\def\\HALUNITPreAmb{This assessment of the Atlantic halibut \(\\textit{Hippoglossus hippoglossus}\)  stock is an update of the existing 2012 benchmark assessment \(NEFSC 2010\)  and the last update assessment \(NEFSC 2012\). This assessment updates commercial fishery catch data, research survey indices of abundance, and the replacement yield assessment model through 2014. Additionally, stock projections have been updated through 2018. Reference points have not been updated. } \\def\\HALUNITSoS{ \\textbf{State of Stock: }{}Based on this updated assessment, Atlantic halibut \(\\textit{Hippoglossus hippoglossus}\)  stock is unknown and unknown \(Figures \\ref{HALUNITSSB\_plot1}\-\\ref{HALUNITF\_plot1}\){}. Retrospective adjustments were not made to the model results.  Biomass \(SSB\)  in 2014 was estimated to be 96,464 \(mt\)  which is 199\\% of the biomass target \(\$SSB\_{MSY}\${} \\textit{proxy}{} = 48,509\;  Figure \\ref{HALUNITSSB\_plot1}{}\).  The 2014 fully selected fishing mortality was estimated to be 0.001 which is 1\\% of the overfishing threshold proxy \(\$F\_{MSY}\${} \\textit{proxy}{} = 0.073\;  Figure \\ref{HALUNITF\_plot1}{}\).} \\def\\HALUNITProj{ \\textbf{Projections: }{} Short term projections were based on a constant F =  \$F\_{MSY}\${} \\textit{proxy}{} = 0.073.  Projections use the assessment model \(replacement yield\)  and maintain all other model assumptions.} \\def\\HALUNITSpecCmt{ \\textbf{Special Comments: } \\begin{itemize}{} \\item{}What are the most important sources of uncertainty in this stock assessment?  Explain, and describe qualitatively how they affect the assessment results \(such as estimates of biomass, F, recruitment, and population projections\).  \\linebreak{} \\hspace\*{0.5cm} \\textit{The assessment model used for Atlantic halibut is highly uncertain.  It estimates one parameter, the initial biomass, and  proceeds deterministically from 1800 to 2014.  The model is highly sensitive to the initial biomass.  The model is  tuned to the survey index, which is inefficient for Atlantic halibut, catches very few animals and is therefore noisy.   The RYM model assumes no immigration or emmigration and that the population both began, and tends to, equilibrium.   These assumptions are unlikely to be true for Atlantic halibut. The model estimates a biomass that is approximately equal  to unfished biomass, which is not credible. Catch has been very low for at least 100 years relative  to the landings reported early in the time series, despite a strong market and high value  relative to other groundfish.  The low catch throughout the century implies that the Atlantic halibut stock is very likely  depleted relative to it\'s unfished condition and is therefore likely to be overfished, even if its current biomass is  unknown.}  \\item{} Does this assessment model have a retrospective pattern? If so, is the pattern minor, or major? \(A major retrospective pattern occurs when the adjusted SSB or  \$F\_{Full}\${} lies outside of the approximate  joint confidence region for SSB and  \$F\_{Full}\${}\; see  Figure \\ref{RhoDecision\_tab}{}\). \\linebreak{} \\hspace\*{0.5cm} \\textit{ The model used to determine the status of this stock does not allow estimation of a retrospective pattern. }  \\item{}Based on this stock assessment, are population projections well determined or uncertain? \\linebreak{} \\hspace\*{0.5cm} \\textit{Population projections for Atlantic halibut are uncertain because biomass cannot be reasonably determined using  the current assessment model.}  \\item{}Describe any changes that were made to the current stock assessment, beyond incorporating additional years of data  and the affect these changes had on the assessment and stock status. \\linebreak{} \\hspace\*{0.5cm} \\textit{ The catch data were slightly altered due to the exclusion of catch made in international waters and the  re\-estiamtion of average discard ratio after 1998 \(due to the incorporation of more years of data\).}  \\item{}If the stock status has changed a lot since the previous assessment, explain why this occurred.  \\linebreak{} \\hspace\*{0.5cm} \\textit{The overfishing and overfished status of Atlantic halibut cannot be determined using the current assessment.  This  occurred because diagnostics showed the model was unreliable.  }  \\item{}Indicate what data or studies are currently lacking and which would be needed most to improve this stock assessment in the future.  \\linebreak{} \\hspace\*{0.5cm} \\textit{The Atlantic halibut assessment could be improved with additional studies on stock structure, additional age and length data,  a more precise and accurrate survey, and an investigation of alternate assessment models.}  \\item{}Are there other important issues? \\linebreak{} \\hspace\*{0.5cm} \\textit{Atlantic halibut are clearly depleted relative to their unfished state.  Catches have been far below historical landings  for more than 100 years, despite a lack of regulation before 1999 and a strong commercial market.  The current  assessment model implies that Atlantic halibut is near or above its unfished biomass and could support removals  commensurate with MSY.  The current assessment should probably not be used to inform management decisions.} \\end{itemize}{}} \\def\\HALUNITRefr{ \\textbf{References: }{} \\linebreak{} Northeast Fisheries Science Center. 2012. Assessment or Data Updates of 13 Northeast Groundfish Stocks  through 2010. US Dept Commer, Northeast Fish Sci Cent Ref Doc. 12\-06\; 789 p. Available from: National  Marine Fisheries Service, 166 Water Street, Woods Hole, MA 02543\-1026, or online at  http:\/\/nefsc.noaa.gov\/publications\/ \\linebreak{} \\linebreak{}Col, L.A., Legault, C.M. 2009. The 2008 Assessment of Atlantic halibut in the Gulf of Maine Georges Bank region.  US Dept Commer, Northeast Fish Sci Cent Ref Doc. 09\-08\; 39 p. Available from: National Marine Fisheries Service, 166 Water Street, Woods Hole, MA 02543\-1026, or online at http:\/\/www.nefsc.noaa.gov\/nefsc\/publications\/ } \\def\\HALUNITDraft{} \\def\\HALUNITSPPname{Atlantic halibut} \\def\\HALUNITSPPnameT{Atlantic halibut} \\def\\HALUNITRptYr{2015} \\def\\HALUNITAuthor{Daniel Hennen} \\def\\HALUNITReviewerComments{/home/dhennen/EIEIO/BigReport/HAL_UNIT/latex}  \\def\\CODGMMyPathTab{/home/dhennen/EIEIO/BigReport/COD_GM/tables} \\def\\CODGMMyPathFig{/home/dhennen/EIEIO/BigReport/COD_GM/figures} \\def\\CODGMfigFishCap{Total catch of Gulf of Maine Atlantic cod between 1982 and 2014 by fleet \(commercial and recreational\)  and disposition \(landings and discards\).} \\def\\CODGMfigSSBCap{Estimated trends in the spawning stock biomass \(SSB\)  of Gulf of Maine Atlantic cod between 1982 and 2014 from the current  \(solid line\)  and previous \(dashed line\)  assessment and the corresponding  \$SSB\_{Threshold}\${} \(\$\\dfrac{1}{2}\${} \$SSB\_{MSY}\${}\; horizontal dashed line\)  as well as  \$SSB\_{Target}\${} \$SSB\_{MSY}\${}\; horizontal dotted line\)   based on the 2015 M=0.2 \(A\)  and M\-ramp \(B\)  assessment models. The 90\\% lognormal confidence intervals are shown. The red dot indicates the rho\-adjusted SSB values that would have resulted had a retrospective adjusment been made to either model \(see Special Comments section\).} \\def\\CODGMfigFCap{Estimated trends in the fully selected fishing mortality \(F\)  of Gulf of Maine Atlantic cod between 1982 and 2014 from the current  \(solid line\)  and previous \(dashed line\)  assessment and the corresponding  \$F\_{Threshold}\${} \(0.185 \(M=0.2\), 0.187 \(M\-ramp\)\; dashed line\)  based on the 2015 M=0.2 \(A\)  and M\-ramp \(B\)  assessment models. The 90\\% lognormal confidence intervals are shown. The red dot indicates the rho\-adjusted F values that would have resulted had a retrospective adjusment been made to either model \(see Special Comments section\).} \\def\\CODGMfigRecrCap{Estimated trends in age\-1 recruitment  \(000s\)  of Gulf of Maine Atlantic cod between 1982 and 2014 from the current \(solid line\)  and previous \(dashed line\)  M=0.2 \(A\)  and M\-ramp \(B\)  assessment models. The 90\\% lognormal confidence intervals are shown.} \\def\\CODGMfigSurvCap{Indices of biomass for the Gulf of Maine Atlantic cod between 1963 and 2015 for the Northeast Fisheries Science Center \(NEFSC\)  spring and fall bottom trawl surveys and Massachusetts Division of Marine Fisheries \(MADMF\)  spring bottom trawl survey.  The 90\\% lognormal confidence intervals are shown.} \\def\\CODGMPreAmb{This assessment of the Gulf of Maine Atlantic cod \(\\textit{Gadus morhua}\)  stock is an update of the existing 2014 assessment \(Palmer 2014\). This assessment updates commercial and recreational fishery catch data, research survey indices of abundance, and the analytical ASAPassessment models through 2014. Additionally, stock projections have been updated through 2018. In what follows, there are two population assessment models brought forward from the most recent benchmark assessment \(2012\), the M=0.2 \(natural mortality = 0.2\)  and the M\-ramp \(M ramps from 0.2 to 0.4\)  assessment models \(see NEFSC 2013 for a full description of the model formulations\).} \\def\\CODGMSoS{ \\textbf{State of Stock: }{}Based on this updated assessment, the Gulf of Maine Atlantic cod \(\\textit{Gadus morhua}\)  stock is overfished and overfishing is occurring \(Figures \\ref{CODGMSSB\_plot1}\-\\ref{CODGMF\_plot1}\){}. Retrospective adjustments were not made to the model results \(see Special Comments section of this report\). Spawning stock biomass \(SSB\)  in 2014 was estimated to be 2,225 \(mt\)  under the M=0.2 model and 2,536 \(mt\)  under the M\-ramp model scenario \(Table \\ref{CODGMCatch\_Status\_Table}{}\)  which is 6 and 4\\% \(respectively\)  of the biomass target,  \$SSB\_{MSY}\${} \\textit{proxy}{} \(40,187 \(mt\)  and 59,045 \(mt\)\;  Figure \\ref{CODGMSSB\_plot1}{}\).  The 2014 fully selected fishing mortality was estimated to be 0.956 and 0.932 which is 517 and 498\\% of the  \$F\_{MSY}\${} \\textit{proxy}{}\(\$F\_{40\\\%}\${}\; 0.185 and 0.187\;  Figure \\ref{CODGMF\_plot1}{}\).} \\def\\CODGMProj{ \\textbf{Projections: }{} Short term projections of median total fishery yield and spawning stock biomass for Gulf of Maine Atlantic cod were conducted based on a harvest scenario of fishing at the FMSY proxy between 2016 and 2018. Catch in 2015 was estimated at 279 mt. Recruitment was sampled from a cumulative distribution function derived from ASAP estimated age\-1 recruitment between 1982 and 2012.  The projection recruitment model declines linearly to zero when SSB is below 6.3 kmt under the M=0.2 model and 7.9 kmt under the M\-ramp model. The 2015 age\-1 recruitment was estimated from the geometric mean of the 2010\-2014 ASAP recruitment estimates. No retrospective adjustments were applied in the projections as the retrospective patterns are similar to the 2014 update for which no retrospective adjustments were made\; however, the 2015 assessment review panel recommended that that M=0.2 projections with retrospective adjustments be brought forward to the SSC for consideration in the evaluation of uncertainty when setting catch advice \(provided in the Supplemental Information Report, \\href{http:\/\/www.nefsc.noaa.gov\/saw\/sasi\/sasi\_report\_options.php}{SASINF}{}\). Assumed weights are based on an average of the most recent three years. For the M\-ramp model, projections are shown under two assumptions of short\-term natural mortality: M=0.2 and M=0.4.} \\def\\CODGMSpecCmt{ \\textbf{Special Comments: } \\begin{itemize}{} \\item{}What are the most important sources of uncertainty in this stock assessment?  Explain, and describe qualitatively how they affect the assessment results \(such as estimates of biomass, F, recruitment, and population projections\).  \\linebreak{} \\hspace\*{0.5cm} \\textit{The largest source of uncertainty is the estimate of natural mortality. Past investigations into changes in natural mortality over time have been inconclusive \(NEFSC 2013\). Different assumptions about natural mortality affect the scale of the biomass, recruitment, and fishing mortality estimates. Other areas of uncertainty include the retrospective error in the M=0.2 model, residual patterns in the model fits to some of the survey series \(e.g., aggregate MADMF spring survey\)  and stock structure.}  \\item{}Does this assessment model have a retrospective pattern? If so, is the pattern minor, or major? \(A major retrospective pattern occurs when the adjusted SSB or  \$F\_{Full}\${} lie outside of the approximate joint confidence region for SSB and  \$F\_{Full}\${}\). \\linebreak{} \\hspace\*{0.5cm} \\textit{The M=0.2 model has a major retrospective pattern \(7\-year Mohn\'s rho SSB=0.54, F=\-0.31\)  and the M\-ramp model has a minor retrospective pattern \(7\-year Mohn\'s rho SSB=0.20, F=\-0.08\). The 7\-year Mohn\'s rho values from the current assessment are similar to those from the 2014 assessment \(M=0.2: SSB=0.53, F=\-0.33\; M\-ramp: SSB=0.17, F=\-0.05\)  where the M=0.2 model had a major retrospective pattern and the M\-ramp model had a minor pattern. No retrospective adjustment have been to the terminal model results or in the base catch projections following the recommendations of the SARC 55 and 2014 assessment review panels. The 2015 assessment review panel supported this decision noting that the most recent retrospective \'peel\' suggested that an adjustment using the 7\-year average may not be appropriate. However, the 2015 review panel highlighted the retrospective error in the M=0.2 model as a source of uncertainty \- it should be noted that the retrospective error of the most recent peel is larger for the M\-ramp model. Should the retrospective patterns continue then the models may have overestimated spawning stock size and underestimated fishing mortality.}  \\item{}Based on this stock assessment, are population projections well determined or uncertain? \\linebreak{} \\hspace\*{0.5cm} \\textit{Population projections for Gulf of Maine Atlantic cod are reasonably well determined and projected boimass from the last assessment  was within the confidence bounds of the biomass estimated in the current assessment. }  \\item{}Describe any changes that were made to the current stock assessment, beyond incorporating additional years of data  and the affect these changes had on the assessment and stock status. \\linebreak{} \\hspace\*{0.5cm} \\textit{ This update included several minor changes to model input data including: \(1\)  re\-estimation of recreational catch from 2004\-2014 to account for recent updates to the MRIP data\; \(2\)  a revised assumption on recreational discard mortality from 30\\% to 15\\% following a Capizzano et al. 2015 study \(unpublished\)\; and \(3\)  re\-estimation of 2009\-2014 NEFSC spring and fall survey time series using the TOGA station acceptance criterion. Additionally, the ASAP assessment model was run with the likelihood constants option turned off. All of these changes had minimal impacts on model results \- summaries of the impacts of these changes are provided in the Supplemental Information Report \(\\href{http:\/\/www.nefsc.noaa.gov\/saw\/sasi\/sasi\_report\_options.php}{SASINF}{}\).}  \\item{}If the stock status has changed a lot since the previous assessment, explain why this occurred.  \\linebreak{} \\hspace\*{0.5cm} \\textit{There has been no change in stock status since the 2014 udpate assessment.}  \\item{}Indicate what data or studies are currently lacking and which would be needed most to improve this stock assessment in the future.  \\linebreak{} \\hspace\*{0.5cm} \\textit{The Gulf of Maine Atlantic cod assessment could be improved with additional studies on natural mortality and stock structure. Additionally, future assessments should consider possible changes in recent fishery selectivity patterns and exlore alternative methods for estimating recruitment. Potential causes of low stock productivity \(i.e., low recruitment\)  should also be investigated.}  \\item{}Are there other important issues? \\linebreak{} \\hspace\*{0.5cm} \\textit{ When setting catch advice careful attention should be given to the retrospective error present in both models, particularly given the poor performance of previous stock projections. Additionally, it is unclear as to which level of natural mortality \(M=0.2 or 0.4\)  to assume for the short\-term projections under the M\-ramp model.} \\end{itemize}{}} \\def\\CODGMRefr{ \\textbf{References: }{} \\linebreak{}Northeast Fisheries Science Center. 2013. 55$^{th}$ Northeast Regional Stock Assessment Workshop \(55$^{th}$ SAW\)  Assessment Summary Report. US Dept Commer, Northeast Fish Sci Cent Ref Doc. 13\-01\; 41 p. Available from: National Marine Fisheries Service, 166 Water Street, Woods Hole, MA 02543\-1026 \\linebreak{} \\linebreak{}Palmer MC. 2014. 2014 Assessment update report of the Gulf of Maine Atlantic cod stock. US Dept Commer, Northeast Fish Sci Cent Ref Doc. 14\-14\; 119 p. Available from: National Marine Fisheries Service,166 Water Street, Woods Hole, MA 02543\-1026 } \\def\\CODGMDraft{} \\def\\CODGMSPPname{Gulf of Maine Atlantic cod} \\def\\CODGMSPPnameT{Gulf of Maine Atlantic cod} \\def\\CODGMRptYr{2015} \\def\\CODGMAuthor{Michael Palmer} \\def\\CODGMReviewerComments{/home/dhennen/EIEIO/BigReport/COD_GM/latex}  \\def\\CODGBMyPathTab{/home/dhennen/EIEIO/BigReport/COD_GB/tables} \\def\\CODGBMyPathFig{/home/dhennen/EIEIO/BigReport/COD_GB/figures} \\def\\CODGBfigFishCap{Total catch of Georges Bank Atlantic Cod between 1978 and 2014 by fleet \(US commercial, US recreational, or Canadian\)  and disposition \(landings and discards\).} \\def\\CODGBfigSSBCap{Trends in spawning stock biomass of Georges Bank Atlantic Cod between 1978 and 2014 from the current  \(solid line\)  and previous \(dashed line\)  assessment and the corresponding  \$SSB\_{Threshold}\${} \(\$\\dfrac{1}{2}\${} \$SSB\_{MSY}\${} \\textit{proxy}{}\; horizontal dashed line\)  as well as  \$SSB\_{Target}\${} \(\$SSB\_{MSY}\${} \\textit{proxy}{}\; horizontal dotted line\)   based on the 2015 assessment.  Biomass was adjusted for a retrospective pattern  and the adjustment is shown in red.  The approximate 90\\% lognormal confidence intervals are shown.} \\def\\CODGBfigFCap{Trends in the fully selected fishing mortality \(\$F\_{Full}\${}\)  of Georges Bank Atlantic Cod between 1978 and 2014 from the current  \(solid line\)  and previous \(dashed line\)  assessment and the corresponding  \$F\_{Threshold}\${} \(\$F\_{MSY}\${} \\textit{proxy}{}=0.169\; horizontal dashed line\).  \$F\_{Full}\${} was adjusted for a retrospective pattern  and the adjustment is shown in red,  based on the 2015 assessment. The approximate 90\\% lognormal confidence intervals are shown.} \\def\\CODGBfigRecrCap{Trends in Recruits \(age 1\)  \(000s\)  of Georges Bank Atlantic Cod between 1978 and 2014 from the current \(solid line\)  and previous \(dashed line\)  assessment. The approximate 90\\% lognormal confidence intervals are shown.} \\def\\CODGBfigSurvCap{Indices of biomass for the Georges Bank Atlantic Cod between 1963 and 2015 for the Northeast Fisheries Science Center \(NEFSC\)  spring and fall, and the DFO research bottom trawl surveys.  The approximate 90\\% lognormal confidence intervals are shown.} \\def\\CODGBPreAmb{This assessment of the Georges Bank Atlantic Cod \(\\textit{Gadus morhua}\)  stock is an operational update of the existing 2012 benchmark assessment \(NEFSC 2013\). Based on the previous assessment the stock was overfished, and overfishing was ocurring. This 2015 assessment updates commercial fishery catch data, research survey indices of abundance, the analytical ASAP assessment model, and reference points through 2014. Additionally, stock projections have been updated through 2018.} \\def\\CODGBSoS{ \\textbf{State of Stock: }{}Based on this updated assessment, the Georges Bank Atlantic Cod \(\\textit{Gadus morhua}\)  stock is overfished and overfishing is occurring \(Figures \\ref{CODGBSSB\_plot1}\-\\ref{CODGBF\_plot1}\){}.  Retrospective adjustments were made to the model results.  Spawning stock biomass \(SSB\)  in 2014 was estimated to be 1,804 \(mt\)  which is 1\\% of the biomass target for this stock \(\$SSB\_{MSY}\${} \\textit{proxy}{} = 201,152\;  Figure \\ref{CODGBSSB\_plot1}{}\).  The 2014 fully selected fishing mortality was estimated to be 1.68 which is 994\\% of the overfishing threshold proxy \(\$F\_{MSY}\${} \\textit{proxy}{} = 0.169\;  Figure \\ref{CODGBF\_plot1}{}\).} \\def\\CODGBProj{ \\textbf{Projections: }{}Short term projections of biomass were derived by sampling from a two\-stage cumulative  distribution  function of recruitment estimates from ASAP model results, using a 50,000 mt cutpoint. The annual fishery selectivity, maturity ogive, and mean weights at age used in projections are the most recent 5 year averages\;  retrospective adjustments were applied in the projections.} \\def\\CODGBSpecCmt{ \\textbf{Special Comments: } \\begin{itemize}{} \\item{}What are the most important sources of uncertainty in this stock assessment?  Explain, and describe qualitatively how they affect the assessment results \(such as estimates of biomass, F, recruitment, and population projections\).  \\linebreak{} \\hspace\*{0.5cm} \\textit{The major source of uncertainty is presumbaly the estimate of catch or of natural mortality, considering the magnitude of the retrospective bias. These both affect the scale of the biomass, fishing mortality estimates, and the reference point estimates. The catch estimates do not include all discards \(e.g.,lobster gear\)  and includes uncertain estimates of recreational landings and discards, and of some commercial discards \(e.g., small mesh\). Natural mortality \(M\)  of Georges Bank Atlantic Cod is not well understood and is assumed constant over time in the model. Other sources of uncertainty include possible changes in growth parameters in recent years and how this affects fecundity, the viability of eggs\/sperm, and the success rate of hatching \- all influencing recruitment survival and year class strength.}  \\item{} Does this assessment model have a retrospective pattern? If so, is the pattern minor, or major? \(A major retrospective pattern occurs when the adjusted SSB or  \$F\_{Full}\${} lies outside of the approximate  joint confidence region for SSB and  \$F\_{Full}\${}\; see  Figure \\ref{RhoDecision\_tab}{}\). \\linebreak{} \\hspace\*{0.5cm} \\textit{ The 7\-year Mohn\'s  \\textrho{}, relative to SSB, was 0.68 in the 2012 assessment and was 2.43 in 2014. The 7\-year Mohn\'s  \\textrho{}, relative to F, was \-0.46 in the 2012 assessment and was \-0.72 in 2014. There was a major retrospective pattern for this assessment because the  \\textrho{} adjusted estimates of 2014 SSB \(\$SSB\_{\\rho}\${}=1,804\)  and 2014 F \(\$F\_{\\rho}\${}=1.68\)  were outside the approximate 90\\% confidence regions around SSB \(3,922 \- 10,596\)  and F \(0.251 \- 0.815\).  A retrospective  adjustment was made for both the determination of stock status and for projections of catch in 2016. The retrospective adjustment changed the 2014 SSB from 6,180 to 1,804 and the 2014  \$F\_{Full}\${} from 0.463 to 1.68.}  \\item{}Based on this stock assessment, are population projections well determined or uncertain? \\linebreak{} \\hspace\*{0.5cm} \\textit{Population projections for Georges Bank Atlantic Cod are uncertain and likely optimistic. The projections are based on a biomass cutpoint of 50,000 mt, which has not been produced since 1992. The average recruitment since 1992 has been 4.9 million age 1 fish, whereas during the last 10 years, average recruitment has been about 2.7 million age 1 fish. A sensistivity projection using the most recent 10 years of recruitment was conducted and results presented in the SASINF database. }  \\item{}Describe any changes that were made to the current stock assessment, beyond incorporating additional years of data  and the effect these changes had on the assessment and stock status. \\linebreak{} \\hspace\*{0.5cm} \\textit{ No major changes, other than the addition of recent years of data, were made to the Georges Bank Atlantic Cod assessment for this update. However, recreational catch and commercial discard estimates were revised slightly due to minor changes in the databases, and the application of length frequencies \(annual instead of half year\)  in one instance.}  \\item{}If the stock status has changed a lot since the previous assessment, explain why this occurred.  \\linebreak{} \\hspace\*{0.5cm} \\textit{As in recent assessments for Georges Bank Atlantic Cod the stock remains in an overfishing and overfished status.}  \\item{}Indicate what data or studies are currently lacking and which would be needed most to improve this stock assessment in the future.  \\linebreak{} \\hspace\*{0.5cm} \\textit{The Georges Bank Atlantic Cod assessment could be improved with additional studies on natural mortality, growth, and fecundity. Additionally, more precise estimates of recreational landings and discards, sampling of fish caught by individual recreational anglers, and incorporation of discards in the lobster fishery would decrease uncertainty in the discard esimates.}  \\item{}Are there other important issues? \\linebreak{} \\hspace\*{0.5cm} \\textit{The differences in model assumptions of natural mortality between the SARC GB cod and the TRAC eGB cod assessment is problematic for the recovery of the entire GB cod stock. Model results of the TRAC VPA M=0.8 model are used to determine quota for the eGB management unit, so by default, proportionally more cod are being removed from eastern GB than what the GB cod ASAP model would predict.} \\end{itemize}{}} \\def\\CODGBRefr{ \\textbf{References: }{} \\linebreak{}Northeast Fisheries Science Center. 2013. 55$^{th}$ Northeast Regional Stock AssessmentWorkshop \(55$^{th}$ SAW\)  Assessment Summary Report. Northeast Fisheries Science CenterReference Document 13\-01:43. \\linebreak{} \\linebreak{}} \\def\\CODGBDraft{} \\def\\CODGBSPPname{Georges Bank Atlantic Cod} \\def\\CODGBSPPnameT{Georges Bank Atlantic Cod} \\def\\CODGBRptYr{2015} \\def\\CODGBAuthor{Loretta O\'Brien} \\def\\CODGBReviewerComments{/home/dhennen/EIEIO/BigReport/COD_GB/latex}  \\def\\HADGMMyPathTab{/home/dhennen/EIEIO/BigReport/HAD_GM/tables} \\def\\HADGMMyPathFig{/home/dhennen/EIEIO/BigReport/HAD_GM/figures} \\def\\HADGMfigFishCap{Total catch of Gulf of Maine haddock between 1977 and 2014 by fleet \(commercial, recreational, or foreign\)  and disposition \(landings and discards\).} \\def\\HADGMfigSSBCap{Trends in spawning stock biomass \(SSB\)  of Gulf of Maine haddock between 1977 and 2014 from the current  \(solid line\)  and previous \(dashed line\)  assessment and the corresponding  \$SSB\_{Threshold}\${} \(\$\\dfrac{1}{2}\${} \$SSB\_{MSY}\${} \\textit{proxy}{}\; horizontal dashed line\)  as well as  \$SSB\_{Target}\${} \(\$SSB\_{MSY}\${} \\textit{proxy}{}\; horizontal dotted line\)   based on the 2015 assessment. The approximate 90\\% lognormal confidence intervals are shown. The red dot indicates the rho\-adjusted SSB values that would have resulted had a retrospective adjusment been made to either model \(see Special Comments section\).} \\def\\HADGMfigFCap{Trends in the fully selected fishing mortality \(F\)  of Gulf of Maine haddock between 1977 and 2014 from the current  \(solid line\)  and previous \(dashed line\)  assessment and the corresponding  \$F\_{Threshold}\${} \(\$F\_{MSY}\${} \\textit{proxy}{}=0.468\; horizontal dashed line\)  from the 2015 assessment model. The approximate 90\\% lognormal confidence intervals are shown. The red dot indicates the rho\-adjusted F values that would have resulted had a retrospective adjusment been made to either model \(see Special Comments section\).} \\def\\HADGMfigRecrCap{Trends in Recruits \(age 1\)  \(000s\)  of Gulf of Maine haddock between 1977 and 2014 from the current \(solid line\)  and previous \(dashed line\)  assessment. The approximate 90\\% lognormal confidence intervals are shown.} \\def\\HADGMfigSurvCap{Indices of biomass for the Gulf of Maine haddock between 1963 and 2015 for the Northeast Fisheries Science Center \(NEFSC\)  spring and fall bottom trawl surveys.  The approximate 90\\% lognormal confidence intervals are shown.} \\def\\HADGMPreAmb{This assessment of the Gulf of Maine haddock \(\\textit{Melanogrammus aeglefinus}\)  stock is an operational update of the existing 2014 benchmark assessment \(NEFSC 2014\). Based on the previous assessment, the stock was not overfished, and overfishing was not ocurring. This assessment updates commercial and recreational fishery catch data, research survey indices of abundance, and the analytical ASAP assessment model and reference points through 2014. Additionally, stock projections have been updated through 2018} \\def\\HADGMSoS{ \\textbf{State of Stock: }{}Based on this updated assessment, the Gulf of Maine haddock \(\\textit{Melanogrammus aeglefinus}\)  stock is not overfished and overfishing is not occurring \(Figures \\ref{HADGMSSB\_plot1}\-\\ref{HADGMF\_plot1}\){}. Retrospective adjustments were not made to the model results \(see Special Comments section of this report\). Spawning stock biomass \(SSB\)  in 2014 was estimated to be 10,325 \(mt\)  which is 223\\% of the biomass target \(\$SSB\_{MSY}\${} \\textit{proxy}{} = 4,623\;  Figure \\ref{HADGMSSB\_plot1}{}\).  The 2014 fully selected fishing mortality was estimated to be 0.257 which is 55\\% of the overfishing threshold proxy \(\$F\_{MSY}\${} \\textit{proxy}{} =  \$F\_{40\\\%}\${} = 0.468\;  Figure \\ref{HADGMF\_plot1}{}\).} \\def\\HADGMProj{ \\textbf{Projections: }{}Short term projections of median total fishery yield and spawning stock biomass for Gulf of Maine haddock were conducted based on a harvest scenario of fishing at the  \$F\_{MSY}\${} \\textit{proxy}{} between 2016 and 2018. Catch in 2015 has been estimated at 885 mt. Recruitment was sampled from a cumulative distribution  function of model estimated age\-1 recruitment from 1977\-2012. The age\-1 estimate in 2015 was generated from the geometric mean of the 1977\-2014 recruitment series. The annual fishery selectivity, maturity ogive, and mean weights at age used in the projections  were estimated from the most recent 5 year averages\;  retrospective adjustments were not applied in the projections. Given the uncertainty in the size of the 2012 and 2013 year classes and the model\'s tendency to overestimate large terminal year classes, the 2015 assessment review panel recommended that a sensitivity projection scenario which constrains terminal recruitment \(\'Constrain terminal R\'\)  be brought forward to the New England Fishery Management Council\'s Scientific and Statistical Committee \(NEFMC SSC\)  for consideration when setting catch advice\; these sensitivity projections are provided in the Supplemental Information Report \(\\href{http:\/\/www.nefsc.noaa.gov\/saw\/sasi\/sasi\_report\_options.php}{SASINF}{}\).} \\def\\HADGMSpecCmt{ \\textbf{Special Comments: } \\begin{itemize}{} \\item{}What are the most important sources of uncertainty in this stock assessment?  Explain, and describe qualitatively how they affect the assessment results \(such as estimates of biomass, F, recruitment, and population projections\).  \\linebreak{} \\hspace\*{0.5cm} \\textit{ The largest source of uncertainty in the assessment is the estimated size of the 2012 and 2013 year classes. Based on the estimated selectivity patterns, these year classes are projected to be 30\\% selected to the fishery in 2016 and 2017 respectively. However, recent changes to the commercial and recreational minimum retention size may result in these year classes recruiting to the fishery sooner than projected. The abundance and growth of the 2012 and 2013 year classes should be monitored and frequent model updates would be expected to improve the estimates of year class size and validate projection assumptions.}  \\item{}Does this assessment model have a retrospective pattern? If so, is the pattern minor, or major? \(A major retrospective pattern occurs when the adjusted SSB or  \$F\_{Full}\${} lie outside of the approximate joint confidence region for SSB and  \$F\_{Full}\${}\). \\linebreak{} \\hspace\*{0.5cm} \\textit{This assessment does not exhibit a retrospective pattern and therefore no retrospective adjustments were made to the terminal model results or the short\-term catch projections. The 7\-year Mohn\'s rho values on SSB \(\-0.04\)  and F \(0.03\)  are small and there were no consistent patterns in the directionality of the retrospective \'peels\' \(see the Supplemental Information Report, \\href{http:\/\/www.nefsc.noaa.gov\/saw\/sasi\/sasi\_report\_options.php}{SASINF}{}\).}  \\item{}Based on this stock assessment, are population projections well determined or uncertain?  \\linebreak{} \\hspace\*{0.5cm} \\textit{Population projections for Gulf of Maine haddock, are reasonably well determined. The projected boimass from the last assessment is below the confidence bounds of the biomass estimated in the current assessment\; however, this is primarily due to the positive rescaling of the population size that occured from turning the ASAP model likelihood constants option off \(see next Special Comment\).}  \\item{}Describe any changes that were made to the current stock assessment, beyond incorporating additional years of data  and the affect these changes had on the assessment and stock status. \\linebreak{} \\hspace\*{0.5cm} \\textit{ Recreational catch estimates from 2004\-2014 were re\-estimated as part of this update to account for updates to the MRIP data. Additionally, the ASAP model was revised by turning the likelihood constants off\; sensitivity runs on SAW\/SARC 59 model suggest minor positive rescaling of recruitment and SSB, negative rescaling of F \(sensitivity results are provided in the Supplemental Information Report, \\href{http:\/\/www.nefsc.noaa.gov\/saw\/sasi\/sasi\_report\_options.php}{SASINF}{}\).}  \\item{}If the stock status has changed a lot since the previous assessment, explain why this occurred.  \\linebreak{} \\hspace\*{0.5cm} \\textit{There has been no change in stock status since the previous SAW\/SARC 59 assessment \(2014\).}  \\item{}Indicate what data or studies are currently lacking and which would be needed most to improve this stock assessment in the future.  \\linebreak{} \\hspace\*{0.5cm} \\textit{Currently the assessment assumes 50\\% survival of haddock discarded in the recreational fishery \- directed field research would improve this estimate. Additionally, a better understanding of recruitment processes may help to improve recruitment forecasting.}  \\item{}Are there other important issues? \\linebreak{} \\hspace\*{0.5cm} \\textit{None.} \\end{itemize}{}} \\def\\HADGMRefr{ \\textbf{References: }{} \\linebreak{}Northeast Fisheries Science Center. 2014. 59$^{th}$ Northeast Regional Stock Assessment Workshop \(59$^{th}$ SAW\)  Assessment Report. US Dept Commer, Northeast Fish Sci Cent Ref Doc. 14\-09\; 782 p. Available from: National Marine Fisheries Service, 166 Water Street, Woods Hole, MA 02543\-1026 \\linebreak{} \\linebreak{}} \\def\\HADGMDraft{} \\def\\HADGMSPPname{Gulf of Maine haddock} \\def\\HADGMSPPnameT{Gulf of Maine haddock} \\def\\HADGMRptYr{2015} \\def\\HADGMAuthor{Michael Palmer} \\def\\HADGMReviewerComments{/home/dhennen/EIEIO/BigReport/HAD_GM/latex}  \\def\\YELGBMyPathTab{/home/dhennen/EIEIO/BigReport/YEL_GB/tables} \\def\\YELGBMyPathFig{/home/dhennen/EIEIO/BigReport/YEL_GB/figures} \\def\\YELGBfigFishCap{Total catch of Georges Bank Yellowtail Flounder between 1935 and 2014 by fleet \(US, Canadian, or Other\)  and disposition \(landings or discards\).} \\def\\YELGBfigSSBCap{Trends in average survey biomass \(mt\)  of Georges Bank Yellowtail Flounder between 2010 and 2015 from the current assessment.} \\def\\YELGBfigFCap{Trends in the exploitation rate \(catch\/average survey biomass\)  of Georges Bank Yellowtail Flounder between 2010 and 2014 from the current assessment.} \\def\\YELGBfigRecrCap{} \\def\\YELGBfigSurvCap{Indices of biomass for the Georges Bank Yellowtail Flounder between 1963 and 2015 for the Canadian DFO and Northeast Fisheries Science Center \(NEFSC\)  spring and fall bottom trawl surveys.  The approximate 90\\% lognormal confidence intervals are shown.} \\def\\YELGBPreAmb{This assessment of the Georges Bank Yellowtail Flounder \(\\textit{Limanda ferruginea}\)  stock was reviewed during the July 2015 TRAC meeting \(Legault et al. 2015\). It is an operational update of the existing 2014 update assessment \(Legault et al. 2014\). Based on the previous assessment the stock status was unknown, but stock condition was poor. This assessment updates commercial fishery catch data through 2014 \(Table \\ref{YELGBCatch\_Status\_Table}{},  Figure \\ref{YELGBFish\_plot1}{}\), and updates research survey indices of abundance and the empirical approach assessment through 2015 \(Figure \\ref{YELGBSurv\_plot1}{}\). No stock projections can be computed using the empirical approach.} \\def\\YELGBSoS{ \\textbf{State of Stock: }{}Based on this updated assessment, Georges Bank Yellowtail Flounder \(\\textit{Limanda ferruginea}\)  stock status is unknown due to a lack of biological reference points associated with the empirical approach, but stock condition is poor.  Retrospective adjustments were not made to the model results. The average survey biomass in 2015 \(the arithmetic average of the 2015 DFO, 2015 NEFSC spring, and 2014 NEFSC fall surveys\)  was estimated to be 2,240 \(mt\)  \(Figure \\ref{YELGBSSB\_plot1}{}\).  The 2014 exploitation rate \(2014 catch divided by 2014 average survey biomass\)  was estimated to be 0.071 \(Figure \\ref{YELGBF\_plot1}{}\).} \\def\\YELGBProj{ \\textbf{Projections: }{}Short term projections cannot be computed using the empirical approach. Application of an exploitation rate of 2\\% to 16\\% to the 2015 average survey biomass \(2,240 mt\)  results in catch advice for 2016 of 45 mt to 359 mt.} \\def\\YELGBSpecCmt{ \\textbf{Special Comments: } \\begin{itemize}{} \\item{}What are the most important sources of uncertainty in this stock assessment?  Explain, and describe qualitatively how they affect the assessment results \(such as estimates of biomass, F, recruitment, and population projections\).  \\linebreak{} \\hspace\*{0.5cm} \\textit{The largest source of uncertainty is the estimate of survey catchability, which currently relies on literature values for other species in other regions of the world using different gear. The survey catchability affects the expansion of the stratified mean catch per tow for each survey and is inversely related to the catch advice. Other sources of uncertainty include the appropriate exploitation rate to apply to this stock, which has seen continued decrease in survey biomass despite low exploitation rates. }  \\item{} Does this assessment model have a retrospective pattern? If so, is the pattern minor, or major? \(A major retrospective pattern occurs when the adjusted SSB or  \$F\_{Full}\${} lies outside of the approximate  joint confidence region for SSB and  \$F\_{Full}\${}\; see RhoDecisionTab.ref\). \\linebreak{} \\hspace\*{0.5cm} \\textit{ The model used to estimate status of this stock does not allow estimation of a retrospective pattern. }  \\item{}Based on this stock assessment, are population projections well determined or uncertain? \\linebreak{} \\hspace\*{0.5cm} \\textit{Population projections for Georges Bank Yellowtail Flounder are not computed. Catch advice is derived from applying an exploitation rate to the current estimate of survey biomass. }  \\item{}Describe any changes that were made to the current stock assessment, beyond incorporating additional years of data  and the affect these changes had on the assessment and stock status. \\linebreak{} \\hspace\*{0.5cm} \\textit{The 2014 NMFS spring survey value was changed from 2,684 mt to 2,763 mt due to using preliminary data during the 2014 TRAC meeting. However, this has no impact on the 2015 stock status or 2016 catch advice in this update assessment.}  \\item{}If the stock status has changed a lot since the previous assessment, explain why this occurred.  \\linebreak{} \\hspace\*{0.5cm} \\textit{The stock status of Georges Bank Yellowtail Flounder remains unknown and stock condition continues to be poor.}  \\item{}Indicate what data or studies are currently lacking and which would be needed most to improve this stock assessment in the future.  \\linebreak{} \\hspace\*{0.5cm} \\textit{The Georges Bank Yellowtail Flounder assessment could be improved with studies on NMFS and DFO survey catchability for flatfish.}  \\item{}Are there other important issues? \\linebreak{} \\hspace\*{0.5cm} \\textit{None. } \\end{itemize}{}} \\def\\YELGBRefr{ \\textbf{References: }{} \\linebreak{}Legault, C.M., L. Alade, W.E. Gross, and H.H. Stone. 2014. Stock Assessment of Georges Bank Yellowtail Flounder for 2014. TRAC Ref. Doc. 2014\/01. 214 p. \\linebreak{}Legault, C.M., L. Alade, D. Busawon, and H.H. Stone. 2015. Stock Assessment of Georges Bank Yellowtail Flounder for 2015. TRAC Ref. Doc. 2015\/01. 66 p. \\linebreak{}} \\def\\YELGBDraft{} \\def\\YELGBSPPname{Georges Bank Yellowtail Flounder} \\def\\YELGBSPPnameT{Georges Bank Yellowtail Flounder} \\def\\YELGBRptYr{2015} \\def\\YELGBAuthor{Chris Legault} \\def\\YELGBReviewerComments{/home/dhennen/EIEIO/BigReport/YEL_GB/latex}  \\def\\YELSNEMAMyPathTab{/home/dhennen/EIEIO/BigReport/YEL_SNEMA/tables} \\def\\YELSNEMAMyPathFig{/home/dhennen/EIEIO/BigReport/YEL_SNEMA/figures} \\def\\YELSNEMAfigFishCap{Total catch of Southern New England\-Mid Atlantic Yellowtail flounder between 1973 and 2014 by fleet \(US domestic and foreign catch\)  and disposition \(landings and discards\).} \\def\\YELSNEMAfigSSBCap{Trends in spawning stock biomass of Southern New England\-Mid Atlantic Yellowtail flounder between 1973 and 2014 from the current  \(solid line\)  and previous \(dashed line\)  assessment and the corresponding  \$SSB\_{Threshold}\${} \(\$\\dfrac{1}{2}\${} \$SSB\_{MSY}\${} \\textit{proxy}{}\; horizontal dashed line\)  as well as  \$SSB\_{Target}\${} \(\$SSB\_{MSY}\${} \\textit{proxy}{}\; horizontal dotted line\)   based on the 2015 assessment.  Biomass was adjusted for a retrospective pattern  and the adjustment is shown in red.  The approximate 90\\% lognormal confidence intervals are shown.} \\def\\YELSNEMAfigFCap{Trends in the fully selected fishing mortality \(\$F\_{Full}\${}\)  of Southern New England\-Mid Atlantic Yellowtail flounder between 1973 and 2014 from the current  \(solid line\)  and previous \(dashed line\)  assessment and the corresponding  \$F\_{Threshold}\${} \(\$F\_{MSY}\${} \\textit{proxy}{}=0.35\; horizontal dashed line\).  \$F\_{Full}\${} was adjusted for a retrospective pattern  and the adjustment is shown in red  based on the 2015 assessment. The approximate 90\\% lognormal confidence intervals are shown.} \\def\\YELSNEMAfigRecrCap{Trends in Recruits \(age 1\)  \(000s\)  of Southern New England\-Mid Atlantic Yellowtail flounder between 1973 and 2014 from the current \(solid line\)  and previous \(dashed line\)  assessment. The approximate 90\\% lognormal confidence intervals are shown.} \\def\\YELSNEMAfigSurvCap{Indices of biomass for the Southern New England\-Mid Atlantic Yellowtail flounder between 1973 and 2015 for the Northeast Fisheries Science Center \(NEFSC\)  spring, fall and winter bottom trawl surveys.  The approximate 90\\% lognormal confidence intervals are shown.Note:  Larval index was also used in this assessment and is available in the supplemental documentation} \\def\\YELSNEMAPreAmb{This assessment of the Southern New England\-Mid Atlantic Yellowtail flounder \(\\textit{Limanda ferruginea}\)  stock is an operational update of the existing 2012 benchmark ASAP assessment \(NEFSC 2012\). Based on the previous assessment the stock was not overfished, and overfishing was not ocurring. This assessment updates commercial fishery catch data, research survey indices of abundance, weights at age and the analytical ASAP assessment model and reference points through 2014. Additionally, stock projections have been updated through 2018} \\def\\YELSNEMASoS{ \\textbf{State of Stock: }{}Based on this updated assessment, Southern New England\-Mid Atlantic Yellowtail flounder \(\\textit{Limanda ferruginea}\)  stock is overfished and overfishing is occurring \(Figures \\ref{YELSNEMASSB\_plot1}\-\\ref{YELSNEMAF\_plot1}\){}. Retrospective adjustments were not made to the model results. Spawning stock biomass \(SSB\)  in 2014 was estimated to be 502 \(mt\)  which is 26\\% of the biomass target \(\$SSB\_{MSY}\${} \\textit{proxy}{} = 1,959\;  Figure \\ref{YELSNEMASSB\_plot1}{}\). The 2014 fully selected fishing mortality was estimated to be 1.64 which is 469\\% of the overfishing threshold proxy \(\$F\_{MSY}\${} \\textit{proxy}{} = 0.35\;  Figure \\ref{YELSNEMAF\_plot1}{}\).} \\def\\YELSNEMAProj{ \\textbf{Projections: }{}Short term projections of biomass were derived by sampling from a cumulative  distribution function of recruitment estimates from ASAP.  Following the previous and accepted benchmark formulation, recruitment was based on the more recent estimates of the model time series \(i.e. corresponding to  year classes 1990 through 2013\)  to reflect the low recent pattern in recruitment. The annual fishery selectivity, maturity ogive, and mean weights at age used  in projection  are the most recent 5 year averages\;  retrospective adjustments were not applied in the projections.} \\def\\YELSNEMASpecCmt{ \\textbf{Special Comments: } \\begin{itemize}{} \\item{}What are the most important sources of uncertainty in this stock assessment?  Explain, and describe qualitatively how they affect the assessment results \(such as estimates of biomass, F, recruitment, and population projections\).  \\linebreak{} \\hspace\*{0.5cm} \\textit{The largest source of uncertainty is the emergence of the retrospective in this updated assessment.  This retrospective bias has resulted in the reduction SSB estimates and F estimates to increase with additional years of data  Further, the basis for recruitment assumption for stock status determination and population forecast   \(i.e. the inclusion of historical recruitment values versus contemporary basis of recruitment\)   is another source of uncertainty.  Although recent estmated recruitment likely reflect the realistic conditions for the stock, the basis for recruitment selection is not clearly understood.}  \\item{} Does this assessment model have a retrospective pattern? If so, is the pattern minor, or major? \(A major retrospective pattern occurs when the adjusted SSB or  \$F\_{Full}\${} lies outside of the approximate  joint confidence region for SSB and  \$F\_{Full}\${}\; see RhoDecisionTab.ref\). \\linebreak{} \\hspace\*{0.5cm} \\textit{ The 7\-year Mohn\'s  \\textrho{}, relative to SSB, was 0.14 in the 2012 assessment and was 1.06 in 2014. The 7\-year Mohn\'s  \\textrho{}, relative to F, was \-0.16 in the 2012 assessment and was \-0.53 in 2014. There was a major retrospective pattern for this assessment because the  \\textrho{} adjusted estimates of 2014 SSB \(\$SSB\_{\\rho}\${}=502\)  and 2014 F \(\$F\_{\\rho}\${}=1.64\)  were outside the approximate 90\\% confidence regions around SSB \(355 \- 739\)  and F \(1.053 \- 2.348\).  However, a retrospective adjustment was not made for both the determination of stock status and for projections of catch because of the large proportion of unfeasible projections \(assumed 2015 catch required a fishing mortality rate greater than 5\). This implies the retrospective adjustment was too large or the assumed 2015 catch was too high. The review panel decided to use the unadjusted projections as an upper bound for OFL with the strong suggestion that the OFL estimates were too high \(meaning the ABC buffer should be larger than normal\).}  \\item{}Based on this stock assessment, are population projections well determined or uncertain? \\linebreak{} \\hspace\*{0.5cm} \\textit{Population projections are uncertain with projected biomass from the last assessment above the confidence bounds of the biomass estimate in the current assessment.  Further, the short\-term projections which accounted for retropective adjustment in the starting numbers\-at\-age were unrelaible due to the low percentage of feasible solutions \(33\\%\)  encountered durring the simulation. The feasibility problem in the projections were due to the assumed 2015 projected cacth exceeding the population biomass in several of the iteration caused by the retrospective adjustment. Evaluation of the the estimated January\-1 2015 biomass from the few feasbile projections indicated that the assumed 2015 catch was approximately 98\\% of the stock biomass.  This suggests that the assumed 2015 catch is not sustainable given the low starting abundance in the forecast. Alternatively, the retro unadjusted projections performed well, but it is likely to result in an overly optimistic projection of the fishery yield and population biomass.}  \\item{}Describe any changes that were made to the current stock assessment, beyond incorporating additional years of data  and the affect these changes had on the assessment and stock status. \\linebreak{} \\hspace\*{0.5cm} \\textit{ There were no major changes to the current stock assessment formulation. However, the criterion for determining acceptable tows on the NEFSC surveys were revised for years the Bigelow year \(i.e. 2009\-2011\)  and carried foreward to ensure consistency between the assessment and deck operations.  The influence of the revised protocol on the survey indices was inconsequential.}  \\item{}If the stock status has changed a lot since the previous assessment, explain why this occurred.  \\linebreak{} \\hspace\*{0.5cm} \\textit{The overfishing and biomass stock status have changed since the previous assessment due to increased catches relative to the stock biomass and the very low recruitment of young fish, contributing very little to the adult biomass.}  \\item{}Indicate what data or studies are currently lacking and which would be needed most to improve this stock assessment in the future.  \\linebreak{} \\hspace\*{0.5cm} \\textit{The emergence of retrospective bias in this assessment is not clearly understood and may result from a variety of sources.  Future studiesshould further investigate the source of this retrospective pattern to help improve the underlying diagnostics of the model for providing catch advice for this stock.  Recruitment for Southern New England\-Mid Atlantic yellowtail flounder continues to be weak and it is likely that the stock is in a new productivity regime.  Should this pattern of poor recruitment continue into the future, the ability of the stock to recover will be impeded. Therefore, future studies should build on current knowledge to further understand the underlying ecological mechanisms of poor recruitment in the stock as it may relate to the physical environment.}  \\item{}Are there other important issues? \\linebreak{} \\hspace\*{0.5cm} \\textit{None. } \\end{itemize}{}} \\def\\YELSNEMARefr{ \\textbf{References: }{}  \\linebreak{} Alade, L,  C. Legault, S.Cadrin.  2008.  In.  Northeast Fisheries Science Center. 2008. Assessment of 19 Northeast Groundfish Stocks  through 2007: Report of the 3$^{rd}$ Groundfish Assessment Review Meeting \(GARM III\), Northeast Fisheries Science Center, Woods  Hole, Massachusetts, August 4\-8, 2008. US Dep Commer, NOAA Fisheries, Northeast Fish Sci Cent Ref Doc. 08\-15\; 884 p + xvii.  http:\/\/www.nefsc.noaa.gov\/publications\/crd\/crd0815\/  \\linebreak{}  \\linebreak{} Northeast Fisheries Science Center. 2012.  54$^{th}$ Northeast Regional Stock Assessment Workshop \(54$^{th}$ SAW\)  Assessment Report. US Dept Commer, NOAA Fisheries, Northeast Fish Sci Cent Ref Doc. 12\-18.\; 600 p.  http:\/\/nefsc.noaa.gov\/publications\/crd\/crd1218\/  \\linebreak{}  \\linebreak{}} \\def\\YELSNEMADraft{} \\def\\YELSNEMASPPname{Southern New England-Mid Atlantic Yellowtail flounder} \\def\\YELSNEMASPPnameT{Southern New England-Mid Atlantic Yellowtail flounder} \\def\\YELSNEMARptYr{2015} \\def\\YELSNEMAAuthor{Larry Alade} \\def\\YELSNEMAReviewerComments{/home/dhennen/EIEIO/BigReport/YEL_SNEMA/latex}  \\def\\YELCCGMMyPathTab{/home/dhennen/EIEIO/BigReport/YEL_CCGM/tables} \\def\\YELCCGMMyPathFig{/home/dhennen/EIEIO/BigReport/YEL_CCGM/figures} \\def\\YELCCGMfigFishCap{Total catch of Cape Cod\-Gulf of Maine Yellowtail flounder between 1985 and 2014 by disposition \(landings and discards\).} \\def\\YELCCGMfigSSBCap{Trends in spawning stock biomass of Cape Cod\-Gulf of Maine Yellowtail flounder between 1985 and 2014 from the current  \(solid line\)  and previous \(dashed line\)  assessment and the corresponding  \$SSB\_{Threshold}\${} \(\$\\dfrac{1}{2}\${} \$SSB\_{MSY}\${} \\textit{proxy}{}\; horizontal dashed line\)  as well as  \$SSB\_{Target}\${} \(\$SSB\_{MSY}\${} \\textit{proxy}{}\; horizontal dotted line\)   based on the 2015 assessment.  Biomass was adjusted for a retrospective pattern  and the adjustment is shown in red.   The 90\\% bootstrap probability intervals are shown.} \\def\\YELCCGMfigFCap{Trends in the fully selected fishing mortality \(\$F\_{Full}\${}\)  of Cape Cod\-Gulf of Maine Yellowtail flounder between 1985 and 2014 from the current  \(solid line\)  and previous \(dashed line\)  assessment and the corresponding  \$F\_{Threshold}\${} \(\$F\_{MSY}\${} \\textit{proxy}{}=0.279\; horizontal dashed line\).  \$F\_{Full}\${} was adjusted for a retrospective pattern  and the adjustment is shown in red  based on the 2015 assessment.  The 90\\% bootstrap probability intervals are shown.} \\def\\YELCCGMfigRecrCap{Trends in Recruits \(age 1\)  \(000s\)  of Cape Cod\-Gulf of Maine Yellowtail flounder between 1985 and 2014 from the current \(solid line\)  and previous \(dashed line\)  assessment.  The 90\\% bootstrap probability intervals are shown.} \\def\\YELCCGMfigSurvCap{Indices of biomass for the Cape Cod\-Gulf of Maine Yellowtail flounder between 1985 and 2015 for the Northeast Fisheries Science Center \(NEFSC\)  spring and fall bottom trawl surveys,  Massachusetts Department of Marine Fisheries \(MADMF\)  inshore state spring and fall bottom trawl surveys,and the Maine New Hampshire inshore state spring and fall state surveys  The 90\\% bootstrap probability intervals are shown.} \\def\\YELCCGMPreAmb{This assessment of the Cape Cod\-Gulf of Maine Yellowtail flounder \(\\textit{Limanda ferruginea}\)  stock is an operational update of the existing 2012 VPA assessment \(Legault et al., 2012\). The last benchmark for this stock was in 2008 \(Legault et al., 2008\). Based on the previous assessment the stock was overfished, and overfishing was ocurring. This assessment updates commercial fishery catch data, research survey indices of abundance, weights at age, and the analytical VPA assessment model and reference points through 2014. Additionally, stock projections have been updated through 2018} \\def\\YELCCGMSoS{ \\textbf{State of Stock: }{}Based on this updated assessment, Cape Cod\-Gulf of Maine Yellowtail flounder \(\\textit{Limanda ferruginea}\)  stock is overfished and overfishing is occurring \(Figures \\ref{YELCCGMSSB\_plot1}\-\\ref{YELCCGMF\_plot1}\){}.  Retrospective adjustments were made to the model results.  Spawning stock biomass \(SSB\)  in 2014 was estimated to be 857 \(mt\)  which is 16\\% of the biomass target \(\$SSB\_{MSY}\${} \\textit{proxy}{} = 5,259\;  Figure \\ref{YELCCGMSSB\_plot1}{}\).  The 2014 fully selected fishing mortality was estimated to be 0.64 which is 229\\% of the overfishing threshold proxy \(\$F\_{MSY}\${} \\textit{proxy}{} = 0.279\;  Figure \\ref{YELCCGMF\_plot1}{}\).} \\def\\YELCCGMProj{ \\textbf{Projections: }{}Short term projections of biomass were derived by sampling from a cumulative  distribution function of recruitment estimates from ADAPT VPA. Recruitment estimates were hindcasted based on a simple linear regression between the NEFSC Fall survey abundance at age 1 and the VPA estimate at age 1.  The most recent two years \(2013 and 2014\)  were not included in the series of values due to high uncertainty in these estimates. This resulted in a total of 36 recruitment values: 8 from the hindcast predictions \(years 1977\-1984\)  and 28 from the VPA \(years 1985\-2012\). The annual fishery selectivity, maturity ogive, and mean weights at age used  in projection  are the most recent 5 year averages\;  retrospective adjustments were applied in the projections.} \\def\\YELCCGMSpecCmt{ \\textbf{Special Comments: } \\begin{itemize}{} \\item{}What are the most important sources of uncertainty in this stock assessment?  Explain, and describe qualitatively how they affect the assessment results \(such as estimates of biomass, F, recruitment, and population projections\).  \\linebreak{} \\hspace\*{0.5cm} \\textit{The largest source of uncertainty is the source of the retrospective pattern.This pattern has persisted for a number of years causing SSB estimates to decrease and F estimates to increaseas more years of data are added.}  \\item{} Does this assessment model have a retrospective pattern? If so, is the pattern minor, or major? \(A major retrospective pattern occurs when the adjusted SSB or  \$F\_{Full}\${} lies outside of the approximate  joint confidence region for SSB and  \$F\_{Full}\${}\; see RhoDecisionTab.ref\). \\linebreak{} \\hspace\*{0.5cm} \\textit{ The 7\-year Mohn\'s  \\textrho{}, relative to SSB, was 0.68 in the 2012 assessment and was 0.98 in 2014. The 7\-year Mohn\'s  \\textrho{}, relative to F, was \-0.19 in the 2012 assessment and was \-0.45 in 2014. There was a major retrospective pattern for this assessment because the  \\textrho{} adjusted estimates of 2014 SSB \(\$SSB\_{\\rho}\${}=857\)  and 2014 F \(\$F\_{\\rho}\${}=0.64\)  were outside the approximate 90\\% confidence regions around SSB \(1,375 \- 2,111\)  and F \(0.25 \- 0.52\).  A retrospective  adjustment was made for both the determination of stock status and for projections of catch in 2016. The retrospective adjustment changed the 2014 SSB from 1,695 to 857 and the 2014  \$F\_{Full}\${} from 0.355 to 0.64.}  \\item{}Based on this stock assessment, are population projections well determined or uncertain? \\linebreak{} \\hspace\*{0.5cm} \\textit{Population projections for Cape Cod\-Gulf of Maine Yellowtail flounder, are uncertain with projected biomass from the last assessmentabove the confidence bounds of the biomass estimated in the current assessment.}  \\item{}Describe any changes that were made to the current stock assessment, beyond incorporating additional years of data  and the affect these changes had on the assessment and stock status. \\linebreak{} \\hspace\*{0.5cm} \\textit{ No changes, other than the incorporation of new data were made to the Cape Cod\-Gulf of Maine Yellowtail flounder assessment for this update.}  \\item{}If the stock status has changed a lot since the previous assessment, explain why this occurred.  \\linebreak{} \\hspace\*{0.5cm} \\textit{The stock status has not changed since the previous assessment.}  \\item{}Indicate what data or studies are currently lacking and which would be needed most to improve this stock assessment in the future.  \\linebreak{} \\hspace\*{0.5cm} \\textit{Extensive studies have examined the causes of the retrospective patterns with no definitive conclusions other than a change in model does not resolve the issue.}  \\item{}Are there other important issues? \\linebreak{} \\hspace\*{0.5cm} \\textit{No. } \\end{itemize}{}} \\def\\YELCCGMRefr{ \\textbf{References: }{} \\linebreak{}Legault, C,  L. Alade, S.Cadrin, J. King, and S. Sherman.  2008.  In.  Northeast Fisheries Science Center. 2008. Assessment of 19 Northeast Groundfish Stocks through 2007: Report of the 3$^{rd}$ Groundfish Assessment Review Meeting \(GARM III\), Northeast Fisheries Science Center, Woods Hole, Massachusetts, August 4\-8, 2008. US Dep Commer, NOAA Fisheries, Northeast Fish Sci Cent Ref Doc. 08\-15\; 884 p + xvii. http:\/\/www.nefsc.noaa.gov\/publications\/crd\/crd0815\/ \\linebreak{} \\linebreak{} Legault, C,  L. Alade, S.Emery, J. King, and S. Sherman.  2012.  In.  Northeast Fisheries Science Center. 2012. Assessment or Data Updates of 13 Northeast Groundfish Stocks through 2010. US Dept Commer, NOAA Fisheries, Northeast Fish Sci Cent Ref Doc. 12\-06.\; 789 p. http:\/\/nefsc.noaa.gov\/publications\/crd\/crd1206\/ \\linebreak{} \\linebreak{}} \\def\\YELCCGMDraft{} \\def\\YELCCGMSPPname{Cape Cod-Gulf of Maine Yellowtail flounder} \\def\\YELCCGMSPPnameT{Cape Cod-Gulf of Maine Yellowtail flounder} \\def\\YELCCGMRptYr{2015} \\def\\YELCCGMAuthor{Larry Alade} \\def\\YELCCGMReviewerComments{/home/dhennen/EIEIO/BigReport/YEL_CCGM/latex}  \\def\\FLWGMMyPathTab{/home/dhennen/EIEIO/BigReport/FLW_GM/tables} \\def\\FLWGMMyPathFig{/home/dhennen/EIEIO/BigReport/FLW_GM/figures} \\def\\FLWGMfigFishCap{Total catch of Gulf of Maine Winter Flounder between 2009 and 2014 by fleet \(commercial and recreational\)  and disposition \(landings and discards\). A 15\\% mortality rate is assumed on recreational discards and a 50\\% mortality rate on commercial discards.} \\def\\FLWGMfigSSBCap{Trends in 30+ cm area\-swept biomass of Gulf of Maine Winter Flounder between 2009 and 2014 from the current assessment based on the fall \(MENH, MDMF, NEFSC\)  surveys.  The approximate 90\\% lognormal confidence intervals are shown.} \\def\\FLWGMfigFCap{Trends in the exploitation rates \(\$E\_{Full}\${}\)  of Gulf of Maine Winter Flounder between 2009 and 2014 from the current assessment and the corresponding  \$F\_{Threshold}\${} \(\$E\_{MSY}\${} \\textit{proxy}{}=0.23\; horizontal dashed line\).  The approximate 90\\% lognormal confidence intervals are shown.} \\def\\FLWGMfigRecrCap{} \\def\\FLWGMfigSurvCap{Indices of biomass for the Gulf of Maine Winter Flounder between 1978 and 2015 for the Northeast Fisheries Science Center \(NEFSC\), Massachusetts Division of Marine Fisheries \(MDMF\), and the Maine New Hampshire \(MENH\)  spring and fall bottom trawl surveys. NEFSC indices are calculated with gear and vessel conversion factors where appropriate.  The approximate 90\\% lognormal confidence intervals are shown.} \\def\\FLWGMPreAmb{This assessment of the  Gulf of Maine Winter Flounder  \(\\textit{Pseudopleuronectes americanus}\)   stock is an operational update of the  existing  2014  operational update area\-swept assessment \(NEFSC 2014\).  Based on the previous assessment the biomass status is unknown but overfishing was not occurring.  This assessment  updates commercial and recreational fishery catch data, research survey indices  of abundance, and the area\-swept estimates of 30+ cm biomass based on the fall NEFSC, MDMF, and MENH surveys.} \\def\\FLWGMSoS{ \\textbf{State of Stock: }{}Based on this updated assessment, the Gulf of Maine Winter Flounder \(\\textit{Pseudopleuronectes americanus}\)  stock biomass status is unknown and overfishing is not occurring \(Figures \\ref{FLWGMSSB\_plot1}\-\\ref{FLWGMF\_plot1}\){}. Retrospective adjustments were not made to the model results.  Biomass  \(30+ cm mt\)  in 2014 was estimated to be 4,655 mt \(Figure \\ref{FLWGMSSB\_plot1}{}\). The 2014 30+ cm exploitation rate was estimated to be 0.06 which is 26\\% of the overfishing exploitation threshold proxy \(\$E\_{MSY}\${} \\textit{proxy}{} = 0.23\;  Figure \\ref{FLWGMF\_plot1}{}\).} \\def\\FLWGMProj{ \\textbf{Projections: }{}Projections are not possible with area\-swept based assessments. Catch advice was based on 75\\% of  \$E\_{40\\\%}\${}\(75\\% \$E\_{MSY}\${} \\textit{proxy}{}\)  using the fall area\-swept estimate assuming q=0.6 on the wing spread. Updated 2014 fall 30+ cm area\-swept biomass \(4,655 mt\)  implies an OFL of 1,080 mt based on the  \$E\_{MSY}\${} \\textit{proxy}{} and a catch of 810 mt for 75\\% of the  \$E\_{MSY}\${} \\textit{proxy}{}.} \\def\\FLWGMSpecCmt{ \\textbf{Special Comments: } \\begin{itemize}{} \\item{}What are the most important sources of uncertainty in this stock assessment?  Explain, and describe qualitatively how they affect the assessment results \(such as estimates of biomass, F, recruitment, and population projections\).  \\linebreak{} \\hspace\*{0.5cm} \\textit{The largest source of uncertainty with the direct estimates of stock biomass from survey area\-swept estimates originate from the assumption of survey gear catchability \(q\). Biomass and exploitation rate estimates are sensitive to the survey q assumption \(0.6 on wing spread\). The 2014 empirical benchmark assessement of Georges bank yellowtail flounder based the area\-swept q assumption on an average value taken from the literature for west coast flatfish \(0.37 on door spread\). The yellowtail q assumption corresponds to a value close to 1 on the wing spread which would result in a lower estimate of biomass \(2,995 mt\). Another major source of uncertainty with this method is that biomass based reference points cannot be determined and overfished status is unknown. }  \\item{} Does this assessment model have a retrospective pattern? If so, is the pattern minor, or major? \(A major retrospective pattern occurs when the adjusted SSB or  \$F\_{Full}\${} lies outside of the approximate  joint confidence region for SSB and  \$F\_{Full}\${}\; see  Figure \\ref{RhoDecision\_tab}{}\). \\linebreak{} \\hspace\*{0.5cm} \\textit{ The model used to determine status of this stock does not allow estimation of a retrospective pattern.  An analytical stock assessment model does not exist for Gulf of Maine Winter Flounder.  An analytical model was no longer used for stock status determination at SARC 52 \(2011\)  due to concerns with a strong retrospective pattern.  Models have difficulty with the apparent lack of a relationship between a large decrease in the catch with little change in the indices and age and\/or size structure over time. }  \\item{}Based on this stock assessment, are population projections well determined or uncertain? \\linebreak{} \\hspace\*{0.5cm} \\textit{Population projections for Gulf of Maine Winter Flounder, do not exist for area\-swept assessments. Catch advice from area\-swept estimates tend to vary with interannual variability in the surveys.}  \\item{}Describe any changes that were made to the current stock assessment, beyond incorporating additional years of data  and the affect these changes had on the assessment and stock status. \\linebreak{} \\hspace\*{0.5cm} \\textit{ No changes, other than the incorporation of new data were made to the Gulf of Maine Winter Flounder assessment for this update. However, stabilizing the catch advice may be desired and could be obtained through the averaging of the area\-swept fall and spring survey estimates.}  \\item{}If the stock status has changed a lot since the previous assessment, explain why this occurred.  \\linebreak{} \\hspace\*{0.5cm} \\textit{The overfishing status of Gulf of Maine Winter Flounder has not changed. }  \\item{}Indicate what data or studies are currently lacking and which would be needed most to improve this stock assessment in the future.  \\linebreak{} \\hspace\*{0.5cm} \\textit{Direct area\-swept assessment could be improved with additional studies on survey gear efficiency.  Quantifying the degree of herding between the doors and escapement under the footrope and\/or above the headrope for each survey is needed since area\-swept biomass estimates and catch advice are sensitive to the assumed catchability.}  \\item{}Are there other important issues? \\linebreak{} \\hspace\*{0.5cm} \\textit{The general lack of a response in survey indices and age\/size structure is the primary source of concern with catches remaining far below the overfishing level. } \\end{itemize}{}} \\def\\FLWGMRefr{ \\textbf{References: }{} \\linebreak{}Hendrickson L, Nitschke P, Linton B. 2015. 2014 Operational Stock Assessments for Georges Bank winter flounder, Gulf of Maine winter flounder, and pollock. US Dept Commer, Northeast Fish Sci Cent Ref Doc. 15\-01\; 228 p. Available from: National Marine Fisheries Service, 166 Water Street, Woods Hole, MA 02543\-1026, or online at http:\/\/nefsc.noaa.gov\/publications\/ \\linebreak{} \\linebreak{}Northeast Fisheries Science Center. 2011. 52$^{nd}$ Northeast Regional Stock AssessmentWorkshop \(52$^{nd}$ SAW\)  Assessment Report. US Dept Commer, Northeast Fish SciCent Ref Doc. 11\-17\; 962 p. Available from: National Marine Fisheries Service, 166 Water Street, Woods Hole, MA 02543\-1026, or online at http:\/\/www.nefsc.noaa.gov\/nefsc\/publications\/ \\linebreak{} \\linebreak{}} \\def\\FLWGMDraft{} \\def\\FLWGMSPPname{Gulf of Maine Winter Flounder} \\def\\FLWGMSPPnameT{Gulf of Maine Winter Flounder} \\def\\FLWGMRptYr{2015} \\def\\FLWGMAuthor{Paul Nitschke} \\def\\FLWGMReviewerComments{/home/dhennen/EIEIO/BigReport/FLW_GM/latex}  \\def\\FLWSNEMAMyPathTab{/home/dhennen/EIEIO/BigReport/FLW_SNEMA/tables} \\def\\FLWSNEMAMyPathFig{/home/dhennen/EIEIO/BigReport/FLW_SNEMA/figures} \\def\\FLWSNEMAfigFishCap{Total catch of Southern New England Mid\-Atlantic Winter Flounder between 1981 and 2014 by fleet \(commercial, recreational\)  and disposition \(landings and discards\).} \\def\\FLWSNEMAfigSSBCap{Trends in spawning stock biomass of Southern New England Mid\-Atlantic Winter Flounder between 1981 and 2014 from the current  \(solid line\)  and previous \(dashed line\)  assessment and the corresponding  \$SSB\_{Threshold}\${} \(\$\\dfrac{1}{2}\${} \$SSB\_{MSY}\${} \\textit{proxy}{}\; horizontal dashed line\)  as well as  \$SSB\_{Target}\${} \(\$SSB\_{MSY}\${} \\textit{proxy}{}\; horizontal dotted line\)   based on the 2015 assessment. The approximate 90\\% lognormal confidence intervals are shown.} \\def\\FLWSNEMAfigFCap{Trends in the fully selected fishing mortality \(\$F\_{Full}\${}\)  of Southern New England Mid\-Atlantic Winter Flounder between 1981 and 2014 from the current  \(solid line\)  and previous \(dashed line\)  assessment and the corresponding  \$F\_{Threshold}\${} \(\$F\_{MSY}\${}=0.325\; horizontal dashed line\)   based on the 2015 assessment. The approximate 90\\% lognormal confidence intervals are shown.} \\def\\FLWSNEMAfigRecrCap{Trends in Recruits \(age 1\)  \(000s\)  of Southern New England Mid\-Atlantic Winter Flounder between 1981 and 2014 from the current \(solid line\)  and previous \(dashed line\)  assessment. The approximate 90\\% lognormal confidence intervals are shown.} \\def\\FLWSNEMAfigSurvCap{Indices of biomass for the Southern New England Mid\-Atlantic Winter Flounder between 1963 and 2014 for the Northeast Fisheries Science Center \(NEFSC\)  spring and fall bottom trawl surveys, the MADMF spring survey, and the CT LISTS survey  The approximate 90\\% lognormal confidence intervals are shown.} \\def\\FLWSNEMAPreAmb{This assessment of the Southern New England Mid\-Atlantic Winter Flounder \(\\textit{Pseudopleuronectes americanus}\)  stock is an operational update of the existing 2011 benchmark ASAP assessment \(NEFSC 2011\). Based on the previous assessment the stock was overfished, but overfishing was not ocurring. This assessment updates commercial fishery catch data, recreational fishery catch data, and research survey indices of abundance, and the analytical ASAP assessment models and reference points through 2014. Additionally, stock projections have been updated through 2018} \\def\\FLWSNEMASoS{ \\textbf{State of Stock: }{}Based on this updated assessment, the Southern New England Mid\-Atlantic Winter Flounder \(\\textit{Pseudopleuronectes americanus}\)  stock is overfished but overfishing is not occurring \(Figures \\ref{FLWSNEMASSB\_plot1}\-\\ref{FLWSNEMAF\_plot1}\){}. Spawning stock biomass \(SSB\)  in 2014 was estimated to be 6,151 \(mt\)  which is 23\\% of the biomass target \(26,928 mt\), and 23\\% of the biomass threshold for an overfished stock \(\$SSB\_{Threshold}\${} = 13464 \(mt\)\;  Figure \\ref{FLWSNEMASSB\_plot1}{}\).  The 2014 fully selected fishing mortality was estimated to be 0.16 which is 49\\% of the overfishing threshold \(\$F\_{MSY}\${} = 0.325\;  Figure \\ref{FLWSNEMAF\_plot1}{}\). Retrospective adjustments were not made to the model results. } \\def\\FLWSNEMAProj{ \\textbf{Projections: }{}Short term projections of biomass were derived by sampling from a cumulative  distribution  function of recruitment estimates assuming a Beverton\-Holt stock recruitment relationship. The annual fishery selectivity, maturity ogive, and mean weights at age used  in projection  are the most recent 5 year averages\;  The model exhibited minor retrospective pattern in F and SSB so no retrospective adjustments were applied in the projections.} \\def\\FLWSNEMASpecCmt{ \\textbf{Special Comments: } \\begin{itemize}{} \\item{}What are the most important sources of uncertainty in this stock assessment?  Explain, and describe qualitatively how they affect the assessment results \(such as estimates of biomass, F, recruitment, and population projections\).  \\linebreak{} \\hspace\*{0.5cm} \\textit{A large source of uncertainty is the estimate of natural mortality based on longevity, which is not well studied in Southern New England Mid\-Atlantic Winter Flounder, and assumed constant over time.  Natural mortality affects the scale of the biomass and fishing mortality estimates.  Natural mortality was adjusted upwards from 0.2 to 0.3 during the last benchmark assessment assuming a max age of 16. However, there is still uncertainty in the true max age of the population and the resulting natural mortality estimate. Other sources of uncertainty include length distribution of the recreational discards.  The recreational discards, are a small component of the total catch, but the assessment suffers from very little length information used to characterize the recreational discards \(1 to 2 lengths in recent years\).}  \\item{} Does this assessment model have a retrospective pattern? If so, is the pattern minor, or major? \(A major retrospective pattern occurs when the adjusted SSB or  \$F\_{Full}\${} lies outside of the approximate  joint confidence region for SSB and  \$F\_{Full}\${}\; see  Figure \\ref{RhoDecision\_tab}{}\). \\linebreak{} \\hspace\*{0.5cm} \\textit{ No retrospective adjustment of spawning stock biomass or fishing mortality in 2014 was required. }  \\item{}Based on this stock assessment, are population projections well determined or uncertain? \\linebreak{} \\hspace\*{0.5cm} \\textit{Population projections for Southern New England Mid\-Atlantic Winter Flounder are reasonably well determined. There is uncertainty in the estimates of M. In addition, while the retrospective pattern is considered minor \(within the 90\\% CI of both F and SSB\)  the rho adjusted terminal value is very close to falling out of the bounds, becoming a major retrospective pattern. This would lead to retrospective adjustments being needed for the projections.}  \\item{}Describe any changes that were made to the current stock assessment, beyond incorporating additional years of data  and the affect these changes had on the assessment and stock status. \\linebreak{} \\hspace\*{0.5cm} \\textit{ No changes, other than the incorporation of new data were made to the Southern New England Mid\-Atlantic Winter Flounder assessment for this update.}  \\item{}If the stock status has changed a lot since the previous assessment, explain why this occurred.  \\linebreak{} \\hspace\*{0.5cm} \\textit{The stock status of Southern New England Mid\-Atlantic Winter Flounder has not changed since the previous benchmark in 2011.}  \\item{}Indicate what data or studies are currently lacking and which would be needed most to improve this stock assessment in the future.  \\linebreak{} \\hspace\*{0.5cm} \\textit{The Southern New England Mid\-Atlantic Winter Flounder assessment could be improved with additional studies on maximum age, as well additional information  of recreational discard lengths.  In addition, further investigation into the localized struture\/genetics of the stock is warranted. Also, a future shift to ASAP version 4 will provide the ability to model envirionmental factors that may influence both survey catchability and the modeled S\-R relationship}  \\item{}Are there other important issues? \\linebreak{} \\hspace\*{0.5cm} \\textit{None. } \\end{itemize}{}} \\def\\FLWSNEMARefr{ \\textbf{References: }{} \\linebreak{}Smith, A. and S. Jones.  2008.  In.  Northeast Fisheries Science Center. 2008. Assessment of 19 Northeast Groundfish Stocks through 2007: Report of the 3$^{rd}$ Groundfish Assessment Review Meeting \(GARM III\), Northeast Fisheries Science Center, Woods Hole, Massachusetts, August 4\-8, 2008. US Dep Commer, NOAA Fisheries, Northeast Fish Sci Cent Ref Doc. 08\-15\; 884 p + xvii. http:\/\/www.nefsc.noaa.gov\/publications\/crd\/crd0815\/ \\linebreak{} \\linebreak{}Northeast Fisheries Science Center. 2011. 52$^{nd}$ Northeast Regional Stock AssessmentWorkshop \(52$^{nd}$ SAW\)  Assessment Report. US Dept Commer, Northeast Fish SciCent Ref Doc. 11\-17\; 962 p. Available from: National Marine Fisheries Service, 166Water Street, Woods Hole, MA 02543\-1026, or online at http:\/\/www.nefsc.noaa.gov\/nefsc\/publications\/ \\linebreak{} \\linebreak{}} \\def\\FLWSNEMADraft{} \\def\\FLWSNEMASPPname{Southern New England Mid-Atlantic Winter Flounder} \\def\\FLWSNEMASPPnameT{Southern New England Mid-Atlantic Winter Flounder} \\def\\FLWSNEMARptYr{2015} \\def\\FLWSNEMAAuthor{Anthony Wood} \\def\\FLWSNEMAReviewerComments{/home/dhennen/EIEIO/BigReport/FLW_SNEMA/latex}  \\def\\FLWGBMyPathTab{/home/dhennen/EIEIO/BigReport/FLW_GB/tables} \\def\\FLWGBMyPathFig{/home/dhennen/EIEIO/BigReport/FLW_GB/figures} \\def\\FLWGBfigFishCap{Total catches \(mt\)  of Georges Bank Winter Flounder between 1982 and 2015 by country and disposition \(landings and discards\).} \\def\\FLWGBfigSSBCap{Trends in spawning stock biomass \(mt\)  of Georges Bank Winter Flounder between 1982 and 2014 from the current  \(solid line\)  and previous \(dashed line\)  assessments and the corresponding  \$SSB\_{Threshold}\${} \(\$\\dfrac{1}{2}\${} \$SSB\_{MSY}\${}\; horizontal dashed line\)  as well as  \$SSB\_{Target}\${} \(\$SSB\_{MSY}\${}\; horizontal dotted line\)   based on the 2015 assessment.  Biomass was adjusted for a retrospective pattern  and the adjustment is shown in red.  The approximate 90\\% normal confidence intervals are shown.} \\def\\FLWGBfigFCap{Trends in fully selected fishing mortality \(\$F\_{Full}\${}\)  of Georges Bank Winter Flounder between 1982 and 2014 from the current  \(solid line\)  and previous \(dashed line\)  assessments and the corresponding  \$F\_{Threshold}\${} \(\$F\_{MSY}\${}=0.536\; horizontal dashed line\)  as well as \(\$F\_{Target}\${}= 75\\% of FMSY\;  horizontal dotted line\). \$F\_{Full}\${} was adjusted for a retrospective pattern  and the adjustment is shown in red.  The approximate 90\\% normal confidence intervals are also shown.} \\def\\FLWGBfigRecrCap{Trends in Recruits \(age 1\)  \(000s\)  of Georges Bank Winter Flounder between 1982 and 2014 from the current \(solid line\)  and previous \(dashed line\)  assessments. The approximate 90\\% normal confidence intervals are shown.} \\def\\FLWGBfigSurvCap{Indices of biomass for the Georges Bank Winter Flounder for the Northeast Fisheries Science Center \(NEFSC\)  spring \(1968\-2015\)  and fall \(1963\-2014\)   bottom trawl surveys and the Canadian DFO spring survey \(1987\-2015\).  The approximate 90\\% normal confidence intervals are shown.} \\def\\FLWGBPreAmb{This assessment of the Georges Bank Winter Flounder \(\\textit{Pseudopleuronectes americanus}\)  stock is an operational update of the existing 2014 operational VPA assessment which included data for 1982\-2013 \(Hendrickson et al. 2015\). Based on the previous assessment the stock was not overfished and overfishing was not ocurring. This assessment updates commercial fishery catch data, research survey biomass indices, and the analytical VPA assessment model and reference points through 2014. Additionally, stock projections have been updated through 2018.} \\def\\FLWGBSoS{ \\textbf{State of Stock: }{}Based on this updated assessment, the Georges Bank Winter Flounder \(\\textit{Pseudopleuronectes americanus}\)  stock is overfished and overfishing is occurring \(Figures \\ref{FLWGBSSB\_plot1}\-\\ref{FLWGBF\_plot1}\){}. Retrospective adjustments were made to the model results.  Spawning stock biomass \(SSB\)  in 2014 was estimated to be 2,883 \(mt\)  which is 43\\% of the biomass target for an overfished stock \(\$SSB\_{MSY}\${} = 6,700 with a threshold of 50\\% of SSBMSY\;  Figure \\ref{FLWGBSSB\_plot1}{}\).  The 2014 fully selected fishing mortality \(F\)  was estimated to be 0.778 which is 145\\% of the overfishing threshold \(\$F\_{MSY}\${} = 0.536\;  Figure \\ref{FLWGBF\_plot1}{}\). However, the 2014 point estimate of SSB and F, when adjusted for retrospective error \(83\\% for SSB and \-51\\% for F\), is outside the 90\\% confidence interval of the unadjusted 2014 point estimate. Therefore, the 2014 F and SSB values used in the stock status determination were the retrospective\-adjusted values of 0.778 and 2,883 mt, respectively.} \\def\\FLWGBProj{ \\textbf{Projections: }{}Short\-term projections of biomass were derived by sampling from a cumulative  distribution  function of recruitment estimates \(1982\-2013 YC\)  from the final run of the ADAPT VPA model. The annual fishery selectivity, maturity ogive, and mean weights\-at\-age used in the projection  are the most recent 5 year averages \(2010\-2014\). An SSB retrospective adjustment factor of 0.546 was applied in the projections.} \\def\\FLWGBSpecCmt{ \\textbf{Special Comments: } \\begin{itemize}{} \\item{}What are the most important sources of uncertainty in this stock assessment?  Explain, and describe qualitatively how they affect the assessment results \(such as estimates of biomass, F, recruitment, and population projections\).  \\linebreak{} \\hspace\*{0.5cm} \\textit{The largest source of uncertainty is the estimate of natural mortality based on longevity \(max. age = 20 for this stock\), which is not well studied in Georges Bank Winter Flounder, and assumed constant over time.  Natural mortality affects the scale of the biomass and fishing mortality estimates. Other sources of uncertainty include the underestimation of catches. Discards from the Canadian bottom trawl fleet were not provided by the CA DFO and the precision of the Canadian scallop dredge discard estimates, with only 1\-2 trips per month, are uncertain.The lack of age data for the Canadian spring survey catches requires the use of the US spring survey A\/L keys despite selectivity differences. In addition, there are no length or age composition data from the Canadian landings or discards GB winter flounder.}  \\item{} Does this assessment model have a retrospective pattern? If so, is the pattern minor, or major? \(A major retrospective pattern occurs when the adjusted SSB or  \$F\_{Full}\${} lies outside of the approximate  joint confidence region for SSB and  \$F\_{Full}\${}\; see  Figure \\ref{RhoDecision\_tab}{}\). \\linebreak{} \\hspace\*{0.5cm} \\textit{ The 7\-year Mohn\'s  \\textrho{}, relative to SSB, was 0.26 in the 2014 assessment and was 0.83 in 2014. The 7\-year Mohn\'s  \\textrho{}, relative to F, was \-0.16 in the 2014 assessment and was \-0.51 in 2014. There was a major retrospective pattern for this assessment because the  \\textrho{} adjusted estimates of 2014 SSB \(\$SSB\_{\\rho}\${}=2,883\)  and 2014 F \(\$F\_{\\rho}\${}=0.778\)  were outside the approximate 90\\% confidence region around SSB \(3,783 \- 6,767\)  and F \(0.254 \- 0.504\).  A retrospective  adjustment was made for both the determination of stock status and for projections of catch in 2016. The retrospective adjustment changed the 2014 SSB from 5,275 to 2,883 and the 2014  \$F\_{Full}\${} from 0.379 to 0.778.}  \\item{}Based on this stock assessment, are population projections well determined or uncertain? \\linebreak{} \\hspace\*{0.5cm} \\textit{Population projections for Georges Bank Winter Flounder are reasonably well determined.}  \\item{}Describe any changes that were made to the current stock assessment, beyond incorporating additional years of data  and the affect these changes had on the assessment and stock status. \\linebreak{} \\hspace\*{0.5cm} \\textit{ The only change made to the Georges Bank Winter Flounder assessment, other than the incorporation of an additional  year of data, involved fishery selectivity.  During the 2014 assessment update, stock size estimates of age 1 and age 2 fish were not estimable  in the VPA during year t + 1 \(CVs near 1.0\). When age 2 stock size is not estimated in year t + 1,  the VPA model calculates the stock size of age 1 fish \(i.e., recruitment\)  in the terminal year by  using the age 1 partial recruitment \(PR\)  value to derive the F at age 1 in the terminal year. The  age 1 PR value used in the 2014 assessment update was 0.001. However, when this same age 1 PR value  was used in a VPA run for the current assessment update, the low PR value combined with the low age  1 catch in 2014 resulted in an unlikely high stock size estimate for age 1 recruitment in 2014 \(i.e.,  41,587,000 fish\)  when compared to survey observations of the same cohort \(i.e., age 1 in 2014 and age  2 in 2015\). In order to obtain a more realistic estimate of age 1 recruitment in 2014, I allowed the  VPA model to estimate age 2 stock size in 2015 \(i.e., and thereby avoided the use of an age 1 PR  value in the age 1 stock size calculation for 2014\)  and used the back\-calculated PR values from this  VPA run to derive a new PR\-at\-age vector which was used in the final 2015 VPA run. Similar to the  2014 assessment update, the final 2015 VPA run did not include the estimation of age 2 stock size  and the new PR\-at\-age vector was computed using the same methods as in the 2014 assessment.   Full selectivity occurs at age 4. For the 2015 assessment update, fishery selectivity for ages  1\-3 was changed from the 2014 assessment values of 0.001, 0.10 and 0.43, respectively, to 0.01,  0.08 and 0.55, respectively. Differences between estimates  of F, SSB and R values from the final  2015 VPA run, with the new PR vector, and a 2015 VPA run that utilized the PR vector from the 2014  assessment are shown in Table G30.}  \\item{}If the stock status has changed a lot since the previous assessment, explain why this occurred.  \\linebreak{} \\hspace\*{0.5cm} \\textit{The overfished and overfishing status of Georges Bank Winter Flounder has changed in the current assessment update due to a worsening of the retrospective error associated with fishing mortality and SSB.}  \\item{}Indicate what data or studies are currently lacking and which would be needed most to improve this stock assessment in the future.  \\linebreak{} \\hspace\*{0.5cm} \\textit{The Georges Bank Winter Flounder assessment could be improved with discard estimates from the Canadian bottom trawl fleet and age data from the Canadian spring bottom trawl surveys.}  \\item{}Are there other important issues? \\linebreak{} \\hspace\*{0.5cm} \\textit{None. } \\end{itemize}{}} \\def\\FLWGBRefr{ \\textbf{References: }{} \\linebreak{} Hendrickson L, Nitschke P, Linton B. 2015. 2014 Operational Stock Assessments for Georges Bank winter flounder, Gulf of Maine winter flounder, and pollock. US Dept Commer, Northeast Fish Sci Cent Ref Doc. 15\-01\; 228 p. \\linebreak{} \\linebreak{}} \\def\\FLWGBDraft{} \\def\\FLWGBSPPname{Georges Bank Winter Flounder} \\def\\FLWGBSPPnameT{Georges Bank Winter Flounder} \\def\\FLWGBRptYr{2015} \\def\\FLWGBAuthor{Lisa Hendrickson} \\def\\FLWGBReviewerComments{/home/dhennen/EIEIO/BigReport/FLW_GB/latex}  \\def\\FLDGMGBMyPathTab{/home/dhennen/EIEIO/BigReport/FLD_GMGB/tables} \\def\\FLDGMGBMyPathFig{/home/dhennen/EIEIO/BigReport/FLD_GMGB/figures} \\def\\FLDGMGBfigFishCap{Total catch of northern windowpane flounder between 1975 and 2014 by disposition \(landings and discards\).} \\def\\FLDGMGBfigSSBCap{Trends in the biomass index \(a 3\-year moving average of the NEFSC fall bottom trawl survey index\)  of northern windowpane flounder between 1975 and 2014 from the current  assessment, and the corresponding  \$B\_{Threshold}\${} =  \$\\dfrac{1}{2}\${} \$B\_{MSY}\${} \\textit{proxy}{} = 0.777 kg\/tow \(horizontal dashed line\). } \\def\\FLDGMGBfigFCap{Trends in relative fishing mortality  of northern windowpane flounder between 1975 and 2014 from the current  assessment, and the corresponding  \$F\_{MSY}\${} \\textit{proxy}{}=0.45 \(horizontal dashed line\). } \\def\\FLDGMGBfigRecrCap{} \\def\\FLDGMGBfigSurvCap{NEFSC fall bottom trawl survey indices in kg\/tow for northern windowpane flounder between 1975 and 2014  The approximate 90\\% lognormal confidence intervals are shown.} \\def\\FLDGMGBPreAmb{This assessment of the northern windowpane flounder \(\\textit{Scophthalmus aquosus}\)  stock is an operational update of the 2012 assessment which included updates through 2010 \(NEFSC 2012\). Based on the 2012 assessment the stock was overfished, and overfishing was ocurring. This assessment updates commercial fishery catch data, survey indices of abundance, AIM model results,  and reference points through 2014.} \\def\\FLDGMGBSoS{ \\textbf{State of Stock: }{}Based on this updated assessment, the northern windowpane flounder \(\\textit{Scophthalmus aquosus}\)  stock is overfished but overfishing is not occurring \(Figures \\ref{FLDGMGBSSB\_plot1}\-\\ref{FLDGMGBF\_plot1}\){}. Retrospective adjustments were not made to the model results. The mean NEFSC fall bottom trawl survey index from years 2012, 2013 and 2014 \(a 3\-year moving average is used as a biomass index\)  was 0.535 kg\/tow which is lower than the \$B\_{Threshold}\${} of 0.777 kg\/tow. The 2014 relative fishing mortality was estimated to be 0.393 kt per kg\/tow which is lower than the  \$F\_{MSY}\${} \\textit{proxy}{} of 0.450 kt per kg\/tow.} \\def\\FLDGMGBProj{} \\def\\FLDGMGBSpecCmt{ \\textbf{Special Comments: } \\begin{itemize}{} \\item{}What are the most important sources of uncertainty in this stock assessment?  Explain, and describe qualitatively how they affect the assessment results \(such as estimates of biomass, F, recruitment, and population projections\).  \\linebreak{} \\hspace\*{0.5cm} \\textit{The main source of uncertainty in this assessment is the lack of windowpane discard estimates from Canadian fisheries to add to the catch component of model input. Discard estimates were from the U.S. only. There is overlap between the survey area and Canadian fishing grounds \(Van Eeckhaute et al. 2010\), which means catch from within the stock area was likely underestimated. }  \\item{} Does this assessment model have a retrospective pattern? If so, is the pattern minor, or major? \(A major retrospective pattern occurs when the adjusted SSB or  \$F\_{Full}\${} lies outside of the approximate  joint confidence region for SSB and  \$F\_{Full}\${}\; see  Figure \\ref{RhoDecision\_tab}{}\). \\linebreak{} \\hspace\*{0.5cm} \\textit{ The model used to estimate status of this stock does not allow estimation of a retrospective pattern. }  \\item{}Based on this stock assessment, are population projections well determined or uncertain? \\linebreak{} \\hspace\*{0.5cm} \\textit{N\/A }  \\item{}Describe any changes that were made to the current stock assessment, beyond incorporating additional years of data  and the affect these changes had on the assessment and stock status. \\linebreak{} \\hspace\*{0.5cm} \\textit{No changes were made to the northern windowpane flounder assessment for this update  other than the incorporation of four years of new NEFSC fall bottom trawl survey data and  four years of new U.S. commercial landings and discard data \(2011 \- 2014\). }  \\item{}If the stock status has changed a lot since the previous assessment, explain why this occurred.  \\linebreak{} \\hspace\*{0.5cm} \\textit{The stock status of northern windowpane flounder changed from \'overfished and overfishing is occurring\' to \'overfished and overfishing is not occurring\' due to stable\-to\-decreasing catch since 2008, and an increasing trend in the survey index since 2010. }  \\item{}Indicate what data or studies are currently lacking and which would be needed most to improve this stock assessment in the future.  \\linebreak{} \\hspace\*{0.5cm} \\textit{The northern windowpane flounder assessment could be improved by estimating the Canadian windowpane removals and, although to a lesser degree, the \'general category\' scallop dredge fleet discards from within the stock area and using them as additional catch input to the AIM model.  While the model fit now is reasonable \(the relationship between ln\(relative F\)  and ln\(replacement ratio\), a measure of the relationship between catch and survey index values, has a p\-value of 0.079\)  there are probably removals unaccounted for in the model and the fit can likely be improved. }  \\item{}Are there other important issues? \\linebreak{} \\hspace\*{0.5cm} \\textit{None. } \\end{itemize}{}} \\def\\FLDGMGBRefr{ \\textbf{References: }{} \\linebreak{} Most recent assessment update:  \\linebreak{} Northeast Fisheries Science Center. 2012. Assessment or Data Updates of 13 Northeast Groundfish Stocks through 2010.  US Dept Commer, Northeast Fish Sci Cent Ref Doc. 12\-06\; 789 p. Available online at http:\/\/nefsc.noaa.gov\/publications\/  \\linebreak{} \\linebreak{} Most recent benchmark assessment:  \\linebreak{} Northeast Fisheries Science Center. 2008. Assessment of 19 Northeast Groundfish Stocks through 2007:  Report of the 3$^{rd}$ Groundfish Assessment Review Meeting \(GARM III\), Northeast Fisheries Science Center,  Woods Hole, Massachusetts, August 4\-8, 2008. US Dep Commer, NOAA FIsheries, Northeast Fish Sci Cent Ref Doc. 08\-15\; 884 p + xvii.  \\linebreak{} \\linebreak{} Van Eeckhaute, L., Sameoto, J., and A. Glass. 2010. Discards of Atlantic cod, haddock and yellowtail flounder  from the 2009 Canadian scallop fishery on Georges Bank. TRAC Ref. Doc. 2010\/10. 7p.  \\linebreak{} \\linebreak{}} \\def\\FLDGMGBDraft{} \\def\\FLDGMGBSPPname{northern windowpane flounder} \\def\\FLDGMGBSPPnameT{Northern windowpane flounder} \\def\\FLDGMGBRptYr{2015} \\def\\FLDGMGBAuthor{Toni Chute} \\def\\FLDGMGBReviewerComments{/home/dhennen/EIEIO/BigReport/FLD_GMGB/latex}  \\def\\FLDSNEMAMyPathTab{/home/dhennen/EIEIO/BigReport/FLD_SNEMA/tables} \\def\\FLDSNEMAMyPathFig{/home/dhennen/EIEIO/BigReport/FLD_SNEMA/figures} \\def\\FLDSNEMAfigFishCap{Total catch of southern windowpane flounder between 1975 and 2014 by disposition \(landings and discards\).} \\def\\FLDSNEMAfigSSBCap{Trends in the biomass index \(a 3\-year moving average of the NEFSC fall bottom trawl survey index\)  of southern windowpane flounder between 1975 and 2014 from the current  assessment, and the corresponding  \$B\_{Threshold}\${} =  \$\\dfrac{1}{2}\${} \$B\_{MSY}\${} \\textit{proxy}{} = 0.123 kg\/tow\(horizontal dashed line\). } \\def\\FLDSNEMAfigFCap{Trends in relative fishing mortality  of southern windowpane flounder between 1975 and 2014 from the current  assessment, and the corresponding  \$F\_{MSY}\${} \\textit{proxy}{}=2.027 \(horizontal dashed line\). } \\def\\FLDSNEMAfigRecrCap{} \\def\\FLDSNEMAfigSurvCap{NEFSC fall bottom trawl survey indices in kg\/tow for southern windowpane flounder between 1975 and 2014. The approximate 90\\% lognormal confidence intervals are shown.} \\def\\FLDSNEMAPreAmb{This assessment of the southern windowpane flounder \(\\textit{Scophthalmus aquosus}\)  stock is an operational update of the 2012 assessment which included updates through 2010 \(NEFSC 2012\). Based on the 2012 assessment the stock was not overfished, and overfishing was not ocurring. This assessment updates commercial fishery catch data, survey indices of abundance, AIM model results, and reference points through 2014. } \\def\\FLDSNEMASoS{ \\textbf{State of Stock: }{}Based on this updated assessment, the southern windowpane flounder \(\\textit{Scophthalmus aquosus}\)  stock is not overfished and overfishing is not occurring \(Figures \\ref{FLDSNEMASSB\_plot1}\-\\ref{FLDSNEMAF\_plot1}\){}. Retrospective adjustments were not made to the model results. The mean NEFSC fall bottom trawl survey index from years 2012, 2013, and 2014 \(a 3\-year moving average is used as a biomass index\)  was  0.413 \(kg\/tow\)  which is higher than the \$B\_{Threshold}\${}of 0.123 \(kg\/tow\). The 2014 relative fishing mortality was estimated to be  1.308 \(kt per kg\/tow\)  which is lower than the  \$F\_{MSY}\${} \\textit{proxy}{} of 2.027 \(kt per kg\/tow\). } \\def\\FLDSNEMAProj{} \\def\\FLDSNEMASpecCmt{ \\textbf{Special Comments: } \\begin{itemize}{} \\item{}What are the most important sources of uncertainty in this stock assessment?  Explain, and describe qualitatively how they affect the assessment results \(such as estimates of biomass, F, recruitment, and population projections\).  \\linebreak{} \\hspace\*{0.5cm} \\textit{A source of uncertainty for this assessment is missing commercial discard estimates from the general category scallop dredge fleet that should be added to the catch time series for model input. }  \\item{} Does this assessment model have a retrospective pattern? If so, is the pattern minor, or major? \(A major retrospective pattern occurs when the adjusted SSB or  \$F\_{Full}\${} lies outside of the approximate  joint confidence region for SSB and  \$F\_{Full}\${}\; see  Figure \\ref{RhoDecision\_tab}{}\). \\linebreak{} \\hspace\*{0.5cm} \\textit{ The model used to estimate status of this stock does not allow estimation of a retrospective pattern. }  \\item{}Based on this stock assessment, are population projections well determined or uncertain? \\linebreak{} \\hspace\*{0.5cm} \\textit{N\/A}  \\item{}Describe any changes that were made to the current stock assessment, beyond incorporating additional years of data  and the affect these changes had on the assessment and stock status. \\linebreak{} \\hspace\*{0.5cm} \\textit{ No changes were made to the southern windowpane flounder assessment for this update  other than the incorporation of four years of new NEFSC fall bottom trawl survey data and  four years of new U.S. commercial landings and discard data \(2011 \- 2014\). }  \\item{}If the stock status has changed a lot since the previous assessment, explain why this occurred.  \\linebreak{} \\hspace\*{0.5cm} \\textit{The stock status of southern windowpane flounder has not changed since the previous assessment. }  \\item{}Indicate what data or studies are currently lacking and which would be needed most to improve this stock assessment in the future.  \\linebreak{} \\hspace\*{0.5cm} \\textit{Estimates of discards from the general category scallop dredge fleet should be added to the catch time series for model input. However, the model fit is presently good with a randomization test indicating the correlation between ln\(relative F\)  and ln\(replacement ratio\), a measure of the relationship between catch and survey index values, is significant \(p = 0.002.\)  }  \\item{}Are there other important issues? \\linebreak{} \\hspace\*{0.5cm} \\textit{None. } \\end{itemize}{}} \\def\\FLDSNEMARefr{ \\textbf{References: }{} \\linebreak{} Most recent assessment update:  \\linebreak{} Northeast Fisheries Science Center. 2012. Assessment or Data Updates of 13 Northeast Groundfish Stocks through 2010.  US Dept Commer, Northeast Fish Sci Cent Ref Doc. 12\-06\; 789 p. Available online at http:\/\/nefsc.noaa.gov\/publications\/  \\linebreak{} \\linebreak{} Most recent benchmark assessment:  \\linebreak{} Northeast Fisheries Science Center. 2008. Assessment of 19 Northeast Groundfish Stocks through 2007:  Report of the 3$^{rd}$ Groundfish Assessment Review Meeting \(GARM III\), Northeast Fisheries Science Center,  Woods Hole, MA, August 4\-8, 2008. US Dep Commer, NOAA Fisheries, Northeast Fish Sci Cent Ref Doc. 08\-15\; 884 p + xvii. \\linebreak{} \\linebreak{}} \\def\\FLDSNEMADraft{} \\def\\FLDSNEMASPPname{southern windowpane flounder} \\def\\FLDSNEMASPPnameT{Southern windowpane flounder} \\def\\FLDSNEMARptYr{2015} \\def\\FLDSNEMAAuthor{Toni Chute} \\def\\FLDSNEMAReviewerComments{/home/dhennen/EIEIO/BigReport/FLD_SNEMA/latex}  \\def\\PLAUNITMyPathTab{/home/dhennen/EIEIO/BigReport/PLA_UNIT/tables} \\def\\PLAUNITMyPathFig{/home/dhennen/EIEIO/BigReport/PLA_UNIT/figures} \\def\\PLAUNITfigFishCap{Total catch of Gulf of Maine\-Georges Bank American Plaice between 1980 and 2015 by fleet \(Gulf of Maine, Georges Bank, Southern New England, and Canadian\)  and disposition \(landings and discards\).} \\def\\PLAUNITfigSSBCap{Trends in spawning stock biomass of Gulf of Maine\-Georges Bank American Plaice between 1980 and 2015 from the current  \(solid line\)  and previous \(dashed line\)  assessment and the corresponding  \$SSB\_{Threshold}\${} \(\$\\dfrac{1}{2}\${} \$SSB\_{MSY}\${} \\textit{proxy}{}\; horizontal dashed line\)  as well as  \$SSB\_{Target}\${} \(\$SSB\_{MSY}\${} \\textit{proxy}{}\; horizontal dotted line\)   based on the 2015 assessment.  Biomass was adjusted for a retrospective pattern  and the adjustment is shown in red.  The approximate 90\\% normal confidence intervals are shown.} \\def\\PLAUNITfigFCap{Trends in the fully selected fishing mortality \(\$F\_{Full}\${}\)  of Gulf of Maine\-Georges Bank American Plaice between 1980 and 2015 from the current  \(solid line\)  and previous \(dashed line\)  assessment and the corresponding  \$F\_{Threshold}\${} \(\$F\_{MSY}\${} \\textit{proxy}{}=0.196\; horizontal dashed line\).  \$F\_{Full}\${} was adjusted for a retrospective pattern  and the adjustment is shown in red,  based on the 2015 assessment. The approximate 90\\% normal confidence intervals are shown.} \\def\\PLAUNITfigRecrCap{Trends in Recruits \(age 1\)  \(000s\)  of Gulf of Maine\-Georges Bank American Plaice between 1980 and 2015 from the current \(solid line\)  and previous \(dashed line\)  assessment.} \\def\\PLAUNITfigSurvCap{Indices of biomass for the Gulf of Maine\-Georges Bank American Plaice between 1963 and 2015 for the Northeast Fisheries Science Center \(NEFSC\)  and Massachusetts Division of Marine Fisheries \(MADMF\)  spring and autumn research bottom trawl surveys.  The approximate 90\\% normal confidence intervals are shown.} \\def\\PLAUNITPreAmb{This assessment of the Gulf of Maine\-Georges Bank American Plaice \(\\textit{Hippoglossoides platessoides}\)  stock is an operational update of the existing 2012 benchmark assessment \(O\'Brien et al. 2012\). Based on the previous assessment the stock was not overfished, and overfishing was not ocurring. This 2015 assessment updates commercial fishery catch data, research survey indices of abundance, the analytical VPA assessment model, and reference points through 2014. Additionally, stock projections have been updated through 2018.} \\def\\PLAUNITSoS{ \\textbf{State of Stock: }{}Based on this updated assessment, the Gulf of Maine\-Georges Bank American Plaice \(\\textit{Hippoglossoides platessoides}\)  stock is not overfished and overfishing is not occurring \(Figures \\ref{PLAUNITSSB\_plot1}\-\\ref{PLAUNITF\_plot1}\){}.  Retrospective adjustments were made to the model results.  Spawning stock biomass \(SSB\)  in 2014 was estimated to be 10,915 mt which is 83\\% of the biomass target for this stock \(\$SSB\_{MSY}\${} \\textit{proxy}{} = 13,107\;  Figure \\ref{PLAUNITSSB\_plot1}{}\). The 2014 fully selected fishing mortality was estimated to be 0.118 which is 60\\% of the overfishing threshold proxy \(\$F\_{MSY}\${} \\textit{proxy}{} = 0.196\;  Figure \\ref{PLAUNITF\_plot1}{}\).} \\def\\PLAUNITProj{ \\textbf{Projections: }{}Short term projections of biomass were derived by sampling from an empirical cumulative  distribution  function of 34 recruitment estimates from VPA model results. The annual fishery selectivity, maturity ogive, and mean weights at age used in projections are the most recent 5 year averages\;  retrospective adjustments were applied in the projections.} \\def\\PLAUNITSpecCmt{ \\textbf{Special Comments: } \\begin{itemize}{} \\item{}What are the most important sources of uncertainty in this stock assessment?  Explain, and describe qualitatively how they affect the assessment results \(such as estimates of biomass, F, recruitment, and population projections\).  \\linebreak{} \\hspace\*{0.5cm} \\textit{A source of uncertainty in this assessment are the estimates of historical landings at age, prior to 1984, and the magnitude of  historical discards, prior to 1989. Both of these affect the scale of the biomass and fishing mortality estimates, and influence reference point estimations.}  \\item{} Does this assessment model have a retrospective pattern? If so, is the pattern minor, or major? \(A major retrospective pattern occurs when the adjusted SSB or  \$F\_{Full}\${} lies outside of the approximate  joint confidence region for SSB and  \$F\_{Full}\${}\; see  Figure \\ref{RhoDecision\_tab}{}\). \\linebreak{} \\hspace\*{0.5cm} \\textit{ The 7\-year Mohn\'s  \\textrho{}, relative to SSB, was 0.63 in the 2012 assessment and was 0.32 in 2014. The 7\-year Mohn\'s  \\textrho{}, relative to F, was \-0.35 in the 2012 assessment and was 0.32 in 2014. There was a major retrospective pattern for this assessment because the  \\textrho{} adjusted estimates of 2014 SSB \(\$SSB\_{\\rho}\${}=10,915\)  and 2014 F \(\$F\_{\\rho}\${}=0.118\)  were outside the approximate 90\\% confidence regions around SSB \(12,742 \- 16,439\)  and F \(0.069 \- 0.093\).  A retrospective  adjustment was made for both the determination of stock status and for projections of catch in 2016. The retrospective adjustment changed the 2014 SSB from 14,543 to 10,915 and the 2014  \$F\_{Full}\${} from 0.08 to 0.118.}  \\item{}Based on this stock assessment, are population projections well determined or uncertain? \\linebreak{} \\hspace\*{0.5cm} \\textit{Population projections for Gulf of Maine\-Georges Bank American Plaice are reasonably well determined.}  \\item{}Describe any changes that were made to the current stock assessment, beyond incorporating additional years of data  and the effect these changes had on the assessment and stock status. \\linebreak{} \\hspace\*{0.5cm} \\textit{ No major changes, other than the addition of recent years of data, were made to the Gulf of Maine\-Georges Bank American Plaice assessment for this update. A new version of VPA was used \(V3.3.0\)  which gave very similar results to the 2012 VPA 3.1.0 run, with the same F and slightly lower SSB. The MADMF spring and autumn survey indices were re\-estimated for the time series, accounting for revised stratum areas. The revision occurred in 2007, but was overlooked in the 2012 assessment. A comparison of 2010 terminal year VPAs indicated minimal differences in 2010 SSB \(now slightly lower\)  and no change in F.}  \\item{}If the stock status has changed a lot since the previous assessment, explain why this occurred.  \\linebreak{} \\hspace\*{0.5cm} \\textit{As in recent assessments for Gulf of Maine\-Georges Bank American Plaice the stock status remains as not overfished and overfishing not occurring.}  \\item{}Indicate what data or studies are currently lacking and which would be needed most to improve this stock assessment in the future.  \\linebreak{} \\hspace\*{0.5cm} \\textit{The Gulf of Maine\-Georges Bank American Plaice assessment could be improved with updated studies on growth of Georges Bank and Gulf of Maine fish.}  \\item{}Are there other important issues? \\linebreak{} \\hspace\*{0.5cm} \\textit{A difference in growth between GM and GB fish has been documented, however, historical catch data information for GB may not be sufficient to conduct a separate assessment. Also, the growth difference may not persist in the most recent years. This could all be explored further in an benchmark review.} \\end{itemize}{}} \\def\\PLAUNITRefr{ \\textbf{References: }{} \\linebreak{}O\'Brien, L. and J. Dayton \(2012\). E. Gulf of Maine \- Georges Bank American plaice Assessment for 2012 in Northeast Fisheries Science Center, 2012, Assessment or Data Updates of 13 Northeast Groundfish Stocks through 2010. US Dept Commer, Northeast Fish Sci Cent Ref Doc. 12\-06\; 789 p. http:\/\/www.nefsc.noaa.gov\/publications\/crd\/crd1206\/. \\linebreak{} \\linebreak{}} \\def\\PLAUNITDraft{} \\def\\PLAUNITSPPname{Gulf of Maine-Georges Bank American Plaice} \\def\\PLAUNITSPPnameT{Gulf of Maine-Georges Bank American Plaice} \\def\\PLAUNITRptYr{2015} \\def\\PLAUNITAuthor{Loretta O\'Brien} \\def\\PLAUNITReviewerComments{/home/dhennen/EIEIO/BigReport/PLA_UNIT/latex}  \\def\\WITUNITMyPathTab{/home/dhennen/EIEIO/BigReport/WIT_UNIT/tables} \\def\\WITUNITMyPathFig{/home/dhennen/EIEIO/BigReport/WIT_UNIT/figures} \\def\\WITUNITfigFishCap{Total catch of witch flounder between 1982 and 2014 by fleet \(commercial\)  and disposition \(landings and discards\).} \\def\\WITUNITfigSSBCap{Trends in spawning stock biomass \(mt\)  of witch flounder between 1982 and 2014 from the current  \(solid line\)  and previous \(dashed line\)  assessment and the corresponding  \$SSB\_{Threshold}\${} \(\$\\dfrac{1}{2}\${} \$SSB\_{MSY}\${}\; horizontal dashed line\)  as well as  \$SSB\_{Target}\${} \$SSB\_{MSY}\${}\; horizontal dotted line\)   based on the current assessment. Red solid vertical line indicates rho adjusted SSB. Black solid vertical line indicates 90\\% confidence interval for 2014.} \\def\\WITUNITfigFCap{Trends in the fully selected fishing mortality \(\$F\_{Full}\${}\)  of witch flounder between 1982 and 2014 from the current  \(solid line\)  and previous \(dashed line\)  assessment and the corresponding  \$F\_{Threshold}\${} \(\$F\_{MSY}\${}=0.279\; horizontal dashed line\)  based on the current assessment.  Red solid vertical line indicates rho adjusted  \$F\_{Full}\${}. Black solid vertical line indicates 90\\% confidence interval for 2014.} \\def\\WITUNITfigRecrCap{Trends in Age 3  \(000s\)  of witch flounder between 1982 and 2014 from the current \(solid line\)  and previous \(dashed line\)  assessment.} \\def\\WITUNITfigSurvCap{Indices of biomass \(kg\/tow\)  for the witch flounder between 1963 and 2015 for the Northeast Fisheries Science Center \(NEFSC\)  spring and fall bottom trawl surveys.  The 90\\% lognormal confidence intervals are shown.} \\def\\WITUNITPreAmb{This assessment of the witch flounder \(\\textit{Glyptocephalus cynoglossus}\)  stock is an operational update of the 2012 assessment \(NEFSC 2012\)  and the 2008 benchmark assessment \(NEFSC 2008\). This assessment updates commercial fishery catch data, research survey indices, and the analytical assessment model through 2014. Additionally, stock projections have been updated through 2018. Reference points have been updated. } \\def\\WITUNITSoS{ \\textbf{State of Stock: }{}witch flounder \(\\textit{Glyptocephalus cynoglossus}\)  stock is overfished and overfishing is occurring \(Figures \\ref{WITUNITSSB\_plot1}\-\\ref{WITUNITF\_plot1}\){}. Retrospective adjustments were made to the model results.  Spawning stock biomass \(SSB\)  in 2014 was estimated to be 2,077 \(mt\)  which is 22\\% of the  \$SSB\_{MSY}\${} proxy \(9,473\;  Figure \\ref{WITUNITSSB\_plot1}{}\).  The 2014 fully selected fishing mortality was estimated to be 0.687 which is 246\\% of the  \$F\_{MSY}\${} proxy \(0.279\;  Figure \\ref{WITUNITF\_plot1}{}\). A retrospective adjustment to  \$F\_{Full}\${} and SSB in 2014 was required but did not lead to a change in status.  } \\def\\WITUNITProj{ \\textbf{Projections: }{}Short term projection recruitment was sampled from a cumulative distribution function derived from ADAPT VPA \(with split time series between 1994 and 1995\)  estimated age 3 recruitment between 1982 and 2013.  Average 2010\-2014 partial recruitment, average 2010\-2014 mean weights, and maturation ogive representing 2011\-2015 maturity data were used.} \\def\\WITUNITSpecCmt{ \\textbf{Special Comments: } \\begin{itemize}{} \\item{}What are the most important sources of uncertainty in this stock assessment?  Explain, and describe qualitatively how they affect the assessment results \(such as estimates of biomass, F, recruitment, and population projections\).  \\linebreak{} \\hspace\*{0.5cm} \\textit{An important source of uncertainty is the retrospective pattern where fishing mortality is underestimated and spawning stock biomass and recruitment are overestimated. }  \\item{} Does this assessment model have a retrospective pattern? If so, is the pattern minor, or major? \(A major retrospective pattern occurs when the adjusted SSB or  \$F\_{Full}\${} lies outside of the approximate  joint confidence region for SSB and  \$F\_{Full}\${}\).  \\linebreak{} \\hspace\*{0.5cm} \\textit{ The 7\-year Mohn\'s  \\textrho{}, relative to SSB, was 0.61 in the 2012 assessment and was 0.51 in 2014. The 7\-year Mohn\'s  \\textrho{}, relative to F, was \-0.33 in the 2012 assessment and was \-0.38 in 2014. There was a major retrospective pattern for this assessment because the  \\textrho{} adjusted estimates of 2014 SSB \(\$SSB\_{\\rho}\${}=2,077\)  and 2014 F \(\$F\_{\\rho}\${}=0.687\)  were outside the approximate 90\\% confidence regions around SSB \(2,643 \- 3,864\)  and F \(0.321 \- 0.603\).  A retrospective  adjustment was made for both the determination of stock status and for projections of catch in 2016. The retrospective adjustment changed the 2014 SSB from 3,129 to 2,077 and the 2014  \$F\_{Full}\${} from 0.428 to 0.687.}  \\item{}Based on this stock assessment, are population projections well determined or uncertain? \\linebreak{} \\hspace\*{0.5cm} \\textit{Population projections for witch flounder appear to be optimistic\; the projected rho adjusted biomass from the last assessment  was above the upper confidence bounds of the projected rho adjusted biomass estimated in the current assessment. }  \\item{}Describe any changes that were made to the current stock assessment, beyond incorporating additional years of data  and the effect these changes had on the assessment and stock status.  \\linebreak{} \\hspace\*{0.5cm} \\textit{TOGA \(Type, Operation, Gear, Acquisition\)  values were used for haul criteria for NEFSC surveys for 2009 onward and minor changes in the use of observer data for discard estimates were made to the current witch flounder assessment. These changes had negligible effect on the assessment and stock status.  }  \\item{}If the stock status has changed a lot since the previous assessment, explain why this occurred.  \\linebreak{} \\hspace\*{0.5cm} \\textit{No change in stock status has occurred for witch flounder since the previous assessment. }  \\item{}Indicate what data or studies are currently lacking and which would be needed most to improve this stock assessment in the future.  \\linebreak{} \\hspace\*{0.5cm} \\textit{Extensive studies have examined the causes of retrospective patterns with no definitive conclusions other than a change in model does not resolve the issue. }  \\item{}Are there other important comments? \\linebreak{} \\hspace\*{0.5cm} \\textit{The VPA analysis was performed with survey time series split between 1994 and 1995. This time split corresponds to changes in the commercial reporting methods as well as other regulatory management changes.  } \\end{itemize}{}} \\def\\WITUNITRefr{ \\textbf{References: }{} \\linebreak{}Northeast Fisheries Science Center. 2008. Assessment of 19 Northeast Groundfish Stocks through 2007: Report of the 3$^{rd}$ Groundfish Assessment Review Meeting \(GARM III\), Northeast Fisheries Science Center, Woods Hole, Massachusetts, August 4\-8, 2008. US Dep Commer, NOAA Fisheries, Northeast Fish Sci Cent Ref Doc. 08\-15\; 884 p + xvii. http:\/\/www.nefsc.noaa.gov\/publications\/crd\/crd0815\/ \\linebreak{} \\linebreak{}Northeast Fisheries Science Center. 2012. Assessment or Data Updates of 13 Northeast Groundfish Stocks through 2010.  US Dep Commer, NOAA Fisheries, Northeast Fish Sci Cent Ref Doc. 12\-06\; 789 p. http:\/\/www.nefsc.noaa.gov\/publications\/crd\/crd1206\/ \\linebreak{} \\linebreak{}} \\def\\WITUNITDraft{} \\def\\WITUNITSPPname{witch flounder} \\def\\WITUNITSPPnameT{Witch flounder} \\def\\WITUNITRptYr{2015} \\def\\WITUNITAuthor{Susan Wigley} \\def\\WITUNITReviewerComments{/home/dhennen/EIEIO/BigReport/WIT_UNIT/latex}  \\def\\HKWUNITMyPathTab{/home/dhennen/EIEIO/BigReport/HKW_UNIT/tables} \\def\\HKWUNITMyPathFig{/home/dhennen/EIEIO/BigReport/HKW_UNIT/figures} \\def\\HKWUNITfigFishCap{Total catch of white hake between 1963 and 2014 by fleet \(commercial, recreational, or Canadian\)  and disposition \(landings and discards\).} \\def\\HKWUNITfigSSBCap{Trends in spawning stock biomass of white hake between 1963 and 2014 from the current  \(solid line\)  and previous \(dashed line\)  assessment and the corresponding  \$SSB\_{Threshold}\${} \(\$\\dfrac{1}{2}\${} \$SSB\_{MSY}\${} \\textit{proxy}{}\; horizontal dashed line\)  as well as  \$SSB\_{Target}\${} \(\$SSB\_{MSY}\${} \\textit{proxy}{}\; horizontal dotted line\)   based on the 2014 assessment.  The red dot indicates the rho\-adjusted SSB values that would have resulted had a retrospective  adjusment been made \(see Special Comments section\).  The approximate 90\\% lognormal confidence intervals are shown.} \\def\\HKWUNITfigFCap{Trends in the fully selected fishing mortality \(\$F\_{Full}\${}\)  of white hake between 1963 and 2014 from the current  \(solid line\)  and previous \(dashed line\)  assessment and the corresponding  \$F\_{Threshold}\${} \(\$F\_{MSY}\${} \\textit{proxy}{}=0.188\; horizontal dashed line\).  The red dot indicates the rho\-adjusted SSB values that would have resulted had a retrospective  adjusment been made \(see Special Comments section\).  The approximate 90\\% lognormal confidence intervals are shown.} \\def\\HKWUNITfigRecrCap{Trends in Recruits \(age 1\)  \(000s\)  of white hake between 1963 and 2014 from the current \(solid line\)  and previous \(dashed line\)  assessment. The approximate 90\\% lognormal confidence intervals are shown.} \\def\\HKWUNITfigSurvCap{Indices of biomass for the white hake between 1963 and 2015 for the Northeast Fisheries Science Center \(NEFSC\)  spring and fall bottom trawl surveys.  The approximate 90\\% lognormal confidence intervals are shown.} \\def\\HKWUNITPreAmb{This assessment of the white hake \(\\textit{Urophycis tenuis}\)  stock is an operational update of the existing 2013 benchmark ASAP assessment \(NEFSC 2013\). Based on the previous assessment the stock was not overfished, and overfishing was not ocurring. This assessment updates commercial fishery catch data, research survey indices of abundance, and the ASAP assessment models and reference points through 2014. Additionally, stock projections have been updated through 2018.} \\def\\HKWUNITSoS{ \\textbf{State of Stock: }{}Based on this updated assessment, white hake \(\\textit{Urophycis tenuis}\)  stock is not overfished and overfishing is not occurring \(Figures \\ref{HKWUNITSSB\_plot1}\-\\ref{HKWUNITF\_plot1}\){}. Retrospective adjustments were not made to the model results.  Spawning stock biomass \(SSB\)  in 2014 was estimated to be 28,553 \(mt\)  which is 88\\% of the biomass threshold for an overfished stock \(\$SSB\_{MSY}\${} \\textit{proxy}{} = 32,550\;  Figure \\ref{HKWUNITSSB\_plot1}{}\).  The 2014 fully selected fishing mortality was estimated to be 0.076 which is 40\\% of the overfishing threshold proxy \(\$F\_{MSY}\${} \\textit{proxy}{} = 0.188\;  Figure \\ref{HKWUNITF\_plot1}{}\).} \\def\\HKWUNITProj{ \\textbf{Projections: }{}Short term projections of catch and SSB were derived by sampling from a cumulative  distribution  function of recruitment estimates from ASAP from 1995\-2012. The annual fishery selectivity, maturity ogive, and mean weights at age used in the projection  are the most recent 5 year averages. } \\def\\HKWUNITSpecCmt{ \\textbf{Special Comments: } \\begin{itemize}{} \\item{}What are the most important sources of uncertainty in this stock assessment?  Explain, and describe qualitatively how they affect the assessment results \(such as estimates of biomass, F, recruitment, and population projections\).  \\linebreak{} \\hspace\*{0.5cm} \\textit{1. Catch at age information is not well characterized due to possible mis\-identification of species in the commercial and sea sampling data, particularly in early years, low sampling of commercial landings in  some years, and sparse discard data particularly in early years.  \\linebreak{} \\hspace\*{0.5cm}2. Since the commercial catch is aged primarily with survey age\/length keys, there is considerable augmentation required, mainly for ages 5 and older. The numbers at age and mean weights at age in the catch for these ages may therefore not be well specified.  \\linebreak{} \\hspace\*{0.5cm}3. White hake may move seasonally into and out of the defined stock area.  \\linebreak{} \\hspace\*{0.5cm}4. There are no commercial catch at age data prior to 1989 and the catchability of older ages in the surveys is very low. This results in a large uncertainty in starting numbers at age.  \\linebreak{} \\hspace\*{0.5cm}5. Since 2003, dealers have been culling very large fish out of the large category. However, there was no market category to input into the landings until June 2014. The length compositions are distinct from large and have been identified since 2011. This may bias the age composition of the landings, particularly in 2014 when 2000 of the 5000 large samples were these extra\-large fish.  \\linebreak{} \\hspace\*{0.5cm}6. A pooled age\/length key is used for 1963\-1981, fall 2003 \(second half of commercial key\)  and 2014.Age data were not available for 2014 in time for this assessment. The same pooled key that was used for 1963\-1981 was used for 2014.}  \\item{} Does this assessment model have a retrospective pattern? If so, is the pattern minor, or major? \(A major retrospective pattern occurs when the adjusted SSB or  \$F\_{Full}\${} lies outside of the approximate  joint confidence region for SSB and  \$F\_{Full}\${}\; see  Figure \\ref{RhoDecision\_tab}{}\). \\linebreak{} \\hspace\*{0.5cm} \\textit{ No retrospective adjustment of spawning stock biomass or fishing mortality in 2014 was required.  The pattern in this assessment is considered minor \(Mohn’s rho of 0.18 on SSB, Mohn’s rho of 0.12 on F\)  with the adjusted SSB within the 90 \\% CI of the MCMC. However, the Mohn’s rho for Age 1 estimates is 0.54. This may have an impact on projections if this continues into the future.}  \\item{}Based on this stock assessment, are population projections well determined or uncertain? \\linebreak{} \\hspace\*{0.5cm} \\textit{Population projections for white hake, are not well determined and projected biomass from the last assessment  was outside the confidence bounds of the biomass estimated in the current assessment. }  \\item{}Describe any changes that were made to the current stock assessment, beyond incorporating additional years of data  and the affect these changes had on the assessment and stock status. \\linebreak{} \\hspace\*{0.5cm} \\textit{ The 2011 catch\-at\-length and age were re\-estimated for both landings and discards. For the  landings, two samples were adjusted for dorsal length to total length that had been missed in the previous assessment.}  \\item{}If the stock status has changed a lot since the previous assessment, explain why this occurred.  \\linebreak{} \\hspace\*{0.5cm} \\textit{While stock status of white hake has not changed, the stock has not rebuilt as the projections from the last assessment indicated. This is due to the retrospective in recruitment. The numbers for the 2005\-2009 year classes, which were included in the age 2\-6 starting numbers in the projections, were over\-estimated which led to over\-estimating SSB in 2014.}  \\item{}Indicate what data or studies are currently lacking and which would be needed most to improve this stock assessment in the future.  \\linebreak{} \\hspace\*{0.5cm} \\textit{ Age structures from the observer program are available and should be aged to augment  the survey keys. There is a also a new market category for heads and age structures could be  acquired from these is an otolith length\/total length relationship can be established. }  \\item{}Are there other important issues? \\linebreak{} \\hspace\*{0.5cm} \\textit{None. } \\end{itemize}{}} \\def\\HKWUNITRefr{ \\textbf{References: }{} \\linebreak{} NEFSC. 2013. 56$^{th}$ Northeast Regional Stock Assessment Workshop \(56$^{th}$ SAW\)  Assessment  Report.US Dep Commer, NOAA Fisheries, Northeast Fish Sci Cent Ref Doc. 13\-10\; 868 p.  http:\/\/www.nefsc.noaa.gov\/publications\/crd\/crd1310\/  \\linebreak{} \\linebreak{}} \\def\\HKWUNITDraft{} \\def\\HKWUNITSPPname{white hake} \\def\\HKWUNITSPPnameT{White hake} \\def\\HKWUNITRptYr{2015} \\def\\HKWUNITAuthor{Katherine Sosebee} \\def\\HKWUNITReviewerComments{/home/dhennen/EIEIO/BigReport/HKW_UNIT/latex}