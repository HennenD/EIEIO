 \def\HADGMMyPathTab{/home/dhennen/EIEIO/BigReport/HAD_GM/tables} \def\HADGMMyPathFig{/home/dhennen/EIEIO/BigReport/HAD_GM/figures} \def\HADGMfigFishCap{Total catch of Gulf of Maine haddock between 1977 and 2014 by fleet (commercial, recreational, or foreign)  and disposition (landings and discards).} \def\HADGMfigSSBCap{Trends in spawning stock biomass (SSB)  of Gulf of Maine haddock between 1977 and 2014 from the current  (solid line)  and previous (dashed line)  assessment and the corresponding  $SSB_{Threshold}${} ($\dfrac{1}{2}${} $SSB_{MSY}${} \textit{proxy}{}; horizontal dashed line)  as well as  $SSB_{Target}${} ($SSB_{MSY}${} \textit{proxy}{}; horizontal dotted line)   based on the 2015 assessment. The approximate 90\percent{} lognormal confidence intervals are shown. The red dot indicates the rho-adjusted SSB values that would have resulted had a retrospective adjusment been made to either model (see Special Comments section).} \def\HADGMfigFCap{Trends in the fully selected fishing mortality (F)  of Gulf of Maine haddock between 1977 and 2014 from the current  (solid line)  and previous (dashed line)  assessment and the corresponding  $F_{Threshold}${} ($F_{MSY}${} \textit{proxy}{}=0.468; horizontal dashed line)  from the 2015 assessment model. The approximate 90\percent{} lognormal confidence intervals are shown. The red dot indicates the rho-adjusted F values that would have resulted had a retrospective adjusment been made to either model (see Special Comments section).} \def\HADGMfigRecrCap{Trends in Recruits (age 1)  (000s)  of Gulf of Maine haddock between 1977 and 2014 from the current (solid line)  and previous (dashed line)  assessment. The approximate 90\percent{} lognormal confidence intervals are shown.} \def\HADGMfigSurvCap{Indices of biomass for the Gulf of Maine haddock between 1963 and 2015 for the Northeast Fisheries Science Center (NEFSC)  spring and fall bottom trawl surveys.  The approximate 90\percent{} lognormal confidence intervals are shown.} \def\HADGMPreAmb{This assessment of the Gulf of Maine haddock (\textit{Melanogrammus aeglefinus})  stock is an operational assessment of the existing 2014 benchmark assessment (NEFSC 2014). Based on the previous assessment, the stock was not overfished, and overfishing was not occurring. This assessment updates commercial and recreational fishery catch data, research survey indices of abundance, and the analytical ASAP assessment model and reference points through 2014. Additionally, stock projections have been updated through 2018} \def\HADGMSoS{ \textbf{State of Stock: }{}Based on this updated assessment, the Gulf of Maine haddock (\textit{Melanogrammus aeglefinus})  stock is not overfished and overfishing is not occurring (Figures \ref{HADGMSSB_plot1}-\ref{HADGMF_plot1}){}. Retrospective adjustments were not made to the model results (see Special Comments section of this report). Spawning stock biomass (SSB)  in 2014 was estimated to be 10,325 (mt)  which is 223\percent{} of the biomass target ($SSB_{MSY}${} \textit{proxy}{} = 4,623;  Figure \ref{HADGMSSB_plot1}{}).  The 2014 fully selected fishing mortality was estimated to be 0.257 which is 55\percent{} of the overfishing threshold proxy ($F_{MSY}${} \textit{proxy}{} =  $F_{40\percent{}}${} = 0.468;  Figure \ref{HADGMF_plot1}{}).} \def\HADGMProj{ \textbf{Projections: }{}Short term projections of median total fishery yield and spawning stock biomass for Gulf of Maine haddock were conducted based on a harvest scenario of fishing at the  $F_{MSY}${} \textit{proxy}{} between 2016 and 2018. Catch in 2015 has been estimated at 885 mt. Recruitment was sampled from a cumulative distribution  function of model estimated age-1 recruitment from 1977-2012. The age-1 estimate in 2015 was generated from the geometric mean of the 1977-2014 recruitment series. The annual fishery selectivity, maturity ogive, and mean weights at age used in the projections  were estimated from the most recent 5 year averages;  retrospective adjustments were not applied in the projections. Given the uncertainty in the size of the 2012 and 2013 year classes and the model's tendency to overestimate large terminal year classes, the 2015 assessment review panel recommended that a sensitivity projection scenario which constrains terminal recruitment ('Constrain terminal R')  be brought forward to the New England Fishery Management Council's Scientific and Statistical Committee (NEFMC SSC)  for consideration when setting catch advice; these sensitivity projections are provided in the Supplemental Information Report (\href{http://www.nefsc.noaa.gov/saw/sasi/sasi_report_options.php}{SASINF}{}).} \def\HADGMSpecCmt{ \textbf{Special Comments: } \begin{itemize}{} \item{}What are the most important sources of uncertainty in this stock assessment?  Explain, and describe qualitatively how they affect the assessment results (such as estimates of biomass, F, recruitment, and population projections).  \linebreak{} \hspace*{0.5cm} \textit{ The largest source of uncertainty in the assessment is the estimated size of the 2012 and 2013 year classes. Based on the estimated selectivity patterns, these year classes are projected to be 30\percent{} selected to the fishery in 2016 and 2017 respectively. However, recent changes to the commercial and recreational minimum retention size may result in these year classes recruiting to the fishery sooner than projected. The abundance and growth of the 2012 and 2013 year classes should be monitored and frequent model updates would be expected to improve the estimates of year class size and validate projection assumptions.}  \item{}Does this assessment model have a retrospective pattern? If so, is the pattern minor, or major? (A major retrospective pattern occurs when the adjusted SSB or  $F_{Full}${} lie outside of the approximate joint confidence region for SSB and  $F_{Full}${}). \linebreak{} \hspace*{0.5cm} \textit{This assessment does not exhibit a retrospective pattern and therefore no retrospective adjustments were made to the terminal model results or the short-term catch projections. The 7-year Mohn's rho values on SSB (-0.04)  and F (0.03)  are small and there were no consistent patterns in the directionality of the retrospective 'peels' (see the Supplemental Information Report, \href{http://www.nefsc.noaa.gov/saw/sasi/sasi_report_options.php}{SASINF}{}).}  \item{}Based on this stock assessment, are population projections well determined or uncertain?  \linebreak{} \hspace*{0.5cm} \textit{Population projections for Gulf of Maine haddock are reasonably well determined. The projected biomass from the last assessment is below the confidence bounds of the biomass estimated in the current assessment; however, this is primarily due to the positive rescaling of the population size that occured from turning the ASAP model likelihood constants option off (see next Special Comment).}  \item{}Describe any changes that were made to the current stock assessment beyond incorporating additional years of data,  and the affect these changes had on the assessment and stock status. \linebreak{} \hspace*{0.5cm} \textit{ Recreational catch estimates from 2004-2014 were re-estimated as part of this update to account for updates to the MRIP data. Additionally, the ASAP model was revised by turning the likelihood constants off; sensitivity runs on SAW/SARC 59 model suggest minor positive rescaling of recruitment and SSB, negative rescaling of F (sensitivity results are provided in the Supplemental Information Report, \href{http://www.nefsc.noaa.gov/saw/sasi/sasi_report_options.php}{SASINF}{}).}  \item{}If the stock status has changed a lot since the previous assessment, explain why this occurred.  \linebreak{} \hspace*{0.5cm} \textit{There has been no change in stock status since the previous SAW/SARC 59 assessment (2014).}  \item{}Indicate what data or studies are currently lacking and which would be needed most to improve this stock assessment in the future.  \linebreak{} \hspace*{0.5cm} \textit{Currently the assessment assumes 50\percent{} survival of haddock discarded in the recreational fishery - directed field research would improve this estimate. Additionally, a better understanding of recruitment processes may help to improve recruitment forecasting.}  \item{}Are there other important issues? \linebreak{} \hspace*{0.5cm} \textit{None.} \end{itemize}{}} \def\HADGMRefr{ \textbf{References: }{} \linebreak{}Northeast Fisheries Science Center. 2014. 59$^{th}$ Northeast Regional Stock Assessment Workshop (59$^{th}$ SAW)  Assessment Report. US Dept Commer, Northeast Fish Sci Cent Ref Doc. 14-09; 782 p. Available from: National Marine Fisheries Service, 166 Water Street, Woods Hole, MA 02543-1026. \href{http://nefsc.noaa.gov/publications/crd/crd1409/}{CRD14-09}  \linebreak{} \linebreak{}} \def\HADGMDraft{} \def\HADGMSPPname{Gulf of Maine haddock} \def\HADGMSPPnameT{Gulf of Maine haddock} \def\HADGMRptYr{2015} \def\HADGMAuthor{Michael Palmer} \def\HADGMReviewerComments{/home/dhennen/EIEIO/BigReport/HAD_GM/latex}