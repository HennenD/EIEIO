\def\YELCCGMReviewerComments{/home/dhennen/EIEIO/BigReport/YEL_CCGM/latex}   \def\FLDSNEMAReviewerComments{/home/dhennen/EIEIO/BigReport/FLD_SNEMA/latex}  \def\PLAUNITMyPathTab{/home/dhennen/EIEIO/BigReport/PLA_UNIT/tables} \def\PLAUNITMyPathFig{/home/dhennen/EIEIO/BigReport/PLA_UNIT/figures} \def\PLAUNITfigFishCap{Total catch of Gulf of Maine\-Georges Bank American Plaice between 1980 and 2015 by fleet \(Gulf of Maine, Georges Bank, Southern New England, and Canadian\)  and disposition \(landings and discards\).} \def\PLAUNITfigSSBCap{Trends in spawning stock biomass of Gulf of Maine\-Georges Bank American Plaice between 1980 and 2015 from the current  \(solid line\)  and previous \(dashed line\)  assessment and the corresponding  \$SSB\_{Threshold}\${} \(\$\dfrac{1}{2}\${} \$SSB\_{MSY}\${} \textit{proxy}{}\; horizontal dashed line\)  as well as  \$SSB\_{Target}\${} \(\$SSB\_{MSY}\${} \textit{proxy}{}\; horizontal dotted line\)   based on the 2015 assessment.  Biomass was adjusted for a retrospective pattern  and the adjustment is shown in red.  The approximate 90\% normal confidence intervals are shown.} \def\PLAUNITfigFCap{Trends in the fully selected fishing mortality \(\$F\_{Full}\${}\)  of Gulf of Maine\-Georges Bank American Plaice between 1980 and 2015 from the current  \(solid line\)  and previous \(dashed line\)  assessment and the corresponding  \$F\_{Threshold}\${} \(\$F\_{MSY}\${} \textit{proxy}{}=0.196\; horizontal dashed line\).  \$F\_{Full}\${} was adjusted for a retrospective pattern  and the adjustment is shown in red,  based on the 2015 assessment. The approximate 90\% normal confidence intervals are shown.} \def\PLAUNITfigRecrCap{Trends in Recruits \(age 1\)  \(000s\)  of Gulf of Maine\-Georges Bank American Plaice between 1980 and 2015 from the current \(solid line\)  and previous \(dashed line\)  assessment.} \def\PLAUNITfigSurvCap{Indices of biomass for the Gulf of Maine\-Georges Bank American Plaice between 1963 and 2015 for the Northeast Fisheries Science Center \(NEFSC\)  and Massachusetts Division of Marine Fisheries \(MADMF\)  spring and autumn research bottom trawl surveys.  The approximate 90\% normal confidence intervals are shown.} \def\PLAUNITPreAmb{This assessment of the Gulf of Maine\-Georges Bank American Plaice \(\textit{Hippoglossoides platessoides}\)  stock is an operational update of the existing 2012 benchmark assessment \(O\'Brien et al. 2012\). Based on the previous assessment the stock was not overfished, and overfishing was not ocurring. This 2015 assessment updates commercial fishery catch data, research survey indices of abundance, the analytical VPA assessment model, and reference points through 2014. Additionally, stock projections have been updated through 2018.} \def\PLAUNITSoS{ \textbf{State of Stock: }{}Based on this updated assessment, the Gulf of Maine\-Georges Bank American Plaice \(\textit{Hippoglossoides platessoides}\)  stock is not overfished and overfishing is not occurring \(Figures \ref{PLAUNITSSB\_plot1}\-\ref{PLAUNITF\_plot1}\){}.  Retrospective adjustments were made to the model results.  Spawning stock biomass \(SSB\)  in 2014 was estimated to be 10,915 mt which is 83\% of the biomass target for this stock \(\$SSB\_{MSY}\${} \textit{proxy}{} = 13,107\;  Figure \ref{PLAUNITSSB\_plot1}{}\). The 2014 fully selected fishing mortality was estimated to be 0.118 which is 60\% of the overfishing threshold proxy \(\$F\_{MSY}\${} \textit{proxy}{} = 0.196\;  Figure \ref{PLAUNITF\_plot1}{}\).} \def\PLAUNITProj{ \textbf{Projections: }{}Short term projections of biomass were derived by sampling from an empirical cumulative  distribution  function of 34 recruitment estimates from VPA model results. The annual fishery selectivity, maturity ogive, and mean weights at age used in projections are the most recent 5 year averages\;  retrospective adjustments were applied in the projections.} \def\PLAUNITSpecCmt{ \textbf{Special Comments: } \begin{itemize}{} \item{}What are the most important sources of uncertainty in this stock assessment?  Explain, and describe qualitatively how they affect the assessment results \(such as estimates of biomass, F, recruitment, and population projections\).  \linebreak{} \hspace\*{0.5cm} \textit{A source of uncertainty in this assessment are the estimates of historical landings at age, prior to 1984, and the magnitude of  historical discards, prior to 1989. Both of these affect the scale of the biomass and fishing mortality estimates, and influence reference point estimations.}  \item{} Does this assessment model have a retrospective pattern? If so, is the pattern minor, or major? \(A major retrospective pattern occurs when the adjusted SSB or  \$F\_{Full}\${} lies outside of the approximate  joint confidence region for SSB and  \$F\_{Full}\${}\; see  Figure \ref{RhoDecision\_tab}{}\). \linebreak{} \hspace\*{0.5cm} \textit{ The 7\-year Mohn\'s  \textrho{}, relative to SSB, was 0.63 in the 2012 assessment and was 0.32 in 2014. The 7\-year Mohn\'s  \textrho{}, relative to F, was \-0.35 in the 2012 assessment and was 0.32 in 2014. There was a major retrospective pattern for this assessment because the  \textrho{} adjusted estimates of 2014 SSB \(\$SSB\_{\rho}\${}=10,915\)  and 2014 F \(\$F\_{\rho}\${}=0.118\)  were outside the approximate 90\% confidence regions around SSB \(12,742 \- 16,439\)  and F \(0.069 \- 0.093\).  A retrospective  adjustment was made for both the determination of stock status and for projections of catch in 2016. The retrospective adjustment changed the 2014 SSB from 14,543 to 10,915 and the 2014  \$F\_{Full}\${} from 0.08 to 0.118.}  \item{}Based on this stock assessment, are population projections well determined or uncertain? \linebreak{} \hspace\*{0.5cm} \textit{Population projections for Gulf of Maine\-Georges Bank American Plaice are reasonably well determined.}  \item{}Describe any changes that were made to the current stock assessment, beyond incorporating additional years of data  and the effect these changes had on the assessment and stock status. \linebreak{} \hspace\*{0.5cm} \textit{ No major changes, other than the addition of recent years of data, were made to the Gulf of Maine\-Georges Bank American Plaice assessment for this update. A new version of VPA was used \(V3.3.0\)  which gave very similar results to the 2012 VPA 3.1.0 run, with the same F and slightly lower SSB. The MADMF spring and autumn survey indices were re\-estimated for the time series, accounting for revised stratum areas. The revision occurred in 2007, but was overlooked in the 2012 assessment. A comparison of 2010 terminal year VPAs indicated minimal differences in 2010 SSB \(now slightly lower\)  and no change in F.}  \item{}If the stock status has changed a lot since the previous assessment, explain why this occurred.  \linebreak{} \hspace\*{0.5cm} \textit{As in recent assessments for Gulf of Maine\-Georges Bank American Plaice the stock status remains as not overfished and overfishing not occurring.}  \item{}Indicate what data or studies are currently lacking and which would be needed most to improve this stock assessment in the future.  \linebreak{} \hspace\*{0.5cm} \textit{The Gulf of Maine\-Georges Bank American Plaice assessment could be improved with updated studies on growth of Georges Bank and Gulf of Maine fish.}  \item{}Are there other important issues? \linebreak{} \hspace\*{0.5cm} \textit{A difference in growth between GM and GB fish has been documented, however, historical catch data information for GB may not be sufficient to conduct a separate assessment. Also, the growth difference may not persist in the most recent years. This could all be explored further in an benchmark review.} \end{itemize}{}} \def\PLAUNITRefr{ \textbf{References: }{} \linebreak{}O\'Brien, L. and J. Dayton \(2012\). E. Gulf of Maine \- Georges Bank American plaice Assessment for 2012 in Northeast Fisheries Science Center, 2012, Assessment or Data Updates of 13 Northeast Groundfish Stocks through 2010. US Dept Commer, Northeast Fish Sci Cent Ref Doc. 12\-06\; 789 p. http:\/\/www.nefsc.noaa.gov\/publications\/crd\/crd1206\/. \linebreak{} \linebreak{}} \def\PLAUNITDraft{} \def\PLAUNITSPPname{Gulf of Maine-Georges Bank American Plaice} \def\PLAUNITSPPnameT{Gulf of Maine-Georges Bank American Plaice} \def\PLAUNITRptYr{2015} \def\PLAUNITAuthor{Loretta O\'Brien} \def\PLAUNITReviewerComments{/home/dhennen/EIEIO/BigReport/PLA_UNIT/latex}  \def\WITUNITMyPathTab{/home/dhennen/EIEIO/BigReport/WIT_UNIT/tables} \def\WITUNITMyPathFig{/home/dhennen/EIEIO/BigReport/WIT_UNIT/figures} \def\WITUNITfigFishCap{Total catch of witch flounder between 1982 and 2014 by fleet \(commercial\)  and disposition \(landings and discards\).} \def\WITUNITfigSSBCap{Trends in spawning stock biomass \(mt\)  of witch flounder between 1982 and 2014 from the current  \(solid line\)  and previous \(dashed line\)  assessment and the corresponding  \$SSB\_{Threshold}\${} \(\$\dfrac{1}{2}\${} \$SSB\_{MSY}\${}\; horizontal dashed line\)  as well as  \$SSB\_{Target}\${} \$SSB\_{MSY}\${}\; horizontal dotted line\)   based on the current assessment. Red solid vertical line indicates rho adjusted SSB. Black solid vertical line indicates 90\% confidence interval for 2014.} \def\WITUNITfigFCap{Trends in the fully selected fishing mortality \(\$F\_{Full}\${}\)  of witch flounder between 1982 and 2014 from the current  \(solid line\)  and previous \(dashed line\)  assessment and the corresponding  \$F\_{Threshold}\${} \(\$F\_{MSY}\${}=0.279\; horizontal dashed line\)  based on the current assessment.  Red solid vertical line indicates rho adjusted  \$F\_{Full}\${}. Black solid vertical line indicates 90\% confidence interval for 2014.} \def\WITUNITfigRecrCap{Trends in Age 3  \(000s\)  of witch flounder between 1982 and 2014 from the current \(solid line\)  and previous \(dashed line\)  assessment.} \def\WITUNITfigSurvCap{Indices of biomass \(kg\/tow\)  for the witch flounder between 1963 and 2015 for the Northeast Fisheries Science Center \(NEFSC\)  spring and fall bottom trawl surveys.  The 90\% lognormal confidence intervals are shown.} \def\WITUNITPreAmb{This assessment of the witch flounder \(\textit{Glyptocephalus cynoglossus}\)  stock is an operational update of the 2012 assessment \(NEFSC 2012\)  and the 2008 benchmark assessment \(NEFSC 2008\). This assessment updates commercial fishery catch data, research survey indices, and the analytical assessment model through 2014. Additionally, stock projections have been updated through 2018. Reference points have been updated. } \def\WITUNITSoS{ \textbf{State of Stock: }{}witch flounder \(\textit{Glyptocephalus cynoglossus}\)  stock is overfished and overfishing is occurring \(Figures \ref{WITUNITSSB\_plot1}\-\ref{WITUNITF\_plot1}\){}. Retrospective adjustments were made to the model results.  Spawning stock biomass \(SSB\)  in 2014 was estimated to be 2,077 \(mt\)  which is 22\% of the  \$SSB\_{MSY}\${} proxy \(9,473\;  Figure \ref{WITUNITSSB\_plot1}{}\).  The 2014 fully selected fishing mortality was estimated to be 0.687 which is 246\% of the  \$F\_{MSY}\${} proxy \(0.279\;  Figure \ref{WITUNITF\_plot1}{}\). A retrospective adjustment to  \$F\_{Full}\${} and SSB in 2014 was required but did not lead to a change in status.  } \def\WITUNITProj{ \textbf{Projections: }{}Short term projection recruitment was sampled from a cumulative distribution function derived from ADAPT VPA \(with split time series between 1994 and 1995\)  estimated age 3 recruitment between 1982 and 2013.  Average 2010\-2014 partial recruitment, average 2010\-2014 mean weights, and maturation ogive representing 2011\-2015 maturity data were used.} \def\WITUNITSpecCmt{ \textbf{Special Comments: } \begin{itemize}{} \item{}What are the most important sources of uncertainty in this stock assessment?  Explain, and describe qualitatively how they affect the assessment results \(such as estimates of biomass, F, recruitment, and population projections\).  \linebreak{} \hspace\*{0.5cm} \textit{An important source of uncertainty is the retrospective pattern where fishing mortality is underestimated and spawning stock biomass and recruitment are overestimated. }  \item{} Does this assessment model have a retrospective pattern? If so, is the pattern minor, or major? \(A major retrospective pattern occurs when the adjusted SSB or  \$F\_{Full}\${} lies outside of the approximate  joint confidence region for SSB and  \$F\_{Full}\${}\).  \linebreak{} \hspace\*{0.5cm} \textit{ The 7\-year Mohn\'s  \textrho{}, relative to SSB, was 0.61 in the 2012 assessment and was 0.51 in 2014. The 7\-year Mohn\'s  \textrho{}, relative to F, was \-0.33 in the 2012 assessment and was \-0.38 in 2014. There was a major retrospective pattern for this assessment because the  \textrho{} adjusted estimates of 2014 SSB \(\$SSB\_{\rho}\${}=2,077\)  and 2014 F \(\$F\_{\rho}\${}=0.687\)  were outside the approximate 90\% confidence regions around SSB \(2,643 \- 3,864\)  and F \(0.321 \- 0.603\).  A retrospective  adjustment was made for both the determination of stock status and for projections of catch in 2016. The retrospective adjustment changed the 2014 SSB from 3,129 to 2,077 and the 2014  \$F\_{Full}\${} from 0.428 to 0.687.}  \item{}Based on this stock assessment, are population projections well determined or uncertain? \linebreak{} \hspace\*{0.5cm} \textit{Population projections for witch flounder appear to be optimistic\; the projected rho adjusted biomass from the last assessment  was above the upper confidence bounds of the projected rho adjusted biomass estimated in the current assessment. }  \item{}Describe any changes that were made to the current stock assessment, beyond incorporating additional years of data  and the effect these changes had on the assessment and stock status.  \linebreak{} \hspace\*{0.5cm} \textit{TOGA \(Type, Operation, Gear, Acquisition\)  values were used for haul criteria for NEFSC surveys for 2009 onward and minor changes in the use of observer data for discard estimates were made to the current witch flounder assessment. These changes had negligible effect on the assessment and stock status.  }  \item{}If the stock status has changed a lot since the previous assessment, explain why this occurred.  \linebreak{} \hspace\*{0.5cm} \textit{No change in stock status has occurred for witch flounder since the previous assessment. }  \item{}Indicate what data or studies are currently lacking and which would be needed most to improve this stock assessment in the future.  \linebreak{} \hspace\*{0.5cm} \textit{Extensive studies have examined the causes of retrospective patterns with no definitive conclusions other than a change in model does not resolve the issue. }  \item{}Are there other important comments? \linebreak{} \hspace\*{0.5cm} \textit{The VPA analysis was performed with survey time series split between 1994 and 1995. This time split corresponds to changes in the commercial reporting methods as well as other regulatory management changes.  } \end{itemize}{}} \def\WITUNITRefr{ \textbf{References: }{} \linebreak{}Northeast Fisheries Science Center. 2008. Assessment of 19 Northeast Groundfish Stocks through 2007: Report of the 3$^{rd}$ Groundfish Assessment Review Meeting \(GARM III\), Northeast Fisheries Science Center, Woods Hole, Massachusetts, August 4\-8, 2008. US Dep Commer, NOAA Fisheries, Northeast Fish Sci Cent Ref Doc. 08\-15\; 884 p + xvii. http:\/\/www.nefsc.noaa.gov\/publications\/crd\/crd0815\/ \linebreak{} \linebreak{}Northeast Fisheries Science Center. 2012. Assessment or Data Updates of 13 Northeast Groundfish Stocks through 2010.  US Dep Commer, NOAA Fisheries, Northeast Fish Sci Cent Ref Doc. 12\-06\; 789 p. http:\/\/www.nefsc.noaa.gov\/publications\/crd\/crd1206\/ \linebreak{} \linebreak{}} \def\WITUNITDraft{} \def\WITUNITSPPname{witch flounder} \def\WITUNITSPPnameT{Witch flounder} \def\WITUNITRptYr{2015} \def\WITUNITAuthor{Susan Wigley} \def\WITUNITReviewerComments{/home/dhennen/EIEIO/BigReport/WIT_UNIT/latex}  \def\HKWUNITMyPathTab{/home/dhennen/EIEIO/BigReport/HKW_UNIT/tables} \def\HKWUNITMyPathFig{/home/dhennen/EIEIO/BigReport/HKW_UNIT/figures} \def\HKWUNITfigFishCap{Total catch of white hake between 1963 and 2014 by fleet \(commercial, recreational, or Canadian\)  and disposition \(landings and discards\).} \def\HKWUNITfigSSBCap{Trends in spawning stock biomass of white hake between 1963 and 2014 from the current  \(solid line\)  and previous \(dashed line\)  assessment and the corresponding  \$SSB\_{Threshold}\${} \(\$\dfrac{1}{2}\${} \$SSB\_{MSY}\${} \textit{proxy}{}\; horizontal dashed line\)  as well as  \$SSB\_{Target}\${} \(\$SSB\_{MSY}\${} \textit{proxy}{}\; horizontal dotted line\)   based on the 2014 assessment.  The red dot indicates the rho\-adjusted SSB values that would have resulted had a retrospective  adjusment been made \(see Special Comments section\).  The approximate 90\% lognormal confidence intervals are shown.} \def\HKWUNITfigFCap{Trends in the fully selected fishing mortality \(\$F\_{Full}\${}\)  of white hake between 1963 and 2014 from the current  \(solid line\)  and previous \(dashed line\)  assessment and the corresponding  \$F\_{Threshold}\${} \(\$F\_{MSY}\${} \textit{proxy}{}=0.188\; horizontal dashed line\).  The red dot indicates the rho\-adjusted SSB values that would have resulted had a retrospective  adjusment been made \(see Special Comments section\).  The approximate 90\% lognormal confidence intervals are shown.} \def\HKWUNITfigRecrCap{Trends in Recruits \(age 1\)  \(000s\)  of white hake between 1963 and 2014 from the current \(solid line\)  and previous \(dashed line\)  assessment. The approximate 90\% lognormal confidence intervals are shown.} \def\HKWUNITfigSurvCap{Indices of biomass for the white hake between 1963 and 2015 for the Northeast Fisheries Science Center \(NEFSC\)  spring and fall bottom trawl surveys.  The approximate 90\% lognormal confidence intervals are shown.} \def\HKWUNITPreAmb{This assessment of the white hake \(\textit{Urophycis tenuis}\)  stock is an operational update of the existing 2013 benchmark ASAP assessment \(NEFSC 2013\). Based on the previous assessment the stock was not overfished, and overfishing was not ocurring. This assessment updates commercial fishery catch data, research survey indices of abundance, and the ASAP assessment models and reference points through 2014. Additionally, stock projections have been updated through 2018.} \def\HKWUNITSoS{ \textbf{State of Stock: }{}Based on this updated assessment, white hake \(\textit{Urophycis tenuis}\)  stock is not overfished and overfishing is not occurring \(Figures \ref{HKWUNITSSB\_plot1}\-\ref{HKWUNITF\_plot1}\){}. Retrospective adjustments were not made to the model results.  Spawning stock biomass \(SSB\)  in 2014 was estimated to be 28,553 \(mt\)  which is 88\% of the biomass threshold for an overfished stock \(\$SSB\_{MSY}\${} \textit{proxy}{} = 32,550\;  Figure \ref{HKWUNITSSB\_plot1}{}\).  The 2014 fully selected fishing mortality was estimated to be 0.076 which is 40\% of the overfishing threshold proxy \(\$F\_{MSY}\${} \textit{proxy}{} = 0.188\;  Figure \ref{HKWUNITF\_plot1}{}\).} \def\HKWUNITProj{ \textbf{Projections: }{}Short term projections of catch and SSB were derived by sampling from a cumulative  distribution  function of recruitment estimates from ASAP from 1995\-2012. The annual fishery selectivity, maturity ogive, and mean weights at age used in the projection  are the most recent 5 year averages. } \def\HKWUNITSpecCmt{ \textbf{Special Comments: } \begin{itemize}{} \item{}What are the most important sources of uncertainty in this stock assessment?  Explain, and describe qualitatively how they affect the assessment results \(such as estimates of biomass, F, recruitment, and population projections\).  \linebreak{} \hspace\*{0.5cm} \textit{1. Catch at age information is not well characterized due to possible mis\-identification of species in the commercial and sea sampling data, particularly in early years, low sampling of commercial landings in  some years, and sparse discard data particularly in early years.  \linebreak{} \hspace\*{0.5cm}2. Since the commercial catch is aged primarily with survey age\/length keys, there is considerable augmentation required, mainly for ages 5 and older. The numbers at age and mean weights at age in the catch for these ages may therefore not be well specified.  \linebreak{} \hspace\*{0.5cm}3. White hake may move seasonally into and out of the defined stock area.  \linebreak{} \hspace\*{0.5cm}4. There are no commercial catch at age data prior to 1989 and the catchability of older ages in the surveys is very low. This results in a large uncertainty in starting numbers at age.  \linebreak{} \hspace\*{0.5cm}5. Since 2003, dealers have been culling very large fish out of the large category. However, there was no market category to input into the landings until June 2014. The length compositions are distinct from large and have been identified since 2011. This may bias the age composition of the landings, particularly in 2014 when 2000 of the 5000 large samples were these extra\-large fish.  \linebreak{} \hspace\*{0.5cm}6. A pooled age\/length key is used for 1963\-1981, fall 2003 \(second half of commercial key\)  and 2014.Age data were not available for 2014 in time for this assessment. The same pooled key that was used for 1963\-1981 was used for 2014.}  \item{} Does this assessment model have a retrospective pattern? If so, is the pattern minor, or major? \(A major retrospective pattern occurs when the adjusted SSB or  \$F\_{Full}\${} lies outside of the approximate  joint confidence region for SSB and  \$F\_{Full}\${}\; see  Figure \ref{RhoDecision\_tab}{}\). \linebreak{} \hspace\*{0.5cm} \textit{ No retrospective adjustment of spawning stock biomass or fishing mortality in 2014 was required.  The pattern in this assessment is considered minor \(Mohn’s rho of 0.18 on SSB, Mohn’s rho of 0.12 on F\)  with the adjusted SSB within the 90 \% CI of the MCMC. However, the Mohn’s rho for Age 1 estimates is 0.54. This may have an impact on projections if this continues into the future.}  \item{}Based on this stock assessment, are population projections well determined or uncertain? \linebreak{} \hspace\*{0.5cm} \textit{Population projections for white hake, are not well determined and projected biomass from the last assessment  was outside the confidence bounds of the biomass estimated in the current assessment. }  \item{}Describe any changes that were made to the current stock assessment, beyond incorporating additional years of data  and the affect these changes had on the assessment and stock status. \linebreak{} \hspace\*{0.5cm} \textit{ The 2011 catch\-at\-length and age were re\-estimated for both landings and discards. For the  landings, two samples were adjusted for dorsal length to total length that had been missed in the previous assessment.}  \item{}If the stock status has changed a lot since the previous assessment, explain why this occurred.  \linebreak{} \hspace\*{0.5cm} \textit{While stock status of white hake has not changed, the stock has not rebuilt as the projections from the last assessment indicated. This is due to the retrospective in recruitment. The numbers for the 2005\-2009 year classes, which were included in the age 2\-6 starting numbers in the projections, were over\-estimated which led to over\-estimating SSB in 2014.}  \item{}Indicate what data or studies are currently lacking and which would be needed most to improve this stock assessment in the future.  \linebreak{} \hspace\*{0.5cm} \textit{ Age structures from the observer program are available and should be aged to augment  the survey keys. There is a also a new market category for heads and age structures could be  acquired from these is an otolith length\/total length relationship can be established. }  \item{}Are there other important issues? \linebreak{} \hspace\*{0.5cm} \textit{None. } \end{itemize}{}} \def\HKWUNITRefr{ \textbf{References: }{} \linebreak{} NEFSC. 2013. 56$^{th}$ Northeast Regional Stock Assessment Workshop \(56$^{th}$ SAW\)  Assessment  Report.US Dep Commer, NOAA Fisheries, Northeast Fish Sci Cent Ref Doc. 13\-10\; 868 p.  http:\/\/www.nefsc.noaa.gov\/publications\/crd\/crd1310\/  \linebreak{} \linebreak{}} \def\HKWUNITDraft{} \def\HKWUNITSPPname{white hake} \def\HKWUNITSPPnameT{White hake} \def\HKWUNITRptYr{2015} \def\HKWUNITAuthor{Katherine Sosebee} \def\HKWUNITReviewerComments{/home/dhennen/EIEIO/BigReport/HKW_UNIT/latex}  \def\OPTUNITMyPathTab{/home/dhennen/EIEIO/BigReport/OPT_UNIT/tables} \def\OPTUNITMyPathFig{/home/dhennen/EIEIO/BigReport/OPT_UNIT/figures} \def\OPTUNITfigFishCap{Total catch of ocean pout  between 1968 and 2014 by fleet \(US and Other\)  and disposition \(landings and discards\).} \def\OPTUNITfigSSBCap{Trends in biomass \(kg\/tow\)  of ocean pout  between 1968 and 2014 from the current  \(solid line\)  and previous \(dashed line\)  assessment and the corresponding  \$B\_{Threshold}\${} \(\$\dfrac{1}{2}\${} \$B\_{MSY}\${} \textit{proxy}{}\; horizontal dashed line\)  as well as  \$B\_{Target}\${} \(\$B\_{MSY}\${} \textit{proxy}{}\; horizontal dotted line\)   based on the current assessment. } \def\OPTUNITfigFCap{Trends in the exploitation rate of ocean pout between 1968 and 2014 from the current  \(solid line\)  and previous \(dashed line\)  assessment and the corresponding  \$F\_{Threshold}\${} \(\$F\_{MSY}\${} \textit{proxy}{}=0.76\; horizontal dashed line\)   based on the current assessment. } \def\OPTUNITfigRecrCap{} \def\OPTUNITfigSurvCap{Indices of biomass \(kg\/tow\)  for ocean pout  between 1968 and 2015 for the Northeast Fisheries Science Center \(NEFSC\)  spring survey.   The approximate 90\% lognormal confidence intervals are shown.} \def\OPTUNITPreAmb{This assessment of the ocean pout  \(\textit{Zoarces americanus}\)  stock is an operational update of the 2012 assessment \(NEFSC 2012\)  and the 2008 benchmark assessment \(NEFSC 2008\). Based on the 2012 assessment, the stock was overfished but overfishing was not ocurring. This assessment updates commercial fishery catch data, research survey indices and the exploitation ratios through 2014. There are no stock projections.} \def\OPTUNITSoS{ \textbf{State of Stock: }{}Based on the current assessment, the ocean pout  \(\textit{Zoarces americanus}\)  stock is overfished and overfishing is not occurring \(Figures \ref{OPTUNITSSB\_plot1}\-\ref{OPTUNITF\_plot1}\){}. Retrospective adjustments were not made to the model results. Biomass proxy \(B\)  in 2014 was estimated to be 0.29 \(kg\/tow\)  which is 6\% of the biomass target \(\$B\_{MSY}\${} \textit{proxy}{} = 4.94\;  Figure \ref{OPTUNITSSB\_plot1}{}\).  The 2014 fully selected fishing mortality was estimated to be 0.269 which is 35\% of the overfishing threshold proxy \(\$F\_{MSY}\${} \textit{proxy}{} = 0.76\;  Figure \ref{OPTUNITF\_plot1}{}\).} \def\OPTUNITProj{ \textbf{Projections: }{}The index\-based assessment approach does not support catch projections\; catch advice for ocean pout has been based on the target exploitation rate and the most recent centered 3\-year average biomass index from the NEFSC spring survey. } \def\OPTUNITSpecCmt{ \textbf{Special Comments: } \begin{itemize}{} \item{}What are the most important sources of uncertainty in this stock assessment?  Explain, and describe qualitatively how they affect the assessment results \(such as estimates of biomass, F, recruitment, and population projections\).  \linebreak{} \hspace\*{0.5cm} \textit{ An important source of uncertainty is the stock has not responded to low catch as expected. }  \item{}Does this assessment model have a retrospective pattern? If so, is the pattern minor or major?  \(A major retrospective pattern occurs when the adjusted SSB or  \$F\_{Full}\${} lies outside of the approximate  joint confidence region for SSB and  \$F\_{Full}\${}\; see  Figure \ref{RhoDecision\_tab}{}\). \linebreak{} \hspace\*{0.5cm} \textit{ The model used to estimate status of this stock does not allow estimation of a retrospective pattern. }  \item{}Based on this stock assessment, are population projections well determined or uncertain? \linebreak{} \hspace\*{0.5cm} \textit{ N\/A}  \item{}Describe any changes that were made to the current stock assessment, beyond incorporating additional years of data  and the effect these changes had in the assessment and stock status. \linebreak{} \hspace\*{0.5cm} \textit{TOGA \(Type, Operation, Gear, Acquisition\)  values were used for haul criteria for NEFSC surveys for 2009 onward and minor changes in the use of observer data for discard estimates were made to the current assessment. These changes had a negligible effect on the assessment and stock status.   Recreational landings were updated and found to be negligible \(time series average of recreational landings to total catch was less than 1\%\)  and therefore not included in this assessment.}  \item{}If the stock status has changed a lot since the previous assessment, explain why this occurred.  \linebreak{} \hspace\*{0.5cm} \textit{Ocean pout stock status has not changed since the previous assessment.}  \item{}Indicate what data or studies are currently lacking and which would be needed most to improve this stock assessment in the future.  \linebreak{} \hspace\*{0.5cm} \textit{The ocean pout assessment could be improved with studies that explore why this stock is not rebuilding as expected. }  \item{}Are there other important comments? \linebreak{} \hspace\*{0.5cm} \textit{Biological reference points are based on catch\; the estimated discards used in the catch are based on a mix of direct \(1989 onward\)  and indirect \(1988 and back\)  methods. The catch used to determine MSY is based on indirect methods. } \end{itemize}{}} \def\OPTUNITRefr{ \textbf{References: }{} \linebreak{}Northeast Fisheries Science Center. 2012. Assessment or Data Updates of 13 Northeast Groundfish Stocks through 2010.  US Dep Commer, NOAA Fisheries, Northeast Fish Sci Cent Ref Doc. 12\-06\; 789 p. http:\/\/www.nefsc.noaa.gov\/publications\/crd\/crd1206\/ \linebreak{} \linebreak{}Northeast Fisheries Science Center. 2008. Assessment of 19 Northeast Groundfish Stocks through 2007: Report of the 3$^{rd}$ Groundfish Assessment Review Meeting \(GARM III\), Northeast Fisheries Science Center, Woods Hole, Massachusetts, August 4\-8, 2008. US Dep Commer, NOAA Fisheries, Northeast Fish Sci Cent Ref Doc. 08\-15\; 884 p + xvii. http:\/\/www.nefsc.noaa.gov\/publications\/crd\/crd0815\/ \linebreak{} \linebreak{}} \def\OPTUNITDraft{} \def\OPTUNITSPPname{Ocean Pout} \def\OPTUNITSPPnameT{Ocean Pout} \def\OPTUNITRptYr{2015} \def\OPTUNITAuthor{Susan Wigley} \def\OPTUNITReviewerComments{/home/dhennen/EIEIO/BigReport/OPT_UNIT/latex}  