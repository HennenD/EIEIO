 \def\FLWGBMyPathTab{/home/dhennen/EIEIO/BigReport/FLW_GB/tables} \def\FLWGBMyPathFig{/home/dhennen/EIEIO/BigReport/FLW_GB/figures} \def\FLWGBfigFishCap{Total catches (mt)  of Georges Bank winter flounder between 1982 and 2015 by country and disposition (landings and discards).} \def\FLWGBfigSSBCap{Trends in spawning stock biomass (mt)  of Georges Bank winter flounder between 1982 and 2014 from the current  (solid line)  and previous (dashed line)  assessments and the corresponding  $SSB_{Threshold}${} ($\dfrac{1}{2}${} $SSB_{MSY}${}; horizontal dashed line)  as well as  $SSB_{Target}${} ($SSB_{MSY}${}; horizontal dotted line)   based on the 2015 assessment.  Biomass was adjusted for a retrospective pattern  and the adjustment is shown in red.  The approximate 90\percent{} normal confidence intervals are shown.} \def\FLWGBfigFCap{Trends in fully selected fishing mortality ($F_{Full}${})  of Georges Bank winter flounder between 1982 and 2014 from the current  (solid line)  and previous (dashed line)  assessments and the corresponding  $F_{Threshold}${} ($F_{MSY}${}=0.536; horizontal dashed line)  as well as ($F_{Target}${}= 75\percent{} of FMSY;  horizontal dotted line). $F_{Full}${} was adjusted for a retrospective pattern  and the adjustment is shown in red.  The approximate 90\percent{} normal confidence intervals are also shown.} \def\FLWGBfigRecrCap{Trends in Recruits (age 1)  (000s)  of Georges Bank winter flounder between 1982 and 2014 from the current (solid line)  and previous (dashed line)  assessments. The approximate 90\percent{} normal confidence intervals are shown.} \def\FLWGBfigSurvCap{Indices of biomass for the Georges Bank winter flounder for the Northeast Fisheries Science Center (NEFSC)  spring (1968-2015)  and fall (1963-2014)   bottom trawl surveys and the Canadian DFO spring survey (1987-2015).  The approximate 90\percent{} normal confidence intervals are shown.} \def\FLWGBPreAmb{This assessment of the Georges Bank winter flounder (\textit{Pseudopleuronectes americanus})  stock is an operational assessment of the existing 2014 operational VPA assessment which included data for 1982-2013 (Hendrickson et al. 2015). Based on the previous assessment the stock was not overfished and overfishing was not occurring. This assessment updates commercial fishery catch data, research survey biomass indices, and the analytical VPA assessment model and reference points through 2014. Additionally, stock projections have been updated through 2018.} \def\FLWGBSoS{ \textbf{State of Stock: }{}Based on this updated assessment, the Georges Bank winter flounder (\textit{Pseudopleuronectes americanus})  stock is overfished and overfishing is occurring (Figures \ref{FLWGBSSB_plot1}-\ref{FLWGBF_plot1}){}. Retrospective adjustments were made to the model results.  Spawning stock biomass (SSB)  in 2014 was estimated to be 2,883 (mt)  which is 43\percent{} of the biomass target for an overfished stock ($SSB_{MSY}${} = 6,700 with a threshold of 50\percent{} of SSBMSY;  Figure \ref{FLWGBSSB_plot1}{}).  The 2014 fully selected fishing mortality (F)  was estimated to be 0.778 which is 145\percent{} of the overfishing threshold ($F_{MSY}${} = 0.536;  Figure \ref{FLWGBF_plot1}{}).} \def\FLWGBProj{ \textbf{Projections: }{}Short-term projections of biomass were derived by sampling from a cumulative  distribution  function of recruitment estimates (1982-2013 year classes)  from the final run of the ADAPT VPA model. The annual fishery selectivity, maturity ogive, and mean weights-at-age used in the projection  are the most recent 5 year averages (2010-2014). An SSB retrospective adjustment factor of 0.546 was applied in the projections.} \def\FLWGBSpecCmt{ \textbf{Special Comments: } \begin{itemize}{} \item{}What are the most important sources of uncertainty in this stock assessment?  Explain, and describe qualitatively how they affect the assessment results (such as estimates of biomass, F, recruitment, and population projections).  \linebreak{} \hspace*{0.5cm} \textit{The largest source of uncertainty is the estimate of natural mortality based on longevity (max. age = 20 for  this stock),  which is not well studied in Georges Bank winter flounder, and assumed constant over time.  Natural mortality affects the scale of the biomass and fishing mortality  estimates.  Another source of uncertainty includes the underestimation of catches. Discards from the Canadian bottom trawl fleet were not provided by the CA DFO and the precision of the Canadian scallop dredge discard estimates, with only 1-2 trips per month, are uncertain. The lack of age data for the Canadian spring survey catches requires the use of the US spring survey age/length keys despite selectivity differences. In addition, there are no length or age composition data from the Canadian landings or discards of Georges Bank winter flounder.}  \item{} Does this assessment model have a retrospective pattern? If so, is the pattern minor, or major? (A major retrospective pattern occurs when the adjusted SSB or  $F_{Full}${} lies outside of the approximate  joint confidence region for SSB and  $F_{Full}${}; see  Table \ref{RhoDecision_tab}{}). \linebreak{} \hspace*{0.5cm} \textit{ The 7-year Mohn's  \textrho{}, relative to SSB, was 0.26 in the 2014 assessment and was 0.83 in 2014. The 7-year Mohn's  \textrho{}, relative to F, was -0.16 in the 2014 assessment and was -0.51 in 2014. There was a major retrospective pattern for this assessment because the  \textrho{} adjusted estimates of 2014 SSB ($SSB_{\rho}${}=2,883)  and 2014 F ($F_{\rho}${}=0.778)  were outside the approximate 90\percent{} confidence region around SSB (3,783 - 6,767)  and F (0.254 - 0.504).  A retrospective  adjustment was made for both the determination of stock status and for projections of catch in 2016. The retrospective adjustment changed the 2014 SSB from 5,275 to 2,883 and the 2014  $F_{Full}${} from 0.379 to 0.778.}  \item{}Based on this stock assessment, are population projections well determined or uncertain? \linebreak{} \hspace*{0.5cm} \textit{Population projections for Georges Bank winter flounder are reasonably well determined.}  \item{}Describe any changes that were made to the current stock assessment, beyond incorporating additional years of data  and the effect these changes had on the assessment and stock status. \linebreak{} \hspace*{0.5cm} \textit{ The only change made to the Georges Bank winter flounder assessment, other than the incorporation of an additional  year of data, involved fishery selectivity.  During the 2014 assessment update, stock size estimates of age 1 and age 2 fish were not estimable  in the VPA during year t + 1 (CVs near 1.0). When age 2 stock size is not estimated in year t + 1,  the VPA model calculates the stock size of age 1 fish (i.e., recruitment)  in the terminal year by  using the age 1 partial recruitment (PR)  value to derive the F at age 1 in the terminal year. The  age 1 PR value used in the 2014 assessment update was 0.001. However, when this same age 1 PR value  was used in a VPA run for the current assessment update, the low PR value combined with the low age  1 catch in 2014 resulted in an unlikely high stock size estimate for age 1 recruitment in 2014 (i.e.,  41,587,000 fish)  when compared to survey observations of the same cohort (i.e., age 1 in 2014 and age  2 in 2015). In order to obtain a more realistic estimate of age 1 recruitment in 2014, I allowed the  VPA model to estimate age 2 stock size in 2015 (and thereby avoided the use of an age 1 PR  value in the age 1 stock size calculation for 2014)  and used the back-calculated PR values from this  VPA run to derive a new PR-at-age vector which was used in the final 2015 VPA run. Similar to the  2014 assessment update, the final 2015 VPA run did not include the estimation of age 2 stock size  and the new PR-at-age vector was computed using the same methods as in the 2014 assessment.   Full selectivity occurs at age 4. For the 2015 assessment update, fishery selectivity for ages  1-3 was changed from the 2014 assessment values of 0.001, 0.10 and 0.43, respectively, to 0.01,  0.08 and 0.55, respectively. Differences between estimates  of F, SSB and R values from the final  2015 VPA run, with the new PR vector, and a 2015 VPA run that utilized the PR vector from the 2014  assessment are shown in Table G30 (see \href{http://www.nefsc.noaa.gov/saw/sasi/sasi_report_options.php}{SASINF}{}).}  \item{}If the stock status has changed a lot since the previous assessment, explain why this occurred.  \linebreak{} \hspace*{0.5cm} \textit{The overfished and overfishing status of Georges Bank winter flounder has changed in the current assessment update due to a worsening of the retrospective error associated with fishing mortality and SSB.}  \item{}Indicate what data or studies are currently lacking and which would be needed most to improve this stock assessment in the future.  \linebreak{} \hspace*{0.5cm} \textit{The Georges Bank winter flounder assessment could be improved with discard estimates from the Canadian bottom trawl fleet and age data from the Canadian spring bottom trawl surveys.}  \item{}Are there other important issues? \linebreak{} \hspace*{0.5cm} \textit{None. } \end{itemize}{}} \def\FLWGBRefr{ \textbf{References: }{} \linebreak{}Hendrickson L, Nitschke P, Linton B. 2015. 2014 Operational stock assessments for Georges Bank winter flounder, Gulf of Maine winter flounder, and pollock. US Dept Commer, Northeast Fish Sci Cent Ref Doc. 15-01; 228 p. Available from: National Marine Fisheries Service, 166 Water Street, Woods Hole, MA 02543-1026. \href{http://www.nefsc.noaa.gov/publications/crd/crd1501/}{CRD15-01} \linebreak{} \linebreak{}} \def\FLWGBDraft{} \def\FLWGBSPPname{Georges Bank winter flounder} \def\FLWGBSPPnameT{Georges Bank winter flounder} \def\FLWGBRptYr{2015} \def\FLWGBAuthor{Lisa Hendrickson} \def\FLWGBReviewerComments{/home/dhennen/EIEIO/BigReport/FLW_GB/latex}