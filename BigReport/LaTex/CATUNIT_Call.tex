 \def\CATUNITMyPathTab{/home/dhennen/EIEIO/BigReport/CAT_UNIT/tables} \def\CATUNITMyPathFig{/home/dhennen/EIEIO/BigReport/CAT_UNIT/figures} \def\CATUNITfigFishCap{Total catch of Atlantic wolffish between 1968 and 2014 by fleet (commercial and recreational)  and disposition (landings and discards). Note that a no possession limit was put in place in May 2010.} \def\CATUNITfigSSBCap{Trends in spawning stock biomass of Atlantic wolffish between 1968 and 2014 from the current  (solid line)  and previous (dashed line)  assessment and the corresponding  $SSB_{Threshold}${} ($\dfrac{1}{2}${} $SSB_{MSY}${} \textit{proxy}{}; horizontal dashed line)  as well as  $SSB_{Target}${} ($SSB_{MSY}${} \textit{proxy}{}; horizontal dotted line)   based on the current assessment.} \def\CATUNITfigFCap{Trends in the fully selected fishing mortality ($F_{Full}${})  of Atlantic wolffish between 1968 and 2014 from the current  (solid line)  and previous (dashed line)  assessment and the corresponding  $F_{Threshold}${} ($F_{MSY}${} \textit{proxy}{}=0.243; horizontal dashed line). } \def\CATUNITfigRecrCap{Trends in age 1 recruits of Atlantic wolffish between 1968 and 2014 from the current (solid line)  and previous (dashed line)  assessment.} \def\CATUNITfigSurvCap{Indices of biomass for the Atlantic wolffish between 1968 and 2015 for the Northeast Fisheries Science Center (NEFSC)  spring and fall bottom trawl surveys, and the Massachusetts Division of Marine Fisheries (MADMF)  spring bottom trawl survey. The approximate 90\percent{} lognormal confidence intervals are shown. NEFSC indices for 2009-2015 are calibrated using the ocean pout coefficient from Miller et al. (2010).} \def\CATUNITPreAmb{This assessment of the Atlantic wolffish (\textit{Anarhichas lupus})  stock is an operational assessment of the existing 2012 operational assessment (NEFSC 2012). Based on the previous assessment the stock was overfished, but overfishing was not occurring. This assessment updates commercial fishery catch data, research survey indices of abundance, and the analytical assessment models and reference points through 2014.} \def\CATUNITSoS{ \textbf{State of Stock: }{}Based on this updated assessment, the Atlantic wolffish (\textit{Anarhichas lupus})  stock is overfished and overfishing is not occurring (Figures \ref{CATUNITSSB_plot1}-\ref{CATUNITF_plot1}){}. Retrospective adjustments were not made to the model results. Spawning stock biomass (SSB)  in 2014 was estimated to be 638 (mt)  which is 38\percent{} of the biomass target ($SSB_{MSY}${} \textit{proxy}{} = 1,663;  Figure \ref{CATUNITSSB_plot1}{}).  The 2014 fully selected fishing mortality was estimated to be 0.003 which is 1\percent{} of the overfishing threshold proxy ($F_{MSY}${} \textit{proxy}{} = 0.243;  Figure \ref{CATUNITF_plot1}{}).} \def\CATUNITProj{} \def\CATUNITSpecCmt{ \textbf{Special Comments: } \begin{itemize}{} \item{}What are the most important sources of uncertainty in this stock assessment?  Explain, and describe qualitatively how they affect the assessment results (such as estimates of biomass, F, recruitment, and population projections).  \linebreak{} \hspace*{0.5cm} \textit{The primary sources of uncertainty are the use of the ocean pout calibration coefficient (Atlantic wolffish coefficients are unknown),  and the change to a no possession limit in May 2010. The ocean pout calibration coefficient (4.575)  is one of the largest for any species (Miller et al. 2010), and results in lower biomass estimates. The change to a no possession limit places greater importance on discard mortality. Additionally, it is unclear whether the lack of a recruitment index since 2004 is due to an actual decrease in recruitment, or a change in catchability resulting from the increase in liner mesh size associated with the switch to the Bigelow. Other sources of uncertainty were identified in previous Atlantic wolffish assessments (NDPSWG 2009, NEFSC 2012): the surveys may have reached the limit of wolffish detectability due to the decline in abundance; and the lack of commercial length information results in model estimation difficulties for fishery selectivity.}  \item{} Does this assessment model have a retrospective pattern? If so, is the pattern minor, or major? (A major retrospective pattern occurs when the adjusted SSB or  $F_{Full}${} lies outside of the approximate  joint confidence region for SSB and  $F_{Full}${}; see  Table \ref{RhoDecision_tab}{}). \linebreak{} \hspace*{0.5cm} \textit{This assessment has retrospective patterns with Mohn's rho = 0.83 for SSB and -0.36 for F. Confidence intervals are not available because MCMC is not fully developed for the SCALE model. Thus, retrospective adjustments were not done for this assessment.}  \item{}Based on this stock assessment, are population projections well determined or uncertain? \linebreak{} \hspace*{0.5cm} \textit{Population projections for Atlantic wolffish were not done. Due to the uncertainties in the assessment, the Northeast Data Poor Stocks Working Group (NDPSWG 2009)  concluded that stock projections would be unreliable and should not be conducted.}  \item{}Describe any changes that were made to the current stock assessment, beyond incorporating additional years of data  and the affect these changes had on the assessment and stock status. \linebreak{} \hspace*{0.5cm} \textit{Commercial discards for the entire time series were revised assuming 8\percent{} discard mortality based on a recent study by Grant and Hiscock (2014). A sensitivity run with the revised discard estimates was presented to the Peer Review Panel during the 2015 Operational Assessments. This became the accepted run. There was no change in stock status resulting from the adoption of the 8\percent{} discard mortality run. \linebreak{} \hspace*{0.5cm}Recreational landings for the entire time series were revised due to an updated grand mean, and the MRFSS/MRIP calibration for 1981-2003. This had a negligible effect on the assessment, and there was no change in stock status.}  \item{}If the stock status has changed a lot since the previous assessment, explain why this occurred.  \linebreak{} \hspace*{0.5cm} \textit{Stock status has not changed since the previous assessment.}  \item{}Indicate what data or studies are currently lacking and which would be needed most to improve this stock assessment in the future.  \linebreak{} \hspace*{0.5cm} \textit{The Atlantic wolffish maturity study in the Gulf of Maine is ongoing. Increased sample size following the previous assessment allowed the use of a revised knife edge maturity of 50 cm in this assessment. Continued histological sampling over the next several years should allow for the development of a definitive maturity ogive that can be used in the next assessment.}  \item{}Are there other important issues? \linebreak{} \hspace*{0.5cm} \textit{Recruitment at the end of the time series increases toward the initial recruitment estimate (Table 1; Figure 3)  because there is no information in the model to inform these estimates. There is no indication in the data that recruitment has increased recently.  \linebreak{} \hspace*{0.5cm}Approximate 90\percent{} lognormal confidence intervals are not shown in Figures 1-3 because MCMC is not fully developed for the SCALE model.} \end{itemize}{}} \def\CATUNITRefr{ \textbf{References: }{} \linebreak{} \linebreak{}Grant SM, Hiscock W. 2014. Post-capture survival of Atlantic wolffish  (\textit{Anarhichas lupus})  captured by bottom otter trawl: Can live release programs contribute to the recovery of species at risk? Fish Res 151:169-176.  (See \href{http://www.nefsc.noaa.gov/saw/sasi/sasi_report_options.php}{SASINF}{})  \linebreak{} \linebreak{}Miller TJ, Das C, Politis PJ, Miller AS, Lucey SM, Legault CM, Brown RW, Rago PJ. 2010. Estimation of Albatross IV to Henry B. Bigelow calibration factors. US Dep Commer, Northeast Fish Sci Cent Ref Doc. 10-05; 233 p. \href{http://www.nefsc.noaa.gov/publications/crd/crd1005/}{CRD10-05} \linebreak{} \linebreak{}Northeast Fisheries Science Center (NEFSC). 2012. Assessment or Data Updates of 13 Northeast Groundfish Stocks through 2010. US Dep Commer, Northeast Fish Sci Cent Ref Doc. 12-06; 789 p. \href{http://www.nefsc.noaa.gov/publications/crd/crd1206/}{CRD12-06} \linebreak{} \linebreak{}Northeast Data Poor Stocks Working Group (NDPSWG). 2009. The Northeast Data Poor Stocks Working Group Report, December 8-12, 2008 Meeting. Part A. Skate species complex, deep sea red crab, Atlantic wolffish, scup, and black sea bass. US Dept Commer, Northeast Fish Sci Cent Ref Doc. 09-02; 496 p. \href{http://www.nefsc.noaa.gov/publications/crd/crd0902/}{CRD09-02} \linebreak{} \linebreak{}} \def\CATUNITDraft{} \def\CATUNITSPPname{Atlantic wolffish} \def\CATUNITSPPnameT{Atlantic wolffish} \def\CATUNITRptYr{2015} \def\CATUNITAuthor{Charles Adams} \def\CATUNITReviewerComments{/home/dhennen/EIEIO/BigReport/CAT_UNIT/latex}