 \def\YELSNEMAMyPathTab{/home/dhennen/EIEIO/BigReport/YEL_SNEMA/tables} \def\YELSNEMAMyPathFig{/home/dhennen/EIEIO/BigReport/YEL_SNEMA/figures} \def\YELSNEMAfigFishCap{Total catch of Southern New England-Mid Atlantic yellowtail flounder between 1973 and 2014 by fleet (US domestic and foreign catch)  and disposition (landings and discards).} \def\YELSNEMAfigSSBCap{Trends in spawning stock biomass of Southern New England-Mid Atlantic yellowtail flounder between 1973 and 2014 from the current  (solid line)  and previous (dashed line)  assessment and the corresponding  $SSB_{Threshold}${} ($\dfrac{1}{2}${} $SSB_{MSY}${} \textit{proxy}{}; horizontal dashed line)  as well as  $SSB_{Target}${} ($SSB_{MSY}${} \textit{proxy}{}; horizontal dotted line)   based on the 2015 assessment.  Biomass was adjusted for a retrospective pattern  and the adjustment is shown in red.  The approximate 90\percent{} lognormal confidence intervals are shown.} \def\YELSNEMAfigFCap{Trends in the fully selected fishing mortality ($F_{Full}${})  of Southern New England-Mid Atlantic yellowtail flounder between 1973 and 2014 from the current  (solid line)  and previous (dashed line)  assessment and the corresponding  $F_{Threshold}${} ($F_{MSY}${} \textit{proxy}{}=0.35; horizontal dashed line).  $F_{Full}${} was adjusted for a retrospective pattern  and the adjustment is shown in red  based on the 2015 assessment. The approximate 90\percent{} lognormal confidence intervals are shown.} \def\YELSNEMAfigRecrCap{Trends in Recruits (age 1)  (000s)  of Southern New England-Mid Atlantic yellowtail flounder between 1973 and 2014 from the current (solid line)  and previous (dashed line)  assessment. The approximate 90\percent{} lognormal confidence intervals are shown.} \def\YELSNEMAfigSurvCap{Indices of biomass for the Southern New England-Mid Atlantic yellowtail flounder between 1973 and 2015 for the Northeast Fisheries Science Center (NEFSC)  spring, fall and winter bottom trawl surveys.  The approximate 90\percent{} lognormal confidence intervals are shown. Note:  Larval index was also used in this assessment and is available in the supplemental documentation.} \def\YELSNEMAPreAmb{This assessment of the Southern New England-Mid Atlantic yellowtail flounder (\textit{Limanda ferruginea})  stock is an operational assessment of the existing 2012 benchmark ASAP assessment (NEFSC 2012). Based on the previous assessment the stock was not overfished, and overfishing was not occurring. This assessment updates commercial fishery catch data, research survey indices of abundance, weights at age and the analytical ASAP assessment model and reference points through 2014. Additionally, stock projections have been updated through 2018.} \def\YELSNEMASoS{ \textbf{State of Stock: }{}Based on this updated assessment, Southern New England-Mid Atlantic yellowtail flounder (\textit{Limanda ferruginea})  stock is overfished and overfishing is occurring (Figures \ref{YELSNEMASSB_plot1}-\ref{YELSNEMAF_plot1}){}. Retrospective adjustments were not made to the model results. Spawning stock biomass (SSB)  in 2014 was estimated to be 502 (mt)  which is 26\percent{} of the biomass target ($SSB_{MSY}${} \textit{proxy}{} = 1,959;  Figure \ref{YELSNEMASSB_plot1}{}). The 2014 fully selected fishing mortality was estimated to be 1.64 which is 469\percent{} of the overfishing threshold proxy ($F_{MSY}${} \textit{proxy}{} = 0.35;  Figure \ref{YELSNEMAF_plot1}{}).} \def\YELSNEMAProj{ \textbf{Projections: }{}Short term projections of biomass were derived by sampling from a cumulative  distribution function of recruitment estimates from ASAP.  Following the previous and accepted benchmark formulation, recruitment was based on the more recent estimates of the model time series (i.e. corresponding to  year classes 1990 through 2013)  to reflect the low recent pattern in recruitment. The annual fishery selectivity, maturity ogive, and mean weights at age used  in projection  are the most recent 5 year averages;  retrospective adjustments were not applied in the projections.} \def\YELSNEMASpecCmt{ \textbf{Special Comments: } \begin{itemize}{} \item{}What are the most important sources of uncertainty in this stock assessment?  Explain, and describe qualitatively how they affect the assessment results (such as estimates of biomass, F, recruitment, and population projections).  \linebreak{} \hspace*{0.5cm} \textit{The largest source of uncertainty is the emergence of the retrospective pattern in this operational assessment.  This retrospective bias has resulted in the reduction of SSB estimates and caused F estimates to increase  with additional years of data.  Further, the basis for the recruitment assumption used in stock status determination and population forecast   (i.e. the inclusion of historical recruitment values versus contemporary basis of recruitment)   is another source of uncertainty.  Although recent estimated recruitments likely reflect realistic conditions  for the stock, the basis for recruitment selection is not clearly understood.}  \item{} Does this assessment model have a retrospective pattern? If so, is the pattern minor, or major? (A major retrospective pattern occurs when the adjusted SSB or  $F_{Full}${} lies outside of the approximate  joint confidence region for SSB and  $F_{Full}${}; see RhoDecisionTab.ref). \linebreak{} \hspace*{0.5cm} \textit{ The 7-year Mohn's  \textrho{}, relative to SSB, was 0.14 in the 2012 assessment and was 1.06 in 2014. The 7-year Mohn's  \textrho{}, relative to F, was -0.16 in the 2012 assessment and was -0.53 in 2014. There was a major retrospective pattern for this assessment because the  \textrho{} adjusted estimates of 2014 SSB ($SSB_{\rho}${}=243)  and 2014 F ($F_{\rho}${}=3.53)  were outside the approximate 90\percent{} confidence region around SSB (355 - 739)  and F (1.053 - 2.348).  However, a retrospective adjustment was not made for both the determination of stock status and for projections of catch because of the large proportion of infeasible projections (assumed 2015 catch required a fishing mortality rate greater than 5). This implies the retrospective adjustment was too large or the assumed 2015 catch was too high. The review panel decided to use the unadjusted projections as an upper bound for OFL with the strong suggestion that the OFL estimates were too high (meaning the ABC buffer should be larger than normal - see Reviewer Comments below).}  \item{}Based on this stock assessment, are population projections well determined or uncertain? \linebreak{} \hspace*{0.5cm} \textit{Population projections are uncertain with projected biomass from the last assessment above the confidence bounds of the biomass estimate in the current assessment.  Further, the short-term projections which incorporated the  retropective adjustment in initial numbers-at-age were unreliable due to the low percentage of  feasible solutions (33\percent{})  encountered durring the simulation. The feasibility problem in the projections was caused by the retrospective adjustment,  which led to the assumed 2015 projected catch exceeding the  population biomass in several of the iterations. Evaluation of the the estimated January-1 2015 biomass from the few feasible projections indicated that  the assumed 2015 catch was approximately 98\percent{} of the stock biomass.  This suggests that the assumed 2015  catch is not sustainable given the low starting abundance in the forecast. Alternatively, the unadjusted (for retrospective pattern)  projections performed well, but are likely  to result in an overly  optimistic projection of the fishery yield and population biomass.}  \item{}Describe any changes that were made to the current stock assessment, beyond incorporating additional years of data  and the effect these changes had on the assessment and stock status. \linebreak{} \hspace*{0.5cm} \textit{ There were no major changes to the current stock assessment formulation. However, the criterion for determining acceptable tows on the NEFSC surveys were revised during the Bigelow years (i.e. 2009-2011)  and carried forward to ensure consistency between the assessment and deck operations.  The influence of the revised protocol on the survey indices was inconsequential.}  \item{}If the stock status has changed a lot since the previous assessment, explain why this occurred.  \linebreak{} \hspace*{0.5cm} \textit{The overfishing and biomass stock status have changed since the previous assessment due to increased catches relative to the stock biomass and the very low recruitment of young fish, which are contributing very little to the adult biomass.}  \item{}Indicate what data or studies are currently lacking and which would be needed most to improve this stock assessment in the future.  \linebreak{} \hspace*{0.5cm} \textit{The emergence of retrospective bias in this assessment is not clearly understood and may result from a variety of sources.  Future studies should further investigate the source of this retrospective pattern to help improve the underlying diagnostics of the model for providing catch advice for this stock.  Recruitment for Southern New England-Mid Atlantic yellowtail flounder continues to be weak and it is likely that the stock is in a new productivity regime.  Should this pattern of poor recruitment continue into the future, the ability of the stock to recover will be impeded. Therefore, future studies should build on current knowledge to further understand the underlying ecological mechanisms of poor recruitment in the stock as it may relate to the physical environment.}  \item{}Are there other important issues? \linebreak{} \hspace*{0.5cm} \textit{None. } \end{itemize}{}} \def\YELSNEMARefr{ \textbf{References: }{}  \linebreak{} Alade, L,  C. Legault, S. Cadrin.  2008.  In.  Northeast Fisheries Science Center. 2008. Assessment of 19 Northeast Groundfish Stocks  through 2007: Report of the 3$^{rd}$ Groundfish Assessment Review Meeting (GARM III), Northeast Fisheries Science Center, Woods  Hole, Massachusetts, August 4-8, 2008. US Dep Commer, NOAA Fisheries, Northeast Fish Sci Cent Ref Doc. 08-15; 884 p + xvii.  \href{http://www.nefsc.noaa.gov/publications/crd/crd0815/}{CRD08-15}  \linebreak{}  \linebreak{} Northeast Fisheries Science Center. 2012.  54$^{th}$ Northeast Regional Stock Assessment Workshop (54$^{th}$ SAW)  Assessment Report. US Dept Commer, NOAA Fisheries, Northeast Fish Sci Cent Ref Doc. 12-18.; 600 p.  \href{http://nefsc.noaa.gov/publications/crd/crd1218/}{CRD12-18}  \linebreak{}  \linebreak{}} \def\YELSNEMADraft{} \def\YELSNEMASPPname{Southern New England-Mid Atlantic yellowtail flounder} \def\YELSNEMASPPnameT{Southern New England-Mid Atlantic yellowtail flounder} \def\YELSNEMARptYr{2015} \def\YELSNEMAAuthor{Larry Alade} \def\YELSNEMAReviewerComments{/home/dhennen/EIEIO/BigReport/YEL_SNEMA/latex}