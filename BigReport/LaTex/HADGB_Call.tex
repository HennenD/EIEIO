 \def\HADGBMyPathTab{/home/dhennen/EIEIO/BigReport/HAD_GB/tables} \def\HADGBMyPathFig{/home/dhennen/EIEIO/BigReport/HAD_GB/figures} \def\HADGBfigFishCap{Total catch of Georges Bank haddock between 1931 and 2014 by fleet (US Commercial, Canadian, or foreign fleet)  and disposition (landings and discards).} \def\HADGBfigSSBCap{Trends in spawning stock biomass of Georges Bank haddock between 1931 and 2014 from the current  (solid line)  and previous (dashed line)  assessment and the corresponding  $SSB_{Threshold}${} ($\dfrac{1}{2}${} $SSB_{MSY}${} \textit{proxy}{}; horizontal dashed line)  as well as  $SSB_{Target}${} ($SSB_{MSY}${} \textit{proxy}{}; horizontal dotted line)   based on the 2015 assessment.  Biomass was adjusted for a retrospective pattern  and the adjustment is shown in red.   The 90\percent{} bootstrap probability intervals are shown.} \def\HADGBfigFCap{Trends in the fully selected fishing mortality ($F_{Full}${})  of Georges Bank haddock between 1931 and 2014 from the current  (solid line)  and previous (dashed line)  assessment and the corresponding  $F_{Threshold}${} ($F_{MSY}${} \textit{proxy}{}=0.39; horizontal dashed line)  based on the 2015 assessment.   $F_{Full}${} was adjusted for a retrospective pattern  and the adjustment is shown in red.   The 90\percent{} bootstrap probability intervals are shown.} \def\HADGBfigRecrCap{Trends in Recruits (age 1)  (000s)  of Georges Bank haddock between 1931 and 2014 from the current (solid line)  and previous (dashed line)  assessment.  The 90\percent{} bootstrap probability intervals are shown.} \def\HADGBfigSurvCap{Indices of biomass (Mean kg/tow)  for the Georges Bank haddock stock between 1963 and 2015 for the Northeast Fisheries Science Center (NEFSC)  spring and fall bottom trawl surveys and the DFO winter bottom trawl survey.  The approximate 90\percent{} lognormal confidence intervals are shown.} \def\HADGBPreAmb{This assessment of the Georges Bank haddock (\textit{Melanogrammus aeglefinus})  stock is an operational assessment of the existing 2012 update VPA assessment (Brooks et al., 2012).  The last benchmark for this stock was in 2008 (Brooks et al., 2008).  Based on the previous assessment in 2012, the stock was not overfished, and overfishing was not occurring. This assessment updates commercial fishery catch data, research survey indices of abundance, weights and maturity at age, and the analytical VPA assessment model and reference points through 2014. Additionally, stock projections have been updated through 2018.} \def\HADGBSoS{ \textbf{State of Stock: }{}Based on this updated assessment, the Georges Bank haddock (\textit{Melanogrammus aeglefinus})  stock is not overfished and overfishing is not occurring (Figures \ref{HADGBSSB_plot1}-\ref{HADGBF_plot1}){}.  Retrospective adjustments were made to the model results.  Spawning stock biomass (SSB)  in 2014 was estimated to be 150,053 (mt)  which is 139\percent{} of the biomass target ($SSB_{MSY}${} \textit{proxy}{} = 108,300;  Figure \ref{HADGBSSB_plot1}{}).  The 2014 fully selected fishing mortality was estimated to be 0.241 which is 62\percent{} of the overfishing threshold proxy ($F_{MSY}${} \textit{proxy}{} = 0.39;  Figure \ref{HADGBF_plot1}{}).} \def\HADGBProj{ \textbf{Projections: }{}Short term projections of biomass were derived by sampling from a cumulative  distribution  function of recruitment estimates from ADAPT VPA (corresponding to SSB$>$75,000 mt and dropping the extremely large 1963, 2003, and 2010 year classes, as well as the two final year class estimates for 2013 and 2014). The annual fishery selectivity, maturity ogive, and mean weights at age used in this projection  are the most recent 5 year averages;  retrospective adjustments were applied to the starting numbers at age (2015)  in the projections.} \def\HADGBSpecCmt{ \textbf{Special Comments: } \begin{itemize}{} \item{}What are the most important sources of uncertainty in this stock assessment?  Explain, and describe qualitatively how they affect the assessment results (such as estimates of biomass, F, recruitment, and population projections).  \linebreak{} \hspace*{0.5cm} \textit{The largest source of uncertainty is the estimate of 2013 recruitment, which accounts for a substantial portion of catch and SSB in projections.  The rho adjusted projections reduce all starting   numbers at age to 67\percent{} of unadjusted values (i.e., all 2015 numbers at age are multiplied by 0.667).  Two other exceptionally large year classes were observed in 2003 and 2010.  The 2003 year class is now estimated to be only 28\percent{} of its initial model estimate, while the 2010 year class is now estimated to be 63\percent{} of its initial estimate.  Given that only 5 years of data are available to estimate the 2010 year class, it is possible that there may be further revisions to the magnitude of this year class estimate with more years of data.  Therefore, it remains uncertain if the scalar applied to all age classes in these projections (0.667, based on Mohn's rho for SSB)  is sufficient to account for future revisions to the 2013 year class estimate.  In addition, the median recruitment in the projections (the proxy for recruitment at MSY)  is 53.4 million, which is greater than 7 of the last 10 recruitments even though SSB is above the SSBMSY proxy (Table 1). While projections of catch and SSB in the near-term are mostly driven by the 2013 year class, it is worth noting the magnitude of median projected recruitment relative to recent recruitment observations.}  \item{} Does this assessment model have a retrospective pattern? If so, is the pattern minor, or major? (A major retrospective pattern occurs when the adjusted SSB or  $F_{Full}${} lies outside of the approximate  joint confidence region for SSB and  $F_{Full}${}; see  Table \ref{RhoDecision_tab}{}). \linebreak{} \hspace*{0.5cm} \textit{ The 7-year Mohn's  \textrho{}, relative to SSB, was 0.20 in the 2012 assessment and was 0.50 in 2014. The 7-year Mohn's  \textrho{}, relative to F, was -0.15 in the 2012 assessment and was -0.34 in 2014. There was a major retrospective pattern for this assessment because the  \textrho{} adjusted estimates of 2014 SSB ($SSB_{\rho}${}=150,053)  and 2014 F ($F_{\rho}${}=0.241)  were outside the approximate 90\percent{} confidence region around SSB (171,911 - 301,282)  and F (0.13 - 0.203).  A retrospective  adjustment was made for both the determination of stock status and for projections of catch in 2016. The retrospective adjustment changed the 2014 SSB from 225,080 to 150,053 and the 2014  $F_{Full}${} from 0.159 to 0.241.}  \item{}Based on this stock assessment, are population projections well determined or uncertain? \linebreak{} \hspace*{0.5cm} \textit{As noted above, population projections for Georges Bank haddock are uncertain due to uncertainty about the size of  the 2013 year class.  Two sensitivity projections were conducted.  The first sensitivity used biological parameters and fishery selectivity values from the 2010 year class for the 2013 year class.  A second sensitivity projection was made that used the same  biological and selectivity parameters as the first sensitivity, and in addition it  doubled the rho-adjustment on the 2013 year class (age 2 at the start of 2015)  by multiplying it by 0.33.  These sensitivity runs are available on the Stock Assessment Supplementary Information  website (\href{http://www.nefsc.noaa.gov/saw/sasi/sasi_report_options.php}{SASINF}{}), in the sensitivity slides appended to the end of the background presentation.}  \item{}Describe any changes that were made to the current stock assessment, beyond incorporating additional years of data  and the effect these changes had on the assessment and stock status. \linebreak{} \hspace*{0.5cm} \textit{ No changes, other than the incorporation of new data, were made to the Georges Bank haddock assessment for this update. However, the criterion for determining acceptable tows on NEFSC surveys used the TOGA protocol rather than the SHG protocol  (TOGA=132x).}  \item{}If the stock status has changed a lot since the previous assessment, explain why this occurred.  \linebreak{} \hspace*{0.5cm} \textit{The stock status of Georges Bank haddock has not changed.}  \item{}Indicate what data or studies are currently lacking and which would be needed most to improve this stock assessment in the future.  \linebreak{} \hspace*{0.5cm} \textit{Projection advice and reference points for  Georges Bank haddock are strongly dependent on recruitment.  A decade ago, extremely large year classes were considered anomalies (e.g., 1963 and 2003).   However, since 2003, there have been two more extremely large (2010 and 2013)  and one very large (2012)  year classes.  Future work could focus on recruitment forecasting and providing robust catch advice.}  \item{}Are there other important issues? \linebreak{} \hspace*{0.5cm} \textit{The Georges Bank haddock assessment has recently developed a major retrospective pattern.  This stock assessment has historically performed  very consistently.  This should continue to be monitored.  Density-dependent responses in growth should also continue to be monitored.  The switch from SHG to TOGA was ruled out as the cause of the retrospective pattern.} \end{itemize}{}} \def\HADGBRefr{ \textbf{References: }{} \linebreak{}Brooks, E.N, M.L. Traver, S.J. Sutherland, L. Van Eeckhaute, and L. Col.  2008.  In.  Northeast Fisheries Science Center. 2008. Assessment of 19 Northeast Groundfish Stocks through 2007: Report of the 3$^{rd}$ Groundfish Assessment Review Meeting (GARM III), Northeast Fisheries Science Center, Woods Hole, Massachusetts, August 4-8, 2008. US Dep Commer, NOAA Fisheries, Northeast Fish Sci Cent Ref Doc. 08-15; 884 p + xvii. \href{http://www.nefsc.noaa.gov/publications/crd/crd0815/}{CRD08-15} \linebreak{} \linebreak{}Brooks, E.N, S.J. Sutherland, L. Van Eeckhaute, and M. Palmer.  2012.  In.  Northeast Fisheries Science Center. 2012. Assessment or Data Updates of 13 Northeast Groundfish Stocks through 2010. US Dept Commer, NOAA Fisheries, Northeast Fish Sci Cent Ref Doc. 12-06.; 789 p. \href{http://nefsc.noaa.gov/publications/crd/crd1206/}{CRD12-06} \linebreak{} \linebreak{}} \def\HADGBDraft{} \def\HADGBSPPname{Georges Bank haddock} \def\HADGBSPPnameT{Georges Bank haddock} \def\HADGBRptYr{2015} \def\HADGBAuthor{Liz Brooks} \def\HADGBReviewerComments{/home/dhennen/EIEIO/BigReport/HAD_GB/latex}