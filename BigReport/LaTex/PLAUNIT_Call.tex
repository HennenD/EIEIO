 \def\PLAUNITMyPathTab{/home/dhennen/EIEIO/BigReport/PLA_UNIT/tables} \def\PLAUNITMyPathFig{/home/dhennen/EIEIO/BigReport/PLA_UNIT/figures} \def\PLAUNITfigFishCap{Total catch of Gulf of Maine-Georges Bank American plaice between 1980 and 2015 by fleet (Gulf of Maine, Georges Bank, Southern New England, and Canadian)  and disposition (landings and discards).} \def\PLAUNITfigSSBCap{Trends in spawning stock biomass of Gulf of Maine-Georges Bank American plaice between 1980 and 2015 from the current  (solid line)  and previous (dashed line)  assessment and the corresponding  $SSB_{Threshold}${} ($\dfrac{1}{2}${} $SSB_{MSY}${} \textit{proxy}{}; horizontal dashed line)  as well as  $SSB_{Target}${} ($SSB_{MSY}${} \textit{proxy}{}; horizontal dotted line)   based on the 2015 assessment.  Biomass was adjusted for a retrospective pattern  and the adjustment is shown in red.  The approximate 90\percent{} normal confidence intervals are shown.} \def\PLAUNITfigFCap{Trends in the fully selected fishing mortality ($F_{Full}${})  of Gulf of Maine-Georges Bank American plaice between 1980 and 2015 from the current  (solid line)  and previous (dashed line)  assessment and the corresponding  $F_{Threshold}${} ($F_{MSY}${} \textit{proxy}{}=0.196; horizontal dashed line).  $F_{Full}${} was adjusted for a retrospective pattern  and the adjustment is shown in red,  based on the 2015 assessment. The approximate 90\percent{} normal confidence intervals are shown.} \def\PLAUNITfigRecrCap{Trends in Recruits (age 1)  (000s)  of Gulf of Maine-Georges Bank American plaice between 1980 and 2015 from the current (solid line)  and previous (dashed line)  assessment.} \def\PLAUNITfigSurvCap{Indices of biomass for the Gulf of Maine-Georges Bank American plaice between 1963 and 2015 for the Northeast Fisheries Science Center (NEFSC)  and Massachusetts Division of Marine Fisheries (MADMF)  spring and autumn research bottom trawl surveys.  The approximate 90\percent{} normal confidence intervals are shown.} \def\PLAUNITPreAmb{This assessment of the Gulf of Maine-Georges Bank American plaice (\textit{Hippoglossoides platessoides})  stock is an operational assessment of the existing 2012 benchmark assessment (O'Brien et al. 2012). Based on the previous assessment the stock was not overfished, and overfishing was not occurring. This 2015 assessment updates commercial fishery catch data, research survey indices of abundance, the analytical VPA assessment model, and reference points through 2014. Additionally, stock projections have been updated through 2018.} \def\PLAUNITSoS{ \textbf{State of Stock: }{}Based on this updated assessment, the Gulf of Maine-Georges Bank American plaice (\textit{Hippoglossoides platessoides})  stock is not overfished and overfishing is not occurring (Figures \ref{PLAUNITSSB_plot1}-\ref{PLAUNITF_plot1}){}.  Retrospective adjustments were made to the model results.  Spawning stock biomass (SSB)  in 2014 was estimated to be 10,977 mt which is 84\percent{} of the biomass target for this stock ($SSB_{MSY}${} \textit{proxy}{} = 13,107;  Figure \ref{PLAUNITSSB_plot1}{}). The 2014 fully selected fishing mortality was estimated to be 0.116 which is 59\percent{} of the overfishing threshold proxy ($F_{MSY}${} \textit{proxy}{} = 0.196;  Figure \ref{PLAUNITF_plot1}{}).} \def\PLAUNITProj{ \textbf{Projections: }{}Short term projections of biomass were derived by sampling from an empirical cumulative  distribution  function of 34 recruitment estimates from VPA model results. The annual fishery selectivity, maturity ogive, and mean weights at age used in projections are the most recent 5 year averages;  retrospective adjustments were applied in the projections.} \def\PLAUNITSpecCmt{ \textbf{Special Comments: } \begin{itemize}{} \item{}What are the most important sources of uncertainty in this stock assessment?  Explain, and describe qualitatively how they affect the assessment results (such as estimates of biomass, F, recruitment, and population projections).  \linebreak{} \hspace*{0.5cm} \textit{Sources of uncertainty in this assessment are the estimates of historical landings at age, prior to 1984, and the magnitude of  historical discards, prior to 1989. Both of these affect the scale of the biomass and fishing mortality estimates, and influence reference point estimations.}  \item{} Does this assessment model have a retrospective pattern? If so, is the pattern minor, or major? (A major retrospective pattern occurs when the adjusted SSB or  $F_{Full}${} lies outside of the approximate  joint confidence region for SSB and  $F_{Full}${}; see  Table \ref{RhoDecision_tab}{}). \linebreak{} \hspace*{0.5cm} \textit{ The 7-year Mohn's  \textrho{}, relative to SSB, was 0.63 in the 2012 assessment and was 0.325 in 2014. The 7-year Mohn's  \textrho{}, relative to F, was -0.35 in the 2012 assessment and was -0.324 in 2014. There was a major retrospective pattern for this assessment because the  \textrho{} adjusted estimates of 2014 SSB ($SSB_{\rho}${}=10,977)  and 2014 F ($F_{\rho}${}=0.116)  were outside the approximate 90\percent{} confidence region around SSB (12,742 - 16,439)  and F (0.069 - 0.093).  A retrospective  adjustment was made for both the determination of stock status and for projections of catch in 2016. The retrospective adjustment changed the 2014 SSB from 14,543 to 10,977 and the 2014  $F_{Full}${} from 0.08 to 0.116.}  \item{}Based on this stock assessment, are population projections well determined or uncertain? \linebreak{} \hspace*{0.5cm} \textit{Population projections for Gulf of Maine-Georges Bank American plaice are reasonably well determined.}  \item{}Describe any changes that were made to the current stock assessment, beyond incorporating additional years of data  and the effect these changes had on the assessment and stock status. \linebreak{} \hspace*{0.5cm} \textit{ No major changes, other than the addition of recent years of data, were made to the Gulf of Maine-Georges Bank American plaice assessment for this update. A new version of VPA was used (V3.3.0)  which gave very similar results to the 2012 VPA 3.1.0 run, with the same F and slightly lower SSB. The MADMF spring and autumn survey indices were re-estimated for the time series, accounting for revised stratum areas. The revision occurred in 2007, but was overlooked in the 2012 assessment. A comparison of 2010 terminal year VPAs indicated minimal differences in 2010 SSB (now slightly lower)  and no change in F.}  \item{}If the stock status has changed a lot since the previous assessment, explain why this occurred.  \linebreak{} \hspace*{0.5cm} \textit{As in recent assessments for Gulf of Maine-Georges Bank American plaice the stock status remains not overfished and overfishing is not occurring.}  \item{}Indicate what data or studies are currently lacking and which would be needed most to improve this stock assessment in the future.  \linebreak{} \hspace*{0.5cm} \textit{The Gulf of Maine-Georges Bank American plaice assessment could be improved with updated studies on growth of Georges Bank and Gulf of Maine fish.}  \item{}Are there other important issues? \linebreak{} \hspace*{0.5cm} \textit{A difference in growth between GM and GB fish has been documented; however, historical catch data for GB may not be sufficient to conduct a separate assessment. Also, the growth difference may not persist in the most recent years. This could all be explored further in a benchmark review.} \end{itemize}{}} \def\PLAUNITRefr{ \textbf{References: }{} \linebreak{}O'Brien, L. and J. Dayton (2012). E. Gulf of Maine - Georges Bank American plaice Assessment  for 2012 in Assessment or Data Updates of 13 Northeast Groundfish Stocks through 2010.  US Dep Commer, NOAA Fisheries, Northeast Fish Sci Cent Ref Doc. 12-06; 789 p. \href{http://www.nefsc.noaa.gov/publications/crd/crd1206/}{CRD12-06} \linebreak{} \linebreak{}} \def\PLAUNITDraft{} \def\PLAUNITSPPname{Gulf of Maine-Georges Bank American plaice} \def\PLAUNITSPPnameT{Gulf of Maine-Georges Bank American plaice} \def\PLAUNITRptYr{2015} \def\PLAUNITAuthor{Loretta O'Brien} \def\PLAUNITReviewerComments{/home/dhennen/EIEIO/BigReport/PLA_UNIT/latex}