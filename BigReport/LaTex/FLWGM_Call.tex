 \def\FLWGMMyPathTab{/home/dhennen/EIEIO/BigReport/FLW_GM/tables} \def\FLWGMMyPathFig{/home/dhennen/EIEIO/BigReport/FLW_GM/figures} \def\FLWGMfigFishCap{Total catch of Gulf of Maine winter flounder between 2009 and 2014 by fleet (commercial and recreational)  and disposition (landings and discards). A 15\percent{} mortality rate is assumed on recreational discards and a 50\percent{} mortality rate on commercial discards.} \def\FLWGMfigSSBCap{Trends in 30+ cm area-swept biomass of Gulf of Maine winter flounder between 2009 and 2014 from the current assessment based on the fall (MENH, MDMF, NEFSC)  surveys.  The approximate 90\percent{} lognormal confidence intervals are shown.} \def\FLWGMfigFCap{Trends in the exploitation rates ($E_{Full}${})  of Gulf of Maine winter flounder between 2009 and 2014 from the current assessment and the corresponding  $F_{Threshold}${} ($E_{MSY}${} \textit{proxy}{}=0.23; horizontal dashed line).  The approximate 90\percent{} lognormal confidence intervals are shown.} \def\FLWGMfigRecrCap{} \def\FLWGMfigSurvCap{Indices of biomass for the Gulf of Maine winter flounder between 1978 and 2015 for the Northeast Fisheries Science Center (NEFSC), Massachusetts Division of Marine Fisheries (MDMF), and the Maine New Hampshire (MENH)  spring and fall bottom trawl surveys. NEFSC indices are calculated with gear and vessel conversion factors where appropriate.  The approximate 90\percent{} lognormal confidence intervals are shown.} \def\FLWGMPreAmb{This assessment of the  Gulf of Maine winter flounder  (\textit{Pseudopleuronectes americanus})   stock is an operational assessment of the  existing  2014  operational assessment area-swept assessment (NEFSC 2014).  Based on the previous assessment the biomass status is unknown but overfishing was not occurring.  This assessment  updates commercial and recreational fishery catch data, research survey indices  of abundance, and the area-swept estimates of 30+ cm biomass based on the fall NEFSC, MDMF, and MENH surveys.} \def\FLWGMSoS{ \textbf{State of Stock: }{}Based on this updated assessment, the Gulf of Maine winter flounder (\textit{Pseudopleuronectes americanus})  stock biomass status is unknown and overfishing is not occurring (Figures \ref{FLWGMSSB_plot1}-\ref{FLWGMF_plot1}){}. Retrospective adjustments were not made to the model results.  Biomass  (30+ cm mt)  in 2014 was estimated to be 4,655 mt (Figure \ref{FLWGMSSB_plot1}{}). The 2014 30+ cm exploitation rate was estimated to be 0.06 which is 26\percent{} of the overfishing exploitation threshold proxy ($E_{MSY}${} \textit{proxy}{} = 0.23;  Figure \ref{FLWGMF_plot1}{}).} \def\FLWGMProj{ \textbf{Projections: }{}Projections are not possible with area-swept based assessments. Catch advice was based on 75\percent{} of  $E_{40\percent{}}${}(75\percent{} $E_{MSY}${} \textit{proxy}{})  using the fall area-swept estimate assuming q=0.6 on the wing spread. Updated 2014 fall 30+ cm area-swept biomass (4,655 mt)  implies an OFL of 1,080 mt based on the  $E_{MSY}${} \textit{proxy}{} and a catch of 810 mt for 75\percent{} of the  $E_{MSY}${} \textit{proxy}{}.} \def\FLWGMSpecCmt{ \textbf{Special Comments: } \begin{itemize}{} \item{}What are the most important sources of uncertainty in this stock assessment?  Explain, and describe qualitatively how they affect the assessment results (such as estimates of biomass, F, recruitment, and population projections).  \linebreak{} \hspace*{0.5cm} \textit{The largest source of uncertainty with the direct estimates of stock biomass from survey area-swept estimates originates from the assumption of survey gear catchability (q). Biomass and exploitation rate estimates are sensitive to the survey q assumption (0.6 on wing spread). The 2014 empirical benchmark assessement of Georges Bank yellowtail flounder based the area-swept q assumption on an average value taken from the literature for west coast flatfish (0.37 on door spread). The yellowtail q assumption corresponds to a value close to 1 on the wing spread which would result in a lower estimate of biomass (2,995 mt). Another major source of uncertainty with this method is that biomass based reference points cannot be determined and overfished status is unknown. }  \item{} Does this assessment model have a retrospective pattern? If so, is the pattern minor, or major? (A major retrospective pattern occurs when the adjusted SSB or  $F_{Full}${} lies outside of the approximate  joint confidence region for SSB and  $F_{Full}${}; see  Table \ref{RhoDecision_tab}{}). \linebreak{} \hspace*{0.5cm} \textit{ The model used to determine status of this stock does not allow estimation of a retrospective pattern.  An analytical stock assessment model does not exist for Gulf of Maine winter flounder.  An analytical model was no longer used for stock status determination at SARC 52 (2011)  due to concerns with a strong retrospective pattern.  Models have difficulty with the apparent lack of a relationship between a large decrease in the catch with little change in the indices and age and/or size structure over time. }  \item{}Based on this stock assessment, are population projections well determined or uncertain? \linebreak{} \hspace*{0.5cm} \textit{Population projections for Gulf of Maine winter flounder do not exist for area-swept assessments. Catch advice from area-swept estimates tend to vary with interannual variability in the surveys.}  \item{}Describe any changes that were made to the current stock assessment, beyond incorporating additional years of data  and the effect these changes had on the assessment and stock status. \linebreak{} \hspace*{0.5cm} \textit{ No changes, other than the incorporation of new data, were made to the Gulf of Maine winter flounder assessment for this update. However, stabilizing the catch advice may be desired and could be obtained through the averaging of the area-swept fall and spring survey estimates.}  \item{}If the stock status has changed a lot since the previous assessment, explain why this occurred.  \linebreak{} \hspace*{0.5cm} \textit{The overfishing status of Gulf of Maine winter flounder has not changed. }  \item{}Indicate what data or studies are currently lacking and which would be needed most to improve this stock assessment in the future.  \linebreak{} \hspace*{0.5cm} \textit{Direct area-swept assessment could be improved with additional studies on survey gear efficiency.  Quantifying the degree of herding between the doors and escapement under the footrope and/or above the headrope for each survey is needed since area-swept biomass estimates and catch advice are sensitive to the assumed catchability.}  \item{}Are there other important issues? \linebreak{} \hspace*{0.5cm} \textit{The general lack of a response in survey indices and age/size structure are the primary sources of concern with catches remaining far below the overfishing level. } \end{itemize}{}} \def\FLWGMRefr{ \textbf{References: }{} \linebreak{}Hendrickson L, Nitschke P, Linton B. 2015. 2014 Operational stock assessments for Georges Bank winter flounder, Gulf of Maine winter flounder, and pollock. US Dept Commer, Northeast Fish Sci Cent Ref Doc. 15-01; 228 p. Available from: National Marine Fisheries Service, 166 Water Street, Woods Hole, MA 02543-1026. \href{http://www.nefsc.noaa.gov/publications/crd/crd1501/}{CRD15-01} \linebreak{} \linebreak{}Northeast Fisheries Science Center. 2011. 52$^{nd}$ Northeast Regional Stock Assessment Workshop (52$^{nd}$ SAW)  Assessment Report. US Dept Commer, Northeast Fish Sci Cent Ref Doc. 11-17; 962 p. Available from: National Marine Fisheries Service, 166  Water Street, Woods Hole, MA 02543-1026. \href{http://www.nefsc.noaa.gov/saw/saw52/crd1117.pdf}{CRD11-17} \linebreak{} \linebreak{}} \def\FLWGMDraft{} \def\FLWGMSPPname{Gulf of Maine winter flounder} \def\FLWGMSPPnameT{Gulf of Maine winter flounder} \def\FLWGMRptYr{2015} \def\FLWGMAuthor{Paul Nitschke} \def\FLWGMReviewerComments{/home/dhennen/EIEIO/BigReport/FLW_GM/latex}