 \def\FLWSNEMAMyPathTab{/home/dhennen/EIEIO/BigReport/FLW_SNEMA/tables} \def\FLWSNEMAMyPathFig{/home/dhennen/EIEIO/BigReport/FLW_SNEMA/figures} \def\FLWSNEMAfigFishCap{Total catch of Southern New England Mid-Atlantic winter flounder between 1981 and 2014 by fleet (commercial, recreational)  and disposition (landings and discards).} \def\FLWSNEMAfigSSBCap{Trends in spawning stock biomass of Southern New England Mid-Atlantic winter flounder between 1981 and 2014 from the current  (solid line)  and previous (dashed line)  assessment and the corresponding  $SSB_{Threshold}${} ($\dfrac{1}{2}${} $SSB_{MSY}${} \textit{proxy}{}; horizontal dashed line)  as well as  $SSB_{Target}${} ($SSB_{MSY}${} \textit{proxy}{}; horizontal dotted line)   based on the 2015 assessment. The approximate 90\percent{} lognormal confidence intervals are shown.} \def\FLWSNEMAfigFCap{Trends in the fully selected fishing mortality ($F_{Full}${})  of Southern New England Mid-Atlantic winter flounder between 1981 and 2014 from the current  (solid line)  and previous (dashed line)  assessment and the corresponding  $F_{Threshold}${} ($F_{MSY}${}=0.325; horizontal dashed line)   based on the 2015 assessment. The approximate 90\percent{} lognormal confidence intervals are shown.} \def\FLWSNEMAfigRecrCap{Trends in Recruits (age 1)  (000s)  of Southern New England Mid-Atlantic winter flounder between 1981 and 2014 from the current (solid line)  and previous (dashed line)  assessment. The approximate 90\percent{} lognormal confidence intervals are shown.} \def\FLWSNEMAfigSurvCap{Indices of biomass for the Southern New England Mid-Atlantic winter flounder between 1963 and 2014 for the Northeast Fisheries Science Center (NEFSC)  spring and fall bottom trawl surveys, the MADMF spring survey, and the CT LISTS survey  The approximate 90\percent{} lognormal confidence intervals are shown.} \def\FLWSNEMAPreAmb{This assessment of the Southern New England Mid-Atlantic winter flounder (\textit{Pseudopleuronectes americanus})  stock is an operational assessment of the existing 2011 benchmark ASAP assessment (NEFSC 2011). Based on the previous assessment the stock was overfished, but overfishing was not occurring. This assessment updates commercial fishery catch data, recreational fishery catch data, and research survey indices of abundance, and the analytical ASAP assessment models and reference points through 2014. Additionally, stock projections have been updated through 2018.} \def\FLWSNEMASoS{ \textbf{State of Stock: }{}Based on this updated assessment, the Southern New England Mid-Atlantic winter flounder (\textit{Pseudopleuronectes americanus})  stock is overfished but overfishing is not occurring (Figures \ref{FLWSNEMASSB_plot1}-\ref{FLWSNEMAF_plot1}){}.  Retrospective adjustments were not made to the model results. Spawning stock biomass (SSB)  in 2014 was estimated to be 6,151 (mt)  which is 23\percent{} of the biomass target (26,928 mt), and 23\percent{} of the biomass threshold for an overfished stock ($SSB_{Threshold}${} = 13464 (mt);  Figure \ref{FLWSNEMASSB_plot1}{}).  The 2014 fully selected fishing mortality was estimated to be 0.16 which is 49\percent{} of the overfishing threshold ($F_{MSY}${} = 0.325;  Figure \ref{FLWSNEMAF_plot1}{}).} \def\FLWSNEMAProj{ \textbf{Projections: }{}Short term projections of biomass were derived by sampling from a cumulative  distribution  function of recruitment estimates assuming a Beverton-Holt stock recruitment relationship. The annual fishery selectivity, maturity ogive, and mean weights at age used in the projection  are the most recent 5 year averages;  The model exhibited a minor retrospective pattern in F and SSB so no retrospective adjustments were applied in the projections.} \def\FLWSNEMASpecCmt{ \textbf{Special Comments: } \begin{itemize}{} \item{}What are the most important sources of uncertainty in this stock assessment?  Explain, and describe qualitatively how they affect the assessment results (such as estimates of biomass, F, recruitment, and population projections).  \linebreak{} \hspace*{0.5cm} \textit{A large source of uncertainty is the estimate of natural mortality based on longevity, which is not well studied in Southern New England Mid-Atlantic winter flounder, and assumed constant over time.  Natural mortality affects the scale of the biomass and fishing mortality estimates.  Natural mortality was adjusted upwards from 0.2 to 0.3 during the last benchmark assessment assuming a max age of 16. However, there is still uncertainty in the true max age of the population and the resulting natural mortality estimate. Other sources of uncertainty include length distribution of the recreational discards.  The recreational discards are a small component of the total catch, but the assessment suffers from very little length information used to characterize the recreational discards (1 to 2 lengths in recent years).}  \item{} Does this assessment model have a retrospective pattern? If so, is the pattern minor, or major? (A major retrospective pattern occurs when the adjusted SSB or  $F_{Full}${} lies outside of the approximate  joint confidence region for SSB and  $F_{Full}${}; see  Table \ref{RhoDecision_tab}{}). \linebreak{} \hspace*{0.5cm} \textit{ No retrospective adjustment of spawning stock biomass or fishing mortality in 2014 was required. }  \item{}Based on this stock assessment, are population projections well determined or uncertain? \linebreak{} \hspace*{0.5cm} \textit{Population projections for Southern New England Mid-Atlantic winter flounder are reasonably well determined. There is uncertainty in the estimates of M. In addition, while the retrospective pattern is considered minor (within the 90\percent{} CI of both F and SSB), the rho adjusted terminal value is very close to falling outside of the bounds, becoming a major retrospective pattern. This would lead to retrospective adjustments being needed for the projections.}  \item{}Describe any changes that were made to the current stock assessment, beyond incorporating additional years of data  and the effect these changes had on the assessment and stock status. \linebreak{} \hspace*{0.5cm} \textit{ No changes, other than the incorporation of new data, were made to the Southern New England Mid-Atlantic winter flounder assessment for this update.}  \item{}If the stock status has changed a lot since the previous assessment, explain why this occurred.  \linebreak{} \hspace*{0.5cm} \textit{The stock status of Southern New England Mid-Atlantic winter flounder has not changed since the previous benchmark in 2011.}  \item{}Indicate what data or studies are currently lacking and which would be needed most to improve this stock assessment in the future.  \linebreak{} \hspace*{0.5cm} \textit{The Southern New England Mid-Atlantic winter flounder assessment could be improved with additional studies on maximum age, as well additional  recreational discard lengths.  In addition, further investigation into the localized struture/genetics of the stock is warranted. Also, a future shift to ASAP version 4 will provide the ability to model envirionmental factors that may influence both survey catchability and the modeled S-R relationship.}  \item{}Are there other important issues? \linebreak{} \hspace*{0.5cm} \textit{None. } \end{itemize}{}} \def\FLWSNEMARefr{ \textbf{References: }{} \linebreak{}Smith, A. and S. Jones.  2008.  In.  Northeast Fisheries Science Center. 2008. Assessment of 19 Northeast Groundfish Stocks  through 2007: Report of the 3$^{rd}$ Groundfish Assessment Review Meeting (GARM III), Northeast Fisheries Science Center, Woods Hole, Massachusetts, August 4-8, 2008. US Dep Commer, NOAA Fisheries, Northeast Fish Sci Cent Ref Doc. 08-15; 884 p + xvii.  \href{http://www.nefsc.noaa.gov/publications/crd/crd0815/}{CRD08-15} \linebreak{} \linebreak{}Northeast Fisheries Science Center. 2011. 52$^{nd}$ Northeast Regional Stock Assessment Workshop (52$^{nd}$ SAW)  Assessment Report. US Dept Commer, Northeast Fish Sci Cent Ref Doc. 11-17; 962 p. Available from: National Marine Fisheries Service, 166 Water Street, Woods Hole, MA 02543-1026. \href{http://www.nefsc.noaa.gov/saw/saw52/crd1117.pdf}{CRD11-17} \linebreak{} \linebreak{}} \def\FLWSNEMADraft{} \def\FLWSNEMASPPname{Southern New England Mid-Atlantic winter flounder} \def\FLWSNEMASPPnameT{Southern New England Mid-Atlantic winter flounder} \def\FLWSNEMARptYr{2015} \def\FLWSNEMAAuthor{Anthony Wood} \def\FLWSNEMAReviewerComments{/home/dhennen/EIEIO/BigReport/FLW_SNEMA/latex}