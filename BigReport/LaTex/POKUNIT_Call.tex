 \def\POKUNITMyPathTab{/home/dhennen/EIEIO/BigReport/POK_UNIT/tables} \def\POKUNITMyPathFig{/home/dhennen/EIEIO/BigReport/POK_UNIT/figures} \def\POKUNITfigFishCap{Total catch of pollock between 1970 and 2014 by fleet (commercial, Canadian, distant water fleet, and recreational)  and disposition (landings and discards).} \def\POKUNITfigSSBCap{Estimated trends in the spawning stock biomass of pollock between 1970 and 2014 from the current  (solid line)  and previous (dashed line)  assessment and the corresponding  $SSB_{Threshold}${} (0.5 * $SSB_{MSY}${} proxy; horizontal dashed line)  as well as  $SSB_{Target}${} ($SSB_{MSY}${} proxy; horizontal dotted line)   based on the 2015 assessment models base (A)  and flat sel sensitivity (B). Biomass was adjusted for a retrospective pattern and the adjustment is shown in red. The approximate 90\percent{} lognormal confidence intervals are shown.} \def\POKUNITfigFCap{Estimated trends in age 5 to 7 average F ($F_{AVG}${})  of pollock between 1970 and 2014 from the current  (solid line)  and previous (dashed line)  assessment and the corresponding  $F_{Threshold}${} ($F_{MSY}${} proxy; dashed line)  based on the 2015 assessment models base (A)  and flat sel sensitivity (B).  $F_{AVG}${} was adjusted for a retrospective pattern and the adjustment is shown in red. The approximate 90\percent{} lognormal confidence intervals are shown.} \def\POKUNITfigRecrCap{Estimated trends in age 1 recruitment  (000s)  of pollock between 1970 and 2014 from the current (solid line)  and previous (dashed line)  assessment for the assessment models base (A)  and flat sel sensitivity (B).  The approximate 90\percent{} lognormal confidence intervals are shown.} \def\POKUNITfigSurvCap{Indices of abundance for pollock from the Northeast Fisheries Science Center (NEFSC)  spring (1970 to 2015)  and fall (1970 to 2014)  bottom trawl surveys. The approximate 90\percent{} lognormal confidence intervals are shown.} \def\POKUNITPreAmb{This assessment of the pollock (\textit{Pollachius virens})  stock is an operational assessment of the existing 2014 operational assessment (Hendrickson et al. 2015). This assessment updates commercial and recreational fishery catch data, research survey indices of abundance, the ASAP analytical models, and biological reference points through 2014. Additionally, stock projections have been updated through 2018. In what follows, there are two population assessment models brought forward from the 2014 operational assessment: the base model (dome-shaped survey selectivity), which is used to provide management advice; and the flat sel sensitivity model (flat-topped survey selectivity), which is included for the sole purpose of demonstrating the sensitivity of assessment results to survey selectivity assumptions. The most recent benchmark assessment of the pollock stock was in 2010 as part of the 50$^{th}$ Stock Assessment Review Committee (SARC 50; NEFSC 2010), which includes a full description of the model formulations.} \def\POKUNITSoS{ \textbf{State of Stock: }{} The pollock (\textit{Pollachius virens})  stock is not overfished and overfishing is not occurring (Figures \ref{POKUNITSSB_plot1}-\ref{POKUNITF_plot1}){}. Retrospective adjustments were made to the model results. Retrospective adjusted spawning stock biomass (SSB)  in 2014 was estimated to be 154,919 (mt)  under the base model and 32,040 (mt)  under the flat sel sensitivity model which is 147 and 58\percent{} (respectively)  of the biomass target, an  $SSB_{MSY}${} proxy of SSB at  $F_{40\percent{}}${} (105,226 and 54,900  (mt);  Figure \ref{POKUNITSSB_plot1}{}). Retrospective adjusted 2014 age 5 to 7 average fishing mortality (F)   was estimated to be 0.07 under the base model and 0.233 under the flat sel sensitivity model, which is 25 and 92\percent{} (respectively)  of the overfishing threshold, an  $F_{MSY}${} proxy of  $F_{40\percent{}}${} (0.277 and 0.252;  Figure \ref{POKUNITF_plot1}{}).} \def\POKUNITProj{ \textbf{Projections: }{}Short term projections of median total fishery yield and spawning stock biomass for pollock were conducted based on a harvest scenario of fishing at an  $F_{MSY}${} proxy of  $F_{40\percent{}}${} between 2016 and 2018. Catch in 2015 has been estimated at 5,208 (mt). Recruitments were sampled from a cumulative distribution function derived from ASAP estimated age 1 recruitment between 1970 and 2012.  Recruitments in 2013 and 2014 were not included due to uncertainty in those estimates. The annual fishery selectivity, natural mortality, maturity ogive, and mean weights used  in projections are the most recent 5 year averages. Retrospective adjusted age 5 to 7 average F in 2014 fell outside the 90\percent{} confidence intervals of the unadjusted 2014 value under the base model (Figure \ref{POKUNITF_plot1}{}). Retrospective adjusted SSB and age 5 to 7 average F in 2014 fell outside the 90\percent{} confidence intervals of the unadjusted 2014 values under the flat sel sensitivity model  (Figures \ref{POKUNITSSB_plot1}-\ref{POKUNITF_plot1}){}. Therefore, retrospective adjustments were applied in the projections for the base model and the flat sel sensitivity model.} \def\POKUNITSpecCmt{ \textbf{Special Comments: } \begin{itemize}{} \item{}What are the most important sources of uncertainty in this stock assessment?  Explain, and describe qualitatively how they affect the assessment results (such as estimates of biomass, F, recruitment, and population projections).  \linebreak{} \hspace*{0.5cm} \textit{The largest source of uncertainty in the pollock assessment is selectivity, as the base model with dome-shaped survey and fishery selectivities implies the existence of a large cryptic biomass that neither current surveys nor the fishery can confirm. If it is assumed that flat-topped survey selectivities lead to lower estimates of SSB and higher estimates of F  (Figures \ref{POKUNITSSB_plot1}-\ref{POKUNITF_plot1}){}, then stock status is insensitive to the shape of the survey selectivity patterns at older ages.}  \item{} Does this assessment model have a retrospective pattern? If so, is the pattern minor, or major? (A major retrospective pattern occurs when the adjusted SSB or  $F_{AVG}${} lies outside of the approximate  joint confidence region for SSB and  $F_{AVG}${}; see  Table \ref{RhoDecision_tab}{}). \linebreak{} \hspace*{0.5cm} \textit{ The 7-year Mohn's  \textrho{}, relative to SSB, was 0.291 under the base model and 0.66 under the flat sel sensitivity model in the 2014 assessment and was 0.284 and 0.789, respectively, in 2014. The 7-year Mohn's  \textrho{}, relative to F, was -0.252 under the base model and -0.359 under the flat sel sensitivity model in the 2014 assessment and was -0.276 and -0.43, respectively, in 2014. There was a major retrospective pattern for the base model because the  \textrho{} adjusted estimate of 2014 F ($F_{\rho}${}=0.07)  was outside the approximate 90\percent{} confidence region around F (0.035 - 0.066). There was a major retrospective pattern for the flat sel sensitivity model because the  \textrho{} adjusted estimates of 2014 SSB ($SSB_{\rho}${}=32,040)  and 2014 F ($F_{\rho}${}=0.233)  were outside the approximate 90\percent{} confidence region around SSB (37,243 - 77,410  (mt))  and F (0.084 - 0.182). A retrospective adjustment was made for both the determination of stock status and for projections of catch in 2016. The base model retrospective adjustment changed the 2014 SSB from 198,847 to 154,919 and the 2014  $F_{AVG}${} from 0.051 to 0.07. The flat sel sensitivity model retrospective adjustment changed the 2014 SSB from 57,327 to 32,040 and the 2014  $F_{AVG}${} from 0.133 to 0.233.}  \item{}Based on this stock assessment, are population projections well determined or uncertain? \linebreak{} \hspace*{0.5cm} \textit{Population projections for pollock appear to be reasonably well determined for both the base model and the flat sel sensitivity model. }  \item{}Describe any changes that were made to the current stock assessment, beyond incorporating additional years of data  and the affect these changes had on the assessment and stock status. \linebreak{} \hspace*{0.5cm} \textit{Only one major change was made to the pollock assessment as part of this update. Likelihood constants were excluded from likelihood calculations to avoid potential bias caused by one of the recruitment likelihood constants, which is the sum of the log-scale predicted recruitments, and therefore not a constant. Inclusion of this likelihood constant allows the assessment model to minimize the negative log likelihood by estimating lower recruitments. Exclusion of the likelihood constants led to higher estimates of SSB  and lower estimates of F  (Figures \ref{POKUNITSSB_plot1}-\ref{POKUNITF_plot1}){}.}  \item{}If the stock status has changed a lot since the previous assessment, explain why this occurred.  \linebreak{} \hspace*{0.5cm} \textit{Stock status based on the base model has not changed since the previous assessment. Stock status based on the flat sel sensitivity model has changed from 'overfishing is occurring' in the previous assessment to 'overfishing is not occurring' in the current assessment. However, the retrospective adjusted 2014 age 5 to 7 average fishing mortality  from the flat sel sensitivity model (0.233)  is close to the  $F_{MSY}${} proxy (0.252). This change in status likely is due to a decline in predicted F from 2013 to 2014, as well as to the exclusion of the likelihood constants, which led to higher predicted stock productivity.}  \item{}Indicate what data or studies are currently lacking and which would be needed most to improve this stock assessment in the future.  \linebreak{} \hspace*{0.5cm} \textit{The pollock assessment could be improved with additional studies on gear selectivity. These studies could cover topics such as physical selectivity (e.g., multi-mesh gillnet), behavior (e.g., swimming endurance, escape behavior), geographic and vertical distribution by size and age, tag-recovery at size and age, and evaluating information on length-specific selectivity at older ages.}  \item{}Are there other important issues? \linebreak{} \hspace*{0.5cm} \textit{As in the previous assessment, the pollock assessment models had difficulty converging on a solution in some of the retrospective peels. One possible explanation for this convergence issue is that the model may be overparameterized, because the commercial and recreational fleets are modeled separately in this assessment. The possibility of combining the two fleets into a single fleet should be explored during the next benchmark assessment.} \end{itemize}{}} \def\POKUNITRefr{ \textbf{References: }{} \linebreak{}Hendrickson L, Nitschke P, Linton B. 2015. 2014 Operational stock assessments for Georges Bank winter flounder, Gulf of Maine winter flounder, and pollock. US Dept Commer, Northeast Fish Sci Cent Ref Doc. 15-01; 228 p. Available from: National Marine Fisheries Service, 166 Water Street, Woods Hole, MA 02543-1026. \href{http://www.nefsc.noaa.gov/publications/crd/crd1501/}{CRD15-01} \linebreak{} \linebreak{}Northeast Fisheries Science Center. 2010. 50$^{th}$ Northeast Regional Stock Assessment Workshop (50$^{th}$ SAW)  Assessment Report. US Dept Commer, Northeast Fish Sci Cent Ref Doc. 10-17; 844 p. Available from: National Marine Fisheries Service, 166 Water Street, Woods Hole, MA 02543-1026. \href{http://www.nefsc.noaa.gov/publications/crd/crd1017/}{CRD10-17}} \def\POKUNITDraft{} \def\POKUNITSPPname{pollock} \def\POKUNITSPPnameT{Pollock} \def\POKUNITRptYr{2015} \def\POKUNITAuthor{Brian Linton} \def\POKUNITReviewerComments{/home/dhennen/EIEIO/BigReport/POK_UNIT/latex}