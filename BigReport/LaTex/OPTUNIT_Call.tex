 \def\OPTUNITMyPathTab{/home/dhennen/EIEIO/BigReport/OPT_UNIT/tables} \def\OPTUNITMyPathFig{/home/dhennen/EIEIO/BigReport/OPT_UNIT/figures} \def\OPTUNITfigFishCap{Total catch of ocean pout  between 1968 and 2014 by fleet (US and Other)  and disposition (landings and discards).} \def\OPTUNITfigSSBCap{Trends in biomass (kg/tow)  of ocean pout  between 1968 and 2014 from the current  (solid line)  and previous (dashed line)  assessment, and the corresponding  $B_{Threshold}${} ($\dfrac{1}{2}${} $B_{MSY}${} \textit{proxy}{}; horizontal dashed line)  as well as  $B_{Target}${} ($B_{MSY}${} \textit{proxy}{}; horizontal dotted line)   based on the current assessment. } \def\OPTUNITfigFCap{Trends in the exploitation rate of ocean pout between 1968 and 2014 from the current  (solid line)  and previous (dashed line)  assessment and the corresponding  $F_{Threshold}${} ($F_{MSY}${} \textit{proxy}{}=0.76; horizontal dashed line)   based on the current assessment. } \def\OPTUNITfigRecrCap{} \def\OPTUNITfigSurvCap{Indices of biomass (kg/tow)  for ocean pout  between 1968 and 2015 for the Northeast Fisheries Science Center (NEFSC)  spring survey.   The approximate 90\percent{} lognormal confidence intervals are shown.} \def\OPTUNITPreAmb{This assessment of the ocean pout  (\textit{Zoarces americanus})  stock is an operational assessment of the 2012 assessment (NEFSC 2012)  and the 2008 benchmark assessment (NEFSC 2008). Based on the 2012 assessment, the stock was overfished but overfishing was not occurring. This assessment updates commercial fishery catch data, research survey indices and the exploitation ratios through 2014. There are no stock projections.} \def\OPTUNITSoS{ \textbf{State of Stock: }{}Based on the current assessment, the ocean pout  (\textit{Zoarces americanus})  stock is overfished and overfishing is not occurring (Figures \ref{OPTUNITSSB_plot1}-\ref{OPTUNITF_plot1}){}. Retrospective adjustments were not made to the model results. Biomass proxy (B)  in 2014 was estimated to be 0.29 (kg/tow)  which is 6\percent{} of the biomass target ($B_{MSY}${} \textit{proxy}{} = 4.94;  Figure \ref{OPTUNITSSB_plot1}{}).  The 2014 fully selected fishing mortality was estimated to be 0.269 which is 35\percent{} of the overfishing threshold proxy ($F_{MSY}${} \textit{proxy}{} = 0.76;  Figure \ref{OPTUNITF_plot1}{}).} \def\OPTUNITProj{ \textbf{Projections: }{}The index-based assessment approach does not support catch projections; catch advice for ocean pout has been based on the target exploitation rate and the most recent centered 3-year average biomass index from the NEFSC spring survey. } \def\OPTUNITSpecCmt{ \textbf{Special Comments: } \begin{itemize}{} \item{}What are the most important sources of uncertainty in this stock assessment?  Explain, and describe qualitatively how they affect the assessment results (such as estimates of biomass, F, recruitment, and population projections).  \linebreak{} \hspace*{0.5cm} \textit{ An important source of uncertainty is the stock has not responded to low catch as expected. }  \item{}Does this assessment model have a retrospective pattern? If so, is the pattern minor or major?  (A major retrospective pattern occurs when the adjusted SSB or  $F_{Full}${} lies outside of the approximate  joint confidence region for SSB and  $F_{Full}${}; see  Table \ref{RhoDecision_tab}{}). \linebreak{} \hspace*{0.5cm} \textit{ The model used to estimate status of this stock does not allow estimation of a retrospective pattern. }  \item{}Based on this stock assessment, are population projections well determined or uncertain? \linebreak{} \hspace*{0.5cm} \textit{ N/A}  \item{}Describe any changes that were made to the current stock assessment, beyond incorporating additional years of data  and the effect these changes had in the assessment and stock status. \linebreak{} \hspace*{0.5cm} \textit{TOGA (Type, Operation, Gear, Acquisition)  values were used for haul criteria for NEFSC surveys for 2009 onward and minor changes in the use of observer data for discard estimates were made to the current assessment. These changes had a negligible effect on the assessment and stock status.   Recreational landings were updated and found to be negligible (time series average of recreational landings to total catch was less than 1\percent{})  and therefore not included in this assessment.}  \item{}If the stock status has changed a lot since the previous assessment, explain why this occurred.  \linebreak{} \hspace*{0.5cm} \textit{Ocean pout stock status has not changed since the previous assessment.}  \item{}Indicate what data or studies are currently lacking and which would be needed most to improve this stock assessment in the future.  \linebreak{} \hspace*{0.5cm} \textit{The ocean pout assessment could be improved with studies that explore why this stock is not rebuilding as expected. }  \item{}Are there other important comments? \linebreak{} \hspace*{0.5cm} \textit{Biological reference points are based on catch; the estimated discards used in the catch are based on a mix of direct (1989 onward)  and indirect (1988 and back)  methods. The catch used to determine MSY is based on indirect methods. } \end{itemize}{}} \def\OPTUNITRefr{ \textbf{References: }{} \linebreak{}Northeast Fisheries Science Center. 2012. Assessment or Data Updates of 13 Northeast Groundfish Stocks through 2010.  US Dep Commer, NOAA Fisheries, Northeast Fish Sci Cent Ref Doc. 12-06; 789 p. \href{http://www.nefsc.noaa.gov/publications/crd/crd1206/}{CRD12-06} \linebreak{} \linebreak{}Northeast Fisheries Science Center. 2008. Assessment of 19 Northeast Groundfish Stocks through 2007: Report of the 3$^{rd}$ Groundfish Assessment Review Meeting (GARM III), Northeast Fisheries Science Center, Woods Hole, Massachusetts, August 4-8, 2008. US Dep Commer, NOAA Fisheries, Northeast Fish Sci Cent Ref Doc. 08-15; 884 p + xvii. \href{http://www.nefsc.noaa.gov/publications/crd/crd0815/}{CRD08-15} \linebreak{} \linebreak{}} \def\OPTUNITDraft{} \def\OPTUNITSPPname{ocean pout} \def\OPTUNITSPPnameT{Ocean pout} \def\OPTUNITRptYr{2015} \def\OPTUNITAuthor{Susan Wigley} \def\OPTUNITReviewerComments{/home/dhennen/EIEIO/BigReport/OPT_UNIT/latex}