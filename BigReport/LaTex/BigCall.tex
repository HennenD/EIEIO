cd /home/dhennen/EIEIO/BigReport/latex; ls -l; pdflatex -halt-on-error   \def\HALUNITMyPathTab{/home/dhennen/EIEIO/BigReport/HAL_UNIT/tables} \def\HALUNITMyPathFig{/home/dhennen/EIEIO/BigReport/HAL_UNIT/figures} \def\HALUNITfigFishCap{Total catch of Atlantic halibut between 1963 and 2014 by disposition \(landings and discards\).} \def\HALUNITfigSSBCap{Estimated trends in the biomass of Atlantic halibut between 1963 and 2014 from the current  \(solid line\)  and previous \(dashed line\)  assessment and the corresponding  \$B\_{Threshold}\${}= \$\dfrac{1}{2}\${} \$B\_{MSY}\${} \textit{proxy}{}\(horizontal dashed line\)  as well as  \$B\_{Target}\${} \(\$B\_{MSY}\${} \textit{proxy}{}\; horizontal dotted line\)   based on the 2015 assessment.} \def\HALUNITfigFCap{Estimated trends in the fully selected fishing mortality \(\$F\_{Full}\${}\)  of Atlantic halibut between 1963 and 2014 from the current  \(solid line\)  and previous \(dashed line\)  assessment and the corresponding  \$F\_{Threshold}\${} \(0.073\; horizontal dashed line\)  as well as  \$F\_{Target}\${} \(0.8 \* \$F\_{MSY}\${} \textit{proxy}{}\; dotted line\)   based on the 2015 assessment. } \def\HALUNITfigRecrCap{} \def\HALUNITfigSurvCap{Indices of biomass for the Atlantic halibut between 1963 and 2014 for the Northeast Fisheries Science Center \(NEFSC\)  fall bottom trawl survey.  The 90\\percent lognormal confidence intervals are shown.} \def\HALUNITPreAmb{This assessment of the Atlantic halibut \(\textit{Hippoglossus hippoglossus}\)  stock is an update of the existing 2012 benchmark assessment \(NEFSC 2010\)  and the last update assessment \(NEFSC 2012\). This assessment updates commercial fishery catch data, research survey indices of abundance, and the replacement yield assessment model through 2014. Additionally, stock projections have been updated through 2018. Reference points have not been updated. } \def\HALUNITSoS{ \textbf{State of Stock: }{}Based on this updated assessment, Atlantic halibut \(\textit{Hippoglossus hippoglossus}\)  stock is unknown and unknown \(Figures \ref{HALUNITSSB\_plot1}\-\ref{HALUNITF\_plot1}\){}. Retrospective adjustments were not made to the model results.  Biomass \(SSB\)  in 2014 was estimated to be 96,464 \(mt\)  which is 199\\percent of the biomass target \(\$SSB\_{MSY}\${} \textit{proxy}{} = 48,509\;  Figure \ref{HALUNITSSB\_plot1}{}\).  The 2014 fully selected fishing mortality was estimated to be 0.001 which is 1\\percent of the overfishing threshold proxy \(\$F\_{MSY}\${} \textit{proxy}{} = 0.073\;  Figure \ref{HALUNITF\_plot1}{}\).} \def\HALUNITProj{ \textbf{Projections: }{} Short term projections were based on a constant F =  \$F\_{MSY}\${} \textit{proxy}{} = 0.073.  Projections use the assessment model \(replacement yield\)  and maintain all other model assumptions.} \def\HALUNITSpecCmt{ \textbf{Special Comments: } \begin{itemize}{} \item{}What are the most important sources of uncertainty in this stock assessment?  Explain, and describe qualitatively how they affect the assessment results \(such as estimates of biomass, F, recruitment, and population projections\).  \linebreak{} \hspace\*{0.5cm} \textit{The assessment model used for Atlantic halibut is highly uncertain.  It estimates one parameter, the initial biomass, and  proceeds deterministically from 1800 to 2014.  The model is highly sensitive to the initial biomass.  The model is  tuned to the survey index, which is inefficient for Atlantic halibut, catches very few animals and is therefore noisy.   The RYM model assumes no immigration or emmigration and that the population both began, and tends to, equilibrium.   These assumptions are unlikely to be true for Atlantic halibut. The model estimates a biomass that is approximately equal  to unfished biomass, which is not credible. Catch has been very low for at least 100 years relative  to the landings reported early in the time series, despite a strong market and high value  relative to other groundfish.  The low catch throughout the century implies that the Atlantic halibut stock is very likely  depleted relative to it\'s unfished condition and is therefore likely to be overfished, even if its current biomass is  unknown.}  \item{} Does this assessment model have a retrospective pattern? If so, is the pattern minor, or major? \(A major retrospective pattern occurs when the adjusted SSB or  \$F\_{Full}\${} lies outside of the approximate  joint confidence region for SSB and  \$F\_{Full}\${}\; see  Figure \ref{RhoDecision\_tab}{}\). \linebreak{} \hspace\*{0.5cm} \textit{ The model used to determine the status of this stock does not allow estimation of a retrospective pattern. }  \item{}Based on this stock assessment, are population projections well determined or uncertain? \linebreak{} \hspace\*{0.5cm} \textit{Population projections for Atlantic halibut are uncertain because biomass cannot be reasonably determined using  the current assessment model.}  \item{}Describe any changes that were made to the current stock assessment, beyond incorporating additional years of data  and the affect these changes had on the assessment and stock status. \linebreak{} \hspace\*{0.5cm} \textit{ The catch data were slightly altered due to the exclusion of catch made in international waters and the  re\-estiamtion of average discard ratio after 1998 \(due to the incorporation of more years of data\).}  \item{}If the stock status has changed a lot since the previous assessment, explain why this occurred.  \linebreak{} \hspace\*{0.5cm} \textit{The overfishing and overfished status of Atlantic halibut cannot be determined using the current assessment.  This  occurred because diagnostics showed the model was unreliable.  }  \item{}Indicate what data or studies are currently lacking and which would be needed most to improve this stock assessment in the future.  \linebreak{} \hspace\*{0.5cm} \textit{The Atlantic halibut assessment could be improved with additional studies on stock structure, additional age and length data,  a more precise and accurrate survey, and an investigation of alternate assessment models.}  \item{}Are there other important issues? \linebreak{} \hspace\*{0.5cm} \textit{Atlantic halibut are clearly depleted relative to their unfished state.  Catches have been far below historical landings  for more than 100 years, despite a lack of regulation before 1999 and a strong commercial market.  The current  assessment model implies that Atlantic halibut is near or above its unfished biomass and could support removals  commensurate with MSY.  The current assessment should probably not be used to inform management decisions.} \end{itemize}{}} \def\HALUNITRefr{ \textbf{References: }{} \linebreak{} Northeast Fisheries Science Center. 2012. Assessment or Data Updates of 13 Northeast Groundfish Stocks  through 2010. US Dept Commer, Northeast Fish Sci Cent Ref Doc. 12\-06\; 789 p. Available from: National  Marine Fisheries Service, 166 Water Street, Woods Hole, MA 02543\-1026, or online at  http:\/\/nefsc.noaa.gov\/publications\/ \linebreak{} \linebreak{}Col, L.A., Legault, C.M. 2009. The 2008 Assessment of Atlantic halibut in the Gulf of Maine Georges Bank region.  US Dept Commer, Northeast Fish Sci Cent Ref Doc. 09\-08\; 39 p. Available from: National Marine Fisheries Service, 166 Water Street, Woods Hole, MA 02543\-1026, or online at http:\/\/www.nefsc.noaa.gov\/nefsc\/publications\/ } \def\HALUNITDraft{} \def\HALUNITSPPname{Atlantic halibut} \def\HALUNITSPPnameT{Atlantic halibut} \def\HALUNITRptYr{2015} \def\HALUNITAuthor{Daniel Hennen} \def\HALUNITReviewerComments{/home/dhennen/EIEIO/BigReport/HAL_UNIT/latex}  \def\CODGMMyPathTab{/home/dhennen/EIEIO/BigReport/COD_GM/tables} \def\CODGMMyPathFig{/home/dhennen/EIEIO/BigReport/COD_GM/figures} \def\CODGMfigFishCap{Total catch of Gulf of Maine Atlantic cod between 1982 and 2014 by fleet \(commercial and recreational\)  and disposition \(landings and discards\).} \def\CODGMfigSSBCap{Estimated trends in the spawning stock biomass \(SSB\)  of Gulf of Maine Atlantic cod between 1982 and 2014 from the current  \(solid line\)  and previous \(dashed line\)  assessment and the corresponding  \$SSB\_{Threshold}\${} \(\$\dfrac{1}{2}\${} \$SSB\_{MSY}\${}\; horizontal dashed line\)  as well as  \$SSB\_{Target}\${} \$SSB\_{MSY}\${}\; horizontal dotted line\)   based on the 2015 M=0.2 \(A\)  and M\-ramp \(B\)  assessment models. The 90\\percent lognormal confidence intervals are shown. The red dot indicates the rho\-adjusted SSB values that would have resulted had a retrospective adjusment been made to either model \(see Special Comments section\).} \def\CODGMfigFCap{Estimated trends in the fully selected fishing mortality \(F\)  of Gulf of Maine Atlantic cod between 1982 and 2014 from the current  \(solid line\)  and previous \(dashed line\)  assessment and the corresponding  \$F\_{Threshold}\${} \(0.185 \(M=0.2\), 0.187 \(M\-ramp\)\; dashed line\)  based on the 2015 M=0.2 \(A\)  and M\-ramp \(B\)  assessment models. The 90\\percent lognormal confidence intervals are shown. The red dot indicates the rho\-adjusted F values that would have resulted had a retrospective adjusment been made to either model \(see Special Comments section\).} \def\CODGMfigRecrCap{Estimated trends in age\-1 recruitment  \(000s\)  of Gulf of Maine Atlantic cod between 1982 and 2014 from the current \(solid line\)  and previous \(dashed line\)  M=0.2 \(A\)  and M\-ramp \(B\)  assessment models. The 90\\percent lognormal confidence intervals are shown.} \def\CODGMfigSurvCap{Indices of biomass for the Gulf of Maine Atlantic cod between 1963 and 2015 for the Northeast Fisheries Science Center \(NEFSC\)  spring and fall bottom trawl surveys and Massachusetts Division of Marine Fisheries \(MADMF\)  spring bottom trawl survey.  The 90\\percent lognormal confidence intervals are shown.} \def\CODGMPreAmb{This assessment of the Gulf of Maine Atlantic cod \(\textit{Gadus morhua}\)  stock is an update of the existing 2014 assessment \(Palmer 2014\). This assessment updates commercial and recreational fishery catch data, research survey indices of abundance, and the analytical ASAPassessment models through 2014. Additionally, stock projections have been updated through 2018. In what follows, there are two population assessment models brought forward from the most recent benchmark assessment \(2012\), the M=0.2 \(natural mortality = 0.2\)  and the M\-ramp \(M ramps from 0.2 to 0.4\)  assessment models \(see NEFSC 2013 for a full description of the model formulations\).} \def\CODGMSoS{ \textbf{State of Stock: }{}Based on this updated assessment, the Gulf of Maine Atlantic cod \(\textit{Gadus morhua}\)  stock is overfished and overfishing is occurring \(Figures \ref{CODGMSSB\_plot1}\-\ref{CODGMF\_plot1}\){}. Retrospective adjustments were not made to the model results \(see Special Comments section of this report\). Spawning stock biomass \(SSB\)  in 2014 was estimated to be 2,225 \(mt\)  under the M=0.2 model and 2,536 \(mt\)  under the M\-ramp model scenario \(Table \ref{CODGMCatch\_Status\_Table}{}\)  which is 6 and 4\\percent \(respectively\)  of the biomass target,  \$SSB\_{MSY}\${} \textit{proxy}{} \(40,187 \(mt\)  and 59,045 \(mt\)\;  Figure \ref{CODGMSSB\_plot1}{}\).  The 2014 fully selected fishing mortality was estimated to be 0.956 and 0.932 which is 517 and 498\\percent of the  \$F\_{MSY}\${} \textit{proxy}{}\(\$F\_{40\\percent}\${}\; 0.185 and 0.187\;  Figure \ref{CODGMF\_plot1}{}\).} \def\CODGMProj{ \textbf{Projections: }{} Short term projections of median total fishery yield and spawning stock biomass for Gulf of Maine Atlantic cod were conducted based on a harvest scenario of fishing at the FMSY proxy between 2016 and 2018. Catch in 2015 was estimated at 279 mt. Recruitment was sampled from a cumulative distribution function derived from ASAP estimated age\-1 recruitment between 1982 and 2012.  The projection recruitment model declines linearly to zero when SSB is below 6.3 kmt under the M=0.2 model and 7.9 kmt under the M\-ramp model. The 2015 age\-1 recruitment was estimated from the geometric mean of the 2010\-2014 ASAP recruitment estimates. No retrospective adjustments were applied in the projections as the retrospective patterns are similar to the 2014 update for which no retrospective adjustments were made\; however, the 2015 assessment review panel recommended that that M=0.2 projections with retrospective adjustments be brought forward to the SSC for consideration in the evaluation of uncertainty when setting catch advice \(provided in the Supplemental Information Report, \href{http:\/\/www.nefsc.noaa.gov\/saw\/sasi\/sasi\_report\_options.php}{SASINF}{}\). Assumed weights are based on an average of the most recent three years. For the M\-ramp model, projections are shown under two assumptions of short\-term natural mortality: M=0.2 and M=0.4.} \def\CODGMSpecCmt{ \textbf{Special Comments: } \begin{itemize}{} \item{}What are the most important sources of uncertainty in this stock assessment?  Explain, and describe qualitatively how they affect the assessment results \(such as estimates of biomass, F, recruitment, and population projections\).  \linebreak{} \hspace\*{0.5cm} \textit{The largest source of uncertainty is the estimate of natural mortality. Past investigations into changes in natural mortality over time have been inconclusive \(NEFSC 2013\). Different assumptions about natural mortality affect the scale of the biomass, recruitment, and fishing mortality estimates. Other areas of uncertainty include the retrospective error in the M=0.2 model, residual patterns in the model fits to some of the survey series \(e.g., aggregate MADMF spring survey\)  and stock structure.}  \item{}Does this assessment model have a retrospective pattern? If so, is the pattern minor, or major? \(A major retrospective pattern occurs when the adjusted SSB or  \$F\_{Full}\${} lie outside of the approximate joint confidence region for SSB and  \$F\_{Full}\${}\). \linebreak{} \hspace\*{0.5cm} \textit{The M=0.2 model has a major retrospective pattern \(7\-year Mohn\'s rho SSB=0.54, F=\-0.31\)  and the M\-ramp model has a minor retrospective pattern \(7\-year Mohn\'s rho SSB=0.20, F=\-0.08\). The 7\-year Mohn\'s rho values from the current assessment are similar to those from the 2014 assessment \(M=0.2: SSB=0.53, F=\-0.33\; M\-ramp: SSB=0.17, F=\-0.05\)  where the M=0.2 model had a major retrospective pattern and the M\-ramp model had a minor pattern. No retrospective adjustment have been to the terminal model results or in the base catch projections following the recommendations of the SARC 55 and 2014 assessment review panels. The 2015 assessment review panel supported this decision noting that the most recent retrospective \'peel\' suggested that an adjustment using the 7\-year average may not be appropriate. However, the 2015 review panel highlighted the retrospective error in the M=0.2 model as a source of uncertainty \- it should be noted that the retrospective error of the most recent peel is larger for the M\-ramp model. Should the retrospective patterns continue then the models may have overestimated spawning stock size and underestimated fishing mortality.}  \item{}Based on this stock assessment, are population projections well determined or uncertain? \linebreak{} \hspace\*{0.5cm} \textit{Population projections for Gulf of Maine Atlantic cod are reasonably well determined and projected boimass from the last assessment  was within the confidence bounds of the biomass estimated in the current assessment. }  \item{}Describe any changes that were made to the current stock assessment, beyond incorporating additional years of data  and the affect these changes had on the assessment and stock status. \linebreak{} \hspace\*{0.5cm} \textit{ This update included several minor changes to model input data including: \(1\)  re\-estimation of recreational catch from 2004\-2014 to account for recent updates to the MRIP data\; \(2\)  a revised assumption on recreational discard mortality from 30\\percent to 15\\percent following a Capizzano et al. 2015 study \(unpublished\)\; and \(3\)  re\-estimation of 2009\-2014 NEFSC spring and fall survey time series using the TOGA station acceptance criterion. Additionally, the ASAP assessment model was run with the likelihood constants option turned off. All of these changes had minimal impacts on model results \- summaries of the impacts of these changes are provided in the Supplemental Information Report \(\href{http:\/\/www.nefsc.noaa.gov\/saw\/sasi\/sasi\_report\_options.php}{SASINF}{}\).}  \item{}If the stock status has changed a lot since the previous assessment, explain why this occurred.  \linebreak{} \hspace\*{0.5cm} \textit{There has been no change in stock status since the 2014 udpate assessment.}  \item{}Indicate what data or studies are currently lacking and which would be needed most to improve this stock assessment in the future.  \linebreak{} \hspace\*{0.5cm} \textit{The Gulf of Maine Atlantic cod assessment could be improved with additional studies on natural mortality and stock structure. Additionally, future assessments should consider possible changes in recent fishery selectivity patterns and exlore alternative methods for estimating recruitment. Potential causes of low stock productivity \(i.e., low recruitment\)  should also be investigated.}  \item{}Are there other important issues? \linebreak{} \hspace\*{0.5cm} \textit{ When setting catch advice careful attention should be given to the retrospective error present in both models, particularly given the poor performance of previous stock projections. Additionally, it is unclear as to which level of natural mortality \(M=0.2 or 0.4\)  to assume for the short\-term projections under the M\-ramp model.} \end{itemize}{}} \def\CODGMRefr{ \textbf{References: }{} \linebreak{}Northeast Fisheries Science Center. 2013. 55$^{th}$ Northeast Regional Stock Assessment Workshop \(55$^{th}$ SAW\)  Assessment Summary Report. US Dept Commer, Northeast Fish Sci Cent Ref Doc. 13\-01\; 41 p. Available from: National Marine Fisheries Service, 166 Water Street, Woods Hole, MA 02543\-1026 \linebreak{} \linebreak{}Palmer MC. 2014. 2014 Assessment update report of the Gulf of Maine Atlantic cod stock. US Dept Commer, Northeast Fish Sci Cent Ref Doc. 14\-14\; 119 p. Available from: National Marine Fisheries Service,166 Water Street, Woods Hole, MA 02543\-1026 } \def\CODGMDraft{} \def\CODGMSPPname{Gulf of Maine Atlantic cod} \def\CODGMSPPnameT{Gulf of Maine Atlantic cod} \def\CODGMRptYr{2015} \def\CODGMAuthor{Michael Palmer} \def\CODGMReviewerComments{/home/dhennen/EIEIO/BigReport/COD_GM/latex}  \def\CODGBMyPathTab{/home/dhennen/EIEIO/BigReport/COD_GB/tables} \def\CODGBMyPathFig{/home/dhennen/EIEIO/BigReport/COD_GB/figures} \def\CODGBfigFishCap{Total catch of Georges Bank Atlantic Cod between 1978 and 2014 by fleet \(US commercial, US recreational, or Canadian\)  and disposition \(landings and discards\).} \def\CODGBfigSSBCap{Trends in spawning stock biomass of Georges Bank Atlantic Cod between 1978 and 2014 from the current  \(solid line\)  and previous \(dashed line\)  assessment and the corresponding  \$SSB\_{Threshold}\${} \(\$\dfrac{1}{2}\${} \$SSB\_{MSY}\${} \textit{proxy}{}\; horizontal dashed line\)  as well as  \$SSB\_{Target}\${} \(\$SSB\_{MSY}\${} \textit{proxy}{}\; horizontal dotted line\)   based on the 2015 assessment.  Biomass was adjusted for a retrospective pattern  and the adjustment is shown in red.  The approximate 90\\percent lognormal confidence intervals are shown.} \def\CODGBfigFCap{Trends in the fully selected fishing mortality \(\$F\_{Full}\${}\)  of Georges Bank Atlantic Cod between 1978 and 2014 from the current  \(solid line\)  and previous \(dashed line\)  assessment and the corresponding  \$F\_{Threshold}\${} \(\$F\_{MSY}\${} \textit{proxy}{}=0.169\; horizontal dashed line\).  \$F\_{Full}\${} was adjusted for a retrospective pattern  and the adjustment is shown in red,  based on the 2015 assessment. The approximate 90\\percent lognormal confidence intervals are shown.} \def\CODGBfigRecrCap{Trends in Recruits \(age 1\)  \(000s\)  of Georges Bank Atlantic Cod between 1978 and 2014 from the current \(solid line\)  and previous \(dashed line\)  assessment. The approximate 90\\percent lognormal confidence intervals are shown.} \def\CODGBfigSurvCap{Indices of biomass for the Georges Bank Atlantic Cod between 1963 and 2015 for the Northeast Fisheries Science Center \(NEFSC\)  spring and fall, and the DFO research bottom trawl surveys.  The approximate 90\\percent lognormal confidence intervals are shown.} \def\CODGBPreAmb{This assessment of the Georges Bank Atlantic Cod \(\textit{Gadus morhua}\)  stock is an operational update of the existing 2012 benchmark assessment \(NEFSC 2013\). Based on the previous assessment the stock was overfished, and overfishing was ocurring. This 2015 assessment updates commercial fishery catch data, research survey indices of abundance, the analytical ASAP assessment model, and reference points through 2014. Additionally, stock projections have been updated through 2018.} \def\CODGBSoS{ \textbf{State of Stock: }{}Based on this updated assessment, the Georges Bank Atlantic Cod \(\textit{Gadus morhua}\)  stock is overfished and overfishing is occurring \(Figures \ref{CODGBSSB\_plot1}\-\ref{CODGBF\_plot1}\){}.  Retrospective adjustments were made to the model results.  Spawning stock biomass \(SSB\)  in 2014 was estimated to be 1,804 \(mt\)  which is 1\\percent of the biomass target for this stock \(\$SSB\_{MSY}\${} \textit{proxy}{} = 201,152\;  Figure \ref{CODGBSSB\_plot1}{}\).  The 2014 fully selected fishing mortality was estimated to be 1.68 which is 994\\percent of the overfishing threshold proxy \(\$F\_{MSY}\${} \textit{proxy}{} = 0.169\;  Figure \ref{CODGBF\_plot1}{}\).} \def\CODGBProj{ \textbf{Projections: }{}Short term projections of biomass were derived by sampling from a two\-stage cumulative  distribution  function of recruitment estimates from ASAP model results, using a 50,000 mt cutpoint. The annual fishery selectivity, maturity ogive, and mean weights at age used in projections are the most recent 5 year averages\;  retrospective adjustments were applied in the projections.} \def\CODGBSpecCmt{ \textbf{Special Comments: } \begin{itemize}{} \item{}What are the most important sources of uncertainty in this stock assessment?  Explain, and describe qualitatively how they affect the assessment results \(such as estimates of biomass, F, recruitment, and population projections\).  \linebreak{} \hspace\*{0.5cm} \textit{The major source of uncertainty is presumbaly the estimate of catch or of natural mortality, considering the magnitude of the retrospective bias. These both affect the scale of the biomass, fishing mortality estimates, and the reference point estimates. The catch estimates do not include all discards \(e.g.,lobster gear\)  and includes uncertain estimates of recreational landings and discards, and of some commercial discards \(e.g., small mesh\). Natural mortality \(M\)  of Georges Bank Atlantic Cod is not well understood and is assumed constant over time in the model. Other sources of uncertainty include possible changes in growth parameters in recent years and how this affects fecundity, the viability of eggs\/sperm, and the success rate of hatching \- all influencing recruitment survival and year class strength.}  \item{} Does this assessment model have a retrospective pattern? If so, is the pattern minor, or major? \(A major retrospective pattern occurs when the adjusted SSB or  \$F\_{Full}\${} lies outside of the approximate  joint confidence region for SSB and  \$F\_{Full}\${}\; see  Figure \ref{RhoDecision\_tab}{}\). \linebreak{} \hspace\*{0.5cm} \textit{ The 7\-year Mohn\'s  \textrho{}, relative to SSB, was 0.68 in the 2012 assessment and was 2.43 in 2014. The 7\-year Mohn\'s  \textrho{}, relative to F, was \-0.46 in the 2012 assessment and was \-0.72 in 2014. There was a major retrospective pattern for this assessment because the  \textrho{} adjusted estimates of 2014 SSB \(\$SSB\_{\rho}\${}=1,804\)  and 2014 F \(\$F\_{\rho}\${}=1.68\)  were outside the approximate 90\\percent confidence regions around SSB \(3,922 \- 10,596\)  and F \(0.251 \- 0.815\).  A retrospective  adjustment was made for both the determination of stock status and for projections of catch in 2016. The retrospective adjustment changed the 2014 SSB from 6,180 to 1,804 and the 2014  \$F\_{Full}\${} from 0.463 to 1.68.}  \item{}Based on this stock assessment, are population projections well determined or uncertain? \linebreak{} \hspace\*{0.5cm} \textit{Population projections for Georges Bank Atlantic Cod are uncertain and likely optimistic. The projections are based on a biomass cutpoint of 50,000 mt, which has not been produced since 1992. The average recruitment since 1992 has been 4.9 million age 1 fish, whereas during the last 10 years, average recruitment has been about 2.7 million age 1 fish. A sensistivity projection using the most recent 10 years of recruitment was conducted and results presented in the SASINF database. }  \item{}Describe any changes that were made to the current stock assessment, beyond incorporating additional years of data  and the effect these changes had on the assessment and stock status. \linebreak{} \hspace\*{0.5cm} \textit{ No major changes, other than the addition of recent years of data, were made to the Georges Bank Atlantic Cod assessment for this update. However, recreational catch and commercial discard estimates were revised slightly due to minor changes in the databases, and the application of length frequencies \(annual instead of half year\)  in one instance.}  \item{}If the stock status has changed a lot since the previous assessment, explain why this occurred.  \linebreak{} \hspace\*{0.5cm} \textit{As in recent assessments for Georges Bank Atlantic Cod the stock remains in an overfishing and overfished status.}  \item{}Indicate what data or studies are currently lacking and which would be needed most to improve this stock assessment in the future.  \linebreak{} \hspace\*{0.5cm} \textit{The Georges Bank Atlantic Cod assessment could be improved with additional studies on natural mortality, growth, and fecundity. Additionally, more precise estimates of recreational landings and discards, sampling of fish caught by individual recreational anglers, and incorporation of discards in the lobster fishery would decrease uncertainty in the discard esimates.}  \item{}Are there other important issues? \linebreak{} \hspace\*{0.5cm} \textit{The differences in model assumptions of natural mortality between the SARC GB cod and the TRAC eGB cod assessment is problematic for the recovery of the entire GB cod stock. Model results of the TRAC VPA M=0.8 model are used to determine quota for the eGB management unit, so by default, proportionally more cod are being removed from eastern GB than what the GB cod ASAP model would predict.} \end{itemize}{}} \def\CODGBRefr{ \textbf{References: }{} \linebreak{}Northeast Fisheries Science Center. 2013. 55$^{th}$ Northeast Regional Stock AssessmentWorkshop \(55$^{th}$ SAW\)  Assessment Summary Report. Northeast Fisheries Science CenterReference Document 13\-01:43. \linebreak{} \linebreak{}} \def\CODGBDraft{} \def\CODGBSPPname{Georges Bank Atlantic Cod} \def\CODGBSPPnameT{Georges Bank Atlantic Cod} \def\CODGBRptYr{2015} \def\CODGBAuthor{Loretta O\'Brien} \def\CODGBReviewerComments{/home/dhennen/EIEIO/BigReport/COD_GB/latex}  \def\HADGBMyPathTab{/home/dhennen/EIEIO/BigReport/HAD_GB/tables} \def\HADGBMyPathFig{/home/dhennen/EIEIO/BigReport/HAD_GB/figures} \def\HADGBfigFishCap{Total catch of Georges Bank haddock between 1931 and 2014 by fleet \(US Commercial, Canadian, or foreign fleet\)  and disposition \(landings and discards\).} \def\HADGBfigSSBCap{Trends in spawning stock biomass of Georges Bank haddock between 1931 and 2014 from the current  \(solid line\)  and previous \(dashed line\)  assessment and the corresponding  \$SSB\_{Threshold}\${} \(\$\dfrac{1}{2}\${} \$SSB\_{MSY}\${} \textit{proxy}{}\; horizontal dashed line\)  as well as  \$SSB\_{Target}\${} \(\$SSB\_{MSY}\${} \textit{proxy}{}\; horizontal dotted line\)   based on the 2015 assessment.  Biomass was adjusted for a retrospective pattern  and the adjustment is shown in red.   The 90\\percent bootstrap probability intervals are shown.} \def\HADGBfigFCap{Trends in the fully selected fishing mortality \(\$F\_{Full}\${}\)  of Georges Bank haddock between 1931 and 2014 from the current  \(solid line\)  and previous \(dashed line\)  assessment and the corresponding  \$F\_{Threshold}\${} \(\$F\_{MSY}\${} \textit{proxy}{}=0.39\; horizontal dashed line\)  based on the 2015 assessment.   \$F\_{Full}\${} was adjusted for a retrospective pattern  and the adjustment is shown in red.   The 90\\percent bootstrap probability intervals are shown.} \def\HADGBfigRecrCap{Trends in Recruits \(age 1\)  \(000s\)  of Georges Bank haddock between 1931 and 2014 from the current \(solid line\)  and previous \(dashed line\)  assessment.  The 90\\percent bootstrap probability intervals are shown.} \def\HADGBfigSurvCap{Indices of biomass \(Mean kg\/tow\)  for the Georges Bank haddock stock between 1963 and 2015 for the Northeast Fisheries Science Center \(NEFSC\)  spring and fall bottom trawl surveys and the DFO winter bottom trawl survey.  The approximate 90\\percent lognormal confidence intervals are shown.} \def\HADGBPreAmb{This assessment of the Georges Bank haddock \(\textit{Melanogrammus aeglefinus}\)  stock is an operational update of the existing 2012 update VPA assessment \(Brooks et al., 2012\).  The last benchmark for this stock was in 2008 \(Brooks et al., 2008\).  Based on the previous assessment in 2012, the stock was not overfished, and overfishing was not ocurring. This assessment updates commercial fishery catch data, research survey indices of abundance, weights and maturity at age, and the analytical VPA assessment model and reference points through 2014. Additionally, stock projections have been updated through 2018.} \def\HADGBSoS{ \textbf{State of Stock: }{}Based on this updated assessment, the Georges Bank haddock \(\textit{Melanogrammus aeglefinus}\)  stock is not overfished and overfishing is not occurring \(Figures \ref{HADGBSSB\_plot1}\-\ref{HADGBF\_plot1}\){}.  Retrospective adjustments were made to the model results.  Spawning stock biomass \(SSB\)  in 2014 was estimated to be 150,053 \(mt\)  which is 139\\percent of the biomass target \(\$SSB\_{MSY}\${} \textit{proxy}{} = 108,300\;  Figure \ref{HADGBSSB\_plot1}{}\).  The 2014 fully selected fishing mortality was estimated to be 0.241 which is 62\\percent of the overfishing threshold proxy \(\$F\_{MSY}\${} \textit{proxy}{} = 0.39\;  Figure \ref{HADGBF\_plot1}{}\).} \def\HADGBProj{ \textbf{Projections: }{}Short term projections of biomass were derived by sampling from a cumulative  distribution  function of recruitment estimates from ADAPT VPA \(corresponding to SSB\$\>\$75,000 mt and dropping the extremely large 1963, 2003, and 2010 year classes, as well as the two final year class estimates for 2013 and 2014\). The annual fishery selectivity, maturity ogive, and mean weights at age used in this projection  are the most recent 5 year averages\;  retrospective adjustments were applied to the starting numbers at age \(2015\)  in the projections.} \def\HADGBSpecCmt{ \textbf{Special Comments: } \begin{itemize}{} \item{}What are the most important sources of uncertainty in this stock assessment?  Explain, and describe qualitatively how they affect the assessment results \(such as estimates of biomass, F, recruitment, and population projections\).  \linebreak{} \hspace\*{0.5cm} \textit{The largest source of uncertainty is the estimate of 2013 recruitment, which accounts for a substantial portion of catch and SSB in projections.  The rho adjusted projections reduce all starting   numbers at age to 67\\percent of unadjusted values \(i.e., all 2015 numbers at age are multiplied by 0.667\).  Two other exceptionally large year classes were observed in 2003 and 2010.  The 2003 year class is now estimated to be only 28\\percent of its initial model estimate, while the 2010 year class is now estimated to be 63\\percent of it\'s initial estimate.  Given that only 5 years of data are available to estimate the 2010 year class, it is possible that there may be further revisions to the magnitude of this year class estimate with more years of data.  Therefore, it remains uncertain if the scalar applied to all age classes in these projections \(0.667, based on Mohn\'s rho for SSB\)  is sufficient to account for future revisions to the 2013 year class estimate.  In addition, the median recruitment in the projections \(the proxy for recruitment at MSY\)  is 53.4 million, which is greater than 7 of the last 10 recruitments even though SSB is above the SSBMSY proxy \(Table 1\). While projections of catch and SSB in the near\-term are mostly driven by the 2013 year class, it is worth noting the magnitude of median projected recruitment relative to recent recruitment observations.}  \item{} Does this assessment model have a retrospective pattern? If so, is the pattern minor, or major? \(A major retrospective pattern occurs when the adjusted SSB or  \$F\_{Full}\${} lies outside of the approximate  joint confidence region for SSB and  \$F\_{Full}\${}\). \linebreak{} \hspace\*{0.5cm} \textit{ The 7\-year Mohn\'s  \textrho{}, relative to SSB, was 0.20 in the 2012 assessment and was 0.50 in 2014. The 7\-year Mohn\'s  \textrho{}, relative to F, was \-0.15 in the 2012 assessment and was \-0.34 in 2014. There was a major retrospective pattern for this assessment because the  \textrho{} adjusted estimates of 2014 SSB \(\$SSB\_{\rho}\${}=150,053\)  and 2014 F \(\$F\_{\rho}\${}=0.241\)  were outside the approximate 90\\percent confidence regions around SSB \(171,911 \- 301,282\)  and F \(0.13 \- 0.203\).  A retrospective  adjustment was made for both the determination of stock status and for projections of catch in 2016. The retrospective adjustment changed the 2014 SSB from 225,080 to 150,053 and the 2014  \$F\_{Full}\${} from 0.159 to 0.241.}  \item{}Based on this stock assessment, are population projections well determined or uncertain? \linebreak{} \hspace\*{0.5cm} \textit{As noted in \(1\)  above, population projections for Georges Bank haddock are uncertain due to uncertainty about the size of  the 2013 year class.  Two sensitivity projections were conducted.  The first sensitivity used biological parameters and fishery selectivity values from the 2010 year class for the 2013 year class.  A second sensitivity projection was made that used the same  biological and selectivity parameters as the first sensitivity, and in addition it  doubled the rho\-adjustment on the 2013 year class \(age 2 at the start of 2015\)  by multiplying it by 0.33.  These sensitivity runs are available on the Stock Assessment Supplementary Information  website \(\href{http:\/\/www.nefsc.noaa.gov\/saw\/sasi\/sasi\_report\_options.php}{SASINF}{}\), in the sensitivity slides appended to the end of the background presentation.}  \item{}Describe any changes that were made to the current stock assessment, beyond incorporating additional years of data  and the affect these changes had on the assessment and stock status. \linebreak{} \hspace\*{0.5cm} \textit{ No changes, other than the incorporation of new data were made to the Georges Bank haddock assessment for this update. However, the criterion for determining acceptable tows on NEFSC surveys used the TOGA protocol rather than the SHG protocol  \(TOGA=132x\).}  \item{}If the stock status has changed a lot since the previous assessment, explain why this occurred.  \linebreak{} \hspace\*{0.5cm} \textit{The stock status of Georges Bank haddock has not changed.}  \item{}Indicate what data or studies are currently lacking and which would be needed most to improve this stock assessment in the future.  \linebreak{} \hspace\*{0.5cm} \textit{Projection advice and reference points for  Georges Bank haddock are strongly dependent on recruitment.  A decade ago, extremely large year classes were considered anomalies \(e.g., 1963 and 2003\).   However, since 2003, there have been two more extremely large \(2010 and 2013\)  and one very large \(2012\)  year classes.  Future work could focus on recruitment forecasting and providing robust catch advice.}  \item{}Are there other important issues? \linebreak{} \hspace\*{0.5cm} \textit{The Georges Bank haddock assessment has recently developed a major retrospective pattern.  This stock assessment has historically performed  very consistently.  This should continue to be monitored.  Density\-dependent responses in growth should also continue to be monitored.  The switch from SHG to TOGA was ruled out as the cause of the retrospective pattern.} \end{itemize}{}} \def\HADGBRefr{ \textbf{References: }{} \linebreak{}Brooks, E.N, M.L. Traver, S.J. Sutherland, L. Van Eeckhaute, and L. Col.  2008.  In.  Northeast Fisheries Science Center. 2008. Assessment of 19 Northeast Groundfish Stocks through 2007: Report of the 3$^{rd}$ Groundfish Assessment Review Meeting \(GARM III\), Northeast Fisheries Science Center, Woods Hole, Massachusetts, August 4\-8, 2008. US Dep Commer, NOAA Fisheries, Northeast Fish Sci Cent Ref Doc. 08\-15\; 884 p + xvii. http:\/\/www.nefsc.noaa.gov\/publications\/crd\/crd0815\/ \linebreak{} \linebreak{}Brooks, E.N, S.J. Sutherland, L. Van Eeckhaute, and M. Palmer.  2012.  In.  Northeast Fisheries Science Center. 2012. Assessment or Data Updates of 13 Northeast Groundfish Stocks through 2010. US Dept Commer, NOAA Fisheries, Northeast Fish Sci Cent Ref Doc. 12\-06.\; 789 p. http:\/\/nefsc.noaa.gov\/publications\/crd\/crd1206\/ \linebreak{} \linebreak{}} \def\HADGBDraft{} \def\HADGBSPPname{Georges Bank haddock} \def\HADGBSPPnameT{Georges Bank haddock} \def\HADGBRptYr{2015} \def\HADGBAuthor{Liz Brooks} \def\HADGBReviewerComments{/home/dhennen/EIEIO/BigReport/HAD_GB/latex}  \def\HADGMMyPathTab{/home/dhennen/EIEIO/BigReport/HAD_GM/tables} \def\HADGMMyPathFig{/home/dhennen/EIEIO/BigReport/HAD_GM/figures} \def\HADGMfigFishCap{Total catch of Gulf of Maine haddock between 1977 and 2014 by fleet \(commercial, recreational, or foreign\)  and disposition \(landings and discards\).} \def\HADGMfigSSBCap{Trends in spawning stock biomass \(SSB\)  of Gulf of Maine haddock between 1977 and 2014 from the current  \(solid line\)  and previous \(dashed line\)  assessment and the corresponding  \$SSB\_{Threshold}\${} \(\$\dfrac{1}{2}\${} \$SSB\_{MSY}\${} \textit{proxy}{}\; horizontal dashed line\)  as well as  \$SSB\_{Target}\${} \(\$SSB\_{MSY}\${} \textit{proxy}{}\; horizontal dotted line\)   based on the 2015 assessment. The approximate 90\\percent lognormal confidence intervals are shown. The red dot indicates the rho\-adjusted SSB values that would have resulted had a retrospective adjusment been made to either model \(see Special Comments section\).} \def\HADGMfigFCap{Trends in the fully selected fishing mortality \(F\)  of Gulf of Maine haddock between 1977 and 2014 from the current  \(solid line\)  and previous \(dashed line\)  assessment and the corresponding  \$F\_{Threshold}\${} \(\$F\_{MSY}\${} \textit{proxy}{}=0.468\; horizontal dashed line\)  from the 2015 assessment model. The approximate 90\\percent lognormal confidence intervals are shown. The red dot indicates the rho\-adjusted F values that would have resulted had a retrospective adjusment been made to either model \(see Special Comments section\).} \def\HADGMfigRecrCap{Trends in Recruits \(age 1\)  \(000s\)  of Gulf of Maine haddock between 1977 and 2014 from the current \(solid line\)  and previous \(dashed line\)  assessment. The approximate 90\\percent lognormal confidence intervals are shown.} \def\HADGMfigSurvCap{Indices of biomass for the Gulf of Maine haddock between 1963 and 2015 for the Northeast Fisheries Science Center \(NEFSC\)  spring and fall bottom trawl surveys.  The approximate 90\\percent lognormal confidence intervals are shown.} \def\HADGMPreAmb{This assessment of the Gulf of Maine haddock \(\textit{Melanogrammus aeglefinus}\)  stock is an operational update of the existing 2014 benchmark assessment \(NEFSC 2014\). Based on the previous assessment, the stock was not overfished, and overfishing was not ocurring. This assessment updates commercial and recreational fishery catch data, research survey indices of abundance, and the analytical ASAP assessment model and reference points through 2014. Additionally, stock projections have been updated through 2018} \def\HADGMSoS{ \textbf{State of Stock: }{}Based on this updated assessment, the Gulf of Maine haddock \(\textit{Melanogrammus aeglefinus}\)  stock is not overfished and overfishing is not occurring \(Figures \ref{HADGMSSB\_plot1}\-\ref{HADGMF\_plot1}\){}. Retrospective adjustments were not made to the model results \(see Special Comments section of this report\). Spawning stock biomass \(SSB\)  in 2014 was estimated to be 10,325 \(mt\)  which is 223\\percent of the biomass target \(\$SSB\_{MSY}\${} \textit{proxy}{} = 4,623\;  Figure \ref{HADGMSSB\_plot1}{}\).  The 2014 fully selected fishing mortality was estimated to be 0.257 which is 55\\percent of the overfishing threshold proxy \(\$F\_{MSY}\${} \textit{proxy}{} =  \$F\_{40\\percent}\${} = 0.468\;  Figure \ref{HADGMF\_plot1}{}\).} \def\HADGMProj{ \textbf{Projections: }{}Short term projections of median total fishery yield and spawning stock biomass for Gulf of Maine haddock were conducted based on a harvest scenario of fishing at the  \$F\_{MSY}\${} \textit{proxy}{} between 2016 and 2018. Catch in 2015 has been estimated at 885 mt. Recruitment was sampled from a cumulative distribution  function of model estimated age\-1 recruitment from 1977\-2012. The age\-1 estimate in 2015 was generated from the geometric mean of the 1977\-2014 recruitment series. The annual fishery selectivity, maturity ogive, and mean weights at age used in the projections  were estimated from the most recent 5 year averages\;  retrospective adjustments were not applied in the projections. Given the uncertainty in the size of the 2012 and 2013 year classes and the model\'s tendency to overestimate large terminal year classes, the 2015 assessment review panel recommended that a sensitivity projection scenario which constrains terminal recruitment \(\'Constrain terminal R\'\)  be brought forward to the New England Fishery Management Council\'s Scientific and Statistical Committee \(NEFMC SSC\)  for consideration when setting catch advice\; these sensitivity projections are provided in the Supplemental Information Report \(\href{http:\/\/www.nefsc.noaa.gov\/saw\/sasi\/sasi\_report\_options.php}{SASINF}{}\).} \def\HADGMSpecCmt{ \textbf{Special Comments: } \begin{itemize}{} \item{}What are the most important sources of uncertainty in this stock assessment?  Explain, and describe qualitatively how they affect the assessment results \(such as estimates of biomass, F, recruitment, and population projections\).  \linebreak{} \hspace\*{0.5cm} \textit{ The largest source of uncertainty in the assessment is the estimated size of the 2012 and 2013 year classes. Based on the estimated selectivity patterns, these year classes are projected to be 30\\percent selected to the fishery in 2016 and 2017 respectively. However, recent changes to the commercial and recreational minimum retention size may result in these year classes recruiting to the fishery sooner than projected. The abundance and growth of the 2012 and 2013 year classes should be monitored and frequent model updates would be expected to improve the estimates of year class size and validate projection assumptions.}  \item{}Does this assessment model have a retrospective pattern? If so, is the pattern minor, or major? \(A major retrospective pattern occurs when the adjusted SSB or  \$F\_{Full}\${} lie outside of the approximate joint confidence region for SSB and  \$F\_{Full}\${}\). \linebreak{} \hspace\*{0.5cm} \textit{This assessment does not exhibit a retrospective pattern and therefore no retrospective adjustments were made to the terminal model results or the short\-term catch projections. The 7\-year Mohn\'s rho values on SSB \(\-0.04\)  and F \(0.03\)  are small and there were no consistent patterns in the directionality of the retrospective \'peels\' \(see the Supplemental Information Report, \href{http:\/\/www.nefsc.noaa.gov\/saw\/sasi\/sasi\_report\_options.php}{SASINF}{}\).}  \item{}Based on this stock assessment, are population projections well determined or uncertain?  \linebreak{} \hspace\*{0.5cm} \textit{Population projections for Gulf of Maine haddock, are reasonably well determined. The projected boimass from the last assessment is below the confidence bounds of the biomass estimated in the current assessment\; however, this is primarily due to the positive rescaling of the population size that occured from turning the ASAP model likelihood constants option off \(see next Special Comment\).}  \item{}Describe any changes that were made to the current stock assessment, beyond incorporating additional years of data  and the affect these changes had on the assessment and stock status. \linebreak{} \hspace\*{0.5cm} \textit{ Recreational catch estimates from 2004\-2014 were re\-estimated as part of this update to account for updates to the MRIP data. Additionally, the ASAP model was revised by turning the likelihood constants off\; sensitivity runs on SAW\/SARC 59 model suggest minor positive rescaling of recruitment and SSB, negative rescaling of F \(sensitivity results are provided in the Supplemental Information Report, \href{http:\/\/www.nefsc.noaa.gov\/saw\/sasi\/sasi\_report\_options.php}{SASINF}{}\).}  \item{}If the stock status has changed a lot since the previous assessment, explain why this occurred.  \linebreak{} \hspace\*{0.5cm} \textit{There has been no change in stock status since the previous SAW\/SARC 59 assessment \(2014\).}  \item{}Indicate what data or studies are currently lacking and which would be needed most to improve this stock assessment in the future.  \linebreak{} \hspace\*{0.5cm} \textit{Currently the assessment assumes 50\\percent survival of haddock discarded in the recreational fishery \- directed field research would improve this estimate. Additionally, a better understanding of recruitment processes may help to improve recruitment forecasting.}  \item{}Are there other important issues? \linebreak{} \hspace\*{0.5cm} \textit{None.} \end{itemize}{}} \def\HADGMRefr{ \textbf{References: }{} \linebreak{}Northeast Fisheries Science Center. 2014. 59$^{th}$ Northeast Regional Stock Assessment Workshop \(59$^{th}$ SAW\)  Assessment Report. US Dept Commer, Northeast Fish Sci Cent Ref Doc. 14\-09\; 782 p. Available from: National Marine Fisheries Service, 166 Water Street, Woods Hole, MA 02543\-1026 \linebreak{} \linebreak{}} \def\HADGMDraft{} \def\HADGMSPPname{Gulf of Maine haddock} \def\HADGMSPPnameT{Gulf of Maine haddock} \def\HADGMRptYr{2015} \def\HADGMAuthor{Michael Palmer} \def\HADGMReviewerComments{/home/dhennen/EIEIO/BigReport/HAD_GM/latex}  \def\YELGBMyPathTab{/home/dhennen/EIEIO/BigReport/YEL_GB/tables} \def\YELGBMyPathFig{/home/dhennen/EIEIO/BigReport/YEL_GB/figures} \def\YELGBfigFishCap{Total catch of Georges Bank Yellowtail Flounder between 1935 and 2014 by fleet \(US, Canadian, or Other\)  and disposition \(landings or discards\).} \def\YELGBfigSSBCap{Trends in average survey biomass \(mt\)  of Georges Bank Yellowtail Flounder between 2010 and 2015 from the current assessment.} \def\YELGBfigFCap{Trends in the exploitation rate \(catch\/average survey biomass\)  of Georges Bank Yellowtail Flounder between 2010 and 2014 from the current assessment.} \def\YELGBfigRecrCap{} \def\YELGBfigSurvCap{Indices of biomass for the Georges Bank Yellowtail Flounder between 1963 and 2015 for the Canadian DFO and Northeast Fisheries Science Center \(NEFSC\)  spring and fall bottom trawl surveys.  The approximate 90\\percent lognormal confidence intervals are shown.} \def\YELGBPreAmb{This assessment of the Georges Bank Yellowtail Flounder \(\textit{Limanda ferruginea}\)  stock was reviewed during the July 2015 TRAC meeting \(Legault et al. 2015\). It is an operational update of the existing 2014 update assessment \(Legault et al. 2014\). Based on the previous assessment the stock status was unknown, but stock condition was poor. This assessment updates commercial fishery catch data through 2014 \(Table \ref{YELGBCatch\_Status\_Table}{},  Figure \ref{YELGBFish\_plot1}{}\), and updates research survey indices of abundance and the empirical approach assessment through 2015 \(Figure \ref{YELGBSurv\_plot1}{}\). No stock projections can be computed using the empirical approach.} \def\YELGBSoS{ \textbf{State of Stock: }{}Based on this updated assessment, Georges Bank Yellowtail Flounder \(\textit{Limanda ferruginea}\)  stock status is unknown due to a lack of biological reference points associated with the empirical approach, but stock condition is poor.  Retrospective adjustments were not made to the model results. The average survey biomass in 2015 \(the arithmetic average of the 2015 DFO, 2015 NEFSC spring, and 2014 NEFSC fall surveys\)  was estimated to be 2,240 \(mt\)  \(Figure \ref{YELGBSSB\_plot1}{}\).  The 2014 exploitation rate \(2014 catch divided by 2014 average survey biomass\)  was estimated to be 0.071 \(Figure \ref{YELGBF\_plot1}{}\).} \def\YELGBProj{ \textbf{Projections: }{}Short term projections cannot be computed using the empirical approach. Application of an exploitation rate of 2\\percent to 16\\percent to the 2015 average survey biomass \(2,240 mt\)  results in catch advice for 2016 of 45 mt to 359 mt.} \def\YELGBSpecCmt{ \textbf{Special Comments: } \begin{itemize}{} \item{}What are the most important sources of uncertainty in this stock assessment?  Explain, and describe qualitatively how they affect the assessment results \(such as estimates of biomass, F, recruitment, and population projections\).  \linebreak{} \hspace\*{0.5cm} \textit{The largest source of uncertainty is the estimate of survey catchability, which currently relies on literature values for other species in other regions of the world using different gear. The survey catchability affects the expansion of the stratified mean catch per tow for each survey and is inversely related to the catch advice. Other sources of uncertainty include the appropriate exploitation rate to apply to this stock, which has seen continued decrease in survey biomass despite low exploitation rates. }  \item{} Does this assessment model have a retrospective pattern? If so, is the pattern minor, or major? \(A major retrospective pattern occurs when the adjusted SSB or  \$F\_{Full}\${} lies outside of the approximate  joint confidence region for SSB and  \$F\_{Full}\${}\; see RhoDecisionTab.ref\). \linebreak{} \hspace\*{0.5cm} \textit{ The model used to estimate status of this stock does not allow estimation of a retrospective pattern. }  \item{}Based on this stock assessment, are population projections well determined or uncertain? \linebreak{} \hspace\*{0.5cm} \textit{Population projections for Georges Bank Yellowtail Flounder are not computed. Catch advice is derived from applying an exploitation rate to the current estimate of survey biomass. }  \item{}Describe any changes that were made to the current stock assessment, beyond incorporating additional years of data  and the affect these changes had on the assessment and stock status. \linebreak{} \hspace\*{0.5cm} \textit{The 2014 NMFS spring survey value was changed from 2,684 mt to 2,763 mt due to using preliminary data during the 2014 TRAC meeting. However, this has no impact on the 2015 stock status or 2016 catch advice in this update assessment.}  \item{}If the stock status has changed a lot since the previous assessment, explain why this occurred.  \linebreak{} \hspace\*{0.5cm} \textit{The stock status of Georges Bank Yellowtail Flounder remains unknown and stock condition continues to be poor.}  \item{}Indicate what data or studies are currently lacking and which would be needed most to improve this stock assessment in the future.  \linebreak{} \hspace\*{0.5cm} \textit{The Georges Bank Yellowtail Flounder assessment could be improved with studies on NMFS and DFO survey catchability for flatfish.}  \item{}Are there other important issues? \linebreak{} \hspace\*{0.5cm} \textit{None. } \end{itemize}{}} \def\YELGBRefr{ \textbf{References: }{} \linebreak{}Legault, C.M., L. Alade, W.E. Gross, and H.H. Stone. 2014. Stock Assessment of Georges Bank Yellowtail Flounder for 2014. TRAC Ref. Doc. 2014\/01. 214 p. \linebreak{}Legault, C.M., L. Alade, D. Busawon, and H.H. Stone. 2015. Stock Assessment of Georges Bank Yellowtail Flounder for 2015. TRAC Ref. Doc. 2015\/01. 66 p. \linebreak{}} \def\YELGBDraft{} \def\YELGBSPPname{Georges Bank Yellowtail Flounder} \def\YELGBSPPnameT{Georges Bank Yellowtail Flounder} \def\YELGBRptYr{2015} \def\YELGBAuthor{Chris Legault} \def\YELGBReviewerComments{/home/dhennen/EIEIO/BigReport/YEL_GB/latex}  \def\YELSNEMAMyPathTab{/home/dhennen/EIEIO/BigReport/YEL_SNEMA/tables} \def\YELSNEMAMyPathFig{/home/dhennen/EIEIO/BigReport/YEL_SNEMA/figures} \def\YELSNEMAfigFishCap{Total catch of Southern New England\-Mid Atlantic Yellowtail flounder between 1973 and 2014 by fleet \(US domestic and foreign catch\)  and disposition \(landings and discards\).} \def\YELSNEMAfigSSBCap{Trends in spawning stock biomass of Southern New England\-Mid Atlantic Yellowtail flounder between 1973 and 2014 from the current  \(solid line\)  and previous \(dashed line\)  assessment and the corresponding  \$SSB\_{Threshold}\${} \(\$\dfrac{1}{2}\${} \$SSB\_{MSY}\${} \textit{proxy}{}\; horizontal dashed line\)  as well as  \$SSB\_{Target}\${} \(\$SSB\_{MSY}\${} \textit{proxy}{}\; horizontal dotted line\)   based on the 2015 assessment.  Biomass was adjusted for a retrospective pattern  and the adjustment is shown in red.  The approximate 90\\percent lognormal confidence intervals are shown.} \def\YELSNEMAfigFCap{Trends in the fully selected fishing mortality \(\$F\_{Full}\${}\)  of Southern New England\-Mid Atlantic Yellowtail flounder between 1973 and 2014 from the current  \(solid line\)  and previous \(dashed line\)  assessment and the corresponding  \$F\_{Threshold}\${} \(\$F\_{MSY}\${} \textit{proxy}{}=0.35\; horizontal dashed line\).  \$F\_{Full}\${} was adjusted for a retrospective pattern  and the adjustment is shown in red  based on the 2015 assessment. The approximate 90\\percent lognormal confidence intervals are shown.} \def\YELSNEMAfigRecrCap{Trends in Recruits \(age 1\)  \(000s\)  of Southern New England\-Mid Atlantic Yellowtail flounder between 1973 and 2014 from the current \(solid line\)  and previous \(dashed line\)  assessment. The approximate 90\\percent lognormal confidence intervals are shown.} \def\YELSNEMAfigSurvCap{Indices of biomass for the Southern New England\-Mid Atlantic Yellowtail flounder between 1973 and 2015 for the Northeast Fisheries Science Center \(NEFSC\)  spring, fall and winter bottom trawl surveys.  The approximate 90\\percent lognormal confidence intervals are shown.Note:  Larval index was also used in this assessment and is available in the supplemental documentation} \def\YELSNEMAPreAmb{This assessment of the Southern New England\-Mid Atlantic Yellowtail flounder \(\textit{Limanda ferruginea}\)  stock is an operational update of the existing 2012 benchmark ASAP assessment \(NEFSC 2012\). Based on the previous assessment the stock was not overfished, and overfishing was not ocurring. This assessment updates commercial fishery catch data, research survey indices of abundance, weights at age and the analytical ASAP assessment model and reference points through 2014. Additionally, stock projections have been updated through 2018} \def\YELSNEMASoS{ \textbf{State of Stock: }{}Based on this updated assessment, Southern New England\-Mid Atlantic Yellowtail flounder \(\textit{Limanda ferruginea}\)  stock is overfished and overfishing is occurring \(Figures \ref{YELSNEMASSB\_plot1}\-\ref{YELSNEMAF\_plot1}\){}. Retrospective adjustments were not made to the model results. Spawning stock biomass \(SSB\)  in 2014 was estimated to be 502 \(mt\)  which is 26\\percent of the biomass target \(\$SSB\_{MSY}\${} \textit{proxy}{} = 1,959\;  Figure \ref{YELSNEMASSB\_plot1}{}\). The 2014 fully selected fishing mortality was estimated to be 1.64 which is 469\\percent of the overfishing threshold proxy \(\$F\_{MSY}\${} \textit{proxy}{} = 0.35\;  Figure \ref{YELSNEMAF\_plot1}{}\).} \def\YELSNEMAProj{ \textbf{Projections: }{}Short term projections of biomass were derived by sampling from a cumulative  distribution function of recruitment estimates from ASAP.  Following the previous and accepted benchmark formulation, recruitment was based on the more recent estimates of the model time series \(i.e. corresponding to  year classes 1990 through 2013\)  to reflect the low recent pattern in recruitment. The annual fishery selectivity, maturity ogive, and mean weights at age used  in projection  are the most recent 5 year averages\;  retrospective adjustments were not applied in the projections.} \def\YELSNEMASpecCmt{ \textbf{Special Comments: } \begin{itemize}{} \item{}What are the most important sources of uncertainty in this stock assessment?  Explain, and describe qualitatively how they affect the assessment results \(such as estimates of biomass, F, recruitment, and population projections\).  \linebreak{} \hspace\*{0.5cm} \textit{The largest source of uncertainty is the emergence of the retrospective in this updated assessment.  This retrospective bias has resulted in the reduction SSB estimates and F estimates to increase with additional years of data  Further, the basis for recruitment assumption for stock status determination and population forecast   \(i.e. the inclusion of historical recruitment values versus contemporary basis of recruitment\)   is another source of uncertainty.  Although recent estmated recruitment likely reflect the realistic conditions for the stock, the basis for recruitment selection is not clearly understood.}  \item{} Does this assessment model have a retrospective pattern? If so, is the pattern minor, or major? \(A major retrospective pattern occurs when the adjusted SSB or  \$F\_{Full}\${} lies outside of the approximate  joint confidence region for SSB and  \$F\_{Full}\${}\; see RhoDecisionTab.ref\). \linebreak{} \hspace\*{0.5cm} \textit{ The 7\-year Mohn\'s  \textrho{}, relative to SSB, was 0.14 in the 2012 assessment and was 1.06 in 2014. The 7\-year Mohn\'s  \textrho{}, relative to F, was \-0.16 in the 2012 assessment and was \-0.53 in 2014. There was a major retrospective pattern for this assessment because the  \textrho{} adjusted estimates of 2014 SSB \(\$SSB\_{\rho}\${}=502\)  and 2014 F \(\$F\_{\rho}\${}=1.64\)  were outside the approximate 90\\percent confidence regions around SSB \(355 \- 739\)  and F \(1.053 \- 2.348\).  However, a retrospective adjustment was not made for both the determination of stock status and for projections of catch because of the large proportion of unfeasible projections \(assumed 2015 catch required a fishing mortality rate greater than 5\). This implies the retrospective adjustment was too large or the assumed 2015 catch was too high. The review panel decided to use the unadjusted projections as an upper bound for OFL with the strong suggestion that the OFL estimates were too high \(meaning the ABC buffer should be larger than normal\).}  \item{}Based on this stock assessment, are population projections well determined or uncertain? \linebreak{} \hspace\*{0.5cm} \textit{Population projections are uncertain with projected biomass from the last assessment above the confidence bounds of the biomass estimate in the current assessment.  Further, the short\-term projections which accounted for retropective adjustment in the starting numbers\-at\-age were unrelaible due to the low percentage of feasible solutions \(33\\percent\)  encountered durring the simulation. The feasibility problem in the projections were due to the assumed 2015 projected cacth exceeding the population biomass in several of the iteration caused by the retrospective adjustment. Evaluation of the the estimated January\-1 2015 biomass from the few feasbile projections indicated that the assumed 2015 catch was approximately 98\\percent of the stock biomass.  This suggests that the assumed 2015 catch is not sustainable given the low starting abundance in the forecast. Alternatively, the retro unadjusted projections performed well, but it is likely to result in an overly optimistic projection of the fishery yield and population biomass.}  \item{}Describe any changes that were made to the current stock assessment, beyond incorporating additional years of data  and the affect these changes had on the assessment and stock status. \linebreak{} \hspace\*{0.5cm} \textit{ There were no major changes to the current stock assessment formulation. However, the criterion for determining acceptable tows on the NEFSC surveys were revised for years the Bigelow year \(i.e. 2009\-2011\)  and carried foreward to ensure consistency between the assessment and deck operations.  The influence of the revised protocol on the survey indices was inconsequential.}  \item{}If the stock status has changed a lot since the previous assessment, explain why this occurred.  \linebreak{} \hspace\*{0.5cm} \textit{The overfishing and biomass stock status have changed since the previous assessment due to increased catches relative to the stock biomass and the very low recruitment of young fish, contributing very little to the adult biomass.}  \item{}Indicate what data or studies are currently lacking and which would be needed most to improve this stock assessment in the future.  \linebreak{} \hspace\*{0.5cm} \textit{The emergence of retrospective bias in this assessment is not clearly understood and may result from a variety of sources.  Future studiesshould further investigate the source of this retrospective pattern to help improve the underlying diagnostics of the model for providing catch advice for this stock.  Recruitment for Southern New England\-Mid Atlantic yellowtail flounder continues to be weak and it is likely that the stock is in a new productivity regime.  Should this pattern of poor recruitment continue into the future, the ability of the stock to recover will be impeded. Therefore, future studies should build on current knowledge to further understand the underlying ecological mechanisms of poor recruitment in the stock as it may relate to the physical environment.}  \item{}Are there other important issues? \linebreak{} \hspace\*{0.5cm} \textit{None. } \end{itemize}{}} \def\YELSNEMARefr{ \textbf{References: }{}  \linebreak{} Alade, L,  C. Legault, S.Cadrin.  2008.  In.  Northeast Fisheries Science Center. 2008. Assessment of 19 Northeast Groundfish Stocks  through 2007: Report of the 3$^{rd}$ Groundfish Assessment Review Meeting \(GARM III\), Northeast Fisheries Science Center, Woods  Hole, Massachusetts, August 4\-8, 2008. US Dep Commer, NOAA Fisheries, Northeast Fish Sci Cent Ref Doc. 08\-15\; 884 p + xvii.  http:\/\/www.nefsc.noaa.gov\/publications\/crd\/crd0815\/  \linebreak{}  \linebreak{} Northeast Fisheries Science Center. 2012.  54$^{th}$ Northeast Regional Stock Assessment Workshop \(54$^{th}$ SAW\)  Assessment Report. US Dept Commer, NOAA Fisheries, Northeast Fish Sci Cent Ref Doc. 12\-18.\; 600 p.  http:\/\/nefsc.noaa.gov\/publications\/crd\/crd1218\/  \linebreak{}  \linebreak{}} \def\YELSNEMADraft{} \def\YELSNEMASPPname{Southern New England-Mid Atlantic Yellowtail flounder} \def\YELSNEMASPPnameT{Southern New England-Mid Atlantic Yellowtail flounder} \def\YELSNEMARptYr{2015} \def\YELSNEMAAuthor{Larry Alade} \def\YELSNEMAReviewerComments{/home/dhennen/EIEIO/BigReport/YEL_SNEMA/latex}  \def\YELCCGMMyPathTab{/home/dhennen/EIEIO/BigReport/YEL_CCGM/tables} \def\YELCCGMMyPathFig{/home/dhennen/EIEIO/BigReport/YEL_CCGM/figures} \def\YELCCGMfigFishCap{Total catch of Cape Cod\-Gulf of Maine Yellowtail flounder between 1985 and 2014 by disposition \(landings and discards\).} \def\YELCCGMfigSSBCap{Trends in spawning stock biomass of Cape Cod\-Gulf of Maine Yellowtail flounder between 1985 and 2014 from the current  \(solid line\)  and previous \(dashed line\)  assessment and the corresponding  \$SSB\_{Threshold}\${} \(\$\dfrac{1}{2}\${} \$SSB\_{MSY}\${} \textit{proxy}{}\; horizontal dashed line\)  as well as  \$SSB\_{Target}\${} \(\$SSB\_{MSY}\${} \textit{proxy}{}\; horizontal dotted line\)   based on the 2015 assessment.  Biomass was adjusted for a retrospective pattern  and the adjustment is shown in red.   The 90\\percent bootstrap probability intervals are shown.} \def\YELCCGMfigFCap{Trends in the fully selected fishing mortality \(\$F\_{Full}\${}\)  of Cape Cod\-Gulf of Maine Yellowtail flounder between 1985 and 2014 from the current  \(solid line\)  and previous \(dashed line\)  assessment and the corresponding  \$F\_{Threshold}\${} \(\$F\_{MSY}\${} \textit{proxy}{}=0.279\; horizontal dashed line\).  \$F\_{Full}\${} was adjusted for a retrospective pattern  and the adjustment is shown in red  based on the 2015 assessment.  The 90\\percent bootstrap probability intervals are shown.} \def\YELCCGMfigRecrCap{Trends in Recruits \(age 1\)  \(000s\)  of Cape Cod\-Gulf of Maine Yellowtail flounder between 1985 and 2014 from the current \(solid line\)  and previous \(dashed line\)  assessment.  The 90\\percent bootstrap probability intervals are shown.} \def\YELCCGMfigSurvCap{Indices of biomass for the Cape Cod\-Gulf of Maine Yellowtail flounder between 1985 and 2015 for the Northeast Fisheries Science Center \(NEFSC\)  spring and fall bottom trawl surveys,  Massachusetts Department of Marine Fisheries \(MADMF\)  inshore state spring and fall bottom trawl surveys,and the Maine New Hampshire inshore state spring and fall state surveys  The 90\\percent bootstrap probability intervals are shown.} \def\YELCCGMPreAmb{This assessment of the Cape Cod\-Gulf of Maine Yellowtail flounder \(\textit{Limanda ferruginea}\)  stock is an operational update of the existing 2012 VPA assessment \(Legault et al., 2012\). The last benchmark for this stock was in 2008 \(Legault et al., 2008\). Based on the previous assessment the stock was overfished, and overfishing was ocurring. This assessment updates commercial fishery catch data, research survey indices of abundance, weights at age, and the analytical VPA assessment model and reference points through 2014. Additionally, stock projections have been updated through 2018} \def\YELCCGMSoS{ \textbf{State of Stock: }{}Based on this updated assessment, Cape Cod\-Gulf of Maine Yellowtail flounder \(\textit{Limanda ferruginea}\)  stock is overfished and overfishing is occurring \(Figures \ref{YELCCGMSSB\_plot1}\-\ref{YELCCGMF\_plot1}\){}.  Retrospective adjustments were made to the model results.  Spawning stock biomass \(SSB\)  in 2014 was estimated to be 857 \(mt\)  which is 16\\percent of the biomass target \(\$SSB\_{MSY}\${} \textit{proxy}{} = 5,259\;  Figure \ref{YELCCGMSSB\_plot1}{}\).  The 2014 fully selected fishing mortality was estimated to be 0.64 which is 229\\percent of the overfishing threshold proxy \(\$F\_{MSY}\${} \textit{proxy}{} = 0.279\;  Figure \ref{YELCCGMF\_plot1}{}\).} \def\YELCCGMProj{ \textbf{Projections: }{}Short term projections of biomass were derived by sampling from a cumulative  distribution function of recruitment estimates from ADAPT VPA. Recruitment estimates were hindcasted based on a simple linear regression between the NEFSC Fall survey abundance at age 1 and the VPA estimate at age 1.  The most recent two years \(2013 and 2014\)  were not included in the series of values due to high uncertainty in these estimates. This resulted in a total of 36 recruitment values: 8 from the hindcast predictions \(years 1977\-1984\)  and 28 from the VPA \(years 1985\-2012\). The annual fishery selectivity, maturity ogive, and mean weights at age used  in projection  are the most recent 5 year averages\;  retrospective adjustments were applied in the projections.} \def\YELCCGMSpecCmt{ \textbf{Special Comments: } \begin{itemize}{} \item{}What are the most important sources of uncertainty in this stock assessment?  Explain, and describe qualitatively how they affect the assessment results \(such as estimates of biomass, F, recruitment, and population projections\).  \linebreak{} \hspace\*{0.5cm} \textit{The largest source of uncertainty is the source of the retrospective pattern.This pattern has persisted for a number of years causing SSB estimates to decrease and F estimates to increaseas more years of data are added.}  \item{} Does this assessment model have a retrospective pattern? If so, is the pattern minor, or major? \(A major retrospective pattern occurs when the adjusted SSB or  \$F\_{Full}\${} lies outside of the approximate  joint confidence region for SSB and  \$F\_{Full}\${}\; see RhoDecisionTab.ref\). \linebreak{} \hspace\*{0.5cm} \textit{ The 7\-year Mohn\'s  \textrho{}, relative to SSB, was 0.68 in the 2012 assessment and was 0.98 in 2014. The 7\-year Mohn\'s  \textrho{}, relative to F, was \-0.19 in the 2012 assessment and was \-0.45 in 2014. There was a major retrospective pattern for this assessment because the  \textrho{} adjusted estimates of 2014 SSB \(\$SSB\_{\rho}\${}=857\)  and 2014 F \(\$F\_{\rho}\${}=0.64\)  were outside the approximate 90\\percent confidence regions around SSB \(1,375 \- 2,111\)  and F \(0.25 \- 0.52\).  A retrospective  adjustment was made for both the determination of stock status and for projections of catch in 2016. The retrospective adjustment changed the 2014 SSB from 1,695 to 857 and the 2014  \$F\_{Full}\${} from 0.355 to 0.64.}  \item{}Based on this stock assessment, are population projections well determined or uncertain? \linebreak{} \hspace\*{0.5cm} \textit{Population projections for Cape Cod\-Gulf of Maine Yellowtail flounder, are uncertain with projected biomass from the last assessmentabove the confidence bounds of the biomass estimated in the current assessment.}  \item{}Describe any changes that were made to the current stock assessment, beyond incorporating additional years of data  and the affect these changes had on the assessment and stock status. \linebreak{} \hspace\*{0.5cm} \textit{ No changes, other than the incorporation of new data were made to the Cape Cod\-Gulf of Maine Yellowtail flounder assessment for this update.}  \item{}If the stock status has changed a lot since the previous assessment, explain why this occurred.  \linebreak{} \hspace\*{0.5cm} \textit{The stock status has not changed since the previous assessment.}  \item{}Indicate what data or studies are currently lacking and which would be needed most to improve this stock assessment in the future.  \linebreak{} \hspace\*{0.5cm} \textit{Extensive studies have examined the causes of the retrospective patterns with no definitive conclusions other than a change in model does not resolve the issue.}  \item{}Are there other important issues? \linebreak{} \hspace\*{0.5cm} \textit{No. } \end{itemize}{}} \def\YELCCGMRefr{ \textbf{References: }{} \linebreak{}Legault, C,  L. Alade, S.Cadrin, J. King, and S. Sherman.  2008.  In.  Northeast Fisheries Science Center. 2008. Assessment of 19 Northeast Groundfish Stocks through 2007: Report of the 3$^{rd}$ Groundfish Assessment Review Meeting \(GARM III\), Northeast Fisheries Science Center, Woods Hole, Massachusetts, August 4\-8, 2008. US Dep Commer, NOAA Fisheries, Northeast Fish Sci Cent Ref Doc. 08\-15\; 884 p + xvii. http:\/\/www.nefsc.noaa.gov\/publications\/crd\/crd0815\/ \linebreak{} \linebreak{} Legault, C,  L. Alade, S.Emery, J. King, and S. Sherman.  2012.  In.  Northeast Fisheries Science Center. 2012. Assessment or Data Updates of 13 Northeast Groundfish Stocks through 2010. US Dept Commer, NOAA Fisheries, Northeast Fish Sci Cent Ref Doc. 12\-06.\; 789 p. http:\/\/nefsc.noaa.gov\/publications\/crd\/crd1206\/ \linebreak{} \linebreak{}} \def\YELCCGMDraft{} \def\YELCCGMSPPname{Cape Cod-Gulf of Maine Yellowtail flounder} \def\YELCCGMSPPnameT{Cape Cod-Gulf of Maine Yellowtail flounder} \def\YELCCGMRptYr{2015} \def\YELCCGMAuthor{Larry Alade} \def\YELCCGMReviewerComments{/home/dhennen/EIEIO/BigReport/YEL_CCGM/latex}  \def\FLWGMMyPathTab{/home/dhennen/EIEIO/BigReport/FLW_GM/tables} \def\FLWGMMyPathFig{/home/dhennen/EIEIO/BigReport/FLW_GM/figures} \def\FLWGMfigFishCap{Total catch of Gulf of Maine Winter Flounder between 2009 and 2014 by fleet \(commercial and recreational\)  and disposition \(landings and discards\). A 15\\percent mortality rate is assumed on recreational discards and a 50\\percent mortality rate on commercial discards.} \def\FLWGMfigSSBCap{Trends in 30+ cm area\-swept biomass of Gulf of Maine Winter Flounder between 2009 and 2014 from the current assessment based on the fall \(MENH, MDMF, NEFSC\)  surveys.  The approximate 90\\percent lognormal confidence intervals are shown.} \def\FLWGMfigFCap{Trends in the exploitation rates \(\$E\_{Full}\${}\)  of Gulf of Maine Winter Flounder between 2009 and 2014 from the current assessment and the corresponding  \$F\_{Threshold}\${} \(\$E\_{MSY}\${} \textit{proxy}{}=0.23\; horizontal dashed line\).  The approximate 90\\percent lognormal confidence intervals are shown.} \def\FLWGMfigRecrCap{} \def\FLWGMfigSurvCap{Indices of biomass for the Gulf of Maine Winter Flounder between 1978 and 2015 for the Northeast Fisheries Science Center \(NEFSC\), Massachusetts Division of Marine Fisheries \(MDMF\), and the Maine New Hampshire \(MENH\)  spring and fall bottom trawl surveys. NEFSC indices are calculated with gear and vessel conversion factors where appropriate.  The approximate 90\\percent lognormal confidence intervals are shown.} \def\FLWGMPreAmb{This assessment of the  Gulf of Maine Winter Flounder  \(\textit{Pseudopleuronectes americanus}\)   stock is an operational update of the  existing  2014  operational update area\-swept assessment \(NEFSC 2014\).  Based on the previous assessment the biomass status is unknown but overfishing was not occurring.  This assessment  updates commercial and recreational fishery catch data, research survey indices  of abundance, and the area\-swept estimates of 30+ cm biomass based on the fall NEFSC, MDMF, and MENH surveys.} \def\FLWGMSoS{ \textbf{State of Stock: }{}Based on this updated assessment, the Gulf of Maine Winter Flounder \(\textit{Pseudopleuronectes americanus}\)  stock biomass status is unknown and overfishing is not occurring \(Figures \ref{FLWGMSSB\_plot1}\-\ref{FLWGMF\_plot1}\){}. Retrospective adjustments were not made to the model results.  Biomass  \(30+ cm mt\)  in 2014 was estimated to be 4,655 mt \(Figure \ref{FLWGMSSB\_plot1}{}\). The 2014 30+ cm exploitation rate was estimated to be 0.06 which is 26\\percent of the overfishing exploitation threshold proxy \(\$E\_{MSY}\${} \textit{proxy}{} = 0.23\;  Figure \ref{FLWGMF\_plot1}{}\).} \def\FLWGMProj{ \textbf{Projections: }{}Projections are not possible with area\-swept based assessments. Catch advice was based on 75\\percent of  \$E\_{40\\percent}\${}\(75\\percent \$E\_{MSY}\${} \textit{proxy}{}\)  using the fall area\-swept estimate assuming q=0.6 on the wing spread. Updated 2014 fall 30+ cm area\-swept biomass \(4,655 mt\)  implies an OFL of 1,080 mt based on the  \$E\_{MSY}\${} \textit{proxy}{} and a catch of 810 mt for 75\\percent of the  \$E\_{MSY}\${} \textit{proxy}{}.} \def\FLWGMSpecCmt{ \textbf{Special Comments: } \begin{itemize}{} \item{}What are the most important sources of uncertainty in this stock assessment?  Explain, and describe qualitatively how they affect the assessment results \(such as estimates of biomass, F, recruitment, and population projections\).  \linebreak{} \hspace\*{0.5cm} \textit{The largest source of uncertainty with the direct estimates of stock biomass from survey area\-swept estimates originate from the assumption of survey gear catchability \(q\). Biomass and exploitation rate estimates are sensitive to the survey q assumption \(0.6 on wing spread\). The 2014 empirical benchmark assessement of Georges bank yellowtail flounder based the area\-swept q assumption on an average value taken from the literature for west coast flatfish \(0.37 on door spread\). The yellowtail q assumption corresponds to a value close to 1 on the wing spread which would result in a lower estimate of biomass \(2,995 mt\). Another major source of uncertainty with this method is that biomass based reference points cannot be determined and overfished status is unknown. }  \item{} Does this assessment model have a retrospective pattern? If so, is the pattern minor, or major? \(A major retrospective pattern occurs when the adjusted SSB or  \$F\_{Full}\${} lies outside of the approximate  joint confidence region for SSB and  \$F\_{Full}\${}\; see  Figure \ref{RhoDecision\_tab}{}\). \linebreak{} \hspace\*{0.5cm} \textit{ The model used to determine status of this stock does not allow estimation of a retrospective pattern.  An analytical stock assessment model does not exist for Gulf of Maine Winter Flounder.  An analytical model was no longer used for stock status determination at SARC 52 \(2011\)  due to concerns with a strong retrospective pattern.  Models have difficulty with the apparent lack of a relationship between a large decrease in the catch with little change in the indices and age and\/or size structure over time. }  \item{}Based on this stock assessment, are population projections well determined or uncertain? \linebreak{} \hspace\*{0.5cm} \textit{Population projections for Gulf of Maine Winter Flounder, do not exist for area\-swept assessments. Catch advice from area\-swept estimates tend to vary with interannual variability in the surveys.}  \item{}Describe any changes that were made to the current stock assessment, beyond incorporating additional years of data  and the affect these changes had on the assessment and stock status. \linebreak{} \hspace\*{0.5cm} \textit{ No changes, other than the incorporation of new data were made to the Gulf of Maine Winter Flounder assessment for this update. However, stabilizing the catch advice may be desired and could be obtained through the averaging of the area\-swept fall and spring survey estimates.}  \item{}If the stock status has changed a lot since the previous assessment, explain why this occurred.  \linebreak{} \hspace\*{0.5cm} \textit{The overfishing status of Gulf of Maine Winter Flounder has not changed. }  \item{}Indicate what data or studies are currently lacking and which would be needed most to improve this stock assessment in the future.  \linebreak{} \hspace\*{0.5cm} \textit{Direct area\-swept assessment could be improved with additional studies on survey gear efficiency.  Quantifying the degree of herding between the doors and escapement under the footrope and\/or above the headrope for each survey is needed since area\-swept biomass estimates and catch advice are sensitive to the assumed catchability.}  \item{}Are there other important issues? \linebreak{} \hspace\*{0.5cm} \textit{The general lack of a response in survey indices and age\/size structure is the primary source of concern with catches remaining far below the overfishing level. } \end{itemize}{}} \def\FLWGMRefr{ \textbf{References: }{} \linebreak{}Hendrickson L, Nitschke P, Linton B. 2015. 2014 Operational Stock Assessments for Georges Bank winter flounder, Gulf of Maine winter flounder, and pollock. US Dept Commer, Northeast Fish Sci Cent Ref Doc. 15\-01\; 228 p. Available from: National Marine Fisheries Service, 166 Water Street, Woods Hole, MA 02543\-1026, or online at http:\/\/nefsc.noaa.gov\/publications\/ \linebreak{} \linebreak{}Northeast Fisheries Science Center. 2011. 52$^{nd}$ Northeast Regional Stock AssessmentWorkshop \(52$^{nd}$ SAW\)  Assessment Report. US Dept Commer, Northeast Fish SciCent Ref Doc. 11\-17\; 962 p. Available from: National Marine Fisheries Service, 166 Water Street, Woods Hole, MA 02543\-1026, or online at http:\/\/www.nefsc.noaa.gov\/nefsc\/publications\/ \linebreak{} \linebreak{}} \def\FLWGMDraft{} \def\FLWGMSPPname{Gulf of Maine Winter Flounder} \def\FLWGMSPPnameT{Gulf of Maine Winter Flounder} \def\FLWGMRptYr{2015} \def\FLWGMAuthor{Paul Nitschke} \def\FLWGMReviewerComments{/home/dhennen/EIEIO/BigReport/FLW_GM/latex}  \def\FLWSNEMAMyPathTab{/home/dhennen/EIEIO/BigReport/FLW_SNEMA/tables} \def\FLWSNEMAMyPathFig{/home/dhennen/EIEIO/BigReport/FLW_SNEMA/figures} \def\FLWSNEMAfigFishCap{Total catch of Southern New England Mid\-Atlantic Winter Flounder between 1981 and 2014 by fleet \(commercial, recreational\)  and disposition \(landings and discards\).} \def\FLWSNEMAfigSSBCap{Trends in spawning stock biomass of Southern New England Mid\-Atlantic Winter Flounder between 1981 and 2014 from the current  \(solid line\)  and previous \(dashed line\)  assessment and the corresponding  \$SSB\_{Threshold}\${} \(\$\dfrac{1}{2}\${} \$SSB\_{MSY}\${} \textit{proxy}{}\; horizontal dashed line\)  as well as  \$SSB\_{Target}\${} \(\$SSB\_{MSY}\${} \textit{proxy}{}\; horizontal dotted line\)   based on the 2015 assessment. The approximate 90\\percent lognormal confidence intervals are shown.} \def\FLWSNEMAfigFCap{Trends in the fully selected fishing mortality \(\$F\_{Full}\${}\)  of Southern New England Mid\-Atlantic Winter Flounder between 1981 and 2014 from the current  \(solid line\)  and previous \(dashed line\)  assessment and the corresponding  \$F\_{Threshold}\${} \(\$F\_{MSY}\${}=0.325\; horizontal dashed line\)   based on the 2015 assessment. The approximate 90\\percent lognormal confidence intervals are shown.} \def\FLWSNEMAfigRecrCap{Trends in Recruits \(age 1\)  \(000s\)  of Southern New England Mid\-Atlantic Winter Flounder between 1981 and 2014 from the current \(solid line\)  and previous \(dashed line\)  assessment. The approximate 90\\percent lognormal confidence intervals are shown.} \def\FLWSNEMAfigSurvCap{Indices of biomass for the Southern New England Mid\-Atlantic Winter Flounder between 1963 and 2014 for the Northeast Fisheries Science Center \(NEFSC\)  spring and fall bottom trawl surveys, the MADMF spring survey, and the CT LISTS survey  The approximate 90\\percent lognormal confidence intervals are shown.} \def\FLWSNEMAPreAmb{This assessment of the Southern New England Mid\-Atlantic Winter Flounder \(\textit{Pseudopleuronectes americanus}\)  stock is an operational update of the existing 2011 benchmark ASAP assessment \(NEFSC 2011\). Based on the previous assessment the stock was overfished, but overfishing was not ocurring. This assessment updates commercial fishery catch data, recreational fishery catch data, and research survey indices of abundance, and the analytical ASAP assessment models and reference points through 2014. Additionally, stock projections have been updated through 2018} \def\FLWSNEMASoS{ \textbf{State of Stock: }{}Based on this updated assessment, the Southern New England Mid\-Atlantic Winter Flounder \(\textit{Pseudopleuronectes americanus}\)  stock is overfished but overfishing is not occurring \(Figures \ref{FLWSNEMASSB\_plot1}\-\ref{FLWSNEMAF\_plot1}\){}. Spawning stock biomass \(SSB\)  in 2014 was estimated to be 6,151 \(mt\)  which is 23\\percent of the biomass target \(26,928 mt\), and 23\\percent of the biomass threshold for an overfished stock \(\$SSB\_{Threshold}\${} = 13464 \(mt\)\;  Figure \ref{FLWSNEMASSB\_plot1}{}\).  The 2014 fully selected fishing mortality was estimated to be 0.16 which is 49\\percent of the overfishing threshold \(\$F\_{MSY}\${} = 0.325\;  Figure \ref{FLWSNEMAF\_plot1}{}\). Retrospective adjustments were not made to the model results. } \def\FLWSNEMAProj{ \textbf{Projections: }{}Short term projections of biomass were derived by sampling from a cumulative  distribution  function of recruitment estimates assuming a Beverton\-Holt stock recruitment relationship. The annual fishery selectivity, maturity ogive, and mean weights at age used  in projection  are the most recent 5 year averages\;  The model exhibited minor retrospective pattern in F and SSB so no retrospective adjustments were applied in the projections.} \def\FLWSNEMASpecCmt{ \textbf{Special Comments: } \begin{itemize}{} \item{}What are the most important sources of uncertainty in this stock assessment?  Explain, and describe qualitatively how they affect the assessment results \(such as estimates of biomass, F, recruitment, and population projections\).  \linebreak{} \hspace\*{0.5cm} \textit{A large source of uncertainty is the estimate of natural mortality based on longevity, which is not well studied in Southern New England Mid\-Atlantic Winter Flounder, and assumed constant over time.  Natural mortality affects the scale of the biomass and fishing mortality estimates.  Natural mortality was adjusted upwards from 0.2 to 0.3 during the last benchmark assessment assuming a max age of 16. However, there is still uncertainty in the true max age of the population and the resulting natural mortality estimate. Other sources of uncertainty include length distribution of the recreational discards.  The recreational discards, are a small component of the total catch, but the assessment suffers from very little length information used to characterize the recreational discards \(1 to 2 lengths in recent years\).}  \item{} Does this assessment model have a retrospective pattern? If so, is the pattern minor, or major? \(A major retrospective pattern occurs when the adjusted SSB or  \$F\_{Full}\${} lies outside of the approximate  joint confidence region for SSB and  \$F\_{Full}\${}\; see  Figure \ref{RhoDecision\_tab}{}\). \linebreak{} \hspace\*{0.5cm} \textit{ No retrospective adjustment of spawning stock biomass or fishing mortality in 2014 was required. }  \item{}Based on this stock assessment, are population projections well determined or uncertain? \linebreak{} \hspace\*{0.5cm} \textit{Population projections for Southern New England Mid\-Atlantic Winter Flounder are reasonably well determined. There is uncertainty in the estimates of M. In addition, while the retrospective pattern is considered minor \(within the 90\\percent CI of both F and SSB\)  the rho adjusted terminal value is very close to falling out of the bounds, becoming a major retrospective pattern. This would lead to retrospective adjustments being needed for the projections.}  \item{}Describe any changes that were made to the current stock assessment, beyond incorporating additional years of data  and the affect these changes had on the assessment and stock status. \linebreak{} \hspace\*{0.5cm} \textit{ No changes, other than the incorporation of new data were made to the Southern New England Mid\-Atlantic Winter Flounder assessment for this update.}  \item{}If the stock status has changed a lot since the previous assessment, explain why this occurred.  \linebreak{} \hspace\*{0.5cm} \textit{The stock status of Southern New England Mid\-Atlantic Winter Flounder has not changed since the previous benchmark in 2011.}  \item{}Indicate what data or studies are currently lacking and which would be needed most to improve this stock assessment in the future.  \linebreak{} \hspace\*{0.5cm} \textit{The Southern New England Mid\-Atlantic Winter Flounder assessment could be improved with additional studies on maximum age, as well additional information  of recreational discard lengths.  In addition, further investigation into the localized struture\/genetics of the stock is warranted. Also, a future shift to ASAP version 4 will provide the ability to model envirionmental factors that may influence both survey catchability and the modeled S\-R relationship}  \item{}Are there other important issues? \linebreak{} \hspace\*{0.5cm} \textit{None. } \end{itemize}{}} \def\FLWSNEMARefr{ \textbf{References: }{} \linebreak{}Smith, A. and S. Jones.  2008.  In.  Northeast Fisheries Science Center. 2008. Assessment of 19 Northeast Groundfish Stocks through 2007: Report of the 3$^{rd}$ Groundfish Assessment Review Meeting \(GARM III\), Northeast Fisheries Science Center, Woods Hole, Massachusetts, August 4\-8, 2008. US Dep Commer, NOAA Fisheries, Northeast Fish Sci Cent Ref Doc. 08\-15\; 884 p + xvii. http:\/\/www.nefsc.noaa.gov\/publications\/crd\/crd0815\/ \linebreak{} \linebreak{}Northeast Fisheries Science Center. 2011. 52$^{nd}$ Northeast Regional Stock AssessmentWorkshop \(52$^{nd}$ SAW\)  Assessment Report. US Dept Commer, Northeast Fish SciCent Ref Doc. 11\-17\; 962 p. Available from: National Marine Fisheries Service, 166Water Street, Woods Hole, MA 02543\-1026, or online at http:\/\/www.nefsc.noaa.gov\/nefsc\/publications\/ \linebreak{} \linebreak{}} \def\FLWSNEMADraft{} \def\FLWSNEMASPPname{Southern New England Mid-Atlantic Winter Flounder} \def\FLWSNEMASPPnameT{Southern New England Mid-Atlantic Winter Flounder} \def\FLWSNEMARptYr{2015} \def\FLWSNEMAAuthor{Anthony Wood} \def\FLWSNEMAReviewerComments{/home/dhennen/EIEIO/BigReport/FLW_SNEMA/latex}  \def\FLWGBMyPathTab{/home/dhennen/EIEIO/BigReport/FLW_GB/tables} \def\FLWGBMyPathFig{/home/dhennen/EIEIO/BigReport/FLW_GB/figures} \def\FLWGBfigFishCap{Total catches \(mt\)  of Georges Bank Winter Flounder between 1982 and 2015 by country and disposition \(landings and discards\).} \def\FLWGBfigSSBCap{Trends in spawning stock biomass \(mt\)  of Georges Bank Winter Flounder between 1982 and 2014 from the current  \(solid line\)  and previous \(dashed line\)  assessments and the corresponding  \$SSB\_{Threshold}\${} \(\$\dfrac{1}{2}\${} \$SSB\_{MSY}\${}\; horizontal dashed line\)  as well as  \$SSB\_{Target}\${} \(\$SSB\_{MSY}\${}\; horizontal dotted line\)   based on the 2015 assessment.  Biomass was adjusted for a retrospective pattern  and the adjustment is shown in red.  The approximate 90\\percent normal confidence intervals are shown.} \def\FLWGBfigFCap{Trends in fully selected fishing mortality \(\$F\_{Full}\${}\)  of Georges Bank Winter Flounder between 1982 and 2014 from the current  \(solid line\)  and previous \(dashed line\)  assessments and the corresponding  \$F\_{Threshold}\${} \(\$F\_{MSY}\${}=0.536\; horizontal dashed line\)  as well as \(\$F\_{Target}\${}= 75\\percent of FMSY\;  horizontal dotted line\). \$F\_{Full}\${} was adjusted for a retrospective pattern  and the adjustment is shown in red.  The approximate 90\\percent normal confidence intervals are also shown.} \def\FLWGBfigRecrCap{Trends in Recruits \(age 1\)  \(000s\)  of Georges Bank Winter Flounder between 1982 and 2014 from the current \(solid line\)  and previous \(dashed line\)  assessments. The approximate 90\\percent normal confidence intervals are shown.} \def\FLWGBfigSurvCap{Indices of biomass for the Georges Bank Winter Flounder for the Northeast Fisheries Science Center \(NEFSC\)  spring \(1968\-2015\)  and fall \(1963\-2014\)   bottom trawl surveys and the Canadian DFO spring survey \(1987\-2015\).  The approximate 90\\percent normal confidence intervals are shown.} \def\FLWGBPreAmb{This assessment of the Georges Bank Winter Flounder \(\textit{Pseudopleuronectes americanus}\)  stock is an operational update of the existing 2014 operational VPA assessment which included data for 1982\-2013 \(Hendrickson et al. 2015\). Based on the previous assessment the stock was not overfished and overfishing was not ocurring. This assessment updates commercial fishery catch data, research survey biomass indices, and the analytical VPA assessment model and reference points through 2014. Additionally, stock projections have been updated through 2018.} \def\FLWGBSoS{ \textbf{State of Stock: }{}Based on this updated assessment, the Georges Bank Winter Flounder \(\textit{Pseudopleuronectes americanus}\)  stock is overfished and overfishing is occurring \(Figures \ref{FLWGBSSB\_plot1}\-\ref{FLWGBF\_plot1}\){}. Retrospective adjustments were made to the model results.  Spawning stock biomass \(SSB\)  in 2014 was estimated to be 2,883 \(mt\)  which is 43\\percent of the biomass target for an overfished stock \(\$SSB\_{MSY}\${} = 6,700 with a threshold of 50\\percent of SSBMSY\;  Figure \ref{FLWGBSSB\_plot1}{}\).  The 2014 fully selected fishing mortality \(F\)  was estimated to be 0.778 which is 145\\percent of the overfishing threshold \(\$F\_{MSY}\${} = 0.536\;  Figure \ref{FLWGBF\_plot1}{}\). However, the 2014 point estimate of SSB and F, when adjusted for retrospective error \(83\\percent for SSB and \-51\\percent for F\), is outside the 90\\percent confidence interval of the unadjusted 2014 point estimate. Therefore, the 2014 F and SSB values used in the stock status determination were the retrospective\-adjusted values of 0.778 and 2,883 mt, respectively.} \def\FLWGBProj{ \textbf{Projections: }{}Short\-term projections of biomass were derived by sampling from a cumulative  distribution  function of recruitment estimates \(1982\-2013 YC\)  from the final run of the ADAPT VPA model. The annual fishery selectivity, maturity ogive, and mean weights\-at\-age used in the projection  are the most recent 5 year averages \(2010\-2014\). An SSB retrospective adjustment factor of 0.546 was applied in the projections.} \def\FLWGBSpecCmt{ \textbf{Special Comments: } \begin{itemize}{} \item{}What are the most important sources of uncertainty in this stock assessment?  Explain, and describe qualitatively how they affect the assessment results \(such as estimates of biomass, F, recruitment, and population projections\).  \linebreak{} \hspace\*{0.5cm} \textit{The largest source of uncertainty is the estimate of natural mortality based on longevity \(max. age = 20 for this stock\), which is not well studied in Georges Bank Winter Flounder, and assumed constant over time.  Natural mortality affects the scale of the biomass and fishing mortality estimates. Other sources of uncertainty include the underestimation of catches. Discards from the Canadian bottom trawl fleet were not provided by the CA DFO and the precision of the Canadian scallop dredge discard estimates, with only 1\-2 trips per month, are uncertain.The lack of age data for the Canadian spring survey catches requires the use of the US spring survey A\/L keys despite selectivity differences. In addition, there are no length or age composition data from the Canadian landings or discards GB winter flounder.}  \item{} Does this assessment model have a retrospective pattern? If so, is the pattern minor, or major? \(A major retrospective pattern occurs when the adjusted SSB or  \$F\_{Full}\${} lies outside of the approximate  joint confidence region for SSB and  \$F\_{Full}\${}\; see  Figure \ref{RhoDecision\_tab}{}\). \linebreak{} \hspace\*{0.5cm} \textit{ The 7\-year Mohn\'s  \textrho{}, relative to SSB, was 0.26 in the 2014 assessment and was 0.83 in 2014. The 7\-year Mohn\'s  \textrho{}, relative to F, was \-0.16 in the 2014 assessment and was \-0.51 in 2014. There was a major retrospective pattern for this assessment because the  \textrho{} adjusted estimates of 2014 SSB \(\$SSB\_{\rho}\${}=2,883\)  and 2014 F \(\$F\_{\rho}\${}=0.778\)  were outside the approximate 90\\percent confidence region around SSB \(3,783 \- 6,767\)  and F \(0.254 \- 0.504\).  A retrospective  adjustment was made for both the determination of stock status and for projections of catch in 2016. The retrospective adjustment changed the 2014 SSB from 5,275 to 2,883 and the 2014  \$F\_{Full}\${} from 0.379 to 0.778.}  \item{}Based on this stock assessment, are population projections well determined or uncertain? \linebreak{} \hspace\*{0.5cm} \textit{Population projections for Georges Bank Winter Flounder are reasonably well determined.}  \item{}Describe any changes that were made to the current stock assessment, beyond incorporating additional years of data  and the affect these changes had on the assessment and stock status. \linebreak{} \hspace\*{0.5cm} \textit{ The only change made to the Georges Bank Winter Flounder assessment, other than the incorporation of an additional  year of data, involved fishery selectivity.  During the 2014 assessment update, stock size estimates of age 1 and age 2 fish were not estimable  in the VPA during year t + 1 \(CVs near 1.0\). When age 2 stock size is not estimated in year t + 1,  the VPA model calculates the stock size of age 1 fish \(i.e., recruitment\)  in the terminal year by  using the age 1 partial recruitment \(PR\)  value to derive the F at age 1 in the terminal year. The  age 1 PR value used in the 2014 assessment update was 0.001. However, when this same age 1 PR value  was used in a VPA run for the current assessment update, the low PR value combined with the low age  1 catch in 2014 resulted in an unlikely high stock size estimate for age 1 recruitment in 2014 \(i.e.,  41,587,000 fish\)  when compared to survey observations of the same cohort \(i.e., age 1 in 2014 and age  2 in 2015\). In order to obtain a more realistic estimate of age 1 recruitment in 2014, I allowed the  VPA model to estimate age 2 stock size in 2015 \(i.e., and thereby avoided the use of an age 1 PR  value in the age 1 stock size calculation for 2014\)  and used the back\-calculated PR values from this  VPA run to derive a new PR\-at\-age vector which was used in the final 2015 VPA run. Similar to the  2014 assessment update, the final 2015 VPA run did not include the estimation of age 2 stock size  and the new PR\-at\-age vector was computed using the same methods as in the 2014 assessment.   Full selectivity occurs at age 4. For the 2015 assessment update, fishery selectivity for ages  1\-3 was changed from the 2014 assessment values of 0.001, 0.10 and 0.43, respectively, to 0.01,  0.08 and 0.55, respectively. Differences between estimates  of F, SSB and R values from the final  2015 VPA run, with the new PR vector, and a 2015 VPA run that utilized the PR vector from the 2014  assessment are shown in Table G30.}  \item{}If the stock status has changed a lot since the previous assessment, explain why this occurred.  \linebreak{} \hspace\*{0.5cm} \textit{The overfished and overfishing status of Georges Bank Winter Flounder has changed in the current assessment update due to a worsening of the retrospective error associated with fishing mortality and SSB.}  \item{}Indicate what data or studies are currently lacking and which would be needed most to improve this stock assessment in the future.  \linebreak{} \hspace\*{0.5cm} \textit{The Georges Bank Winter Flounder assessment could be improved with discard estimates from the Canadian bottom trawl fleet and age data from the Canadian spring bottom trawl surveys.}  \item{}Are there other important issues? \linebreak{} \hspace\*{0.5cm} \textit{None. } \end{itemize}{}} \def\FLWGBRefr{ \textbf{References: }{} \linebreak{} Hendrickson L, Nitschke P, Linton B. 2015. 2014 Operational Stock Assessments for Georges Bank winter flounder, Gulf of Maine winter flounder, and pollock. US Dept Commer, Northeast Fish Sci Cent Ref Doc. 15\-01\; 228 p. \linebreak{} \linebreak{}} \def\FLWGBDraft{} \def\FLWGBSPPname{Georges Bank Winter Flounder} \def\FLWGBSPPnameT{Georges Bank Winter Flounder} \def\FLWGBRptYr{2015} \def\FLWGBAuthor{Lisa Hendrickson} \def\FLWGBReviewerComments{/home/dhennen/EIEIO/BigReport/FLW_GB/latex}  \def\FLDGMGBMyPathTab{/home/dhennen/EIEIO/BigReport/FLD_GMGB/tables} \def\FLDGMGBMyPathFig{/home/dhennen/EIEIO/BigReport/FLD_GMGB/figures} \def\FLDGMGBfigFishCap{Total catch of northern windowpane flounder between 1975 and 2014 by disposition \(landings and discards\).} \def\FLDGMGBfigSSBCap{Trends in the biomass index \(a 3\-year moving average of the NEFSC fall bottom trawl survey index\)  of northern windowpane flounder between 1975 and 2014 from the current  assessment, and the corresponding  \$B\_{Threshold}\${} =  \$\dfrac{1}{2}\${} \$B\_{MSY}\${} \textit{proxy}{} = 0.777 kg\/tow \(horizontal dashed line\). } \def\FLDGMGBfigFCap{Trends in relative fishing mortality  of northern windowpane flounder between 1975 and 2014 from the current  assessment, and the corresponding  \$F\_{MSY}\${} \textit{proxy}{}=0.45 \(horizontal dashed line\). } \def\FLDGMGBfigRecrCap{} \def\FLDGMGBfigSurvCap{NEFSC fall bottom trawl survey indices in kg\/tow for northern windowpane flounder between 1975 and 2014  The approximate 90\\percent lognormal confidence intervals are shown.} \def\FLDGMGBPreAmb{This assessment of the northern windowpane flounder \(\textit{Scophthalmus aquosus}\)  stock is an operational update of the 2012 assessment which included updates through 2010 \(NEFSC 2012\). Based on the 2012 assessment the stock was overfished, and overfishing was ocurring. This assessment updates commercial fishery catch data, survey indices of abundance, AIM model results,  and reference points through 2014.} \def\FLDGMGBSoS{ \textbf{State of Stock: }{}Based on this updated assessment, the northern windowpane flounder \(\textit{Scophthalmus aquosus}\)  stock is overfished but overfishing is not occurring \(Figures \ref{FLDGMGBSSB\_plot1}\-\ref{FLDGMGBF\_plot1}\){}. Retrospective adjustments were not made to the model results. The mean NEFSC fall bottom trawl survey index from years 2012, 2013 and 2014 \(a 3\-year moving average is used as a biomass index\)  was 0.535 kg\/tow which is lower than the \$B\_{Threshold}\${} of 0.777 kg\/tow. The 2014 relative fishing mortality was estimated to be 0.393 kt per kg\/tow which is lower than the  \$F\_{MSY}\${} \textit{proxy}{} of 0.450 kt per kg\/tow.} \def\FLDGMGBProj{} \def\FLDGMGBSpecCmt{ \textbf{Special Comments: } \begin{itemize}{} \item{}What are the most important sources of uncertainty in this stock assessment?  Explain, and describe qualitatively how they affect the assessment results \(such as estimates of biomass, F, recruitment, and population projections\).  \linebreak{} \hspace\*{0.5cm} \textit{The main source of uncertainty in this assessment is the lack of windowpane discard estimates from Canadian fisheries to add to the catch component of model input. Discard estimates were from the U.S. only. There is overlap between the survey area and Canadian fishing grounds \(Van Eeckhaute et al. 2010\), which means catch from within the stock area was likely underestimated. }  \item{} Does this assessment model have a retrospective pattern? If so, is the pattern minor, or major? \(A major retrospective pattern occurs when the adjusted SSB or  \$F\_{Full}\${} lies outside of the approximate  joint confidence region for SSB and  \$F\_{Full}\${}\; see  Figure \ref{RhoDecision\_tab}{}\). \linebreak{} \hspace\*{0.5cm} \textit{ The model used to estimate status of this stock does not allow estimation of a retrospective pattern. }  \item{}Based on this stock assessment, are population projections well determined or uncertain? \linebreak{} \hspace\*{0.5cm} \textit{N\/A }  \item{}Describe any changes that were made to the current stock assessment, beyond incorporating additional years of data  and the affect these changes had on the assessment and stock status. \linebreak{} \hspace\*{0.5cm} \textit{No changes were made to the northern windowpane flounder assessment for this update  other than the incorporation of four years of new NEFSC fall bottom trawl survey data and  four years of new U.S. commercial landings and discard data \(2011 \- 2014\). }  \item{}If the stock status has changed a lot since the previous assessment, explain why this occurred.  \linebreak{} \hspace\*{0.5cm} \textit{The stock status of northern windowpane flounder changed from \'overfished and overfishing is occurring\' to \'overfished and overfishing is not occurring\' due to stable\-to\-decreasing catch since 2008, and an increasing trend in the survey index since 2010. }  \item{}Indicate what data or studies are currently lacking and which would be needed most to improve this stock assessment in the future.  \linebreak{} \hspace\*{0.5cm} \textit{The northern windowpane flounder assessment could be improved by estimating the Canadian windowpane removals and, although to a lesser degree, the \'general category\' scallop dredge fleet discards from within the stock area and using them as additional catch input to the AIM model.  While the model fit now is reasonable \(the relationship between ln\(relative F\)  and ln\(replacement ratio\), a measure of the relationship between catch and survey index values, has a p\-value of 0.079\)  there are probably removals unaccounted for in the model and the fit can likely be improved. }  \item{}Are there other important issues? \linebreak{} \hspace\*{0.5cm} \textit{None. } \end{itemize}{}} \def\FLDGMGBRefr{ \textbf{References: }{} \linebreak{} Most recent assessment update:  \linebreak{} Northeast Fisheries Science Center. 2012. Assessment or Data Updates of 13 Northeast Groundfish Stocks through 2010.  US Dept Commer, Northeast Fish Sci Cent Ref Doc. 12\-06\; 789 p. Available online at http:\/\/nefsc.noaa.gov\/publications\/  \linebreak{} \linebreak{} Most recent benchmark assessment:  \linebreak{} Northeast Fisheries Science Center. 2008. Assessment of 19 Northeast Groundfish Stocks through 2007:  Report of the 3$^{rd}$ Groundfish Assessment Review Meeting \(GARM III\), Northeast Fisheries Science Center,  Woods Hole, Massachusetts, August 4\-8, 2008. US Dep Commer, NOAA FIsheries, Northeast Fish Sci Cent Ref Doc. 08\-15\; 884 p + xvii.  \linebreak{} \linebreak{} Van Eeckhaute, L., Sameoto, J., and A. Glass. 2010. Discards of Atlantic cod, haddock and yellowtail flounder  from the 2009 Canadian scallop fishery on Georges Bank. TRAC Ref. Doc. 2010\/10. 7p.  \linebreak{} \linebreak{}} \def\FLDGMGBDraft{} \def\FLDGMGBSPPname{northern windowpane flounder} \def\FLDGMGBSPPnameT{Northern windowpane flounder} \def\FLDGMGBRptYr{2015} \def\FLDGMGBAuthor{Toni Chute} \def\FLDGMGBReviewerComments{/home/dhennen/EIEIO/BigReport/FLD_GMGB/latex}  \def\FLDSNEMAMyPathTab{/home/dhennen/EIEIO/BigReport/FLD_SNEMA/tables} \def\FLDSNEMAMyPathFig{/home/dhennen/EIEIO/BigReport/FLD_SNEMA/figures} \def\FLDSNEMAfigFishCap{Total catch of southern windowpane flounder between 1975 and 2014 by disposition \(landings and discards\).} \def\FLDSNEMAfigSSBCap{Trends in the biomass index \(a 3\-year moving average of the NEFSC fall bottom trawl survey index\)  of southern windowpane flounder between 1975 and 2014 from the current  assessment, and the corresponding  \$B\_{Threshold}\${} =  \$\dfrac{1}{2}\${} \$B\_{MSY}\${} \textit{proxy}{} = 0.123 kg\/tow\(horizontal dashed line\). } \def\FLDSNEMAfigFCap{Trends in relative fishing mortality  of southern windowpane flounder between 1975 and 2014 from the current  assessment, and the corresponding  \$F\_{MSY}\${} \textit{proxy}{}=2.027 \(horizontal dashed line\). } \def\FLDSNEMAfigRecrCap{} \def\FLDSNEMAfigSurvCap{NEFSC fall bottom trawl survey indices in kg\/tow for southern windowpane flounder between 1975 and 2014. The approximate 90\\percent lognormal confidence intervals are shown.} \def\FLDSNEMAPreAmb{This assessment of the southern windowpane flounder \(\textit{Scophthalmus aquosus}\)  stock is an operational update of the 2012 assessment which included updates through 2010 \(NEFSC 2012\). Based on the 2012 assessment the stock was not overfished, and overfishing was not ocurring. This assessment updates commercial fishery catch data, survey indices of abundance, AIM model results, and reference points through 2014. } \def\FLDSNEMASoS{ \textbf{State of Stock: }{}Based on this updated assessment, the southern windowpane flounder \(\textit{Scophthalmus aquosus}\)  stock is not overfished and overfishing is not occurring \(Figures \ref{FLDSNEMASSB\_plot1}\-\ref{FLDSNEMAF\_plot1}\){}. Retrospective adjustments were not made to the model results. The mean NEFSC fall bottom trawl survey index from years 2012, 2013, and 2014 \(a 3\-year moving average is used as a biomass index\)  was  0.413 \(kg\/tow\)  which is higher than the \$B\_{Threshold}\${}of 0.123 \(kg\/tow\). The 2014 relative fishing mortality was estimated to be  1.308 \(kt per kg\/tow\)  which is lower than the  \$F\_{MSY}\${} \textit{proxy}{} of 2.027 \(kt per kg\/tow\). } \def\FLDSNEMAProj{} \def\FLDSNEMASpecCmt{ \textbf{Special Comments: } \begin{itemize}{} \item{}What are the most important sources of uncertainty in this stock assessment?  Explain, and describe qualitatively how they affect the assessment results \(such as estimates of biomass, F, recruitment, and population projections\).  \linebreak{} \hspace\*{0.5cm} \textit{A source of uncertainty for this assessment is missing commercial discard estimates from the general category scallop dredge fleet that should be added to the catch time series for model input. }  \item{} Does this assessment model have a retrospective pattern? If so, is the pattern minor, or major? \(A major retrospective pattern occurs when the adjusted SSB or  \$F\_{Full}\${} lies outside of the approximate  joint confidence region for SSB and  \$F\_{Full}\${}\; see  Figure \ref{RhoDecision\_tab}{}\). \linebreak{} \hspace\*{0.5cm} \textit{ The model used to estimate status of this stock does not allow estimation of a retrospective pattern. }  \item{}Based on this stock assessment, are population projections well determined or uncertain? \linebreak{} \hspace\*{0.5cm} \textit{N\/A}  \item{}Describe any changes that were made to the current stock assessment, beyond incorporating additional years of data  and the affect these changes had on the assessment and stock status. \linebreak{} \hspace\*{0.5cm} \textit{ No changes were made to the southern windowpane flounder assessment for this update  other than the incorporation of four years of new NEFSC fall bottom trawl survey data and  four years of new U.S. commercial landings and discard data \(2011 \- 2014\). }  \item{}If the stock status has changed a lot since the previous assessment, explain why this occurred.  \linebreak{} \hspace\*{0.5cm} \textit{The stock status of southern windowpane flounder has not changed since the previous assessment. }  \item{}Indicate what data or studies are currently lacking and which would be needed most to improve this stock assessment in the future.  \linebreak{} \hspace\*{0.5cm} \textit{Estimates of discards from the general category scallop dredge fleet should be added to the catch time series for model input. However, the model fit is presently good with a randomization test indicating the correlation between ln\(relative F\)  and ln\(replacement ratio\), a measure of the relationship between catch and survey index values, is significant \(p = 0.002.\)  }  \item{}Are there other important issues? \linebreak{} \hspace\*{0.5cm} \textit{None. } \end{itemize}{}} \def\FLDSNEMARefr{ \textbf{References: }{} \linebreak{} Most recent assessment update:  \linebreak{} Northeast Fisheries Science Center. 2012. Assessment or Data Updates of 13 Northeast Groundfish Stocks through 2010.  US Dept Commer, Northeast Fish Sci Cent Ref Doc. 12\-06\; 789 p. Available online at http:\/\/nefsc.noaa.gov\/publications\/  \linebreak{} \linebreak{} Most recent benchmark assessment:  \linebreak{} Northeast Fisheries Science Center. 2008. Assessment of 19 Northeast Groundfish Stocks through 2007:  Report of the 3$^{rd}$ Groundfish Assessment Review Meeting \(GARM III\), Northeast Fisheries Science Center,  Woods Hole, MA, August 4\-8, 2008. US Dep Commer, NOAA Fisheries, Northeast Fish Sci Cent Ref Doc. 08\-15\; 884 p + xvii. \linebreak{} \linebreak{}} \def\FLDSNEMADraft{} \def\FLDSNEMASPPname{southern windowpane flounder} \def\FLDSNEMASPPnameT{Southern windowpane flounder} \def\FLDSNEMARptYr{2015} \def\FLDSNEMAAuthor{Toni Chute} \def\FLDSNEMAReviewerComments{/home/dhennen/EIEIO/BigReport/FLD_SNEMA/latex}  \def\PLAUNITMyPathTab{/home/dhennen/EIEIO/BigReport/PLA_UNIT/tables} \def\PLAUNITMyPathFig{/home/dhennen/EIEIO/BigReport/PLA_UNIT/figures} \def\PLAUNITfigFishCap{Total catch of Gulf of Maine\-Georges Bank American Plaice between 1980 and 2015 by fleet \(Gulf of Maine, Georges Bank, Southern New England, and Canadian\)  and disposition \(landings and discards\).} \def\PLAUNITfigSSBCap{Trends in spawning stock biomass of Gulf of Maine\-Georges Bank American Plaice between 1980 and 2015 from the current  \(solid line\)  and previous \(dashed line\)  assessment and the corresponding  \$SSB\_{Threshold}\${} \(\$\dfrac{1}{2}\${} \$SSB\_{MSY}\${} \textit{proxy}{}\; horizontal dashed line\)  as well as  \$SSB\_{Target}\${} \(\$SSB\_{MSY}\${} \textit{proxy}{}\; horizontal dotted line\)   based on the 2015 assessment.  Biomass was adjusted for a retrospective pattern  and the adjustment is shown in red.  The approximate 90\\percent normal confidence intervals are shown.} \def\PLAUNITfigFCap{Trends in the fully selected fishing mortality \(\$F\_{Full}\${}\)  of Gulf of Maine\-Georges Bank American Plaice between 1980 and 2015 from the current  \(solid line\)  and previous \(dashed line\)  assessment and the corresponding  \$F\_{Threshold}\${} \(\$F\_{MSY}\${} \textit{proxy}{}=0.196\; horizontal dashed line\).  \$F\_{Full}\${} was adjusted for a retrospective pattern  and the adjustment is shown in red,  based on the 2015 assessment. The approximate 90\\percent normal confidence intervals are shown.} \def\PLAUNITfigRecrCap{Trends in Recruits \(age 1\)  \(000s\)  of Gulf of Maine\-Georges Bank American Plaice between 1980 and 2015 from the current \(solid line\)  and previous \(dashed line\)  assessment.} \def\PLAUNITfigSurvCap{Indices of biomass for the Gulf of Maine\-Georges Bank American Plaice between 1963 and 2015 for the Northeast Fisheries Science Center \(NEFSC\)  and Massachusetts Division of Marine Fisheries \(MADMF\)  spring and autumn research bottom trawl surveys.  The approximate 90\\percent normal confidence intervals are shown.} \def\PLAUNITPreAmb{This assessment of the Gulf of Maine\-Georges Bank American Plaice \(\textit{Hippoglossoides platessoides}\)  stock is an operational update of the existing 2012 benchmark assessment \(O\'Brien et al. 2012\). Based on the previous assessment the stock was not overfished, and overfishing was not ocurring. This 2015 assessment updates commercial fishery catch data, research survey indices of abundance, the analytical VPA assessment model, and reference points through 2014. Additionally, stock projections have been updated through 2018.} \def\PLAUNITSoS{ \textbf{State of Stock: }{}Based on this updated assessment, the Gulf of Maine\-Georges Bank American Plaice \(\textit{Hippoglossoides platessoides}\)  stock is not overfished and overfishing is not occurring \(Figures \ref{PLAUNITSSB\_plot1}\-\ref{PLAUNITF\_plot1}\){}.  Retrospective adjustments were made to the model results.  Spawning stock biomass \(SSB\)  in 2014 was estimated to be 10,915 mt which is 83\\percent of the biomass target for this stock \(\$SSB\_{MSY}\${} \textit{proxy}{} = 13,107\;  Figure \ref{PLAUNITSSB\_plot1}{}\). The 2014 fully selected fishing mortality was estimated to be 0.118 which is 60\\percent of the overfishing threshold proxy \(\$F\_{MSY}\${} \textit{proxy}{} = 0.196\;  Figure \ref{PLAUNITF\_plot1}{}\).} \def\PLAUNITProj{ \textbf{Projections: }{}Short term projections of biomass were derived by sampling from an empirical cumulative  distribution  function of 34 recruitment estimates from VPA model results. The annual fishery selectivity, maturity ogive, and mean weights at age used in projections are the most recent 5 year averages\;  retrospective adjustments were applied in the projections.} \def\PLAUNITSpecCmt{ \textbf{Special Comments: } \begin{itemize}{} \item{}What are the most important sources of uncertainty in this stock assessment?  Explain, and describe qualitatively how they affect the assessment results \(such as estimates of biomass, F, recruitment, and population projections\).  \linebreak{} \hspace\*{0.5cm} \textit{A source of uncertainty in this assessment are the estimates of historical landings at age, prior to 1984, and the magnitude of  historical discards, prior to 1989. Both of these affect the scale of the biomass and fishing mortality estimates, and influence reference point estimations.}  \item{} Does this assessment model have a retrospective pattern? If so, is the pattern minor, or major? \(A major retrospective pattern occurs when the adjusted SSB or  \$F\_{Full}\${} lies outside of the approximate  joint confidence region for SSB and  \$F\_{Full}\${}\; see  Figure \ref{RhoDecision\_tab}{}\). \linebreak{} \hspace\*{0.5cm} \textit{ The 7\-year Mohn\'s  \textrho{}, relative to SSB, was 0.63 in the 2012 assessment and was 0.32 in 2014. The 7\-year Mohn\'s  \textrho{}, relative to F, was \-0.35 in the 2012 assessment and was 0.32 in 2014. There was a major retrospective pattern for this assessment because the  \textrho{} adjusted estimates of 2014 SSB \(\$SSB\_{\rho}\${}=10,915\)  and 2014 F \(\$F\_{\rho}\${}=0.118\)  were outside the approximate 90\\percent confidence regions around SSB \(12,742 \- 16,439\)  and F \(0.069 \- 0.093\).  A retrospective  adjustment was made for both the determination of stock status and for projections of catch in 2016. The retrospective adjustment changed the 2014 SSB from 14,543 to 10,915 and the 2014  \$F\_{Full}\${} from 0.08 to 0.118.}  \item{}Based on this stock assessment, are population projections well determined or uncertain? \linebreak{} \hspace\*{0.5cm} \textit{Population projections for Gulf of Maine\-Georges Bank American Plaice are reasonably well determined.}  \item{}Describe any changes that were made to the current stock assessment, beyond incorporating additional years of data  and the effect these changes had on the assessment and stock status. \linebreak{} \hspace\*{0.5cm} \textit{ No major changes, other than the addition of recent years of data, were made to the Gulf of Maine\-Georges Bank American Plaice assessment for this update. A new version of VPA was used \(V3.3.0\)  which gave very similar results to the 2012 VPA 3.1.0 run, with the same F and slightly lower SSB. The MADMF spring and autumn survey indices were re\-estimated for the time series, accounting for revised stratum areas. The revision occurred in 2007, but was overlooked in the 2012 assessment. A comparison of 2010 terminal year VPAs indicated minimal differences in 2010 SSB \(now slightly lower\)  and no change in F.}  \item{}If the stock status has changed a lot since the previous assessment, explain why this occurred.  \linebreak{} \hspace\*{0.5cm} \textit{As in recent assessments for Gulf of Maine\-Georges Bank American Plaice the stock status remains as not overfished and overfishing not occurring.}  \item{}Indicate what data or studies are currently lacking and which would be needed most to improve this stock assessment in the future.  \linebreak{} \hspace\*{0.5cm} \textit{The Gulf of Maine\-Georges Bank American Plaice assessment could be improved with updated studies on growth of Georges Bank and Gulf of Maine fish.}  \item{}Are there other important issues? \linebreak{} \hspace\*{0.5cm} \textit{A difference in growth between GM and GB fish has been documented, however, historical catch data information for GB may not be sufficient to conduct a separate assessment. Also, the growth difference may not persist in the most recent years. This could all be explored further in an benchmark review.} \end{itemize}{}} \def\PLAUNITRefr{ \textbf{References: }{} \linebreak{}O\'Brien, L. and J. Dayton \(2012\). E. Gulf of Maine \- Georges Bank American plaice Assessment for 2012 in Northeast Fisheries Science Center, 2012, Assessment or Data Updates of 13 Northeast Groundfish Stocks through 2010. US Dept Commer, Northeast Fish Sci Cent Ref Doc. 12\-06\; 789 p. http:\/\/www.nefsc.noaa.gov\/publications\/crd\/crd1206\/. \linebreak{} \linebreak{}} \def\PLAUNITDraft{} \def\PLAUNITSPPname{Gulf of Maine-Georges Bank American Plaice} \def\PLAUNITSPPnameT{Gulf of Maine-Georges Bank American Plaice} \def\PLAUNITRptYr{2015} \def\PLAUNITAuthor{Loretta O\'Brien} \def\PLAUNITReviewerComments{/home/dhennen/EIEIO/BigReport/PLA_UNIT/latex}  \def\WITUNITMyPathTab{/home/dhennen/EIEIO/BigReport/WIT_UNIT/tables} \def\WITUNITMyPathFig{/home/dhennen/EIEIO/BigReport/WIT_UNIT/figures} \def\WITUNITfigFishCap{Total catch of witch flounder between 1982 and 2014 by fleet \(commercial\)  and disposition \(landings and discards\).} \def\WITUNITfigSSBCap{Trends in spawning stock biomass \(mt\)  of witch flounder between 1982 and 2014 from the current  \(solid line\)  and previous \(dashed line\)  assessment and the corresponding  \$SSB\_{Threshold}\${} \(\$\dfrac{1}{2}\${} \$SSB\_{MSY}\${}\; horizontal dashed line\)  as well as  \$SSB\_{Target}\${} \$SSB\_{MSY}\${}\; horizontal dotted line\)   based on the current assessment. Red solid vertical line indicates rho adjusted SSB. Black solid vertical line indicates 90\\percent confidence interval for 2014.} \def\WITUNITfigFCap{Trends in the fully selected fishing mortality \(\$F\_{Full}\${}\)  of witch flounder between 1982 and 2014 from the current  \(solid line\)  and previous \(dashed line\)  assessment and the corresponding  \$F\_{Threshold}\${} \(\$F\_{MSY}\${}=0.279\; horizontal dashed line\)  based on the current assessment.  Red solid vertical line indicates rho adjusted  \$F\_{Full}\${}. Black solid vertical line indicates 90\\percent confidence interval for 2014.} \def\WITUNITfigRecrCap{Trends in Age 3  \(000s\)  of witch flounder between 1982 and 2014 from the current \(solid line\)  and previous \(dashed line\)  assessment.} \def\WITUNITfigSurvCap{Indices of biomass \(kg\/tow\)  for the witch flounder between 1963 and 2015 for the Northeast Fisheries Science Center \(NEFSC\)  spring and fall bottom trawl surveys.  The 90\\percent lognormal confidence intervals are shown.} \def\WITUNITPreAmb{This assessment of the witch flounder \(\textit{Glyptocephalus cynoglossus}\)  stock is an operational update of the 2012 assessment \(NEFSC 2012\)  and the 2008 benchmark assessment \(NEFSC 2008\). This assessment updates commercial fishery catch data, research survey indices, and the analytical assessment model through 2014. Additionally, stock projections have been updated through 2018. Reference points have been updated. } \def\WITUNITSoS{ \textbf{State of Stock: }{}witch flounder \(\textit{Glyptocephalus cynoglossus}\)  stock is overfished and overfishing is occurring \(Figures \ref{WITUNITSSB\_plot1}\-\ref{WITUNITF\_plot1}\){}. Retrospective adjustments were made to the model results.  Spawning stock biomass \(SSB\)  in 2014 was estimated to be 2,077 \(mt\)  which is 22\\percent of the  \$SSB\_{MSY}\${} proxy \(9,473\;  Figure \ref{WITUNITSSB\_plot1}{}\).  The 2014 fully selected fishing mortality was estimated to be 0.687 which is 246\\percent of the  \$F\_{MSY}\${} proxy \(0.279\;  Figure \ref{WITUNITF\_plot1}{}\). A retrospective adjustment to  \$F\_{Full}\${} and SSB in 2014 was required but did not lead to a change in status.  } \def\WITUNITProj{ \textbf{Projections: }{}Short term projection recruitment was sampled from a cumulative distribution function derived from ADAPT VPA \(with split time series between 1994 and 1995\)  estimated age 3 recruitment between 1982 and 2013.  Average 2010\-2014 partial recruitment, average 2010\-2014 mean weights, and maturation ogive representing 2011\-2015 maturity data were used.} \def\WITUNITSpecCmt{ \textbf{Special Comments: } \begin{itemize}{} \item{}What are the most important sources of uncertainty in this stock assessment?  Explain, and describe qualitatively how they affect the assessment results \(such as estimates of biomass, F, recruitment, and population projections\).  \linebreak{} \hspace\*{0.5cm} \textit{An important source of uncertainty is the retrospective pattern where fishing mortality is underestimated and spawning stock biomass and recruitment are overestimated. }  \item{} Does this assessment model have a retrospective pattern? If so, is the pattern minor, or major? \(A major retrospective pattern occurs when the adjusted SSB or  \$F\_{Full}\${} lies outside of the approximate  joint confidence region for SSB and  \$F\_{Full}\${}\).  \linebreak{} \hspace\*{0.5cm} \textit{ The 7\-year Mohn\'s  \textrho{}, relative to SSB, was 0.61 in the 2012 assessment and was 0.51 in 2014. The 7\-year Mohn\'s  \textrho{}, relative to F, was \-0.33 in the 2012 assessment and was \-0.38 in 2014. There was a major retrospective pattern for this assessment because the  \textrho{} adjusted estimates of 2014 SSB \(\$SSB\_{\rho}\${}=2,077\)  and 2014 F \(\$F\_{\rho}\${}=0.687\)  were outside the approximate 90\\percent confidence regions around SSB \(2,643 \- 3,864\)  and F \(0.321 \- 0.603\).  A retrospective  adjustment was made for both the determination of stock status and for projections of catch in 2016. The retrospective adjustment changed the 2014 SSB from 3,129 to 2,077 and the 2014  \$F\_{Full}\${} from 0.428 to 0.687.}  \item{}Based on this stock assessment, are population projections well determined or uncertain? \linebreak{} \hspace\*{0.5cm} \textit{Population projections for witch flounder appear to be optimistic\; the projected rho adjusted biomass from the last assessment  was above the upper confidence bounds of the projected rho adjusted biomass estimated in the current assessment. }  \item{}Describe any changes that were made to the current stock assessment, beyond incorporating additional years of data  and the effect these changes had on the assessment and stock status.  \linebreak{} \hspace\*{0.5cm} \textit{TOGA \(Type, Operation, Gear, Acquisition\)  values were used for haul criteria for NEFSC surveys for 2009 onward and minor changes in the use of observer data for discard estimates were made to the current witch flounder assessment. These changes had negligible effect on the assessment and stock status.  }  \item{}If the stock status has changed a lot since the previous assessment, explain why this occurred.  \linebreak{} \hspace\*{0.5cm} \textit{No change in stock status has occurred for witch flounder since the previous assessment. }  \item{}Indicate what data or studies are currently lacking and which would be needed most to improve this stock assessment in the future.  \linebreak{} \hspace\*{0.5cm} \textit{Extensive studies have examined the causes of retrospective patterns with no definitive conclusions other than a change in model does not resolve the issue. }  \item{}Are there other important comments? \linebreak{} \hspace\*{0.5cm} \textit{The VPA analysis was performed with survey time series split between 1994 and 1995. This time split corresponds to changes in the commercial reporting methods as well as other regulatory management changes.  } \end{itemize}{}} \def\WITUNITRefr{ \textbf{References: }{} \linebreak{}Northeast Fisheries Science Center. 2008. Assessment of 19 Northeast Groundfish Stocks through 2007: Report of the 3$^{rd}$ Groundfish Assessment Review Meeting \(GARM III\), Northeast Fisheries Science Center, Woods Hole, Massachusetts, August 4\-8, 2008. US Dep Commer, NOAA Fisheries, Northeast Fish Sci Cent Ref Doc. 08\-15\; 884 p + xvii. http:\/\/www.nefsc.noaa.gov\/publications\/crd\/crd0815\/ \linebreak{} \linebreak{}Northeast Fisheries Science Center. 2012. Assessment or Data Updates of 13 Northeast Groundfish Stocks through 2010.  US Dep Commer, NOAA Fisheries, Northeast Fish Sci Cent Ref Doc. 12\-06\; 789 p. http:\/\/www.nefsc.noaa.gov\/publications\/crd\/crd1206\/ \linebreak{} \linebreak{}} \def\WITUNITDraft{} \def\WITUNITSPPname{witch flounder} \def\WITUNITSPPnameT{Witch flounder} \def\WITUNITRptYr{2015} \def\WITUNITAuthor{Susan Wigley} \def\WITUNITReviewerComments{/home/dhennen/EIEIO/BigReport/WIT_UNIT/latex}  \def\HKWUNITMyPathTab{/home/dhennen/EIEIO/BigReport/HKW_UNIT/tables} \def\HKWUNITMyPathFig{/home/dhennen/EIEIO/BigReport/HKW_UNIT/figures} \def\HKWUNITfigFishCap{Total catch of white hake between 1963 and 2014 by fleet \(commercial, recreational, or Canadian\)  and disposition \(landings and discards\).} \def\HKWUNITfigSSBCap{Trends in spawning stock biomass of white hake between 1963 and 2014 from the current  \(solid line\)  and previous \(dashed line\)  assessment and the corresponding  \$SSB\_{Threshold}\${} \(\$\dfrac{1}{2}\${} \$SSB\_{MSY}\${} \textit{proxy}{}\; horizontal dashed line\)  as well as  \$SSB\_{Target}\${} \(\$SSB\_{MSY}\${} \textit{proxy}{}\; horizontal dotted line\)   based on the 2014 assessment.  The red dot indicates the rho\-adjusted SSB values that would have resulted had a retrospective  adjusment been made \(see Special Comments section\).  The approximate 90\\percent lognormal confidence intervals are shown.} \def\HKWUNITfigFCap{Trends in the fully selected fishing mortality \(\$F\_{Full}\${}\)  of white hake between 1963 and 2014 from the current  \(solid line\)  and previous \(dashed line\)  assessment and the corresponding  \$F\_{Threshold}\${} \(\$F\_{MSY}\${} \textit{proxy}{}=0.188\; horizontal dashed line\).  The red dot indicates the rho\-adjusted SSB values that would have resulted had a retrospective  adjusment been made \(see Special Comments section\).  The approximate 90\\percent lognormal confidence intervals are shown.} \def\HKWUNITfigRecrCap{Trends in Recruits \(age 1\)  \(000s\)  of white hake between 1963 and 2014 from the current \(solid line\)  and previous \(dashed line\)  assessment. The approximate 90\\percent lognormal confidence intervals are shown.} \def\HKWUNITfigSurvCap{Indices of biomass for the white hake between 1963 and 2015 for the Northeast Fisheries Science Center \(NEFSC\)  spring and fall bottom trawl surveys.  The approximate 90\\percent lognormal confidence intervals are shown.} \def\HKWUNITPreAmb{This assessment of the white hake \(\textit{Urophycis tenuis}\)  stock is an operational update of the existing 2013 benchmark ASAP assessment \(NEFSC 2013\). Based on the previous assessment the stock was not overfished, and overfishing was not ocurring. This assessment updates commercial fishery catch data, research survey indices of abundance, and the ASAP assessment models and reference points through 2014. Additionally, stock projections have been updated through 2018.} \def\HKWUNITSoS{ \textbf{State of Stock: }{}Based on this updated assessment, white hake \(\textit{Urophycis tenuis}\)  stock is not overfished and overfishing is not occurring \(Figures \ref{HKWUNITSSB\_plot1}\-\ref{HKWUNITF\_plot1}\){}. Retrospective adjustments were not made to the model results.  Spawning stock biomass \(SSB\)  in 2014 was estimated to be 28,553 \(mt\)  which is 88\\percent of the biomass threshold for an overfished stock \(\$SSB\_{MSY}\${} \textit{proxy}{} = 32,550\;  Figure \ref{HKWUNITSSB\_plot1}{}\).  The 2014 fully selected fishing mortality was estimated to be 0.076 which is 40\\percent of the overfishing threshold proxy \(\$F\_{MSY}\${} \textit{proxy}{} = 0.188\;  Figure \ref{HKWUNITF\_plot1}{}\).} \def\HKWUNITProj{ \textbf{Projections: }{}Short term projections of catch and SSB were derived by sampling from a cumulative  distribution  function of recruitment estimates from ASAP from 1995\-2012. The annual fishery selectivity, maturity ogive, and mean weights at age used in the projection  are the most recent 5 year averages. } \def\HKWUNITSpecCmt{ \textbf{Special Comments: } \begin{itemize}{} \item{}What are the most important sources of uncertainty in this stock assessment?  Explain, and describe qualitatively how they affect the assessment results \(such as estimates of biomass, F, recruitment, and population projections\).  \linebreak{} \hspace\*{0.5cm} \textit{1. Catch at age information is not well characterized due to possible mis\-identification of species in the commercial and sea sampling data, particularly in early years, low sampling of commercial landings in  some years, and sparse discard data particularly in early years.  \linebreak{} \hspace\*{0.5cm}2. Since the commercial catch is aged primarily with survey age\/length keys, there is considerable augmentation required, mainly for ages 5 and older. The numbers at age and mean weights at age in the catch for these ages may therefore not be well specified.  \linebreak{} \hspace\*{0.5cm}3. White hake may move seasonally into and out of the defined stock area.  \linebreak{} \hspace\*{0.5cm}4. There are no commercial catch at age data prior to 1989 and the catchability of older ages in the surveys is very low. This results in a large uncertainty in starting numbers at age.  \linebreak{} \hspace\*{0.5cm}5. Since 2003, dealers have been culling very large fish out of the large category. However, there was no market category to input into the landings until June 2014. The length compositions are distinct from large and have been identified since 2011. This may bias the age composition of the landings, particularly in 2014 when 2000 of the 5000 large samples were these extra\-large fish.  \linebreak{} \hspace\*{0.5cm}6. A pooled age\/length key is used for 1963\-1981, fall 2003 \(second half of commercial key\)  and 2014.Age data were not available for 2014 in time for this assessment. The same pooled key that was used for 1963\-1981 was used for 2014.}  \item{} Does this assessment model have a retrospective pattern? If so, is the pattern minor, or major? \(A major retrospective pattern occurs when the adjusted SSB or  \$F\_{Full}\${} lies outside of the approximate  joint confidence region for SSB and  \$F\_{Full}\${}\; see  Figure \ref{RhoDecision\_tab}{}\). \linebreak{} \hspace\*{0.5cm} \textit{ No retrospective adjustment of spawning stock biomass or fishing mortality in 2014 was required.  The pattern in this assessment is considered minor \(Mohn’s rho of 0.18 on SSB, Mohn’s rho of 0.12 on F\)  with the adjusted SSB within the 90 \\percent CI of the MCMC. However, the Mohn’s rho for Age 1 estimates is 0.54. This may have an impact on projections if this continues into the future.}  \item{}Based on this stock assessment, are population projections well determined or uncertain? \linebreak{} \hspace\*{0.5cm} \textit{Population projections for white hake, are not well determined and projected biomass from the last assessment  was outside the confidence bounds of the biomass estimated in the current assessment. }  \item{}Describe any changes that were made to the current stock assessment, beyond incorporating additional years of data  and the affect these changes had on the assessment and stock status. \linebreak{} \hspace\*{0.5cm} \textit{ The 2011 catch\-at\-length and age were re\-estimated for both landings and discards. For the  landings, two samples were adjusted for dorsal length to total length that had been missed in the previous assessment.}  \item{}If the stock status has changed a lot since the previous assessment, explain why this occurred.  \linebreak{} \hspace\*{0.5cm} \textit{While stock status of white hake has not changed, the stock has not rebuilt as the projections from the last assessment indicated. This is due to the retrospective in recruitment. The numbers for the 2005\-2009 year classes, which were included in the age 2\-6 starting numbers in the projections, were over\-estimated which led to over\-estimating SSB in 2014.}  \item{}Indicate what data or studies are currently lacking and which would be needed most to improve this stock assessment in the future.  \linebreak{} \hspace\*{0.5cm} \textit{ Age structures from the observer program are available and should be aged to augment  the survey keys. There is a also a new market category for heads and age structures could be  acquired from these is an otolith length\/total length relationship can be established. }  \item{}Are there other important issues? \linebreak{} \hspace\*{0.5cm} \textit{None. } \end{itemize}{}} \def\HKWUNITRefr{ \textbf{References: }{} \linebreak{} NEFSC. 2013. 56$^{th}$ Northeast Regional Stock Assessment Workshop \(56$^{th}$ SAW\)  Assessment  Report.US Dep Commer, NOAA Fisheries, Northeast Fish Sci Cent Ref Doc. 13\-10\; 868 p.  http:\/\/www.nefsc.noaa.gov\/publications\/crd\/crd1310\/  \linebreak{} \linebreak{}} \def\HKWUNITDraft{} \def\HKWUNITSPPname{white hake} \def\HKWUNITSPPnameT{White hake} \def\HKWUNITRptYr{2015} \def\HKWUNITAuthor{Katherine Sosebee} \def\HKWUNITReviewerComments{/home/dhennen/EIEIO/BigReport/HKW_UNIT/latex}  \def\OPTUNITMyPathTab{/home/dhennen/EIEIO/BigReport/OPT_UNIT/tables} \def\OPTUNITMyPathFig{/home/dhennen/EIEIO/BigReport/OPT_UNIT/figures} \def\OPTUNITfigFishCap{Total catch of ocean pout  between 1968 and 2014 by fleet \(US and Other\)  and disposition \(landings and discards\).} \def\OPTUNITfigSSBCap{Trends in biomass \(kg\/tow\)  of ocean pout  between 1968 and 2014 from the current  \(solid line\)  and previous \(dashed line\)  assessment and the corresponding  \$B\_{Threshold}\${} \(\$\dfrac{1}{2}\${} \$B\_{MSY}\${} \textit{proxy}{}\; horizontal dashed line\)  as well as  \$B\_{Target}\${} \(\$B\_{MSY}\${} \textit{proxy}{}\; horizontal dotted line\)   based on the current assessment. } \def\OPTUNITfigFCap{Trends in the exploitation rate of ocean pout between 1968 and 2014 from the current  \(solid line\)  and previous \(dashed line\)  assessment and the corresponding  \$F\_{Threshold}\${} \(\$F\_{MSY}\${} \textit{proxy}{}=0.76\; horizontal dashed line\)   based on the current assessment. } \def\OPTUNITfigRecrCap{} \def\OPTUNITfigSurvCap{Indices of biomass \(kg\/tow\)  for ocean pout  between 1968 and 2015 for the Northeast Fisheries Science Center \(NEFSC\)  spring survey.   The approximate 90\\percent lognormal confidence intervals are shown.} \def\OPTUNITPreAmb{This assessment of the ocean pout  \(\textit{Zoarces americanus}\)  stock is an operational update of the 2012 assessment \(NEFSC 2012\)  and the 2008 benchmark assessment \(NEFSC 2008\). Based on the 2012 assessment, the stock was overfished but overfishing was not ocurring. This assessment updates commercial fishery catch data, research survey indices and the exploitation ratios through 2014. There are no stock projections.} \def\OPTUNITSoS{ \textbf{State of Stock: }{}Based on the current assessment, the ocean pout  \(\textit{Zoarces americanus}\)  stock is overfished and overfishing is not occurring \(Figures \ref{OPTUNITSSB\_plot1}\-\ref{OPTUNITF\_plot1}\){}. Retrospective adjustments were not made to the model results. Biomass proxy \(B\)  in 2014 was estimated to be 0.29 \(kg\/tow\)  which is 6\\percent of the biomass target \(\$B\_{MSY}\${} \textit{proxy}{} = 4.94\;  Figure \ref{OPTUNITSSB\_plot1}{}\).  The 2014 fully selected fishing mortality was estimated to be 0.269 which is 35\\percent of the overfishing threshold proxy \(\$F\_{MSY}\${} \textit{proxy}{} = 0.76\;  Figure \ref{OPTUNITF\_plot1}{}\).} \def\OPTUNITProj{ \textbf{Projections: }{}The index\-based assessment approach does not support catch projections\; catch advice for ocean pout has been based on the target exploitation rate and the most recent centered 3\-year average biomass index from the NEFSC spring survey. } \def\OPTUNITSpecCmt{ \textbf{Special Comments: } \begin{itemize}{} \item{}What are the most important sources of uncertainty in this stock assessment?  Explain, and describe qualitatively how they affect the assessment results \(such as estimates of biomass, F, recruitment, and population projections\).  \linebreak{} \hspace\*{0.5cm} \textit{ An important source of uncertainty is the stock has not responded to low catch as expected. }  \item{}Does this assessment model have a retrospective pattern? If so, is the pattern minor or major?  \(A major retrospective pattern occurs when the adjusted SSB or  \$F\_{Full}\${} lies outside of the approximate  joint confidence region for SSB and  \$F\_{Full}\${}\; see  Figure \ref{RhoDecision\_tab}{}\). \linebreak{} \hspace\*{0.5cm} \textit{ The model used to estimate status of this stock does not allow estimation of a retrospective pattern. }  \item{}Based on this stock assessment, are population projections well determined or uncertain? \linebreak{} \hspace\*{0.5cm} \textit{ N\/A}  \item{}Describe any changes that were made to the current stock assessment, beyond incorporating additional years of data  and the effect these changes had in the assessment and stock status. \linebreak{} \hspace\*{0.5cm} \textit{TOGA \(Type, Operation, Gear, Acquisition\)  values were used for haul criteria for NEFSC surveys for 2009 onward and minor changes in the use of observer data for discard estimates were made to the current assessment. These changes had a negligible effect on the assessment and stock status.   Recreational landings were updated and found to be negligible \(time series average of recreational landings to total catch was less than 1\\percent\)  and therefore not included in this assessment.}  \item{}If the stock status has changed a lot since the previous assessment, explain why this occurred.  \linebreak{} \hspace\*{0.5cm} \textit{Ocean pout stock status has not changed since the previous assessment.}  \item{}Indicate what data or studies are currently lacking and which would be needed most to improve this stock assessment in the future.  \linebreak{} \hspace\*{0.5cm} \textit{The ocean pout assessment could be improved with studies that explore why this stock is not rebuilding as expected. }  \item{}Are there other important comments? \linebreak{} \hspace\*{0.5cm} \textit{Biological reference points are based on catch\; the estimated discards used in the catch are based on a mix of direct \(1989 onward\)  and indirect \(1988 and back\)  methods. The catch used to determine MSY is based on indirect methods. } \end{itemize}{}} \def\OPTUNITRefr{ \textbf{References: }{} \linebreak{}Northeast Fisheries Science Center. 2012. Assessment or Data Updates of 13 Northeast Groundfish Stocks through 2010.  US Dep Commer, NOAA Fisheries, Northeast Fish Sci Cent Ref Doc. 12\-06\; 789 p. http:\/\/www.nefsc.noaa.gov\/publications\/crd\/crd1206\/ \linebreak{} \linebreak{}Northeast Fisheries Science Center. 2008. Assessment of 19 Northeast Groundfish Stocks through 2007: Report of the 3$^{rd}$ Groundfish Assessment Review Meeting \(GARM III\), Northeast Fisheries Science Center, Woods Hole, Massachusetts, August 4\-8, 2008. US Dep Commer, NOAA Fisheries, Northeast Fish Sci Cent Ref Doc. 08\-15\; 884 p + xvii. http:\/\/www.nefsc.noaa.gov\/publications\/crd\/crd0815\/ \linebreak{} \linebreak{}} \def\OPTUNITDraft{} \def\OPTUNITSPPname{Ocean Pout} \def\OPTUNITSPPnameT{Ocean Pout} \def\OPTUNITRptYr{2015} \def\OPTUNITAuthor{Susan Wigley} \def\OPTUNITReviewerComments{/home/dhennen/EIEIO/BigReport/OPT_UNIT/latex}  \def\POKUNITMyPathTab{/home/dhennen/EIEIO/BigReport/POK_UNIT/tables} \def\POKUNITMyPathFig{/home/dhennen/EIEIO/BigReport/POK_UNIT/figures} \def\POKUNITfigFishCap{Total catch of pollock between 1970 and 2014 by fleet \(commercial, Canadian, distant water fleet, and recreational\)  and disposition \(landings and discards\).} \def\POKUNITfigSSBCap{Estimated trends in the spawning stock biomass of pollock between 1970 and 2014 from the current  \(solid line\)  and previous \(dashed line\)  assessment and the corresponding  \$SSB\_{Threshold}\${} \(0.5 \* \$SSB\_{MSY}\${} proxy\; horizontal dashed line\)  as well as  \$SSB\_{Target}\${} \(\$SSB\_{MSY}\${} proxy\; horizontal dotted line\)   based on the 2015 assessment models base \(A\)  and flat sel sensitivity \(B\). Biomass was adjusted for a retrospective pattern and the adjustment is shown in red. The approximate 90\\percent lognormal confidence intervals are shown.} \def\POKUNITfigFCap{Estimated trends in age 5 to 7 average F \(\$F\_{AVG}\${}\)  of pollock between 1970 and 2014 from the current  \(solid line\)  and previous \(dashed line\)  assessment and the corresponding  \$F\_{Threshold}\${} \(\$F\_{MSY}\${} proxy\; dashed line\)  based on the 2015 assessment models base \(A\)  and flat sel sensitivity \(B\).  \$F\_{AVG}\${} was adjusted for a retrospective pattern and the adjustment is shown in red. The approximate 90\\percent lognormal confidence intervals are shown.} \def\POKUNITfigRecrCap{Estimated trends in age 1 recruitment  \(000s\)  of pollock between 1970 and 2014 from the current \(solid line\)  and previous \(dashed line\)  assessment for the assessment models base \(A\)  and flat sel sensitivity \(B\).  The approximate 90\\percent lognormal confidence intervals are shown.} \def\POKUNITfigSurvCap{Indices of abundance for pollock from the Northeast Fisheries Science Center \(NEFSC\)  spring \(1970 to 2015\)  and fall \(1970 to 2014\)  bottom trawl surveys. The approximate 90\\percent lognormal confidence intervals are shown.} \def\POKUNITPreAmb{This assessment of the pollock \(\textit{Pollachius virens}\)  stock is an update of the existing 2014 operational assessment \(Hendrickson et al. 2015\). This assessment updates commercial and recreational fishery catch data, research survey indices of abundance, the ASAP analytical models, and biological reference points through 2014. Additionally, stock projections have been updated through 2018. In what follows, there are two population assessment models brought forward from the 2014 operational assessment, the base model \(dome\-shaped survey selectivity\)  , which is used to provide management advice, and the flat sel sensitivity model \(flat\-topped survey selectivity\), which is included for the sole purpose of demonstrating the sensitivity of assessment results to survey selectivity assumptions. The most recent benchmark assessment of the pollock stock was in 2010 as part of the 50$^{th}$ Stock Assessment Review Committee \(SARC 50\; NEFSC 2010\), which includes a full description of the model formulations.} \def\POKUNITSoS{ \textbf{State of Stock: }{} The pollock \(\textit{Pollachius virens}\)  stock is not overfished and overfishing is not occurring \(Figures \ref{POKUNITSSB\_plot1}\-\ref{POKUNITF\_plot1}\){}. Retrospective adjustments were made to the model results. Retrospective adjusted spawning stock biomass \(SSB\)  in 2014 was estimated to be 154,919 \(mt\)  under the base model and 32,040 \(mt\)  under the flat sel sensitivity model which is 147 and 58\\percent \(respectively\)  of the biomass target, an  \$SSB\_{MSY}\${} proxy of SSB at  \$F\_{40\\percent}\${} \(105,226 and 54,900  \(mt\)\;  Figure \ref{POKUNITSSB\_plot1}{}\). Retrospective adjusted 2014 age 5 to 7 average fishing mortality \(F\)   was estimated to be 0.07 under the base model and 0.233 under the flat sel sensitivity model which is 25 and 92\\percent \(respectively\)  of the overfishing threshold, an  \$F\_{MSY}\${} proxy of  \$F\_{40\\percent}\${} \(0.277 and 0.252\;  Figure \ref{POKUNITF\_plot1}{}\).} \def\POKUNITProj{ \textbf{Projections: }{}Short term projections of median total fishery yield and spawning stock biomass for pollock were conducted based on a harvest scenario of fishing at an  \$F\_{MSY}\${} proxy of  \$F\_{40\\percent}\${} between 2016 and 2018. Catch in 2015 has been estimated at 5,208 \(mt\). Recruitments were sampled from a cumulative distribution function derived from ASAP estimated age 1 recruitment between 1970 and 2012.  Recruitments in 2013 and 2014 were not included due to uncertainty in those estimates. The annual fishery selectivity, natural mortality, maturity ogive, and mean weights used  in projections are the most recent 5 year averages. Retrospective adjusted age 5 to 7 average F in 2014 fell outside the 90\\percent confidence intervals of the unadjusted 2014 value under the base model \(Figure \ref{POKUNITF\_plot1}{}\). Retrospective adjusted SSB and age 5 to 7 average F in 2014 fell outside the 90\\percent confidence intervals of the unadjusted 2014 values under the flat sel sensitivity model  \(Figures \ref{POKUNITSSB\_plot1}\-\ref{POKUNITF\_plot1}\){}. Therefore, retrospective adjustments were applied in the projections for the base model and the flat sel sensitivity model.} \def\POKUNITSpecCmt{ \textbf{Special Comments: } \begin{itemize}{} \item{}What are the most important sources of uncertainty in this stock assessment?  Explain, and describe qualitatively how they affect the assessment results \(such as estimates of biomass, F, recruitment, and population projections\).  \linebreak{} \hspace\*{0.5cm} \textit{The largest source of uncertainty in the pollock assessment is selectivity, as the base model with dome\-shaped survey and fishery selectivities implies the existence of a large cryptic biomass that neither current surveys nor the fishery can confirm. Assuming flat\-topped survey selectivities leads to lower estimates of SSB and higher estimates of F  \(Figures \ref{POKUNITSSB\_plot1}\-\ref{POKUNITF\_plot1}\){}. Stock status is insensitive to the shape of the survey selectivity patterns at older ages.}  \item{} Does this assessment model have a retrospective pattern? If so, is the pattern minor, or major? \(A major retrospective pattern occurs when the adjusted SSB or  \$F\_{AVG}\${} lies outside of the approximate  joint confidence region for SSB and  \$F\_{AVG}\${}\; see  Figure \ref{RhoDecision\_tab}{}\). \linebreak{} \hspace\*{0.5cm} \textit{ The 7\-year Mohn\'s  \textrho{}, relative to SSB, was 0.291 under the base model and 0.66 under the flat sel sensitivity model in the 2014 assessment and was 0.284 and 0.789, respectively, in 2014. The 7\-year Mohn\'s  \textrho{}, relative to F, was \-0.252 under the base model and \-0.359 under the flat sel sensitivity model in the 2014 assessment and was \-0.276 and \-0.43, respectively, in 2014. There was a major retrospective pattern for the base model because the  \textrho{} adjusted estimate of 2014 F \(\$F\_{\rho}\${}=0.07\)  was outside the approximate 90\\percent confidence regions around F \(0.035 \- 0.066\). There was a major retrospective pattern for the flat sel sensitivity model because the  \textrho{} adjusted estimates of 2014 SSB \(\$SSB\_{\rho}\${}=32,040\)  and 2014 F \(\$F\_{\rho}\${}=0.233\)  were outside the approximate 90\\percent confidence regions around SSB \(37,243 \- 77,410  \(mt\)\)  and F \(0.084 \- 0.182\). A retrospective adjustment was made for both the determination of stock status and for projections of catch in 2016. The base model retrospective adjustment changed the 2014 SSB from 198,847 to 154,919 and the 2014  \$F\_{AVG}\${} from 0.051 to 0.07. The flat sel sensitivity model retrospective adjustment changed the 2014 SSB from 57,327 to 32,040 and the 2014  \$F\_{AVG}\${} from 0.133 to 0.233.}  \item{}Based on this stock assessment, are population projections well determined or uncertain? \linebreak{} \hspace\*{0.5cm} \textit{Population projections for pollock, appear to be reasonably well determined for both the base model and the flat sel sensitivity model. }  \item{}Describe any changes that were made to the current stock assessment, beyond incorporating additional years of data  and the affect these changes had on the assessment and stock status. \linebreak{} \hspace\*{0.5cm} \textit{Only one major change was made to the pollock assessment as part of this update. Likelihood constants were excluded from likelihood calculations to avoid potential bias caused by one of the recruitment likelihood constants, which is the sum of the log\-scale predicted recruitments, and therefore not a constant. Inclusion of this likelihood constant allows the assessment model to minimize the negative log likelihood by estimating lower recruitments. Exclusion of the likelihood constants led to higher estimates of SSB  and lower estimates of F  \(Figures \ref{POKUNITSSB\_plot1}\-\ref{POKUNITF\_plot1}\){}.}  \item{}If the stock status has changed a lot since the previous assessment, explain why this occurred.  \linebreak{} \hspace\*{0.5cm} \textit{Stock status based on the base model has not changed since the previous assessment. Stock status based on the flat sel sensitivity model has changed from \'overfishing is occurring\' in the previous assessment to \'overfishing is not occurring\' in the current assessment. Though, the retrospective adjusted 2014 age 5 to 7 average fishing mortality  from the flat sel sensitivity model \(0.233\)  is close to the  \$F\_{MSY}\${} proxy \(0.252\). This change in status likely is due to a decline in predicted F from 2013 to 2014, as well as to the exclusion of the likelihood constants, which led to higher predicted stock productivity.}  \item{}Indicate what data or studies are currently lacking and which would be needed most to improve this stock assessment in the future.  \linebreak{} \hspace\*{0.5cm} \textit{The pollock assessment could be improved with additional studies on gear selectivity. These studies could cover topics such as physical selectivity \(e.g., multi\-mesh gillnet\), behavior \(e.g., swimming endurance, escape behavior\), geographic and vertical distribution by size and age, tag\-recovery at size and age, and evaluating information on length\-specific selectivity at older ages.}  \item{}Are there other important issues? \linebreak{} \hspace\*{0.5cm} \textit{As in the previous assessment, the pollock assessment models had difficulty converging on a solution in some of the retrospective peels. One possible explanation for this convergence issue is that the model may be overparameterized, because the commercial and recreational fleets are modeled separately in this assessment. The possibility of combining the two fleets into a single fleet should be explored during the next benchmark assessment.} \end{itemize}{}} \def\POKUNITRefr{ \textbf{References: }{} \linebreak{}Hendrickson L, Nitschke P, Linton B. 2015. 2014 Operational stock assessments for Georges Bank winter flounder, Gulf of Maine winter flounder, and pollock. US Dept Commer, Northeast Fish Sci Cent Ref Doc. 15\-01\; 228 p. Available from: NationalMarine Fisheries Service, 166 Water Street, Woods Hole, MA 02543\-1026, or online at http:\/\/www.nefsc.noaa.gov\/publications\/ \linebreak{} \linebreak{}Northeast Fisheries Science Center. 2010. 50$^{th}$ Northeast Regional Stock Assessment Workshop \(50$^{th}$ SAW\)  Assessment Report. US Dept Commer, Northeast Fish Sci Cent Ref Doc. 10\-17\; 844 p. Available from: National Marine Fisheries Service, 166 Water Street, Woods Hole, MA 02543\-1026, or online at http:\/\/www.nefsc.noaa.gov\/nefsc\/publications\/ } \def\POKUNITDraft{} \def\POKUNITSPPname{pollock} \def\POKUNITSPPnameT{Pollock} \def\POKUNITRptYr{2015} \def\POKUNITAuthor{Brian Linton} \def\POKUNITReviewerComments{/home/dhennen/EIEIO/BigReport/POK_UNIT/latex}  \def\REDUNITMyPathTab{/home/dhennen/EIEIO/BigReport/RED_UNIT/tables} \def\REDUNITMyPathFig{/home/dhennen/EIEIO/BigReport/RED_UNIT/figures} \def\REDUNITfigFishCap{Total catch of Acadian redfish between 1913 and 2014 by fleet \(commercial and other\)  and disposition \(landings and discards\).} \def\REDUNITfigSSBCap{Trends in spawning stock biomass of Acadian redfish between 1913 and 2014 from the current  \(solid line\)  and previous \(dashed line\)  assessment and the corresponding  \$SSB\_{Threshold}\${} \(0.5 \* \$SSB\_{MSY}\${} \textit{proxy}{}\; horizontal dashed line\)  as well as  \$SSB\_{Target}\${} \(\$SSB\_{MSY}\${} \textit{proxy}{}\; horizontal dotted line\)  based on the 2015 assessment. Biomass was adjusted for a retrospective pattern and the adjustment is shown in red. The approximate 90\\percent lognormal confidence intervals are shown.} \def\REDUNITfigFCap{Trends in the fully selected fishing mortality \(\$F\_{Full}\${}\)  of Acadian redfish between 1913 and 2014 from the current \(solid line\)  and previous \(dashed line\)  assessment and the corresponding  \$F\_{Threshold}\${} \(\$F\_{MSY}\${} \textit{proxy}{}=0.038\; horizontal dashed line\)  based on the 2015 assessment.  \$F\_{Full}\${} was adjusted for a retrospective pattern and the adjustment is shown in red. The approximate 90\\percent lognormal confidence intervals are shown.} \def\REDUNITfigRecrCap{Trends in Recruits \(age 1\)  \(000s\)  of Acadian redfish between 1913 and 2014 from the current \(solid line\)  and previous \(dashed line\)  assessment. The approximate 90\\percent lognormal confidence intervals are shown.} \def\REDUNITfigSurvCap{Indices of abundance for Acadian redfish from the Northeast Fisheries Science Center \(NEFSC\)  spring \(1963 to 2015\)  and fall \(1963 to 2014\)  bottom trawl surveys. The approximate 90\\percent lognormal confidence intervals are shown.} \def\REDUNITPreAmb{This assessment of the Acadian redfish \(\textit{Sebastes fasciatus}\)  stock is an update of the existing 2012 operational assessment \(NEFSC 2012\). This assessment updates commercial fishery catch data, research survey indices of abundance, the ASAP analytical model, and biological reference points through 2014. Additionally, stock projections have been updated through 2018. The most recent benchmark assessment of the Acadian redfish stock was in 2008 as part of the 3$^{rd}$ Groundfish Assessment Review Meeting \(GARM III\; NEFSC 2008\), which includes a full description of the model formulations.} \def\REDUNITSoS{ \textbf{State of Stock: }{}Based on this updated assessment, the Acadian redfish \(\textit{Sebastes fasciatus}\)  stock is not overfished and overfishing is not occurring \(Figures \ref{REDUNITSSB\_plot1}\-\ref{REDUNITF\_plot1}\){}. Retrospective adjustments were made to the model results. Retrospective adjusted spawning stock biomass \(SSB\)  in 2014 was estimated to be 330,004 \(mt\)  which is 117\\percent of the biomass target \(\$SSB\_{MSY}\${} \textit{proxy}{} of SSB at  \$F\_{50\\percent}\${} = 281,112\;  Figure \ref{REDUNITSSB\_plot1}{}\).  The retrospective adjusted 2014 fully selected fishing mortality \(F\)  was estimated to be 0.015 which is 39\\percent of the overfishing threshold \(\$F\_{MSY}\${} \textit{proxy}{} of  \$F\_{50\\percent}\${} = 0.038\;  Figure \ref{REDUNITF\_plot1}{}\).} \def\REDUNITProj{ \textbf{Projections: }{}Short term projections of median total fishery yield and spawning stock biomass for Acadian redfish were conducted based on a harvest scenario of fishing at the  \$F\_{MSY}\${} \textit{proxy}{} between 2016 and 2018. Catch in 2015 has been estimated at 5,204 \(mt\). Recruitments were sampled from a cumulative distribution function derived from ASAP estimated age 1 recruitment between 1969 and 2014. The annual fishery selectivity, natural mortality, maturity ogive, and mean weights used  in projections are the same as those used in the assessment model. Retrospective adjusted SSB and fully selected F in 2014 fell outside the 90\\percent confidence intervals of the unadjusted 2014 values. Therefore, retrospective adjustments were applied in the projections. } \def\REDUNITSpecCmt{ \textbf{Special Comments: } \begin{itemize}{} \item{}What are the most important sources of uncertainty in this stock assessment?  Explain, and describe qualitatively how they affect the assessment results \(such as estimates of biomass, F, recruitment, and population projections\).  \linebreak{} \hspace\*{0.5cm} \textit{The largest source of uncertainty in the Acadian redfish assessment is the lack of age data, particularly from the commercial fishery. Age measurements from landings halted after 1985, due to relatively low landings. Current landings have increased to levels seen in the mid\-1980s. If landings continue to increase, then age data from the fishery will become increasingly important. Dimorphic growth is another source of uncertainty in this assessment, with females growing faster than males. The use of female weights at age in the stock projections may lead to overestimation of stock productivity, as well as having an unknown effect on biological reference points.}  \item{} Does this assessment model have a retrospective pattern? If so, is the pattern minor, or major? \(A major retrospective pattern occurs when the adjusted SSB or  \$F\_{Full}\${} lies outside of the approximate  joint confidence region for SSB and  \$F\_{Full}\${}\; see  Figure \ref{RhoDecision\_tab}{}\). \linebreak{} \hspace\*{0.5cm} \textit{ The 7\-year Mohn\'s  \textrho{}, relative to SSB, was 0.036 in the 2012 assessment and was 0.256 in 2014. The 7\-year Mohn\'s  \textrho{}, relative to F, was \-0.035 in the 2012 assessment and was \-0.190 in 2014. There was a major retrospective pattern for this assessment because the  \textrho{} adjusted estimates of 2014 SSB \(\$SSB\_{\rho}\${}=330,004\)  and 2014 F \(\$F\_{\rho}\${}=0.015\)  were outside the approximate 90\\percent confidence regions around SSB \(368,906 \- 465,828\)  and F \(0.011 \- 0.014\).  A retrospective  adjustment was made for both the determination of stock status and for projections of catch in 2016. The retrospective adjustment changed the 2014 SSB from 414,544 to 330,004 and the 2014  \$F\_{Full}\${} from 0.012 to 0.015.}  \item{}Based on this stock assessment, are population projections well determined or uncertain? \linebreak{} \hspace\*{0.5cm} \textit{Population projections for Acadian redfish appear to be reasonably well determined. }  \item{}Describe any changes that were made to the current stock assessment, beyond incorporating additional years of data  and the affect these changes had on the assessment and stock status. \linebreak{} \hspace\*{0.5cm} \textit{Only one major change was made to the Acadian redfish assessment as part of this update. Likelihood constants were excluded from likelihood calculations to avoid potential bias caused by one of the recruitment likelihood constants, which is the sum of the log\-scale predicted recruitments, and therefore not a constant. Inclusion of this likelihood constant allows the assessment model to minimize the negative log likelihood by estimating lower recruitments. Exclusion of the likelihood constants led to slightly higher estimates of SSB in recent years. }  \item{}If the stock status has changed a lot since the previous assessment, explain why this occurred.  \linebreak{} \hspace\*{0.5cm} \textit{There has been no change in the stock status of Acadian redfish since the previous assessment.}  \item{}Indicate what data or studies are currently lacking and which would be needed most to improve this stock assessment in the future.  \linebreak{} \hspace\*{0.5cm} \textit{The Acadian redfish assessment could be improved by 1\)  including additional age data, particularly from the commercial fishery, and 2\)  investigating the sensitivity of biological reference points and stock projections to the mean weights at age. }  \item{}Are there other important issues? \linebreak{} \hspace\*{0.5cm} \textit{Northeast Fisheries Science Center \(NEFSC\)  fall bottom trawl index values for 2013 and 2014 are lower than in previous years \(Figure \ref{REDUNITSurv\_plot1}{}\), but the current assessment model continues to predict an increase in SSB for the last two years \(Figure \ref{REDUNITSSB\_plot1}{}\). If future index values remain low \(i.e., if the index is responding to a change in abundance, rather than interannual variability\), then the predicted trend in SSB may change abruptly in a future assessment. Such an abrupt change may lead to an increase in the retrospective pattern.} \end{itemize}{}} \def\REDUNITRefr{ \textbf{References: }{} \linebreak{}Northeast Fisheries Science Center. 2008. Assessment of 19 Northeast Groundfish Stocks through 2007: Report of the 3$^{rd}$ Groundfish Assessment Review Meeting \(GARM III\), Northeast Fisheries Science Center, Woods Hole, Massachusetts, August 4\-8, 2008. US Dept Commer, Northeast Fish Sci Cent Ref Doc. 08\-15\; 884 p + xvii. Available from: National Marine Fisheries Service, 166 Water Street, Woods Hole, MA 02543\-1026, or online at http:\/\/www.nefsc.noaa.gov\/nefsc\/publications\/ \linebreak{} \linebreak{}Northeast Fisheries Science Center. 2012. Assessment or Data Updates of 13 Northeast Groundfish Stocks through 2010. US Dept Commer, Northeast Fish Sci Cent Ref Doc. 12\-06\; 789 p. Available from: National Marine Fisheries Service, 166 Water Street, Woods Hole, MA 02543\-1026, or online at http:\/\/www.nefsc.noaa.gov\/nefsc\/publications\/} \def\REDUNITDraft{} \def\REDUNITSPPname{Acadian redfish} \def\REDUNITSPPnameT{Acadian redfish} \def\REDUNITRptYr{2015} \def\REDUNITAuthor{Brian Linton} \def\REDUNITReviewerComments{/home/dhennen/EIEIO/BigReport/RED_UNIT/latex}  \def\CATUNITMyPathTab{/home/dhennen/EIEIO/BigReport/CAT_UNIT/tables} \def\CATUNITMyPathFig{/home/dhennen/EIEIO/BigReport/CAT_UNIT/figures} \def\CATUNITfigFishCap{Total catch of Atlantic wolffish between 1968 and 2014 by fleet \(commercial and recreational\)  and disposition \(landings and discards\). Note that a no possession limit was put in place in May 2010.} \def\CATUNITfigSSBCap{Trends in spawning stock biomass of Atlantic wolffish between 1968 and 2014 from the current  \(solid line\)  and previous \(dashed line\)  assessment and the corresponding  \$SSB\_{Threshold}\${} \(\$\dfrac{1}{2}\${} \$SSB\_{MSY}\${} \textit{proxy}{}\; horizontal dashed line\)  as well as  \$SSB\_{Target}\${} \(\$SSB\_{MSY}\${} \textit{proxy}{}\; horizontal dotted line\)   based on the current assessment.} \def\CATUNITfigFCap{Trends in the fully selected fishing mortality \(\$F\_{Full}\${}\)  of Atlantic wolffish between 1968 and 2014 from the current  \(solid line\)  and previous \(dashed line\)  assessment and the corresponding  \$F\_{Threshold}\${} \(\$F\_{MSY}\${} \textit{proxy}{}=0.243\; horizontal dashed line\). } \def\CATUNITfigRecrCap{Trends in age 1 recruits of Atlantic wolffish between 1968 and 2014 from the current \(solid line\)  and previous \(dashed line\)  assessment.} \def\CATUNITfigSurvCap{Indices of biomass for the Atlantic wolffish between 1968 and 2015 for the Northeast Fisheries Science Center \(NEFSC\)  spring and fall bottom trawl surveys, and the Massachusetts Division of Marine Fisheries \(MADMF\)  spring bottom trawl survey. The approximate 90\\percent lognormal confidence intervals are shown. NEFSC indices for 2009\-2015 are calibrated using the ocean pout coefficient from Miller et al. \(2010\).} \def\CATUNITPreAmb{This assessment of the Atlantic wolffish \(\textit{Anarhichas lupus}\)  stock is an update of the existing 2012 operational assessment \(NEFSC 2012\). Based on the previous assessment the stock was overfished, but overfishing was not occurring. This assessment updates commercial fishery catch data, research survey indices of abundance, and the analytical assessment models and reference points through 2014.} \def\CATUNITSoS{ \textbf{State of Stock: }{}Based on this updated assessment, the Atlantic wolffish \(\textit{Anarhichas lupus}\)  stock is overfished and overfishing is not occurring \(Figures \ref{CATUNITSSB\_plot1}\-\ref{CATUNITF\_plot1}\){}. Retrospective adjustments were not made to the model results. Spawning stock biomass \(SSB\)  in 2014 was estimated to be 638 \(mt\)  which is 38\\percent of the biomass target \(\$SSB\_{MSY}\${} \textit{proxy}{} = 1,663\;  Figure \ref{CATUNITSSB\_plot1}{}\).  The 2014 fully selected fishing mortality was estimated to be 0.003 which is 1\\percent of the overfishing threshold proxy \(\$F\_{MSY}\${} \textit{proxy}{} = 0.243\;  Figure \ref{CATUNITF\_plot1}{}\).} \def\CATUNITProj{} \def\CATUNITSpecCmt{ \textbf{Special Comments: } \begin{itemize}{} \item{}What are the most important sources of uncertainty in this stock assessment?  Explain, and describe qualitatively how they affect the assessment results \(such as estimates of biomass, F, recruitment, and population projections\).  \linebreak{} \hspace\*{0.5cm} \textit{The primary sources of uncertainty are the use of the ocean pout calibration coefficient, and the change to a no possession limit in May 2010. The ocean pout calibration coefficient \(4.575\)  is one of the largest for any species \(Miller et al. 2010\), and results in lower biomass estimates. The change to a no possession limit places greater importance on discard mortality. Additionally, it is unclear whether the lack of a recruitment index since 2004 is due to an actual decrease in recruitment, or a change in catchability resulting from the increase in liner mesh size associated with the switch to the Bigelow. Other sources of uncertainty were identified in previous Atlantic wolffish assessments \(NDPSWG 2009, NEFSC 2012\): the surveys may have reached the limit of wolffish detectability due to the decline in abundance\; and the lack of commercial length information results in model estimation difficulties for fishery selectivity.}  \item{} Does this assessment model have a retrospective pattern? If so, is the pattern minor, or major? \(A major retrospective pattern occurs when the adjusted SSB or  \$F\_{Full}\${} lies outside of the approximate  joint confidence region for SSB and  \$F\_{Full}\${}\; see  Figure \ref{RhoDecision\_tab}{}\). \linebreak{} \hspace\*{0.5cm} \textit{This assessment has retrospective patterns with Mohn\'s rho = 0.83 for SSB and \-0.36 for F. Confidence intervals are not available because MCMC is not fully developed for the SCALE model. Thus, retrospective adjustments were not done for this assessment.}  \item{}Based on this stock assessment, are population projections well determined or uncertain? \linebreak{} \hspace\*{0.5cm} \textit{Population projections for Atlantic wolffish were not done. Due to the uncertainties in the assessment, the Northeast Data Poor Stocks Working Group \(NDPSWG 2009\)  concluded that stock projections would be unreliable and should not be conducted.}  \item{}Describe any changes that were made to the current stock assessment, beyond incorporating additional years of data  and the affect these changes had on the assessment and stock status. \linebreak{} \hspace\*{0.5cm} \textit{Commercial discards for the entire time series were revised assuming 8\\percent discard mortality based on a recent study by Grant and Hiscock \(2014\). A sensitivity run with the revised discard estimates was presented to the Peer Review Panel during the 2015 Operational Assessments. This became the accepted run. There was no change in stock status resulting from the adoption of the 8\\percent discard mortality run. \linebreak{} \hspace\*{0.5cm}Recreational landings for the entire time series were revised due to an updated grand mean, and the MRFSS\/MRIP calibration for 1981\-2003. This had a negligible effect on the assessment, and there was no change in stock status.}  \item{}If the stock status has changed a lot since the previous assessment, explain why this occurred.  \linebreak{} \hspace\*{0.5cm} \textit{Stock status has not changed since the previous assessment.}  \item{}Indicate what data or studies are currently lacking and which would be needed most to improve this stock assessment in the future.  \linebreak{} \hspace\*{0.5cm} \textit{The Atlantic wolffish maturity study in the Gulf of Maine is ongoing. Increased sample size since the previous assessment allowed the use of a revised knife edge maturity of 50 cm in this assessment. Continued histological sampling over the next several years should allow for the development of a definitive maturity ogive that can be used in the next assessment.}  \item{}Are there other important issues? \linebreak{} \hspace\*{0.5cm} \textit{Recruitment at the end of the time series increases toward the initial recruitment estimate \(Table 1\; Figure 3\)  because there is no information in the model to inform these estimates. There is no indication in the data that recruitment has increased recently.  \linebreak{} \hspace\*{0.5cm}Approximate 90\\percent lognormal confidence intervals are not shown in Figures 1\-3 because MCMC is not fully developed for the SCALE model.} \end{itemize}{}} \def\CATUNITRefr{ \textbf{References: }{} \linebreak{} \linebreak{}Grant SM, Hiscock W. 2014. Post\-capture survival of Atlantic wolffish  \(\textit{Anarhichas lupus}\)  captured by bottom otter trawl: Can live release programs contribute to the recovery of species at risk? Fish Res 151:169\-176 \linebreak{} \linebreak{}Miller TJ, Das C, Politis PJ, Miller AS, Lucey SM, Legault CM, Brown RW, Rago PJ. 2010. Estimation of Albatross IV to Henry B. Bigelow calibration factors. US Dep Commer, Northeast Fish Sci Cent Ref Doc. 10\-05\; 233 p. http:\/\/www.nefsc.noaa.gov\/publications\/crd\/crd1005\/ \linebreak{} \linebreak{}Northeast Fisheries Science Center \(NEFSC\). 2012. Assessment or Data Updates of 13 Northeast Groundfish Stocks through 2010. US Dep Commer, Northeast Fish Sci Cent Ref Doc. 12\-06\; 789 p. http:\/\/www.nefsc.noaa.gov\/publications\/crd\/crd1206\/ \linebreak{} \linebreak{}Northeast Data Poor Stocks Working Group \(NDPSWG\). 2009. The Northeast Data Poor Stocks Working Group Report, December 8\-12, 2008 Meeting. Part A. Skate species complex, deep sea red crab, Atlantic wolffish, scup, and black sea bass. US Dept Commer, Northeast Fish Sci Cent Ref Doc. 09\-02\; 496 p. http:\/\/www.nefsc.noaa.gov\/publications\/crd\/crd0902\/ \linebreak{} \linebreak{}} \def\CATUNITDraft{} \def\CATUNITSPPname{Atlantic wolffish} \def\CATUNITSPPnameT{Atlantic wolffish} \def\CATUNITRptYr{2015} \def\CATUNITAuthor{Charles Adams} \def\CATUNITReviewerComments{/home/dhennen/EIEIO/BigReport/CAT_UNIT/latex} 
%Test combining several pdf doc together...
\documentclass[10pt]{article}
\usepackage{multicol}
\usepackage[pdftex]{graphicx}
\usepackage{setspace}
\usepackage{tabu}
\usepackage{amsmath}
\usepackage{longtable}
\usepackage{pdflscape}
%\usepackage[width=\textwidth]{caption}
\usepackage[width=5in]{caption}   %%%Adjusted 10 5 15
		\captionsetup[table]{skip=5pt}
\usepackage{threeparttablex}
\usepackage{rotating}
\usepackage{tabularx}
\usepackage[table]{xcolor}
\usepackage[showframe=false]{geometry}
\usepackage{changepage}
\usepackage{fancyhdr}
\usepackage{titling}
\usepackage{datetime}
\usepackage{adjustbox}
\usepackage{float}
\usepackage{textgreek}
\usepackage{lastpage}
\usepackage{pdfpages}
\floatstyle{plaintop}
%\restylefloat{table}
\setlength{\headheight}{15.2pt}
\setlength\parindent{0pt}

%-----  Styles for section headers-----
\usepackage{sectsty}                           % Use to change section fonts
\sectionfont{\raggedright\bfseries\sffamily\large}
\subsectionfont{\bfseries\sffamily\normalsize}
\subsubsectionfont{\mdseries\sffamily\normalsize}
\paragraphfont{\mdseries\sffamily\normalsize}
\subparagraphfont{\mdseries\sffamily\normalsize}
\renewcommand{\captionfont}{\mdseries\sffamily\normalsize} %Figure and table caption fonts to match



%-----Page layout commands-----
%\usepackage[letterpaper]{geometry}
%\geometry{height=8.75in,width=7in,marginratio=1:1}
%----- Paragraph formatting -----
\setlength{\parindent}{0pt}
\setlength{\parskip}{2.0ex plus 2.0ex minus 1.0ex}


%----- Define CRD details -----
\newcommand{\crdfontfamily}{\sffamily}
\newcommand{\crdnumber}{15-XXXX}         %reference document number
\newcommand{\crdauthors}{Northeast Fisheries Science Center} %authors
\newcommand{\crdauthorsabbrev}{NEFSC}
\newcommand{\crdtitle}{Stock Assessment Update of 20 Northeast Groundfish Stocks Through 2014} %title
\newcommand{\crdyear}{2015} %year
\newcommand{\crddate}{October \crdyear} % DATE
%citation of crd on 3rd cover page
\newcommand{\crdciteas}{\crdauthorsabbrev\ \crdyear. \crdtitle. US Dept Commer, Northeast
Fish Sci Cent Ref Doc. \crdnumber; \pageref*{LastPage} p. Available from: National Marine Fisheries Service,
166 Water Street, Woods Hole, MA 02543-1026, or online at \href{http://www.nefsc.noaa.gov/nefsc/publications/}{http://www.nefsc.noaa.gov/nefsc/publications/}}

%%%%%%%%%%% BEGIN Hyperref %%%%%%%%%%%%
\usepackage[colorlinks,bookmarks,bookmarksnumbered,linktocpage]{hyperref}
\definecolor{colora}{rgb}{0.0, 0.0, 0.7}
\definecolor{colorb}{rgb}{0.0, 0.4, 0.2}
\definecolor{colorc}{rgb}{0.5, 0.0, 0.4}
\hypersetup{pdftitle=Assessment Update of 20 Groundfish Stocks Through 2014}
\hypersetup{pdfauthor={\crdauthors}}
\hypersetup{linkcolor=colora}
\hypersetup{citecolor=colorb}
\hypersetup{urlcolor=colorc}
\hypersetup{pdfstartview=Fit}
\hypersetup{pdfpagemode=none}
%%%%%%%%%%% END Hyperref %%%%%%%%%%%%


%----- Section numbering setup -----
%begin section numbering at "1"
\setcounter{section}{0}
%-Set the paragraph section type to display counter - starts with "B"
\setcounter{secnumdepth}{5}
%\renewcommand{\thetable}{B\arabic{table}}
%\renewcommand{\thefigure}{B\arabic{figure}}
\renewcommand{\thetable}{\arabic{table}}
\renewcommand{\thefigure}{\arabic{figure}}


%%%%%%%%%%% Begin header - footer definition%%%%%%%%%%%
\pagestyle{fancy}
\renewcommand{\headrulewidth}{0pt}
\renewcommand{\footrulewidth}{0pt}
%\fancypagestyle{empty}
\lhead{ }
\chead{ }
\rhead{ }
%\rhead{\sarcfontfamily \SSS}
\lfoot{{Groundfish Operational Assessments \the\year{}}}
%\rfoot{\sarcfontfamily Assessment Report}
%\rfoot{\textit{Draft report for peer review only}} %for draft document
\rfoot{\textit{\leftmark}} %adds the section name to the footer (with number)
%\cfoot{\sarcfontfamily \thepage}
\cfoot{\thepage}
%%%%%%%%%%%%%%%%%%%%%%%%%%%%%%%%%%%%%
\renewcommand\sectionmark[1]{\markboth{#1}{}} %hopefully remove the section number from footer!
%hyphenation suggestions
\hyphenation{Massachusetts Mass-a-chu-setts}


\begin{document}

\pagenumbering{roman}
\setcounter{page}{1}  %beginning page number

%--------Here are the Ref. Doc coverpages -------

\begin{titlingpage}
\crdfontfamily
%\includegraphics[height=2in,keepaspectratio=true]{NOAA_foam_dome.jpg}
\includegraphics[height=2in,keepaspectratio=true]{NOAA_Transparent_Logo.png}
\raisebox{1.5in}[2in][0in]{\large \textbf{Northeast Fisheries Science Center Reference Document \crdnumber}}
\vspace*{\fill}
\begin{center}
\Large
{\LARGE \textbf{\crdtitle}}\\
\vspace*{11ex}
by \crdauthors\\[3ex]
\vspace*{\fill}
\crddate
\end{center}
\end{titlingpage}

\begin{titlingpage}
\begin{center}
\Large \crdfontfamily 
{\large  \textbf{Northeast Fisheries Science Center Reference Document \crdnumber}}\\
\vspace*{\fill}
{\LARGE \textbf{\crdtitle}} \\
\vspace*{11ex}
by \crdauthors\\[3ex]
{\normalsize NOAA, National Marine Fisheries Service,\\
Northeast Fisheries Science Center, 166 Water Street, Woods Hole, MA 02543}\\[20ex]
\textbf{U.S. Department of Commerce}\\
National Oceanic and Atmospheric Administration\\
National Marine Fisheries Service\\
Northeast Fisheries Science Center\\
Woods Hole, Massachusetts\\
\vspace*{\fill}
\crddate
\end{center}
\end{titlingpage}

\begin{titlingpage}
\vspace*{\fill}
\begin{center}
\setlength{\fboxrule}{2pt}
\fbox{
\crdfontfamily
\parbox[b][][c]{0.9\textwidth}{
\parbox[t][][t]{0.9\textwidth}{\Large \center \textbf{Northeast Fisheries Science Center Reference Documents}}\\[4ex]

\parbox[c][][c]{0.9\textwidth}{
\textbf{This series is a secondary scientific series} designed to assure the long-term documentation and
to enable the timely transmission of research results by Center and/or non-Center researchers,
where such results bear upon the research mission of the Center (see the outside back cover for
the mission statement). These documents receive internal scientific review, and most receive
copy editing. The National Marine Fisheries Service does not endorse any proprietary material,
process, or product mentioned in these documents.\vspace{6pt}

All documents issued in this series since April 2001, and several documents issued prior to
that date, have been copublished in both paper and electronic versions. To access the electronic
version of a document in this series, go to http://www.nefsc.noaa.gov/nefsc/publications/. The
electronic version is available in PDF format to permit printing of a paper copy directly from
the Internet. If you do not have Internet access, or if a desired document is one of the pre-April
2001 documents available only in the paper version, you can obtain a paper copy by contacting
the senior Center author of the desired document. Refer to the title page of the document for
the senior Center author's name and mailing address. If there is no Center author, or if there is
corporate (i.e., non-individualized) authorship, then contact the Center's Woods Hole Laboratory
Library (166 Water St., Woods Hole, MA 02543-1026).
\vspace{6pt}

\textbf{Information Quality Act Compliance}: In accordance with section 515 of Public Law 106-
554, the Northeast Fisheries Science Center completed both technical and policy reviews for
this report. These predissemination reviews are on file at the NEFSC Editorial Office.
This document may be cited as:

\vspace*{7ex}
\begin{center}
\parbox[c][][c]{0.7\textwidth}{\small \crdciteas}
\end{center}
\vspace*{7ex}

}}
}
%\end{minipage}
\end{center}
\end{titlingpage}


%-------------- Table of Contents ----------------
%\section*{Table of Contents}
\tableofcontents
\clearpage
\pagenumbering{arabic}

%\input{def2.tex}
%%%%%%%%%%%%%%%%%%%%%%%%%%%%%%%%%%%%%%%%%%%%%%%%%%%%%%%%%%%%%%%%%%%%%%%%%%%%%%%%%%%%%%%%%%%%%%%%%%%%%%%%%%%%
%Executive Summary
%\newcommand{\ExSumPath}{/net/home2/dhennen/testEIEIO/BigReport/ExSum}
\newcommand{\ExSumPath}{../ExSum}
%Executive Summary

\section{Executive Summary}
\textit{Note:Working Paper}


Update assessments were conducted for the twenty stocks in the Northeast Multispecies Fishery Management Plan in 2015 (Table \ref{stock_abbrv_tab}). The updates replicated the methods recommended in the most recent benchmark decisions, as modified by any subsequent operational assessments or updates (Table \ref{stock_info_tab}), with the intention of simply adding years of data (Table \ref{data_used_tab}). However, minor flexibility was allowed to address emerging issues (Table \ref{Assess_type_tab}).

Stock status did not change for 15 of the 20 stocks, worsened for two stocks, improved for one stock, and became more uncertain for two stocks (Table \ref{sos_tab}).

The number of stocks with retrospective adjustments applied increased from the last assessment from 2 to 7 (Table \ref{RhoAdjust_tab}). The previous Georges Bank cod assessment did apply a retrospective adjustment, however, the assessment model was not approved at the 2015 Updates so it has been excluded from these counts.

While the number of overfished stocks and stocks experiencing overfishing has generally decreased since 2007 (Figure \ref{stock_status}), the magnitude of overfishing or depletion for several stocks has worsened considerably (Figures \ref{propFmsy} and \ref{propBmsy}); Gulf of Maine cod, Southern New England/Mid-Atlantic yellowtail flounder, witch flounder and Cape Cod/Gulf of Maine yellowtail flounder). Of those Northeast groundfish stocks for which stock status can be determined, the majority remain below their biomass targets ($69\%$; Figures \ref{stock_status} and \ref{propBmsy}).

Recent NEFSC survey biomass indices for both the spring and fall surveys are below the long term means. For the majority of stocks the average of the most recent five years are below the time series means (Figures \ref{nefscSpringResiduals} and \ref{nefscFallResiduals})

Estimates of overall (aggregate) groundfish minimum swept area biomass are at, or near, all-time highs (Figures \ref{nefscSpringMinSweptAreaBiomass} and \ref{nefscFallMinSweptAreaBiomass}).  However, the current stock diversity of the overall groundfish biomass is less than that seen in the 1960s and 1970s. Current groundfish biomass is dominated by only a few stocks: For example the combined biomass of the Georges Bank haddock, Gulf of Maine haddock, and redfish stocks currently make up more than $80\%$ of the overall groundfish biomass (Figure \ref{propBiomassHaddockRedfish}). 

Information supplemental to the assessment report for each stock can found on the Stock Assessment Support Information (\href{http://www.nefsc.noaa.gov/saw/sasi/sasi_report_options.php}{SASINF}{}) website.

The appendix to this document contains: The letter from the Northeast Regional Coordinating Council providing guidance on the operational assessment procedure (Section \ref{NRCCletter}), a summary of the meeting with the Assessment Oversight Panel during which assessment plans were developed (Section \ref{AOPsum}), a summary of NEFSC outreach on 2015 groundfish operational assessments (Section \ref{OutreachSum}) and statements from fishing industry members (Section \ref{IndLetter}).  

%%%%%%%%%%%%%%%%%%%%%%%%%%%%%%%%%%%%%%%%%%%%%%%%%%%%%%%%%%%%%%%%%%%%%%%%%%%%%%%%%%%%%%%%%%%%%%%%%%%%%%%
%Tables
\clearpage
\begin{table}
	\centering
	
	\caption{ List of stocks included in the groundfish operational assessment and the abbreviations used for each in this document.}
	\label{stock_abbrv_tab}
	\begin{tabular}{ll}
\hline
Stock Abbrev & Stock Name \\
\hline 
CODGM & Gulf of Maine cod \\
CODGB & Georges Bank cod \\
HADGM & Gulf of Maine haddock \\
HADGB & Georges Bank haddock \\
YELCCGM & Cape Cod/Gulf of Maine yellowtail flounder \\
YELSNEMA & Southern New England/Mid-Atlantic yellowtail flounder \\
FLWGB & Georges Bank winter flounder \\
FLWSNEMA & Southern New England/Mid-Atlantic winter flounder \\
REDUNIT & Acadian redfish \\
PLAUNIT & American plaice \\
WITUNIT & Witch flounder \\
HKWUNIT & White hake \\
POLUNIT & Pollock \\
CATUNIT & Wolffish \\
HALUNIT & Atlantic halibut \\
FLDGMGB & Gulf of Maine/Georges Bank windowpane flounder \\
FLDSNEMA & Southern New England/Mid-Atlantic windowpane flounder \\
OPTUNIT & Ocean pout \\
FLWGM & Gulf of Maine winter flounder \\
YELGB & Georges Bank yellowtail flounder \\
\hline
	\end{tabular}
\end{table}
\clearpage
\newcolumntype{L}{>{\centering}m{2cm}} 
\newcolumntype{O}{>{\centering}m{3cm}}
\newcolumntype{P}{>{\centering}m{1.8cm}} 
\begin{sidewaystable}[ht]  
	\centering
	\captionsetup{width=\textwidth}
	\caption{Lead scientist for each stock (current$/$previous if different), information about last assessment, including: the forum for review of the last assessment (Forum), the type of assessment done (Type), publication year (Pub.), the terminal year of the catch data included (Term. yr.), overfished/overfishing status, rebuilding status, and reference. \textit{Note: Op. Assess $=$ Operational Assessment}}
	\label{stock_info_tab}
	\small{	
	\begin{tabular}{
	m{2cm}@{\hspace{.1cm}}
	O@{\hspace{.1cm}}
	m{2cm}@{\hspace{.1cm}}
	m{1.75cm}@{\hspace{.1cm}}
	m{1.cm}@{\hspace{.1cm}}
	m{1cm}@{\hspace{.1cm}}
	P@{\hspace{.1cm}}
	P@{\hspace{.1cm}}
	m{1.5cm}@{\hspace{.1cm}}
	m{1.5cm}@{\hspace{.1cm}}
	}
	\hline

Stock & 
Lead & 
Forum & 
Type & 
Pub. & 
Term. yr. & 
Overfished? & 
Overfishing? & 
Rebuild status & 
Reference \\
	
	\hline
CODGM & Palmer & Op. Assess & Update & 2014 & 2013 & Yes & Yes & By 2024 & \href{http://www.nefsc.noaa.gov/publications/crd/crd1414/}{CRD14-14} \\
CODGB & O'Brien & SARC 55 & Benchmark & 2012 & 2011 & Yes & Yes & By 2026 & \href{http://nefsc.noaa.gov/publications/crd/crd1311/}{CRD13-11} \\
HADGM & Palmer & SARC 59 & Benchmark & 2014 & 2013 & No & No & Rebuilt & \href{http://nefsc.noaa.gov/publications/crd/crd1409/}{CRD14-09} \\
HADGB & Brooks & GARM2012 & Update & 2012 & 2010 & No & No & Rebuilt & \href{http://www.nefsc.noaa.gov/publications/crd/crd1206/}{CRD12-06} \\
YELCCGM & Alade$/$Legault & GARM2012 & Update & 2012 & 2010 & Yes & Yes & By 2023 & \href{http://www.nefsc.noaa.gov/publications/crd/crd1206/}{CRD12-06} \\
YELSNEMA & Alade & SARC 54 & Benchmark & 2012 & 2011 & No & No & Rebuilt & \href{http://www.nefsc.noaa.gov/publications/crd/crd1218/}{CRD12-18} \\
FLWGB & Hendrickson & Op. Assess & Update & 2015 & 2013 & No & No & By 2017 & \href{http://www.nefsc.noaa.gov/publications/crd/crd1501/}{CRD15-01} \\
FLWSNEMA & Wood$/$Terciero & SARC 52 & Benchmark & 2011 & 2010 & Yes & No & By 2023 &  \href{http://www.nefsc.noaa.gov/saw/saw52/crd1117.pdf}{SARC52} \\
REDUNIT & Linton$/$Miller & GARM2012 & Update & 2012 & 2010 & No & No & Rebuilt & \href{http://www.nefsc.noaa.gov/publications/crd/crd1206/}{CRD12-06} \\
PLAUNIT & O'Brien & GARM2012 & Update & 2012 & 2010 & No & No & By 2024 & \href{http://www.nefsc.noaa.gov/publications/crd/crd1206/}{CRD12-06} \\
WITUNIT & Wigley & GARM2012 & Update & 2012 & 2010 & Yes & Yes & By 2017 & \href{http://www.nefsc.noaa.gov/publications/crd/crd1206/}{CRD12-06} \\
HKWUNIT & Sosebee & SARC 56 & Benchmark & 2013 & 2011 & No & No & By 2014 & \href{http://www.nefsc.noaa.gov/publications/crd/crd1310/}{CRD13-10} \\
POLUNIT & Linton & Op. Assess & Update & 2015 & 2013 & No & No & Rebuilt & \href{http://www.nefsc.noaa.gov/publications/crd/crd1501/}{CRD15-01} \\
CATUNIT & Adams$/$Keith & GARM2012 & Update & 2012 & 2010 & Yes & No & Unknown & \href{http://www.nefsc.noaa.gov/publications/crd/crd1206/}{CRD12-06} \\
HALUNIT & Hennen$/$Blaylock & GARM2012 & Update & 2012 & 2010 & Yes & No & By 2055 & \href{http://www.nefsc.noaa.gov/publications/crd/crd1206/}{CRD12-06} \\
FLDGMGB & Chute$/$Hendrickson & GARM2012 & Update & 2012 & 2010 & Yes & Yes & By 2017 & \href{http://www.nefsc.noaa.gov/publications/crd/crd1206/}{CRD12-06} \\
FLDSNEMA & Chute$/$Hendrickson & GARM2012 & Update & 2012 & 2010 & No & No & Rebuilt & \href{http://www.nefsc.noaa.gov/publications/crd/crd1206/}{CRD12-06} \\
OPTUNIT & Wigley & GARM2012 & Update & 2012 & 2010 & Yes & No & By 2014 & \href{http://www.nefsc.noaa.gov/publications/crd/crd1206/}{CRD12-06} \\
FLWGM & Nitschke & Op. Assess & Update & 2015 & 2013 & Unknown & No & Unknown & \href{http://www.nefsc.noaa.gov/publications/crd/crd1501/}{CRD15-01} \\
YELGB & Legault & TRAC 2015 & Update & 2015 & 2014 & Unknown & Unknown & By 2032 & \href{http://www.nefsc.noaa.gov/saw/trac/TSR_2015_GBYellowTailFlounder.pdf}{TRAC2015}\\
	\hline
	\end{tabular}
}
\end{sidewaystable}



\clearpage
\newcommand{\colspc}{.2cm}
\newcolumntype{Q}{c@{\hspace{\colspc}}}

\begin{sidewaystable}[ht]
%\begin{table}
\captionsetup{width=\textwidth}
\centering
\caption{Data used in each assessment. The column heads are US commercial landings (US c-lnd), US commercial discards (US c-dis), US recreational landings (US r-lnd), US recreational discards (US r-dis), Canadian catch (CA cat), Northeast Fisheries Science Center spring, fall and winter surveys (NE S, NE F and NE W), Massachusetts spring and fall surveys (MA S and MA F), Maine/New Hampshire spring and fall surveys (ME/NH S and ME/NH F) and Canadian Department of Fisheries and Oceans February survey (DFO S).} 
\label{data_used_tab}
{\small
\begin{tabular}{l@{\hspace{\colspc}}
Q 
Q 
Q 
Q 
Q
m{1mm} 
Q 
Q 
Q 
Q 
Q 
Q 
Q 
Q 
}

\hline

& \multicolumn{5}{c}{\textbf{Catch}} && \multicolumn{8}{c}{\textbf{Surveys}} \\  
\cline{2-6} \cline{8-15}
Stock & US c-lnd & US c-dis & US r-lnd & US r-dis & CA Cat && NE S & NE F & NE W & MA S & MA F & ME/NH S & ME/NH F & DFO S \\

\hline
CODGM & Yes & Yes & Yes & Yes & No && Yes & Yes & No & Yes & No & No & No & No \\
CODGB & Yes & Yes & Yes & Yes & Yes && Yes & Yes & No & No & No & No & No & Yes \\
HADGM & Yes & Yes & Yes & Yes & No && Yes & Yes & No & No & No & No & No & No \\
HADGB & Yes & Yes & No & No & Yes && Yes & Yes & No & No & No & No & No & Yes \\
YELCCGM & Yes & Yes & No & No & No && Yes & Yes & No & Yes & Yes & Yes & Yes & No \\
YELSNEMA & Yes & Yes & No & No & No && Yes & Yes & Yes & No & No & No & No & No \\
FLWGB & Yes & Yes & No & No & Yes && Yes & Yes & No & No & No & No & No & Yes \\
FLWSNEMA & Yes & Yes & Yes & Yes & No && Yes & Yes & Yes & Yes & No & No & No & No \\
REDUNIT & Yes & Yes & No & No & No && Yes & Yes & No & No & No & No & No & No \\
PLAUNIT & Yes & Yes & No & No & Yes && Yes & Yes & No & Yes & Yes & No & No & No \\
WITUNIT & Yes & Yes & No & No & No && Yes & Yes & No & No & No & No & No & No \\
HKWUNIT & Yes & Yes & No & No & Yes && Yes & Yes & No & No & No & No & No & No \\
POLUNIT & Yes & Yes & Yes & Yes & No && Yes & Yes & No & No & No & No & No & No \\
CATUNIT & Yes & Yes & Yes & No & No && Yes & Yes & No & Yes & No & No & No & No \\
HALUNIT & Yes & Yes & No & No & Yes &&  No & Yes & No & No & No & No & No & No \\
FLDGMGB & Yes & Yes & No & No & No && No & Yes & No & No & No & No & No & No \\
FLDSNEMA & Yes & Yes & No & No & No &&  No & Yes & No & No & No & No & No & No \\
OPTUNIT & Yes & Yes & No & No & No && Yes & No & No & No & No & No & No & No \\
FLDWGM & Yes & Yes & Yes & Yes & No && Yes & Yes & No & Yes & Yes & Yes & Yes & No \\
YELGB & Yes & Yes & No & No & Yes &&  Yes & Yes & No & No & No & No & No & Yes \\
   \hline
\end{tabular}
}

\end{sidewaystable}
%\end{table}
\clearpage
\newcolumntype{L}{>{\centering}m{2cm}}
\newcolumntype{M}{>{\centering}m{2.25cm}}
\newcolumntype{N}{>{\centering}m{1cm}}


\begin{sidewaystable}[ht]  
	\centering
	\captionsetup{width=\textwidth}
	\caption{Assessment type and reference points from previous assessment. Biomass and yield values are in metric tons. \textit{Note: sp=stochastic projection and surv. B = survey biomass.}}
	\label{Assess_type_tab}
	\small{	
	\begin{tabular}{
	m{2.5cm}@{\hspace{.1cm}}
	m{1.25cm}@{\hspace{.1cm}}
	M@{\hspace{.1cm}}   
	L@{\hspace{.2cm}}
	N@{\hspace{.2cm}}   
	L@{\hspace{.1cm}}
	m{0.75cm}@{\hspace{.1cm}}
	L@{\hspace{.2cm}}
	m{1.5cm}@{\hspace{.1cm}}
	L@{\hspace{.2cm}}
	m{1.5cm}@{\hspace{0cm}}
	}

	\hline

Stock & Assess. & Type & F def. & B def. & $F_{MSY}$ type & $F_{MSY}$ value & $B_{MSY}$ type & $B_{MSY}$ value & MSY type & MSY value \\
\hline
CODGM(M=.2) & ASAP & age-based & $F_{Full}$ & SSB & $F_{40\%SPR}$ & 0.18 & sp & 47,184 & sp & 7,753 \\
CODGM($M_{ramp}$) & ASAP & age-based & $F_{Full}$ & SSB & $F_{40\%SPR}$ & 0.18 & sp & 69,621 & sp & 11,388 \\
CODGB & ASAP & age-based & $F_{Full}$ & SSB & $F_{40\%SPR}$ & 0.18 & sp & 186,535 & sp & 30,622 \\
HADGM & ASAP & age-based & $F_{Full}$ & SSB & $F_{40\%SPR}$ & 0.46 & sp & 4,108 & sp & 955 \\
HADGB & VPA & age-based & avg F ages 5-7 & SSB & $F_{40\%SPR}$ & 0.39 & sp & 124,900 & sp & 28,000 \\
YELCCGOM & VPA & age-based & avg F ages 4-6 & SSB & $F_{40\%SPR}$ & 0.26 & sp & 7,080 & sp & 1,600 \\
YELSNEMA & ASAP & age-based & avg F ages 4-5 & SSB & $F_{40\%SPR}$ & 0.32 & sp & 2,995 & sp & 773 \\
FLWGB & VPA & age-based & avg F ages 4-6 & SSB & Fmsy & 0.44 & sp & 8,100 & sp & 3,200 \\
FLWSNEMA & ASAP & age-based & avg F ages 4-5 & SSB & Fmsy & 0.29 & sp & 43,661 & sp & 11,728 \\
REDUNIT & ASAP & age-based & $F_{Full}$ & SSB & $F_{50\%SPR}$ & 0.04 & sp & 238,480 & sp & 8,891 \\
PLAUNIT & VPA & age-based & avg F ages 6-9 & SSB & $F_{40\%SPR}$ & 0.18 & sp & 18,398 & sp & 3,385 \\
WITUNIT & VPA & age-based & avg F ages 8-11 & SSB & $F_{40\%SPR}$ & 0.27 & sp & 10,051 & sp & 2,075 \\
HKWUNIT & ASAP & age-based & $F_{Full}$ & SSB & $F_{40\%SPR}$ & 0.20 & sp & 32,400 & sp & 5,630 \\
POLUNIT & ASAP & age-based & avg F ages 5-7 & SSB & $F_{40\%SPR}$ & 0.27 & sp & 76,879 & sp & 14,791 \\
CATUNIT & SCALE & length-based & $F_{Full}$ & SSB & $F_{40\%SPR}$ & 0.33 & sp & 1,756 & sp & 261 \\
HALUNIT & RYM & \centering{surplus production} & \centering{biomass wted F} & B & F0.1 & 0.07 & deterministic & 48,509 & \center{deterministic} & 3,546 \\
FLDGMGB & AIM & index & $\frac{catch}{surv. B}$ & surv. B & replacement ratio & 0.44 & $\frac{MSY{}\textit{proxy}}{F_{MSY proxy}}$ & 1.60 & median catch 1995-2001 & 700 \\
FLDSNEMA & AIM & index & $\frac{catch}{surv. B}$ & surv. B & replacement ratio & 2.09 & $\frac{MSY{}\textit{proxy}}{F_{MSY proxy}}$ & 0.24 & median catch 1995-2001 & 500 \\
OPTUNIT & index & index & $\frac{catch}{surv. B}$ & surv. B & med. $F_{1977-1985}$ & 0.76 & med. surv. $B_{1977-1985}$ & 4.94 & \centering{$F_{MSY}$ * $B_{MSY}$} & 3,754 \\
FLWGM & empirical & \centering{survey expansion} & $\frac{catch}{B_{30+cm}}$ & surv. B &  \centering{$F_{40\%}$ from YPR} & 0.23 & NA & NA & NA & NA \\
YELGB & empirical & survey expansion & NA & surv. B & NA & NA & NA & NA & NA & NA \\


	\hline
	\end{tabular}
}
\end{sidewaystable}


%\centering{7,753 ($M=0.2$) or 11,388 (Mramp)}
%\centering{47,184 (M=0.2) or 69,621 (Mramp)}
\clearpage
%sos_tab
\begin{table}
	\centering	
	\caption{ Synopsis of status by stock.}
	\label{sos_tab}
	\begin{tabular}{lcccc}
	\hline


Stock & Last Assessment & Status Change? & Overfishing? & Overfished? \tabularnewline
\hline
CODGM & 2014 & \cellcolor{yellow} Same & \cellcolor{red} Yes & \cellcolor{red} Yes \tabularnewline
CODGB & 2012 & \cellcolor{orange} More uncertain & \cellcolor{yellow} Unknown & \cellcolor{red} Yes \tabularnewline
HADGM & 2012 & \cellcolor{yellow} Same & \cellcolor{green} No & \cellcolor{green} No \tabularnewline
HADGB & 2014 & \cellcolor{yellow} Same & \cellcolor{green} No & \cellcolor{green} No \tabularnewline
YELCCGM & 2012 & \cellcolor{yellow} Same & \cellcolor{red} Yes & \cellcolor{red} Yes \tabularnewline
YELSNEMA & 2012 & \cellcolor{red} Worse & \cellcolor{red} Yes & \cellcolor{red} Yes \tabularnewline
FLWGB & 2014 & \cellcolor{red} Worse & \cellcolor{red} Yes & \cellcolor{red} Yes \tabularnewline
FLWSNEMA & 2011 & \cellcolor{yellow} Same & \cellcolor{green} No & \cellcolor{red} Yes \tabularnewline
REDUNIT & 2012 & \cellcolor{yellow} Same & \cellcolor{green} No & \cellcolor{green} No \tabularnewline
PLAUNIT & 2012 & \cellcolor{yellow} Same & \cellcolor{green} No & \cellcolor{green} No \tabularnewline
WITUNIT & 2012 & \cellcolor{yellow} Same & \cellcolor{red} Yes & \cellcolor{red} Yes \tabularnewline
HKWUNIT & 2013 & \cellcolor{yellow} Same & \cellcolor{green} No & \cellcolor{green} No \tabularnewline
POLUNIT & 2014 & \cellcolor{yellow} Same & \cellcolor{green} No & \cellcolor{green} No \tabularnewline
CATUNIT & 2012 & \cellcolor{yellow} Same & \cellcolor{green} No & \cellcolor{red} Yes \tabularnewline
HALUNIT & 2012 & \cellcolor{orange} More uncertain & \cellcolor{yellow} Unknown & \cellcolor{red} Yes \tabularnewline
FLDGMGB & 2012 & \cellcolor{green} Better & \cellcolor{green} No & \cellcolor{red} Yes \tabularnewline
FLDSNEMA & 2012 & \cellcolor{yellow} Same & \cellcolor{green} No & \cellcolor{green} No \tabularnewline
OPTUNIT & 2012 & \cellcolor{yellow} Same & \cellcolor{green} No & \cellcolor{red} Yes \tabularnewline
FLWGM & 2014 & \cellcolor{yellow} Same & \cellcolor{green} No & \cellcolor{yellow} Unknown \tabularnewline
YELGB & 2014 & \cellcolor{yellow} Same & \cellcolor{yellow} Unknown & \cellcolor{yellow} Unknown \tabularnewline

\hline
	\end{tabular}
\end{table}


%\cellcolor{blue} foo & \cellcolor{red}
\clearpage
\begin{sidewaystable}[ht]
%\begin{table}
\captionsetup{width=\textwidth}

\centering
%\caption{}
\caption{Comparison of biomass ($B$) and fishing mortality ($F$) rate Mohn's rho values ($\rho$) by stock between the previous assessment and the 2015 updates. The biomass and fishing mortality rate point estimates and $\rho$ adjusted values (Adj.) are provided for the 2015 operational assessments. The total number of stocks using $\rho$ adjusted values in the last assessment and the 2015 assessments ($\rho$ adj. vs. pt. est. for those stocks that did not use the $\rho$ adjustment), along with the type of $\rho$ adjustment used in the 2015 assessment (NAA$=$numbers at age, SSB$=$spawning stock biomass applied to all ages), are also provided. Only age-based and length-based stocks that could exhibit retrospective patterns are included in this table. \textit{Note: Because the Georges Bank cod assessment was rejected at the 2015 OA it has been excluded from this table.} }
\label{RhoAdjust_tab}
{\small
\begin{tabular}{
l@{\hspace{.1cm}}
c@{\hspace{.1cm}}
c@{\hspace{.2cm}}
c@{\hspace{.2cm}}
c@{\hspace{.2cm}}
c@{\hspace{.1cm}}
m{1mm} 
c@{\hspace{.2cm}}
c@{\hspace{.2cm}}
c@{\hspace{.2cm}}
c@{\hspace{.1cm}}
m{1mm} 
c@{\hspace{.2cm}} 
c@{\hspace{.2cm}}
c@{\hspace{.2cm}} 
}
\hline
 & & &&&&&&&&&& \\[1pt] %blank row
 & & \multicolumn{4}{c}{\textbf{Biomass}} && \multicolumn{4}{c}{\textbf{Fishing Mortality Rate}} && \multicolumn{3}{c}{\textbf{Used}} \\  
\cline{3-6} \cline{8-11} \cline{13-15}
Stock & Model &$\rho_{last}$ & $\rho_{2014}$ & $B_{2014}$ & Adj. && $\rho_{last}$ & $\rho_{2014}$ & 
$F_{2014}$ & Adj. && Last assess. & 2014 & Proj. adj.\\
  \hline
CODGM & ASAP(M=0.2) & 0.53 & 0.54 & 2225 & 1445 && -0.33 & -0.31 & 0.956 & 1.386 && pt. est. & pt. est. & none \\
CODGM & ASAP(M-ramp) & 0.17 & 0.2 & 2536 & 2113 && -0.05 & -0.08 & 0.932 & 1.013 && pt. est. & pt. est. & none \\
HADGM & ASAP & -0.15 & -0.04 & 10325 & 10755 && 0.3 & 0.03 & 0.257 & 0.25 && pt. est. & pt. est. & none \\
HADGB & VPA & 0.2 & 0.5 & 225080 & 150053 && -0.15 & -0.34 & 0.159 & 0.241 && pt. est. & $\rho$ adj. & SSB \\
YELCCGM & VPA & 0.68 & 0.98 & 1695 & 857 && -0.19 & -0.45 & 0.35 & 0.64 && $\rho$ adj. & $\rho$ adj. & NAA \\
YELSNEMA & ASAP & 0.14 & 1.06 & 502 & 243 && -0.16 & -0.53 & 1.64 & 3.53 && pt. est. & pt. est. & none \\
FLWGB & VPA & 0.26 & 0.83 & 5275 & 2883 && -0.16 & -0.51 & 0.379 & 0.778 && pt. est. & $\rho$ adj. & SSB \\
FLWSNEMA & ASAP & 0.35 & 0.21 & 6151 & 5105 && -0.31 & -0.25 & 0.16 & 0.214 && pt. est. & pt. est. & none \\
REDUNIT & ASAP & 0.04 & 0.26 & 414544 & 330004 && -0.04 & -0.19 & 0.012 & 0.015 && pt. est. & $\rho$ adj. & NAA \\
PLAUNIT & VPA & 0.62 & 0.32 & 14439 & 10915 && -0.35 & -0.32 & 0.08 & 0.12 && $\rho$ adj. & $\rho$ adj. & NAA \\
WITUNIT & VPA & 0.61 & 0.51 & 3129 & 2077 && -0.33 & -0.38 & 0.428 & 0.687 && pt. est. & $\rho$ adj. & SSB \\
HKWUNIT & ASAP & 0.15 & 0.18 & 28553 & 24197 && -0.13 & -0.12 & 0.076 & 0.086 && pt. est. & pt. est. & none \\
POLUNIT & ASAP & 0.29 & 0.28 & 198847 & 154865 && -0.25 & -0.28 & 0.051 & 0.07 && pt. est. & $\rho$ adj. & NAA \\
CATUNIT & SCALE & 0.96 & 0.83 & 592 & 324 && -0.55 & -0.36 & 0.003 & 0.005 & pt. est. && pt. est. & none \\
   \hline
\end{tabular}
}

\end{sidewaystable}
%\end{table}

\clearpage
% latex table generated in R 3.2.1 by xtable 1.7-4 package
% Fri Nov  6 14:46:57 2015
\begin{table*}[ht]
\centering
\caption{The biomass ($B$) and exploitation rate ($F$) values used for status determination were  adjusted to account for a retrospective pattern in some stocks.   In general, when the $B$ or $F$ values adjusted for restrospective pattern ($B_{\rho}$ and $F_{\rho}$)  were outside of the approximate $90\%$ confidence interval (Conf. limits), the  $\rho$ adjusted values were used to determine stock status (Adj. $=$ Yes).  There were exceptions however, such as YELSNEMA and CODGM(M=0.2) and details regarding each decision can be found in the  report and reviewer comments sections for each stock.  Only stocks that had both an estimable 7-year Mohn's $\rho$ for $B$ and $F$ and estimable approximate  90\% confidence limits on  terminal year $B$ and $F$ values are included. } 
\label{RhoDecision_tab}
\begin{tabular}{c@{\hspace{.2cm}}c@{\hspace{.2cm}}c@{\hspace{.2cm}}c@{\hspace{.2cm}}c@{\hspace{.2cm}}c@{\hspace{.2cm}}c@{\hspace{.2cm}}c@{\hspace{.2cm}}}
  \hline
Stock & $B_{2014}$ & $B_{\rho}$ & Conf. limits & $F_{2014}$ & $F_{\rho}$ & Conf. limits & Adj? \\ 
  \hline
CODGM(M=0.2) & 2,225 & 1,443 & 1,942 - 2,892 & 0.956 &  1.39 & 0.654 - 1.387 & No \\ 
  CODGM(M ramp) & 2,536 & 2,106 & 1,921 - 3,298 & 0.932 &  1.01 & 0.662 - 1.304 & No \\ 
  HADGB & 225,080 & 150,053 & 171,911 - 301,282 & 0.159 & 0.241 & 0.13 - 0.203 & Yes \\ 
  HADGM & 10,325 & 10,712 & 7,229 - 14,453 & 0.257 &  0.25 & 0.164 - 0.373 & No \\ 
  YELSNEMA & 502 & 243 & 355 - 739 &  1.64 &  3.53 & 1.053 - 2.348 & No \\ 
  YELCCGM & 1,695 & 857 & 1,375 - 2,111 & 0.355 &  0.64 & 0.25 - 0.52 & Yes \\ 
  FLWSNEMA & 6,151 & 5,105 & 5,045 - 7,500 &  0.16 &  0.21 & 0.12 - 0.213 & No \\ 
  FLWGB & 5,275 & 2,883 & 3,783 - 6,767 & 0.379 & 0.778 & 0.254 - 0.504 & Yes \\ 
  PLAUNIT & 14,543 & 10,977 & 12,742 - 16,439 &  0.08 & 0.116 & 0.069 - 0.093 & Yes \\ 
  WITUNIT & 3,129 & 2,077 & 2,643 - 3,864 & 0.428 & 0.687 & 0.321 - 0.603 & Yes \\ 
  HWKUNIT & 28,553 & 24,197 & 24,351 - 33,480 & 0.076 & 0.086 & 0.063 - 0.092 & No \\ 
  POLUNIT & 198,847 & 154,919 & 37,243 - 255,097 & 0.051 &  0.07 & 0.084 - 0.066 & Yes \\ 
  REDUNIT & 414,544 & 330,004 & 368,906 - 465,828 & 0.012 & 0.015 & 0.011 - 0.014 & Yes \\ 
   \hline
\end{tabular}
\end{table*}





% Figures
\clearpage
%\IfFileExists{\ExSumPath/figures/stock_status.png}{
	\begin{sidewaysfigure}
		\centering	
		\adjustimage{max size={.95\textwidth}{.8\textheight}}{\ExSumPath/figures/StockStatus.png}
		%\adjustimage{max size={.95\textwidth}{.8\textheight}}{../../ExSum/figures/StockStatus.png}		
		\captionsetup{width=\textwidth}
		\caption[.]{Status of the Northeast Multispecies Fishery Management Plan (groundfish) stocks in 2007 (GARM III)  and 2014 (OA 2015) with respect to the $F_{MSY}$ and $B_{MSY}$ proxies. The 'Intermediate assessment' represents the last stock assessment conducted prior to the OA 2015 assessment (year varies by stock). Stocks on which overfishing is occurring are those where the $\frac{F_{terminal}}{F_{MSY proxy}}$ ratio is greater than 1 and overfished stocks are those where the $\frac{B_{terminal}}{B_{MSY{}proxy}}$ ratio is less than 0.5. \textit{Notes: (1) the GARM III assessments did not include wolfish; (2) for the intermediate assessments stock status could not be determined for Gulf of Maine winter flounder (OA 2014) or Georges Bank yellowtail (TRAC 2015); and, (3) based on the OA 2015 assessments stock status could not be determined for Atlantic halibut, Gulf of Maine winter flounder and Georges Bank yellowtail flounder. In the OA 2015 assessment, the stock status for Georges Bank cod remained overfished and overfishing is occurring; however, since the assessment was rejected, ratios of terminal conditions to reference points cannot be determined. Species codes: COD-Atlantic cod, HAD-haddock, POL-pollock, RED-redfish, WHK-white hake, OPT-ocean pout, CAT-wolffish, PLA-American plaice, FLW-winter flounder, YEL-yellowtail flounder, WIT-witch flounder, FLD-windowpane flounder, HAL-Atlantic halibut.}}		
		\label{stock_status}
	\end{sidewaysfigure}
	\clearpage
%}{}  %otherwise do nothing




%\IfFileExists{\ExSumPath/figures/stock_status.png}{
	\begin{figure}
		\centering	
		\adjustimage{max size={.95\textwidth}{.7\textheight}}{\ExSumPath/figures/propFmsy.png}
		%\adjustimage{max size={.95\textwidth}{.8\textheight}}{../../ExSum/figures/StockStatus.png}		
		\captionsetup{singlelinecheck=off}
		\caption[.]{Changes in the ratio of fishing mortality to FMSY proxy from 2007 (GARM III) to 2014 (OA 2015) for the twenty Northeast Multispecies Fishery Management Plan (groundfish) stocks. The results from the assessment prior to the OA 2015 assessment are shown for each stock to provide an 'Intermediate' value. Stocks on which overfishing is occurring are those where the $\frac{F_{terminal}}{F_{MSY{}proxy}}$ ratio is greater than 1. \textit{Notes: (1) the GARM III assessments did not include wolfish; (2) stock status in the 'Intermediate' assessment could not be determined for Gulf of Maine winter flounder or Georges Bank yellowtail flounder; and, (3) based on the OA 2015 assessments stock status could not be determined for Atlantic halibut, Gulf of Maine winter flounder and Georges Bank yellowtail flounder. In the OA 2015 assessment, the stock status for Georges Bank cod remained overfished and overfishing is occurring; however, since the assessment was rejected, ratios of terminal conditions to reference points cannot be determined.}}		
		\label{propFmsy}
	\end{figure}
	\clearpage
%}{}  %otherwise do nothing



%\IfFileExists{\ExSumPath/figures/stock_status.png}{
	\begin{figure}
		\centering	
		\adjustimage{max size={.95\textwidth}{.7\textheight}}{\ExSumPath/figures/propBmsy.png}
		%\adjustimage{max size={.95\textwidth}{.8\textheight}}{../../ExSum/figures/StockStatus.png}		
		\captionsetup{singlelinecheck=off}
		\caption[.]{Changes in the ratio of stock biomass to BMSY proxy from 2007 (GARM III) to 2014 (OA 2015) for the twenty Northeast Multispecies Fishery Management Plan (groundfish) stocks. The results from the assessment prior to the OA 2015 assessment are shown for each stock to provide an 'Intermediate' value. Stocks that are overfished stocks are those where the $\frac{B_{terminal}}{B_{MSY{}proxy}}$ ratio is less than 0.5. \textit{Notes: (1) the GARM III assessments did not include wolfish; (2) stock status in the 'Intermediate' assessment could not be determined for Gulf of Maine winter flounder or Georges Bank yellowtail flounder; and, (3) based on the OA 2015 assessments stock status could not be determined for Atlantic halibut, Gulf of Maine winter flounder and Georges Bank yellowtail flounder. In the OA 2015 assessment, the stock status for Georges Bank cod remained overfished and overfishing is occurring; however, since the assessment was rejected, ratios of terminal conditions to reference points cannot be determined.}}		
		\label{propBmsy}
	\end{figure}
	\clearpage
%}{}  %otherwise do nothing


%\IfFileExists{\ExSumPath/figures/stock_status.png}{
	\begin{figure}
		\centering	
		\adjustimage{max size={.95\textwidth}{.8\textheight}}{\ExSumPath/figures/nefscSpringResiduals.png}
		%\adjustimage{max size={.95\textwidth}{.8\textheight}}{../../ExSum/figures/StockStatus.png}		
		\captionsetup{singlelinecheck=off}
		\caption[.]{NEFSC spring bottom trawl survey index standardized anomalies (Z-score) for the Northeast Multispecies Fishery Management Plan (groundfish) stocks from 1968 to 2015. \textit{Note that both the Georges Bank$/$Gulf of Maine and Southern New England$/$Mid-Atlantic windowpane flounder stocks are not included since the spring survey is uninformative as an index of abundance and not used in the stock assessment.}	}	
		\label{nefscSpringResiduals}
	\end{figure}
	\clearpage
%}{}  %otherwise do nothing



%\IfFileExists{\ExSumPath/figures/stock_status.png}{
	\begin{figure}
		\centering	
		\adjustimage{max size={.95\textwidth}{.8\textheight}}{\ExSumPath/figures/nefscFallResiduals.png}
		%\adjustimage{max size={.95\textwidth}{.8\textheight}}{../../ExSum/figures/StockStatus.png}		
		\captionsetup{singlelinecheck=off}
		\caption[.]{NEFSC fall bottom trawl survey index standardized anomalies (Z-score) for the Northeast Multispecies Fishery Management Plan (groundfish) stocks from 1963 to 2014. \textit{Note that ocean pout is not included since the fall survey is uninformative as an index of abundance and not used in the stock assessment.}	}	
		\label{nefscFallResiduals}
	\end{figure}
	\clearpage
%}{}  %otherwise do nothing




%\IfFileExists{\ExSumPath/figures/stock_status.png}{
	\begin{figure}
		\centering	
		\adjustimage{max size={.95\textwidth}{.8\textheight}}{\ExSumPath/figures/nefscSpringMinSweptAreaBiomass.png}
		%\adjustimage{max size={.95\textwidth}{.8\textheight}}{../../ExSum/figures/StockStatus.png}		
		\captionsetup{singlelinecheck=off}
		\caption[.]{NEFSC spring bottom trawl survey minimum swept area biomass (mt) for the Northeast Multispecies Fishery Management Plan (groundfish) stocks from 1968 to 2015, by stock. Minimum swept area estimates assume a trawl swept area of 0.0112 $nm^{2}$) (0.0384 $km^{2}$) based on the wing spread of the trawl net. \textit{Note that both the Georges Bank$/$Gulf of Maine and Southern New England$/$Mid-Atlantic windowpane flounder stocks are not included since the spring survey is uninformative as an index of abundance and not used in the stock assessment.}	}	
		\label{nefscSpringMinSweptAreaBiomass}
	\end{figure}
	\clearpage
%}{}  %otherwise do nothing


%\IfFileExists{\ExSumPath/figures/stock_status.png}{
	\begin{figure}
		\centering	
		\adjustimage{max size={.95\textwidth}{.8\textheight}}{\ExSumPath/figures/nefscFallMinSweptAreaBiomass.png}
		%\adjustimage{max size={.95\textwidth}{.8\textheight}}{../../ExSum/figures/StockStatus.png}		
		\captionsetup{singlelinecheck=off}
		\caption[.]{NEFSC fall bottom trawl survey minimum swept area biomass (mt) for for the Northeast Multispecies Fishery Management Plan (groundfish) stocks from 1963 to 2014, by stock. Minimum swept area estimates assume a trawl swept area of 0.0112 $nm^{2}$ (0.0384 $km^{2}$) based on the wing spread of the trawl net. \textit{Note that ocean pout is not included since the fall survey is uninformative as an index of abundance and not used in the stock assessment.}	}
		\label{nefscFallMinSweptAreaBiomass}
	\end{figure}
	\clearpage
%}{}  %otherwise do nothing


%\IfFileExists{\ExSumPath/figures/stock_status.png}{
	\begin{figure}
		\centering	
		\adjustimage{max size={.95\textwidth}{.8\textheight}}{\ExSumPath/figures/propBiomassHaddockRedfish.png}
		%\adjustimage{max size={.95\textwidth}{.8\textheight}}{../../ExSum/figures/StockStatus.png}		
		\captionsetup{singlelinecheck=off}
		\caption[.]{Proportion of the total groundfish swept minimum swept area biomass contributed by Georges Bank and Gulf of Maine haddock and Redfish based on the NEFSC spring and fall bottom trawl surveys.}
		\label{propBiomassHaddockRedfish}
	\end{figure}
	\clearpage
%}{}  %otherwise do nothing


\clearpage

%%%%%%%%%%%%%%%%%%%%%%%%%%%%%%%%%%%%%%%%%%%%%%%%%%%%%%%%%%%%%%%%%%%%%%%%%%%%%%%%%%%%%%%%%%%%%%%%%%%%%%%%%%%%
\def\YELCCGMMyPathTab{/home/dhennen/EIEIO/BigReport/YEL_CCGM/tables} \def\YELCCGMMyPathFig{/home/dhennen/EIEIO/BigReport/YEL_CCGM/figures} \def\YELCCGMfigFishCap{Total catch of Cape Cod\-Gulf of Maine Yellowtail flounder between 1985 and 2014 by disposition \(landings and discards\).} \def\YELCCGMfigSSBCap{Trends in spawning stock biomass of Cape Cod\-Gulf of Maine Yellowtail flounder between 1985 and 2014 from the current  \(solid line\)  and previous \(dashed line\)  assessment and the corresponding  \$SSB\_{Threshold}\${} \(\$\dfrac{1}{2}\${} \$SSB\_{MSY}\${} \textit{proxy}{}\; horizontal dashed line\)  as well as  \$SSB\_{Target}\${} \(\$SSB\_{MSY}\${} \textit{proxy}{}\; horizontal dotted line\)   based on the 2015 assessment.  Biomass was adjusted for a retrospective pattern  and the adjustment is shown in red.   The 90\% bootstrap probability intervals are shown.} \def\YELCCGMfigFCap{Trends in the fully selected fishing mortality \(\$F\_{Full}\${}\)  of Cape Cod\-Gulf of Maine Yellowtail flounder between 1985 and 2014 from the current  \(solid line\)  and previous \(dashed line\)  assessment and the corresponding  \$F\_{Threshold}\${} \(\$F\_{MSY}\${} \textit{proxy}{}=0.279\; horizontal dashed line\).  \$F\_{Full}\${} was adjusted for a retrospective pattern  and the adjustment is shown in red  based on the 2015 assessment.  The 90\% bootstrap probability intervals are shown.} \def\YELCCGMfigRecrCap{Trends in Recruits \(age 1\)  \(000s\)  of Cape Cod\-Gulf of Maine Yellowtail flounder between 1985 and 2014 from the current \(solid line\)  and previous \(dashed line\)  assessment.  The 90\% bootstrap probability intervals are shown.} \def\YELCCGMfigSurvCap{Indices of biomass for the Cape Cod\-Gulf of Maine Yellowtail flounder between 1985 and 2015 for the Northeast Fisheries Science Center \(NEFSC\)  spring and fall bottom trawl surveys,  Massachusetts Department of Marine Fisheries \(MADMF\)  inshore state spring and fall bottom trawl surveys,and the Maine New Hampshire inshore state spring and fall state surveys  The 90\% bootstrap probability intervals are shown.} \def\YELCCGMPreAmb{This assessment of the Cape Cod\-Gulf of Maine Yellowtail flounder \(\textit{Limanda ferruginea}\)  stock is an operational update of the existing 2012 VPA assessment \(Legault et al., 2012\). The last benchmark for this stock was in 2008 \(Legault et al., 2008\). Based on the previous assessment the stock was overfished, and overfishing was ocurring. This assessment updates commercial fishery catch data, research survey indices of abundance, weights at age, and the analytical VPA assessment model and reference points through 2014. Additionally, stock projections have been updated through 2018} \def\YELCCGMSoS{ \textbf{State of Stock: }{}Based on this updated assessment, Cape Cod\-Gulf of Maine Yellowtail flounder \(\textit{Limanda ferruginea}\)  stock is overfished and overfishing is occurring \(Figures \ref{YELCCGMSSB\_plot1}\-\ref{YELCCGMF\_plot1}\){}.  Retrospective adjustments were made to the model results.  Spawning stock biomass \(SSB\)  in 2014 was estimated to be 857 \(mt\)  which is 16\% of the biomass target \(\$SSB\_{MSY}\${} \textit{proxy}{} = 5,259\;  Figure \ref{YELCCGMSSB\_plot1}{}\).  The 2014 fully selected fishing mortality was estimated to be 0.64 which is 229\% of the overfishing threshold proxy \(\$F\_{MSY}\${} \textit{proxy}{} = 0.279\;  Figure \ref{YELCCGMF\_plot1}{}\).} \def\YELCCGMProj{ \textbf{Projections: }{}Short term projections of biomass were derived by sampling from a cumulative  distribution function of recruitment estimates from ADAPT VPA. Recruitment estimates were hindcasted based on a simple linear regression between the NEFSC Fall survey abundance at age 1 and the VPA estimate at age 1.  The most recent two years \(2013 and 2014\)  were not included in the series of values due to high uncertainty in these estimates. This resulted in a total of 36 recruitment values: 8 from the hindcast predictions \(years 1977\-1984\)  and 28 from the VPA \(years 1985\-2012\). The annual fishery selectivity, maturity ogive, and mean weights at age used  in projection  are the most recent 5 year averages\;  retrospective adjustments were applied in the projections.} \def\YELCCGMSpecCmt{ \textbf{Special Comments: } \begin{itemize}{} \item{}What are the most important sources of uncertainty in this stock assessment?  Explain, and describe qualitatively how they affect the assessment results \(such as estimates of biomass, F, recruitment, and population projections\).  \linebreak{} \hspace\*{0.5cm} \textit{The largest source of uncertainty is the source of the retrospective pattern.This pattern has persisted for a number of years causing SSB estimates to decrease and F estimates to increaseas more years of data are added.}  \item{} Does this assessment model have a retrospective pattern? If so, is the pattern minor, or major? \(A major retrospective pattern occurs when the adjusted SSB or  \$F\_{Full}\${} lies outside of the approximate  joint confidence region for SSB and  \$F\_{Full}\${}\; see RhoDecisionTab.ref\). \linebreak{} \hspace\*{0.5cm} \textit{ The 7\-year Mohn\'s  \textrho{}, relative to SSB, was 0.68 in the 2012 assessment and was 0.98 in 2014. The 7\-year Mohn\'s  \textrho{}, relative to F, was \-0.19 in the 2012 assessment and was \-0.45 in 2014. There was a major retrospective pattern for this assessment because the  \textrho{} adjusted estimates of 2014 SSB \(\$SSB\_{\rho}\${}=857\)  and 2014 F \(\$F\_{\rho}\${}=0.64\)  were outside the approximate 90\% confidence regions around SSB \(1,375 \- 2,111\)  and F \(0.25 \- 0.52\).  A retrospective  adjustment was made for both the determination of stock status and for projections of catch in 2016. The retrospective adjustment changed the 2014 SSB from 1,695 to 857 and the 2014  \$F\_{Full}\${} from 0.355 to 0.64.}  \item{}Based on this stock assessment, are population projections well determined or uncertain? \linebreak{} \hspace\*{0.5cm} \textit{Population projections for Cape Cod\-Gulf of Maine Yellowtail flounder, are uncertain with projected biomass from the last assessmentabove the confidence bounds of the biomass estimated in the current assessment.}  \item{}Describe any changes that were made to the current stock assessment, beyond incorporating additional years of data  and the affect these changes had on the assessment and stock status. \linebreak{} \hspace\*{0.5cm} \textit{ No changes, other than the incorporation of new data were made to the Cape Cod\-Gulf of Maine Yellowtail flounder assessment for this update.}  \item{}If the stock status has changed a lot since the previous assessment, explain why this occurred.  \linebreak{} \hspace\*{0.5cm} \textit{The stock status has not changed since the previous assessment.}  \item{}Indicate what data or studies are currently lacking and which would be needed most to improve this stock assessment in the future.  \linebreak{} \hspace\*{0.5cm} \textit{Extensive studies have examined the causes of the retrospective patterns with no definitive conclusions other than a change in model does not resolve the issue.}  \item{}Are there other important issues? \linebreak{} \hspace\*{0.5cm} \textit{No. } \end{itemize}{}} \def\YELCCGMRefr{ \textbf{References: }{} \linebreak{}Legault, C,  L. Alade, S.Cadrin, J. King, and S. Sherman.  2008.  In.  Northeast Fisheries Science Center. 2008. Assessment of 19 Northeast Groundfish Stocks through 2007: Report of the 3$^{rd}$ Groundfish Assessment Review Meeting \(GARM III\), Northeast Fisheries Science Center, Woods Hole, Massachusetts, August 4\-8, 2008. US Dep Commer, NOAA Fisheries, Northeast Fish Sci Cent Ref Doc. 08\-15\; 884 p + xvii. http:\/\/www.nefsc.noaa.gov\/publications\/crd\/crd0815\/ \linebreak{} \linebreak{} Legault, C,  L. Alade, S.Emery, J. King, and S. Sherman.  2012.  In.  Northeast Fisheries Science Center. 2012. Assessment or Data Updates of 13 Northeast Groundfish Stocks through 2010. US Dept Commer, NOAA Fisheries, Northeast Fish Sci Cent Ref Doc. 12\-06.\; 789 p. http:\/\/nefsc.noaa.gov\/publications\/crd\/crd1206\/ \linebreak{} \linebreak{}} \def\YELCCGMDraft{} \def\YELCCGMSPPname{Cape Cod-Gulf of Maine Yellowtail flounder} \def\YELCCGMSPPnameT{Cape Cod-Gulf of Maine Yellowtail flounder} \def\YELCCGMRptYr{2015} \def\YELCCGMAuthor{Larry Alade} 
%cd /home/dhennen/EIEIO/BigReport/latex; ls -l; pdflatex -halt-on-error   \def\HALUNITMyPathTab{/home/dhennen/EIEIO/BigReport/HAL_UNIT/tables} \def\HALUNITMyPathFig{/home/dhennen/EIEIO/BigReport/HAL_UNIT/figures} \def\HALUNITfigFishCap{Total catch of Atlantic halibut between 1963 and 2014 by disposition \(landings and discards\).} \def\HALUNITfigSSBCap{Estimated trends in the biomass of Atlantic halibut between 1963 and 2014 from the current  \(solid line\)  and previous \(dashed line\)  assessment and the corresponding  \$B\_{Threshold}\${}= \$\dfrac{1}{2}\${} \$B\_{MSY}\${} \textit{proxy}{}\(horizontal dashed line\)  as well as  \$B\_{Target}\${} \(\$B\_{MSY}\${} \textit{proxy}{}\; horizontal dotted line\)   based on the 2015 assessment.} \def\HALUNITfigFCap{Estimated trends in the fully selected fishing mortality \(\$F\_{Full}\${}\)  of Atlantic halibut between 1963 and 2014 from the current  \(solid line\)  and previous \(dashed line\)  assessment and the corresponding  \$F\_{Threshold}\${} \(0.073\; horizontal dashed line\)  as well as  \$F\_{Target}\${} \(0.8 \* \$F\_{MSY}\${} \textit{proxy}{}\; dotted line\)   based on the 2015 assessment. } \def\HALUNITfigRecrCap{} \def\HALUNITfigSurvCap{Indices of biomass for the Atlantic halibut between 1963 and 2014 for the Northeast Fisheries Science Center \(NEFSC\)  fall bottom trawl survey.  The 90\\percent lognormal confidence intervals are shown.} \def\HALUNITPreAmb{This assessment of the Atlantic halibut \(\textit{Hippoglossus hippoglossus}\)  stock is an update of the existing 2012 benchmark assessment \(NEFSC 2010\)  and the last update assessment \(NEFSC 2012\). This assessment updates commercial fishery catch data, research survey indices of abundance, and the replacement yield assessment model through 2014. Additionally, stock projections have been updated through 2018. Reference points have not been updated. } \def\HALUNITSoS{ \textbf{State of Stock: }{}Based on this updated assessment, Atlantic halibut \(\textit{Hippoglossus hippoglossus}\)  stock is unknown and unknown \(Figures \ref{HALUNITSSB\_plot1}\-\ref{HALUNITF\_plot1}\){}. Retrospective adjustments were not made to the model results.  Biomass \(SSB\)  in 2014 was estimated to be 96,464 \(mt\)  which is 199\\percent of the biomass target \(\$SSB\_{MSY}\${} \textit{proxy}{} = 48,509\;  Figure \ref{HALUNITSSB\_plot1}{}\).  The 2014 fully selected fishing mortality was estimated to be 0.001 which is 1\\percent of the overfishing threshold proxy \(\$F\_{MSY}\${} \textit{proxy}{} = 0.073\;  Figure \ref{HALUNITF\_plot1}{}\).} \def\HALUNITProj{ \textbf{Projections: }{} Short term projections were based on a constant F =  \$F\_{MSY}\${} \textit{proxy}{} = 0.073.  Projections use the assessment model \(replacement yield\)  and maintain all other model assumptions.} \def\HALUNITSpecCmt{ \textbf{Special Comments: } \begin{itemize}{} \item{}What are the most important sources of uncertainty in this stock assessment?  Explain, and describe qualitatively how they affect the assessment results \(such as estimates of biomass, F, recruitment, and population projections\).  \linebreak{} \hspace\*{0.5cm} \textit{The assessment model used for Atlantic halibut is highly uncertain.  It estimates one parameter, the initial biomass, and  proceeds deterministically from 1800 to 2014.  The model is highly sensitive to the initial biomass.  The model is  tuned to the survey index, which is inefficient for Atlantic halibut, catches very few animals and is therefore noisy.   The RYM model assumes no immigration or emmigration and that the population both began, and tends to, equilibrium.   These assumptions are unlikely to be true for Atlantic halibut. The model estimates a biomass that is approximately equal  to unfished biomass, which is not credible. Catch has been very low for at least 100 years relative  to the landings reported early in the time series, despite a strong market and high value  relative to other groundfish.  The low catch throughout the century implies that the Atlantic halibut stock is very likely  depleted relative to it\'s unfished condition and is therefore likely to be overfished, even if its current biomass is  unknown.}  \item{} Does this assessment model have a retrospective pattern? If so, is the pattern minor, or major? \(A major retrospective pattern occurs when the adjusted SSB or  \$F\_{Full}\${} lies outside of the approximate  joint confidence region for SSB and  \$F\_{Full}\${}\; see  Figure \ref{RhoDecision\_tab}{}\). \linebreak{} \hspace\*{0.5cm} \textit{ The model used to determine the status of this stock does not allow estimation of a retrospective pattern. }  \item{}Based on this stock assessment, are population projections well determined or uncertain? \linebreak{} \hspace\*{0.5cm} \textit{Population projections for Atlantic halibut are uncertain because biomass cannot be reasonably determined using  the current assessment model.}  \item{}Describe any changes that were made to the current stock assessment, beyond incorporating additional years of data  and the affect these changes had on the assessment and stock status. \linebreak{} \hspace\*{0.5cm} \textit{ The catch data were slightly altered due to the exclusion of catch made in international waters and the  re\-estiamtion of average discard ratio after 1998 \(due to the incorporation of more years of data\).}  \item{}If the stock status has changed a lot since the previous assessment, explain why this occurred.  \linebreak{} \hspace\*{0.5cm} \textit{The overfishing and overfished status of Atlantic halibut cannot be determined using the current assessment.  This  occurred because diagnostics showed the model was unreliable.  }  \item{}Indicate what data or studies are currently lacking and which would be needed most to improve this stock assessment in the future.  \linebreak{} \hspace\*{0.5cm} \textit{The Atlantic halibut assessment could be improved with additional studies on stock structure, additional age and length data,  a more precise and accurrate survey, and an investigation of alternate assessment models.}  \item{}Are there other important issues? \linebreak{} \hspace\*{0.5cm} \textit{Atlantic halibut are clearly depleted relative to their unfished state.  Catches have been far below historical landings  for more than 100 years, despite a lack of regulation before 1999 and a strong commercial market.  The current  assessment model implies that Atlantic halibut is near or above its unfished biomass and could support removals  commensurate with MSY.  The current assessment should probably not be used to inform management decisions.} \end{itemize}{}} \def\HALUNITRefr{ \textbf{References: }{} \linebreak{} Northeast Fisheries Science Center. 2012. Assessment or Data Updates of 13 Northeast Groundfish Stocks  through 2010. US Dept Commer, Northeast Fish Sci Cent Ref Doc. 12\-06\; 789 p. Available from: National  Marine Fisheries Service, 166 Water Street, Woods Hole, MA 02543\-1026, or online at  http:\/\/nefsc.noaa.gov\/publications\/ \linebreak{} \linebreak{}Col, L.A., Legault, C.M. 2009. The 2008 Assessment of Atlantic halibut in the Gulf of Maine Georges Bank region.  US Dept Commer, Northeast Fish Sci Cent Ref Doc. 09\-08\; 39 p. Available from: National Marine Fisheries Service, 166 Water Street, Woods Hole, MA 02543\-1026, or online at http:\/\/www.nefsc.noaa.gov\/nefsc\/publications\/ } \def\HALUNITDraft{} \def\HALUNITSPPname{Atlantic halibut} \def\HALUNITSPPnameT{Atlantic halibut} \def\HALUNITRptYr{2015} \def\HALUNITAuthor{Daniel Hennen} \def\HALUNITReviewerComments{/home/dhennen/EIEIO/BigReport/HAL_UNIT/latex}  \def\CODGMMyPathTab{/home/dhennen/EIEIO/BigReport/COD_GM/tables} \def\CODGMMyPathFig{/home/dhennen/EIEIO/BigReport/COD_GM/figures} \def\CODGMfigFishCap{Total catch of Gulf of Maine Atlantic cod between 1982 and 2014 by fleet \(commercial and recreational\)  and disposition \(landings and discards\).} \def\CODGMfigSSBCap{Estimated trends in the spawning stock biomass \(SSB\)  of Gulf of Maine Atlantic cod between 1982 and 2014 from the current  \(solid line\)  and previous \(dashed line\)  assessment and the corresponding  \$SSB\_{Threshold}\${} \(\$\dfrac{1}{2}\${} \$SSB\_{MSY}\${}\; horizontal dashed line\)  as well as  \$SSB\_{Target}\${} \$SSB\_{MSY}\${}\; horizontal dotted line\)   based on the 2015 M=0.2 \(A\)  and M\-ramp \(B\)  assessment models. The 90\\percent lognormal confidence intervals are shown. The red dot indicates the rho\-adjusted SSB values that would have resulted had a retrospective adjusment been made to either model \(see Special Comments section\).} \def\CODGMfigFCap{Estimated trends in the fully selected fishing mortality \(F\)  of Gulf of Maine Atlantic cod between 1982 and 2014 from the current  \(solid line\)  and previous \(dashed line\)  assessment and the corresponding  \$F\_{Threshold}\${} \(0.185 \(M=0.2\), 0.187 \(M\-ramp\)\; dashed line\)  based on the 2015 M=0.2 \(A\)  and M\-ramp \(B\)  assessment models. The 90\\percent lognormal confidence intervals are shown. The red dot indicates the rho\-adjusted F values that would have resulted had a retrospective adjusment been made to either model \(see Special Comments section\).} \def\CODGMfigRecrCap{Estimated trends in age\-1 recruitment  \(000s\)  of Gulf of Maine Atlantic cod between 1982 and 2014 from the current \(solid line\)  and previous \(dashed line\)  M=0.2 \(A\)  and M\-ramp \(B\)  assessment models. The 90\\percent lognormal confidence intervals are shown.} \def\CODGMfigSurvCap{Indices of biomass for the Gulf of Maine Atlantic cod between 1963 and 2015 for the Northeast Fisheries Science Center \(NEFSC\)  spring and fall bottom trawl surveys and Massachusetts Division of Marine Fisheries \(MADMF\)  spring bottom trawl survey.  The 90\\percent lognormal confidence intervals are shown.} \def\CODGMPreAmb{This assessment of the Gulf of Maine Atlantic cod \(\textit{Gadus morhua}\)  stock is an update of the existing 2014 assessment \(Palmer 2014\). This assessment updates commercial and recreational fishery catch data, research survey indices of abundance, and the analytical ASAPassessment models through 2014. Additionally, stock projections have been updated through 2018. In what follows, there are two population assessment models brought forward from the most recent benchmark assessment \(2012\), the M=0.2 \(natural mortality = 0.2\)  and the M\-ramp \(M ramps from 0.2 to 0.4\)  assessment models \(see NEFSC 2013 for a full description of the model formulations\).} \def\CODGMSoS{ \textbf{State of Stock: }{}Based on this updated assessment, the Gulf of Maine Atlantic cod \(\textit{Gadus morhua}\)  stock is overfished and overfishing is occurring \(Figures \ref{CODGMSSB\_plot1}\-\ref{CODGMF\_plot1}\){}. Retrospective adjustments were not made to the model results \(see Special Comments section of this report\). Spawning stock biomass \(SSB\)  in 2014 was estimated to be 2,225 \(mt\)  under the M=0.2 model and 2,536 \(mt\)  under the M\-ramp model scenario \(Table \ref{CODGMCatch\_Status\_Table}{}\)  which is 6 and 4\\percent \(respectively\)  of the biomass target,  \$SSB\_{MSY}\${} \textit{proxy}{} \(40,187 \(mt\)  and 59,045 \(mt\)\;  Figure \ref{CODGMSSB\_plot1}{}\).  The 2014 fully selected fishing mortality was estimated to be 0.956 and 0.932 which is 517 and 498\\percent of the  \$F\_{MSY}\${} \textit{proxy}{}\(\$F\_{40\\percent}\${}\; 0.185 and 0.187\;  Figure \ref{CODGMF\_plot1}{}\).} \def\CODGMProj{ \textbf{Projections: }{} Short term projections of median total fishery yield and spawning stock biomass for Gulf of Maine Atlantic cod were conducted based on a harvest scenario of fishing at the FMSY proxy between 2016 and 2018. Catch in 2015 was estimated at 279 mt. Recruitment was sampled from a cumulative distribution function derived from ASAP estimated age\-1 recruitment between 1982 and 2012.  The projection recruitment model declines linearly to zero when SSB is below 6.3 kmt under the M=0.2 model and 7.9 kmt under the M\-ramp model. The 2015 age\-1 recruitment was estimated from the geometric mean of the 2010\-2014 ASAP recruitment estimates. No retrospective adjustments were applied in the projections as the retrospective patterns are similar to the 2014 update for which no retrospective adjustments were made\; however, the 2015 assessment review panel recommended that that M=0.2 projections with retrospective adjustments be brought forward to the SSC for consideration in the evaluation of uncertainty when setting catch advice \(provided in the Supplemental Information Report, \href{http:\/\/www.nefsc.noaa.gov\/saw\/sasi\/sasi\_report\_options.php}{SASINF}{}\). Assumed weights are based on an average of the most recent three years. For the M\-ramp model, projections are shown under two assumptions of short\-term natural mortality: M=0.2 and M=0.4.} \def\CODGMSpecCmt{ \textbf{Special Comments: } \begin{itemize}{} \item{}What are the most important sources of uncertainty in this stock assessment?  Explain, and describe qualitatively how they affect the assessment results \(such as estimates of biomass, F, recruitment, and population projections\).  \linebreak{} \hspace\*{0.5cm} \textit{The largest source of uncertainty is the estimate of natural mortality. Past investigations into changes in natural mortality over time have been inconclusive \(NEFSC 2013\). Different assumptions about natural mortality affect the scale of the biomass, recruitment, and fishing mortality estimates. Other areas of uncertainty include the retrospective error in the M=0.2 model, residual patterns in the model fits to some of the survey series \(e.g., aggregate MADMF spring survey\)  and stock structure.}  \item{}Does this assessment model have a retrospective pattern? If so, is the pattern minor, or major? \(A major retrospective pattern occurs when the adjusted SSB or  \$F\_{Full}\${} lie outside of the approximate joint confidence region for SSB and  \$F\_{Full}\${}\). \linebreak{} \hspace\*{0.5cm} \textit{The M=0.2 model has a major retrospective pattern \(7\-year Mohn\'s rho SSB=0.54, F=\-0.31\)  and the M\-ramp model has a minor retrospective pattern \(7\-year Mohn\'s rho SSB=0.20, F=\-0.08\). The 7\-year Mohn\'s rho values from the current assessment are similar to those from the 2014 assessment \(M=0.2: SSB=0.53, F=\-0.33\; M\-ramp: SSB=0.17, F=\-0.05\)  where the M=0.2 model had a major retrospective pattern and the M\-ramp model had a minor pattern. No retrospective adjustment have been to the terminal model results or in the base catch projections following the recommendations of the SARC 55 and 2014 assessment review panels. The 2015 assessment review panel supported this decision noting that the most recent retrospective \'peel\' suggested that an adjustment using the 7\-year average may not be appropriate. However, the 2015 review panel highlighted the retrospective error in the M=0.2 model as a source of uncertainty \- it should be noted that the retrospective error of the most recent peel is larger for the M\-ramp model. Should the retrospective patterns continue then the models may have overestimated spawning stock size and underestimated fishing mortality.}  \item{}Based on this stock assessment, are population projections well determined or uncertain? \linebreak{} \hspace\*{0.5cm} \textit{Population projections for Gulf of Maine Atlantic cod are reasonably well determined and projected boimass from the last assessment  was within the confidence bounds of the biomass estimated in the current assessment. }  \item{}Describe any changes that were made to the current stock assessment, beyond incorporating additional years of data  and the affect these changes had on the assessment and stock status. \linebreak{} \hspace\*{0.5cm} \textit{ This update included several minor changes to model input data including: \(1\)  re\-estimation of recreational catch from 2004\-2014 to account for recent updates to the MRIP data\; \(2\)  a revised assumption on recreational discard mortality from 30\\percent to 15\\percent following a Capizzano et al. 2015 study \(unpublished\)\; and \(3\)  re\-estimation of 2009\-2014 NEFSC spring and fall survey time series using the TOGA station acceptance criterion. Additionally, the ASAP assessment model was run with the likelihood constants option turned off. All of these changes had minimal impacts on model results \- summaries of the impacts of these changes are provided in the Supplemental Information Report \(\href{http:\/\/www.nefsc.noaa.gov\/saw\/sasi\/sasi\_report\_options.php}{SASINF}{}\).}  \item{}If the stock status has changed a lot since the previous assessment, explain why this occurred.  \linebreak{} \hspace\*{0.5cm} \textit{There has been no change in stock status since the 2014 udpate assessment.}  \item{}Indicate what data or studies are currently lacking and which would be needed most to improve this stock assessment in the future.  \linebreak{} \hspace\*{0.5cm} \textit{The Gulf of Maine Atlantic cod assessment could be improved with additional studies on natural mortality and stock structure. Additionally, future assessments should consider possible changes in recent fishery selectivity patterns and exlore alternative methods for estimating recruitment. Potential causes of low stock productivity \(i.e., low recruitment\)  should also be investigated.}  \item{}Are there other important issues? \linebreak{} \hspace\*{0.5cm} \textit{ When setting catch advice careful attention should be given to the retrospective error present in both models, particularly given the poor performance of previous stock projections. Additionally, it is unclear as to which level of natural mortality \(M=0.2 or 0.4\)  to assume for the short\-term projections under the M\-ramp model.} \end{itemize}{}} \def\CODGMRefr{ \textbf{References: }{} \linebreak{}Northeast Fisheries Science Center. 2013. 55$^{th}$ Northeast Regional Stock Assessment Workshop \(55$^{th}$ SAW\)  Assessment Summary Report. US Dept Commer, Northeast Fish Sci Cent Ref Doc. 13\-01\; 41 p. Available from: National Marine Fisheries Service, 166 Water Street, Woods Hole, MA 02543\-1026 \linebreak{} \linebreak{}Palmer MC. 2014. 2014 Assessment update report of the Gulf of Maine Atlantic cod stock. US Dept Commer, Northeast Fish Sci Cent Ref Doc. 14\-14\; 119 p. Available from: National Marine Fisheries Service,166 Water Street, Woods Hole, MA 02543\-1026 } \def\CODGMDraft{} \def\CODGMSPPname{Gulf of Maine Atlantic cod} \def\CODGMSPPnameT{Gulf of Maine Atlantic cod} \def\CODGMRptYr{2015} \def\CODGMAuthor{Michael Palmer} \def\CODGMReviewerComments{/home/dhennen/EIEIO/BigReport/COD_GM/latex}  \def\CODGBMyPathTab{/home/dhennen/EIEIO/BigReport/COD_GB/tables} \def\CODGBMyPathFig{/home/dhennen/EIEIO/BigReport/COD_GB/figures} \def\CODGBfigFishCap{Total catch of Georges Bank Atlantic Cod between 1978 and 2014 by fleet \(US commercial, US recreational, or Canadian\)  and disposition \(landings and discards\).} \def\CODGBfigSSBCap{Trends in spawning stock biomass of Georges Bank Atlantic Cod between 1978 and 2014 from the current  \(solid line\)  and previous \(dashed line\)  assessment and the corresponding  \$SSB\_{Threshold}\${} \(\$\dfrac{1}{2}\${} \$SSB\_{MSY}\${} \textit{proxy}{}\; horizontal dashed line\)  as well as  \$SSB\_{Target}\${} \(\$SSB\_{MSY}\${} \textit{proxy}{}\; horizontal dotted line\)   based on the 2015 assessment.  Biomass was adjusted for a retrospective pattern  and the adjustment is shown in red.  The approximate 90\\percent lognormal confidence intervals are shown.} \def\CODGBfigFCap{Trends in the fully selected fishing mortality \(\$F\_{Full}\${}\)  of Georges Bank Atlantic Cod between 1978 and 2014 from the current  \(solid line\)  and previous \(dashed line\)  assessment and the corresponding  \$F\_{Threshold}\${} \(\$F\_{MSY}\${} \textit{proxy}{}=0.169\; horizontal dashed line\).  \$F\_{Full}\${} was adjusted for a retrospective pattern  and the adjustment is shown in red,  based on the 2015 assessment. The approximate 90\\percent lognormal confidence intervals are shown.} \def\CODGBfigRecrCap{Trends in Recruits \(age 1\)  \(000s\)  of Georges Bank Atlantic Cod between 1978 and 2014 from the current \(solid line\)  and previous \(dashed line\)  assessment. The approximate 90\\percent lognormal confidence intervals are shown.} \def\CODGBfigSurvCap{Indices of biomass for the Georges Bank Atlantic Cod between 1963 and 2015 for the Northeast Fisheries Science Center \(NEFSC\)  spring and fall, and the DFO research bottom trawl surveys.  The approximate 90\\percent lognormal confidence intervals are shown.} \def\CODGBPreAmb{This assessment of the Georges Bank Atlantic Cod \(\textit{Gadus morhua}\)  stock is an operational update of the existing 2012 benchmark assessment \(NEFSC 2013\). Based on the previous assessment the stock was overfished, and overfishing was ocurring. This 2015 assessment updates commercial fishery catch data, research survey indices of abundance, the analytical ASAP assessment model, and reference points through 2014. Additionally, stock projections have been updated through 2018.} \def\CODGBSoS{ \textbf{State of Stock: }{}Based on this updated assessment, the Georges Bank Atlantic Cod \(\textit{Gadus morhua}\)  stock is overfished and overfishing is occurring \(Figures \ref{CODGBSSB\_plot1}\-\ref{CODGBF\_plot1}\){}.  Retrospective adjustments were made to the model results.  Spawning stock biomass \(SSB\)  in 2014 was estimated to be 1,804 \(mt\)  which is 1\\percent of the biomass target for this stock \(\$SSB\_{MSY}\${} \textit{proxy}{} = 201,152\;  Figure \ref{CODGBSSB\_plot1}{}\).  The 2014 fully selected fishing mortality was estimated to be 1.68 which is 994\\percent of the overfishing threshold proxy \(\$F\_{MSY}\${} \textit{proxy}{} = 0.169\;  Figure \ref{CODGBF\_plot1}{}\).} \def\CODGBProj{ \textbf{Projections: }{}Short term projections of biomass were derived by sampling from a two\-stage cumulative  distribution  function of recruitment estimates from ASAP model results, using a 50,000 mt cutpoint. The annual fishery selectivity, maturity ogive, and mean weights at age used in projections are the most recent 5 year averages\;  retrospective adjustments were applied in the projections.} \def\CODGBSpecCmt{ \textbf{Special Comments: } \begin{itemize}{} \item{}What are the most important sources of uncertainty in this stock assessment?  Explain, and describe qualitatively how they affect the assessment results \(such as estimates of biomass, F, recruitment, and population projections\).  \linebreak{} \hspace\*{0.5cm} \textit{The major source of uncertainty is presumbaly the estimate of catch or of natural mortality, considering the magnitude of the retrospective bias. These both affect the scale of the biomass, fishing mortality estimates, and the reference point estimates. The catch estimates do not include all discards \(e.g.,lobster gear\)  and includes uncertain estimates of recreational landings and discards, and of some commercial discards \(e.g., small mesh\). Natural mortality \(M\)  of Georges Bank Atlantic Cod is not well understood and is assumed constant over time in the model. Other sources of uncertainty include possible changes in growth parameters in recent years and how this affects fecundity, the viability of eggs\/sperm, and the success rate of hatching \- all influencing recruitment survival and year class strength.}  \item{} Does this assessment model have a retrospective pattern? If so, is the pattern minor, or major? \(A major retrospective pattern occurs when the adjusted SSB or  \$F\_{Full}\${} lies outside of the approximate  joint confidence region for SSB and  \$F\_{Full}\${}\; see  Figure \ref{RhoDecision\_tab}{}\). \linebreak{} \hspace\*{0.5cm} \textit{ The 7\-year Mohn\'s  \textrho{}, relative to SSB, was 0.68 in the 2012 assessment and was 2.43 in 2014. The 7\-year Mohn\'s  \textrho{}, relative to F, was \-0.46 in the 2012 assessment and was \-0.72 in 2014. There was a major retrospective pattern for this assessment because the  \textrho{} adjusted estimates of 2014 SSB \(\$SSB\_{\rho}\${}=1,804\)  and 2014 F \(\$F\_{\rho}\${}=1.68\)  were outside the approximate 90\\percent confidence regions around SSB \(3,922 \- 10,596\)  and F \(0.251 \- 0.815\).  A retrospective  adjustment was made for both the determination of stock status and for projections of catch in 2016. The retrospective adjustment changed the 2014 SSB from 6,180 to 1,804 and the 2014  \$F\_{Full}\${} from 0.463 to 1.68.}  \item{}Based on this stock assessment, are population projections well determined or uncertain? \linebreak{} \hspace\*{0.5cm} \textit{Population projections for Georges Bank Atlantic Cod are uncertain and likely optimistic. The projections are based on a biomass cutpoint of 50,000 mt, which has not been produced since 1992. The average recruitment since 1992 has been 4.9 million age 1 fish, whereas during the last 10 years, average recruitment has been about 2.7 million age 1 fish. A sensistivity projection using the most recent 10 years of recruitment was conducted and results presented in the SASINF database. }  \item{}Describe any changes that were made to the current stock assessment, beyond incorporating additional years of data  and the effect these changes had on the assessment and stock status. \linebreak{} \hspace\*{0.5cm} \textit{ No major changes, other than the addition of recent years of data, were made to the Georges Bank Atlantic Cod assessment for this update. However, recreational catch and commercial discard estimates were revised slightly due to minor changes in the databases, and the application of length frequencies \(annual instead of half year\)  in one instance.}  \item{}If the stock status has changed a lot since the previous assessment, explain why this occurred.  \linebreak{} \hspace\*{0.5cm} \textit{As in recent assessments for Georges Bank Atlantic Cod the stock remains in an overfishing and overfished status.}  \item{}Indicate what data or studies are currently lacking and which would be needed most to improve this stock assessment in the future.  \linebreak{} \hspace\*{0.5cm} \textit{The Georges Bank Atlantic Cod assessment could be improved with additional studies on natural mortality, growth, and fecundity. Additionally, more precise estimates of recreational landings and discards, sampling of fish caught by individual recreational anglers, and incorporation of discards in the lobster fishery would decrease uncertainty in the discard esimates.}  \item{}Are there other important issues? \linebreak{} \hspace\*{0.5cm} \textit{The differences in model assumptions of natural mortality between the SARC GB cod and the TRAC eGB cod assessment is problematic for the recovery of the entire GB cod stock. Model results of the TRAC VPA M=0.8 model are used to determine quota for the eGB management unit, so by default, proportionally more cod are being removed from eastern GB than what the GB cod ASAP model would predict.} \end{itemize}{}} \def\CODGBRefr{ \textbf{References: }{} \linebreak{}Northeast Fisheries Science Center. 2013. 55$^{th}$ Northeast Regional Stock AssessmentWorkshop \(55$^{th}$ SAW\)  Assessment Summary Report. Northeast Fisheries Science CenterReference Document 13\-01:43. \linebreak{} \linebreak{}} \def\CODGBDraft{} \def\CODGBSPPname{Georges Bank Atlantic Cod} \def\CODGBSPPnameT{Georges Bank Atlantic Cod} \def\CODGBRptYr{2015} \def\CODGBAuthor{Loretta O\'Brien} \def\CODGBReviewerComments{/home/dhennen/EIEIO/BigReport/COD_GB/latex}  \def\HADGBMyPathTab{/home/dhennen/EIEIO/BigReport/HAD_GB/tables} \def\HADGBMyPathFig{/home/dhennen/EIEIO/BigReport/HAD_GB/figures} \def\HADGBfigFishCap{Total catch of Georges Bank haddock between 1931 and 2014 by fleet \(US Commercial, Canadian, or foreign fleet\)  and disposition \(landings and discards\).} \def\HADGBfigSSBCap{Trends in spawning stock biomass of Georges Bank haddock between 1931 and 2014 from the current  \(solid line\)  and previous \(dashed line\)  assessment and the corresponding  \$SSB\_{Threshold}\${} \(\$\dfrac{1}{2}\${} \$SSB\_{MSY}\${} \textit{proxy}{}\; horizontal dashed line\)  as well as  \$SSB\_{Target}\${} \(\$SSB\_{MSY}\${} \textit{proxy}{}\; horizontal dotted line\)   based on the 2015 assessment.  Biomass was adjusted for a retrospective pattern  and the adjustment is shown in red.   The 90\\percent bootstrap probability intervals are shown.} \def\HADGBfigFCap{Trends in the fully selected fishing mortality \(\$F\_{Full}\${}\)  of Georges Bank haddock between 1931 and 2014 from the current  \(solid line\)  and previous \(dashed line\)  assessment and the corresponding  \$F\_{Threshold}\${} \(\$F\_{MSY}\${} \textit{proxy}{}=0.39\; horizontal dashed line\)  based on the 2015 assessment.   \$F\_{Full}\${} was adjusted for a retrospective pattern  and the adjustment is shown in red.   The 90\\percent bootstrap probability intervals are shown.} \def\HADGBfigRecrCap{Trends in Recruits \(age 1\)  \(000s\)  of Georges Bank haddock between 1931 and 2014 from the current \(solid line\)  and previous \(dashed line\)  assessment.  The 90\\percent bootstrap probability intervals are shown.} \def\HADGBfigSurvCap{Indices of biomass \(Mean kg\/tow\)  for the Georges Bank haddock stock between 1963 and 2015 for the Northeast Fisheries Science Center \(NEFSC\)  spring and fall bottom trawl surveys and the DFO winter bottom trawl survey.  The approximate 90\\percent lognormal confidence intervals are shown.} \def\HADGBPreAmb{This assessment of the Georges Bank haddock \(\textit{Melanogrammus aeglefinus}\)  stock is an operational update of the existing 2012 update VPA assessment \(Brooks et al., 2012\).  The last benchmark for this stock was in 2008 \(Brooks et al., 2008\).  Based on the previous assessment in 2012, the stock was not overfished, and overfishing was not ocurring. This assessment updates commercial fishery catch data, research survey indices of abundance, weights and maturity at age, and the analytical VPA assessment model and reference points through 2014. Additionally, stock projections have been updated through 2018.} \def\HADGBSoS{ \textbf{State of Stock: }{}Based on this updated assessment, the Georges Bank haddock \(\textit{Melanogrammus aeglefinus}\)  stock is not overfished and overfishing is not occurring \(Figures \ref{HADGBSSB\_plot1}\-\ref{HADGBF\_plot1}\){}.  Retrospective adjustments were made to the model results.  Spawning stock biomass \(SSB\)  in 2014 was estimated to be 150,053 \(mt\)  which is 139\\percent of the biomass target \(\$SSB\_{MSY}\${} \textit{proxy}{} = 108,300\;  Figure \ref{HADGBSSB\_plot1}{}\).  The 2014 fully selected fishing mortality was estimated to be 0.241 which is 62\\percent of the overfishing threshold proxy \(\$F\_{MSY}\${} \textit{proxy}{} = 0.39\;  Figure \ref{HADGBF\_plot1}{}\).} \def\HADGBProj{ \textbf{Projections: }{}Short term projections of biomass were derived by sampling from a cumulative  distribution  function of recruitment estimates from ADAPT VPA \(corresponding to SSB\$\>\$75,000 mt and dropping the extremely large 1963, 2003, and 2010 year classes, as well as the two final year class estimates for 2013 and 2014\). The annual fishery selectivity, maturity ogive, and mean weights at age used in this projection  are the most recent 5 year averages\;  retrospective adjustments were applied to the starting numbers at age \(2015\)  in the projections.} \def\HADGBSpecCmt{ \textbf{Special Comments: } \begin{itemize}{} \item{}What are the most important sources of uncertainty in this stock assessment?  Explain, and describe qualitatively how they affect the assessment results \(such as estimates of biomass, F, recruitment, and population projections\).  \linebreak{} \hspace\*{0.5cm} \textit{The largest source of uncertainty is the estimate of 2013 recruitment, which accounts for a substantial portion of catch and SSB in projections.  The rho adjusted projections reduce all starting   numbers at age to 67\\percent of unadjusted values \(i.e., all 2015 numbers at age are multiplied by 0.667\).  Two other exceptionally large year classes were observed in 2003 and 2010.  The 2003 year class is now estimated to be only 28\\percent of its initial model estimate, while the 2010 year class is now estimated to be 63\\percent of it\'s initial estimate.  Given that only 5 years of data are available to estimate the 2010 year class, it is possible that there may be further revisions to the magnitude of this year class estimate with more years of data.  Therefore, it remains uncertain if the scalar applied to all age classes in these projections \(0.667, based on Mohn\'s rho for SSB\)  is sufficient to account for future revisions to the 2013 year class estimate.  In addition, the median recruitment in the projections \(the proxy for recruitment at MSY\)  is 53.4 million, which is greater than 7 of the last 10 recruitments even though SSB is above the SSBMSY proxy \(Table 1\). While projections of catch and SSB in the near\-term are mostly driven by the 2013 year class, it is worth noting the magnitude of median projected recruitment relative to recent recruitment observations.}  \item{} Does this assessment model have a retrospective pattern? If so, is the pattern minor, or major? \(A major retrospective pattern occurs when the adjusted SSB or  \$F\_{Full}\${} lies outside of the approximate  joint confidence region for SSB and  \$F\_{Full}\${}\). \linebreak{} \hspace\*{0.5cm} \textit{ The 7\-year Mohn\'s  \textrho{}, relative to SSB, was 0.20 in the 2012 assessment and was 0.50 in 2014. The 7\-year Mohn\'s  \textrho{}, relative to F, was \-0.15 in the 2012 assessment and was \-0.34 in 2014. There was a major retrospective pattern for this assessment because the  \textrho{} adjusted estimates of 2014 SSB \(\$SSB\_{\rho}\${}=150,053\)  and 2014 F \(\$F\_{\rho}\${}=0.241\)  were outside the approximate 90\\percent confidence regions around SSB \(171,911 \- 301,282\)  and F \(0.13 \- 0.203\).  A retrospective  adjustment was made for both the determination of stock status and for projections of catch in 2016. The retrospective adjustment changed the 2014 SSB from 225,080 to 150,053 and the 2014  \$F\_{Full}\${} from 0.159 to 0.241.}  \item{}Based on this stock assessment, are population projections well determined or uncertain? \linebreak{} \hspace\*{0.5cm} \textit{As noted in \(1\)  above, population projections for Georges Bank haddock are uncertain due to uncertainty about the size of  the 2013 year class.  Two sensitivity projections were conducted.  The first sensitivity used biological parameters and fishery selectivity values from the 2010 year class for the 2013 year class.  A second sensitivity projection was made that used the same  biological and selectivity parameters as the first sensitivity, and in addition it  doubled the rho\-adjustment on the 2013 year class \(age 2 at the start of 2015\)  by multiplying it by 0.33.  These sensitivity runs are available on the Stock Assessment Supplementary Information  website \(\href{http:\/\/www.nefsc.noaa.gov\/saw\/sasi\/sasi\_report\_options.php}{SASINF}{}\), in the sensitivity slides appended to the end of the background presentation.}  \item{}Describe any changes that were made to the current stock assessment, beyond incorporating additional years of data  and the affect these changes had on the assessment and stock status. \linebreak{} \hspace\*{0.5cm} \textit{ No changes, other than the incorporation of new data were made to the Georges Bank haddock assessment for this update. However, the criterion for determining acceptable tows on NEFSC surveys used the TOGA protocol rather than the SHG protocol  \(TOGA=132x\).}  \item{}If the stock status has changed a lot since the previous assessment, explain why this occurred.  \linebreak{} \hspace\*{0.5cm} \textit{The stock status of Georges Bank haddock has not changed.}  \item{}Indicate what data or studies are currently lacking and which would be needed most to improve this stock assessment in the future.  \linebreak{} \hspace\*{0.5cm} \textit{Projection advice and reference points for  Georges Bank haddock are strongly dependent on recruitment.  A decade ago, extremely large year classes were considered anomalies \(e.g., 1963 and 2003\).   However, since 2003, there have been two more extremely large \(2010 and 2013\)  and one very large \(2012\)  year classes.  Future work could focus on recruitment forecasting and providing robust catch advice.}  \item{}Are there other important issues? \linebreak{} \hspace\*{0.5cm} \textit{The Georges Bank haddock assessment has recently developed a major retrospective pattern.  This stock assessment has historically performed  very consistently.  This should continue to be monitored.  Density\-dependent responses in growth should also continue to be monitored.  The switch from SHG to TOGA was ruled out as the cause of the retrospective pattern.} \end{itemize}{}} \def\HADGBRefr{ \textbf{References: }{} \linebreak{}Brooks, E.N, M.L. Traver, S.J. Sutherland, L. Van Eeckhaute, and L. Col.  2008.  In.  Northeast Fisheries Science Center. 2008. Assessment of 19 Northeast Groundfish Stocks through 2007: Report of the 3$^{rd}$ Groundfish Assessment Review Meeting \(GARM III\), Northeast Fisheries Science Center, Woods Hole, Massachusetts, August 4\-8, 2008. US Dep Commer, NOAA Fisheries, Northeast Fish Sci Cent Ref Doc. 08\-15\; 884 p + xvii. http:\/\/www.nefsc.noaa.gov\/publications\/crd\/crd0815\/ \linebreak{} \linebreak{}Brooks, E.N, S.J. Sutherland, L. Van Eeckhaute, and M. Palmer.  2012.  In.  Northeast Fisheries Science Center. 2012. Assessment or Data Updates of 13 Northeast Groundfish Stocks through 2010. US Dept Commer, NOAA Fisheries, Northeast Fish Sci Cent Ref Doc. 12\-06.\; 789 p. http:\/\/nefsc.noaa.gov\/publications\/crd\/crd1206\/ \linebreak{} \linebreak{}} \def\HADGBDraft{} \def\HADGBSPPname{Georges Bank haddock} \def\HADGBSPPnameT{Georges Bank haddock} \def\HADGBRptYr{2015} \def\HADGBAuthor{Liz Brooks} \def\HADGBReviewerComments{/home/dhennen/EIEIO/BigReport/HAD_GB/latex}  \def\HADGMMyPathTab{/home/dhennen/EIEIO/BigReport/HAD_GM/tables} \def\HADGMMyPathFig{/home/dhennen/EIEIO/BigReport/HAD_GM/figures} \def\HADGMfigFishCap{Total catch of Gulf of Maine haddock between 1977 and 2014 by fleet \(commercial, recreational, or foreign\)  and disposition \(landings and discards\).} \def\HADGMfigSSBCap{Trends in spawning stock biomass \(SSB\)  of Gulf of Maine haddock between 1977 and 2014 from the current  \(solid line\)  and previous \(dashed line\)  assessment and the corresponding  \$SSB\_{Threshold}\${} \(\$\dfrac{1}{2}\${} \$SSB\_{MSY}\${} \textit{proxy}{}\; horizontal dashed line\)  as well as  \$SSB\_{Target}\${} \(\$SSB\_{MSY}\${} \textit{proxy}{}\; horizontal dotted line\)   based on the 2015 assessment. The approximate 90\\percent lognormal confidence intervals are shown. The red dot indicates the rho\-adjusted SSB values that would have resulted had a retrospective adjusment been made to either model \(see Special Comments section\).} \def\HADGMfigFCap{Trends in the fully selected fishing mortality \(F\)  of Gulf of Maine haddock between 1977 and 2014 from the current  \(solid line\)  and previous \(dashed line\)  assessment and the corresponding  \$F\_{Threshold}\${} \(\$F\_{MSY}\${} \textit{proxy}{}=0.468\; horizontal dashed line\)  from the 2015 assessment model. The approximate 90\\percent lognormal confidence intervals are shown. The red dot indicates the rho\-adjusted F values that would have resulted had a retrospective adjusment been made to either model \(see Special Comments section\).} \def\HADGMfigRecrCap{Trends in Recruits \(age 1\)  \(000s\)  of Gulf of Maine haddock between 1977 and 2014 from the current \(solid line\)  and previous \(dashed line\)  assessment. The approximate 90\\percent lognormal confidence intervals are shown.} \def\HADGMfigSurvCap{Indices of biomass for the Gulf of Maine haddock between 1963 and 2015 for the Northeast Fisheries Science Center \(NEFSC\)  spring and fall bottom trawl surveys.  The approximate 90\\percent lognormal confidence intervals are shown.} \def\HADGMPreAmb{This assessment of the Gulf of Maine haddock \(\textit{Melanogrammus aeglefinus}\)  stock is an operational update of the existing 2014 benchmark assessment \(NEFSC 2014\). Based on the previous assessment, the stock was not overfished, and overfishing was not ocurring. This assessment updates commercial and recreational fishery catch data, research survey indices of abundance, and the analytical ASAP assessment model and reference points through 2014. Additionally, stock projections have been updated through 2018} \def\HADGMSoS{ \textbf{State of Stock: }{}Based on this updated assessment, the Gulf of Maine haddock \(\textit{Melanogrammus aeglefinus}\)  stock is not overfished and overfishing is not occurring \(Figures \ref{HADGMSSB\_plot1}\-\ref{HADGMF\_plot1}\){}. Retrospective adjustments were not made to the model results \(see Special Comments section of this report\). Spawning stock biomass \(SSB\)  in 2014 was estimated to be 10,325 \(mt\)  which is 223\\percent of the biomass target \(\$SSB\_{MSY}\${} \textit{proxy}{} = 4,623\;  Figure \ref{HADGMSSB\_plot1}{}\).  The 2014 fully selected fishing mortality was estimated to be 0.257 which is 55\\percent of the overfishing threshold proxy \(\$F\_{MSY}\${} \textit{proxy}{} =  \$F\_{40\\percent}\${} = 0.468\;  Figure \ref{HADGMF\_plot1}{}\).} \def\HADGMProj{ \textbf{Projections: }{}Short term projections of median total fishery yield and spawning stock biomass for Gulf of Maine haddock were conducted based on a harvest scenario of fishing at the  \$F\_{MSY}\${} \textit{proxy}{} between 2016 and 2018. Catch in 2015 has been estimated at 885 mt. Recruitment was sampled from a cumulative distribution  function of model estimated age\-1 recruitment from 1977\-2012. The age\-1 estimate in 2015 was generated from the geometric mean of the 1977\-2014 recruitment series. The annual fishery selectivity, maturity ogive, and mean weights at age used in the projections  were estimated from the most recent 5 year averages\;  retrospective adjustments were not applied in the projections. Given the uncertainty in the size of the 2012 and 2013 year classes and the model\'s tendency to overestimate large terminal year classes, the 2015 assessment review panel recommended that a sensitivity projection scenario which constrains terminal recruitment \(\'Constrain terminal R\'\)  be brought forward to the New England Fishery Management Council\'s Scientific and Statistical Committee \(NEFMC SSC\)  for consideration when setting catch advice\; these sensitivity projections are provided in the Supplemental Information Report \(\href{http:\/\/www.nefsc.noaa.gov\/saw\/sasi\/sasi\_report\_options.php}{SASINF}{}\).} \def\HADGMSpecCmt{ \textbf{Special Comments: } \begin{itemize}{} \item{}What are the most important sources of uncertainty in this stock assessment?  Explain, and describe qualitatively how they affect the assessment results \(such as estimates of biomass, F, recruitment, and population projections\).  \linebreak{} \hspace\*{0.5cm} \textit{ The largest source of uncertainty in the assessment is the estimated size of the 2012 and 2013 year classes. Based on the estimated selectivity patterns, these year classes are projected to be 30\\percent selected to the fishery in 2016 and 2017 respectively. However, recent changes to the commercial and recreational minimum retention size may result in these year classes recruiting to the fishery sooner than projected. The abundance and growth of the 2012 and 2013 year classes should be monitored and frequent model updates would be expected to improve the estimates of year class size and validate projection assumptions.}  \item{}Does this assessment model have a retrospective pattern? If so, is the pattern minor, or major? \(A major retrospective pattern occurs when the adjusted SSB or  \$F\_{Full}\${} lie outside of the approximate joint confidence region for SSB and  \$F\_{Full}\${}\). \linebreak{} \hspace\*{0.5cm} \textit{This assessment does not exhibit a retrospective pattern and therefore no retrospective adjustments were made to the terminal model results or the short\-term catch projections. The 7\-year Mohn\'s rho values on SSB \(\-0.04\)  and F \(0.03\)  are small and there were no consistent patterns in the directionality of the retrospective \'peels\' \(see the Supplemental Information Report, \href{http:\/\/www.nefsc.noaa.gov\/saw\/sasi\/sasi\_report\_options.php}{SASINF}{}\).}  \item{}Based on this stock assessment, are population projections well determined or uncertain?  \linebreak{} \hspace\*{0.5cm} \textit{Population projections for Gulf of Maine haddock, are reasonably well determined. The projected boimass from the last assessment is below the confidence bounds of the biomass estimated in the current assessment\; however, this is primarily due to the positive rescaling of the population size that occured from turning the ASAP model likelihood constants option off \(see next Special Comment\).}  \item{}Describe any changes that were made to the current stock assessment, beyond incorporating additional years of data  and the affect these changes had on the assessment and stock status. \linebreak{} \hspace\*{0.5cm} \textit{ Recreational catch estimates from 2004\-2014 were re\-estimated as part of this update to account for updates to the MRIP data. Additionally, the ASAP model was revised by turning the likelihood constants off\; sensitivity runs on SAW\/SARC 59 model suggest minor positive rescaling of recruitment and SSB, negative rescaling of F \(sensitivity results are provided in the Supplemental Information Report, \href{http:\/\/www.nefsc.noaa.gov\/saw\/sasi\/sasi\_report\_options.php}{SASINF}{}\).}  \item{}If the stock status has changed a lot since the previous assessment, explain why this occurred.  \linebreak{} \hspace\*{0.5cm} \textit{There has been no change in stock status since the previous SAW\/SARC 59 assessment \(2014\).}  \item{}Indicate what data or studies are currently lacking and which would be needed most to improve this stock assessment in the future.  \linebreak{} \hspace\*{0.5cm} \textit{Currently the assessment assumes 50\\percent survival of haddock discarded in the recreational fishery \- directed field research would improve this estimate. Additionally, a better understanding of recruitment processes may help to improve recruitment forecasting.}  \item{}Are there other important issues? \linebreak{} \hspace\*{0.5cm} \textit{None.} \end{itemize}{}} \def\HADGMRefr{ \textbf{References: }{} \linebreak{}Northeast Fisheries Science Center. 2014. 59$^{th}$ Northeast Regional Stock Assessment Workshop \(59$^{th}$ SAW\)  Assessment Report. US Dept Commer, Northeast Fish Sci Cent Ref Doc. 14\-09\; 782 p. Available from: National Marine Fisheries Service, 166 Water Street, Woods Hole, MA 02543\-1026 \linebreak{} \linebreak{}} \def\HADGMDraft{} \def\HADGMSPPname{Gulf of Maine haddock} \def\HADGMSPPnameT{Gulf of Maine haddock} \def\HADGMRptYr{2015} \def\HADGMAuthor{Michael Palmer} \def\HADGMReviewerComments{/home/dhennen/EIEIO/BigReport/HAD_GM/latex}  \def\YELGBMyPathTab{/home/dhennen/EIEIO/BigReport/YEL_GB/tables} \def\YELGBMyPathFig{/home/dhennen/EIEIO/BigReport/YEL_GB/figures} \def\YELGBfigFishCap{Total catch of Georges Bank Yellowtail Flounder between 1935 and 2014 by fleet \(US, Canadian, or Other\)  and disposition \(landings or discards\).} \def\YELGBfigSSBCap{Trends in average survey biomass \(mt\)  of Georges Bank Yellowtail Flounder between 2010 and 2015 from the current assessment.} \def\YELGBfigFCap{Trends in the exploitation rate \(catch\/average survey biomass\)  of Georges Bank Yellowtail Flounder between 2010 and 2014 from the current assessment.} \def\YELGBfigRecrCap{} \def\YELGBfigSurvCap{Indices of biomass for the Georges Bank Yellowtail Flounder between 1963 and 2015 for the Canadian DFO and Northeast Fisheries Science Center \(NEFSC\)  spring and fall bottom trawl surveys.  The approximate 90\\percent lognormal confidence intervals are shown.} \def\YELGBPreAmb{This assessment of the Georges Bank Yellowtail Flounder \(\textit{Limanda ferruginea}\)  stock was reviewed during the July 2015 TRAC meeting \(Legault et al. 2015\). It is an operational update of the existing 2014 update assessment \(Legault et al. 2014\). Based on the previous assessment the stock status was unknown, but stock condition was poor. This assessment updates commercial fishery catch data through 2014 \(Table \ref{YELGBCatch\_Status\_Table}{},  Figure \ref{YELGBFish\_plot1}{}\), and updates research survey indices of abundance and the empirical approach assessment through 2015 \(Figure \ref{YELGBSurv\_plot1}{}\). No stock projections can be computed using the empirical approach.} \def\YELGBSoS{ \textbf{State of Stock: }{}Based on this updated assessment, Georges Bank Yellowtail Flounder \(\textit{Limanda ferruginea}\)  stock status is unknown due to a lack of biological reference points associated with the empirical approach, but stock condition is poor.  Retrospective adjustments were not made to the model results. The average survey biomass in 2015 \(the arithmetic average of the 2015 DFO, 2015 NEFSC spring, and 2014 NEFSC fall surveys\)  was estimated to be 2,240 \(mt\)  \(Figure \ref{YELGBSSB\_plot1}{}\).  The 2014 exploitation rate \(2014 catch divided by 2014 average survey biomass\)  was estimated to be 0.071 \(Figure \ref{YELGBF\_plot1}{}\).} \def\YELGBProj{ \textbf{Projections: }{}Short term projections cannot be computed using the empirical approach. Application of an exploitation rate of 2\\percent to 16\\percent to the 2015 average survey biomass \(2,240 mt\)  results in catch advice for 2016 of 45 mt to 359 mt.} \def\YELGBSpecCmt{ \textbf{Special Comments: } \begin{itemize}{} \item{}What are the most important sources of uncertainty in this stock assessment?  Explain, and describe qualitatively how they affect the assessment results \(such as estimates of biomass, F, recruitment, and population projections\).  \linebreak{} \hspace\*{0.5cm} \textit{The largest source of uncertainty is the estimate of survey catchability, which currently relies on literature values for other species in other regions of the world using different gear. The survey catchability affects the expansion of the stratified mean catch per tow for each survey and is inversely related to the catch advice. Other sources of uncertainty include the appropriate exploitation rate to apply to this stock, which has seen continued decrease in survey biomass despite low exploitation rates. }  \item{} Does this assessment model have a retrospective pattern? If so, is the pattern minor, or major? \(A major retrospective pattern occurs when the adjusted SSB or  \$F\_{Full}\${} lies outside of the approximate  joint confidence region for SSB and  \$F\_{Full}\${}\; see RhoDecisionTab.ref\). \linebreak{} \hspace\*{0.5cm} \textit{ The model used to estimate status of this stock does not allow estimation of a retrospective pattern. }  \item{}Based on this stock assessment, are population projections well determined or uncertain? \linebreak{} \hspace\*{0.5cm} \textit{Population projections for Georges Bank Yellowtail Flounder are not computed. Catch advice is derived from applying an exploitation rate to the current estimate of survey biomass. }  \item{}Describe any changes that were made to the current stock assessment, beyond incorporating additional years of data  and the affect these changes had on the assessment and stock status. \linebreak{} \hspace\*{0.5cm} \textit{The 2014 NMFS spring survey value was changed from 2,684 mt to 2,763 mt due to using preliminary data during the 2014 TRAC meeting. However, this has no impact on the 2015 stock status or 2016 catch advice in this update assessment.}  \item{}If the stock status has changed a lot since the previous assessment, explain why this occurred.  \linebreak{} \hspace\*{0.5cm} \textit{The stock status of Georges Bank Yellowtail Flounder remains unknown and stock condition continues to be poor.}  \item{}Indicate what data or studies are currently lacking and which would be needed most to improve this stock assessment in the future.  \linebreak{} \hspace\*{0.5cm} \textit{The Georges Bank Yellowtail Flounder assessment could be improved with studies on NMFS and DFO survey catchability for flatfish.}  \item{}Are there other important issues? \linebreak{} \hspace\*{0.5cm} \textit{None. } \end{itemize}{}} \def\YELGBRefr{ \textbf{References: }{} \linebreak{}Legault, C.M., L. Alade, W.E. Gross, and H.H. Stone. 2014. Stock Assessment of Georges Bank Yellowtail Flounder for 2014. TRAC Ref. Doc. 2014\/01. 214 p. \linebreak{}Legault, C.M., L. Alade, D. Busawon, and H.H. Stone. 2015. Stock Assessment of Georges Bank Yellowtail Flounder for 2015. TRAC Ref. Doc. 2015\/01. 66 p. \linebreak{}} \def\YELGBDraft{} \def\YELGBSPPname{Georges Bank Yellowtail Flounder} \def\YELGBSPPnameT{Georges Bank Yellowtail Flounder} \def\YELGBRptYr{2015} \def\YELGBAuthor{Chris Legault} \def\YELGBReviewerComments{/home/dhennen/EIEIO/BigReport/YEL_GB/latex}  \def\YELSNEMAMyPathTab{/home/dhennen/EIEIO/BigReport/YEL_SNEMA/tables} \def\YELSNEMAMyPathFig{/home/dhennen/EIEIO/BigReport/YEL_SNEMA/figures} \def\YELSNEMAfigFishCap{Total catch of Southern New England\-Mid Atlantic Yellowtail flounder between 1973 and 2014 by fleet \(US domestic and foreign catch\)  and disposition \(landings and discards\).} \def\YELSNEMAfigSSBCap{Trends in spawning stock biomass of Southern New England\-Mid Atlantic Yellowtail flounder between 1973 and 2014 from the current  \(solid line\)  and previous \(dashed line\)  assessment and the corresponding  \$SSB\_{Threshold}\${} \(\$\dfrac{1}{2}\${} \$SSB\_{MSY}\${} \textit{proxy}{}\; horizontal dashed line\)  as well as  \$SSB\_{Target}\${} \(\$SSB\_{MSY}\${} \textit{proxy}{}\; horizontal dotted line\)   based on the 2015 assessment.  Biomass was adjusted for a retrospective pattern  and the adjustment is shown in red.  The approximate 90\\percent lognormal confidence intervals are shown.} \def\YELSNEMAfigFCap{Trends in the fully selected fishing mortality \(\$F\_{Full}\${}\)  of Southern New England\-Mid Atlantic Yellowtail flounder between 1973 and 2014 from the current  \(solid line\)  and previous \(dashed line\)  assessment and the corresponding  \$F\_{Threshold}\${} \(\$F\_{MSY}\${} \textit{proxy}{}=0.35\; horizontal dashed line\).  \$F\_{Full}\${} was adjusted for a retrospective pattern  and the adjustment is shown in red  based on the 2015 assessment. The approximate 90\\percent lognormal confidence intervals are shown.} \def\YELSNEMAfigRecrCap{Trends in Recruits \(age 1\)  \(000s\)  of Southern New England\-Mid Atlantic Yellowtail flounder between 1973 and 2014 from the current \(solid line\)  and previous \(dashed line\)  assessment. The approximate 90\\percent lognormal confidence intervals are shown.} \def\YELSNEMAfigSurvCap{Indices of biomass for the Southern New England\-Mid Atlantic Yellowtail flounder between 1973 and 2015 for the Northeast Fisheries Science Center \(NEFSC\)  spring, fall and winter bottom trawl surveys.  The approximate 90\\percent lognormal confidence intervals are shown.Note:  Larval index was also used in this assessment and is available in the supplemental documentation} \def\YELSNEMAPreAmb{This assessment of the Southern New England\-Mid Atlantic Yellowtail flounder \(\textit{Limanda ferruginea}\)  stock is an operational update of the existing 2012 benchmark ASAP assessment \(NEFSC 2012\). Based on the previous assessment the stock was not overfished, and overfishing was not ocurring. This assessment updates commercial fishery catch data, research survey indices of abundance, weights at age and the analytical ASAP assessment model and reference points through 2014. Additionally, stock projections have been updated through 2018} \def\YELSNEMASoS{ \textbf{State of Stock: }{}Based on this updated assessment, Southern New England\-Mid Atlantic Yellowtail flounder \(\textit{Limanda ferruginea}\)  stock is overfished and overfishing is occurring \(Figures \ref{YELSNEMASSB\_plot1}\-\ref{YELSNEMAF\_plot1}\){}. Retrospective adjustments were not made to the model results. Spawning stock biomass \(SSB\)  in 2014 was estimated to be 502 \(mt\)  which is 26\\percent of the biomass target \(\$SSB\_{MSY}\${} \textit{proxy}{} = 1,959\;  Figure \ref{YELSNEMASSB\_plot1}{}\). The 2014 fully selected fishing mortality was estimated to be 1.64 which is 469\\percent of the overfishing threshold proxy \(\$F\_{MSY}\${} \textit{proxy}{} = 0.35\;  Figure \ref{YELSNEMAF\_plot1}{}\).} \def\YELSNEMAProj{ \textbf{Projections: }{}Short term projections of biomass were derived by sampling from a cumulative  distribution function of recruitment estimates from ASAP.  Following the previous and accepted benchmark formulation, recruitment was based on the more recent estimates of the model time series \(i.e. corresponding to  year classes 1990 through 2013\)  to reflect the low recent pattern in recruitment. The annual fishery selectivity, maturity ogive, and mean weights at age used  in projection  are the most recent 5 year averages\;  retrospective adjustments were not applied in the projections.} \def\YELSNEMASpecCmt{ \textbf{Special Comments: } \begin{itemize}{} \item{}What are the most important sources of uncertainty in this stock assessment?  Explain, and describe qualitatively how they affect the assessment results \(such as estimates of biomass, F, recruitment, and population projections\).  \linebreak{} \hspace\*{0.5cm} \textit{The largest source of uncertainty is the emergence of the retrospective in this updated assessment.  This retrospective bias has resulted in the reduction SSB estimates and F estimates to increase with additional years of data  Further, the basis for recruitment assumption for stock status determination and population forecast   \(i.e. the inclusion of historical recruitment values versus contemporary basis of recruitment\)   is another source of uncertainty.  Although recent estmated recruitment likely reflect the realistic conditions for the stock, the basis for recruitment selection is not clearly understood.}  \item{} Does this assessment model have a retrospective pattern? If so, is the pattern minor, or major? \(A major retrospective pattern occurs when the adjusted SSB or  \$F\_{Full}\${} lies outside of the approximate  joint confidence region for SSB and  \$F\_{Full}\${}\; see RhoDecisionTab.ref\). \linebreak{} \hspace\*{0.5cm} \textit{ The 7\-year Mohn\'s  \textrho{}, relative to SSB, was 0.14 in the 2012 assessment and was 1.06 in 2014. The 7\-year Mohn\'s  \textrho{}, relative to F, was \-0.16 in the 2012 assessment and was \-0.53 in 2014. There was a major retrospective pattern for this assessment because the  \textrho{} adjusted estimates of 2014 SSB \(\$SSB\_{\rho}\${}=502\)  and 2014 F \(\$F\_{\rho}\${}=1.64\)  were outside the approximate 90\\percent confidence regions around SSB \(355 \- 739\)  and F \(1.053 \- 2.348\).  However, a retrospective adjustment was not made for both the determination of stock status and for projections of catch because of the large proportion of unfeasible projections \(assumed 2015 catch required a fishing mortality rate greater than 5\). This implies the retrospective adjustment was too large or the assumed 2015 catch was too high. The review panel decided to use the unadjusted projections as an upper bound for OFL with the strong suggestion that the OFL estimates were too high \(meaning the ABC buffer should be larger than normal\).}  \item{}Based on this stock assessment, are population projections well determined or uncertain? \linebreak{} \hspace\*{0.5cm} \textit{Population projections are uncertain with projected biomass from the last assessment above the confidence bounds of the biomass estimate in the current assessment.  Further, the short\-term projections which accounted for retropective adjustment in the starting numbers\-at\-age were unrelaible due to the low percentage of feasible solutions \(33\\percent\)  encountered durring the simulation. The feasibility problem in the projections were due to the assumed 2015 projected cacth exceeding the population biomass in several of the iteration caused by the retrospective adjustment. Evaluation of the the estimated January\-1 2015 biomass from the few feasbile projections indicated that the assumed 2015 catch was approximately 98\\percent of the stock biomass.  This suggests that the assumed 2015 catch is not sustainable given the low starting abundance in the forecast. Alternatively, the retro unadjusted projections performed well, but it is likely to result in an overly optimistic projection of the fishery yield and population biomass.}  \item{}Describe any changes that were made to the current stock assessment, beyond incorporating additional years of data  and the affect these changes had on the assessment and stock status. \linebreak{} \hspace\*{0.5cm} \textit{ There were no major changes to the current stock assessment formulation. However, the criterion for determining acceptable tows on the NEFSC surveys were revised for years the Bigelow year \(i.e. 2009\-2011\)  and carried foreward to ensure consistency between the assessment and deck operations.  The influence of the revised protocol on the survey indices was inconsequential.}  \item{}If the stock status has changed a lot since the previous assessment, explain why this occurred.  \linebreak{} \hspace\*{0.5cm} \textit{The overfishing and biomass stock status have changed since the previous assessment due to increased catches relative to the stock biomass and the very low recruitment of young fish, contributing very little to the adult biomass.}  \item{}Indicate what data or studies are currently lacking and which would be needed most to improve this stock assessment in the future.  \linebreak{} \hspace\*{0.5cm} \textit{The emergence of retrospective bias in this assessment is not clearly understood and may result from a variety of sources.  Future studiesshould further investigate the source of this retrospective pattern to help improve the underlying diagnostics of the model for providing catch advice for this stock.  Recruitment for Southern New England\-Mid Atlantic yellowtail flounder continues to be weak and it is likely that the stock is in a new productivity regime.  Should this pattern of poor recruitment continue into the future, the ability of the stock to recover will be impeded. Therefore, future studies should build on current knowledge to further understand the underlying ecological mechanisms of poor recruitment in the stock as it may relate to the physical environment.}  \item{}Are there other important issues? \linebreak{} \hspace\*{0.5cm} \textit{None. } \end{itemize}{}} \def\YELSNEMARefr{ \textbf{References: }{}  \linebreak{} Alade, L,  C. Legault, S.Cadrin.  2008.  In.  Northeast Fisheries Science Center. 2008. Assessment of 19 Northeast Groundfish Stocks  through 2007: Report of the 3$^{rd}$ Groundfish Assessment Review Meeting \(GARM III\), Northeast Fisheries Science Center, Woods  Hole, Massachusetts, August 4\-8, 2008. US Dep Commer, NOAA Fisheries, Northeast Fish Sci Cent Ref Doc. 08\-15\; 884 p + xvii.  http:\/\/www.nefsc.noaa.gov\/publications\/crd\/crd0815\/  \linebreak{}  \linebreak{} Northeast Fisheries Science Center. 2012.  54$^{th}$ Northeast Regional Stock Assessment Workshop \(54$^{th}$ SAW\)  Assessment Report. US Dept Commer, NOAA Fisheries, Northeast Fish Sci Cent Ref Doc. 12\-18.\; 600 p.  http:\/\/nefsc.noaa.gov\/publications\/crd\/crd1218\/  \linebreak{}  \linebreak{}} \def\YELSNEMADraft{} \def\YELSNEMASPPname{Southern New England-Mid Atlantic Yellowtail flounder} \def\YELSNEMASPPnameT{Southern New England-Mid Atlantic Yellowtail flounder} \def\YELSNEMARptYr{2015} \def\YELSNEMAAuthor{Larry Alade} \def\YELSNEMAReviewerComments{/home/dhennen/EIEIO/BigReport/YEL_SNEMA/latex}  \def\YELCCGMMyPathTab{/home/dhennen/EIEIO/BigReport/YEL_CCGM/tables} \def\YELCCGMMyPathFig{/home/dhennen/EIEIO/BigReport/YEL_CCGM/figures} \def\YELCCGMfigFishCap{Total catch of Cape Cod\-Gulf of Maine Yellowtail flounder between 1985 and 2014 by disposition \(landings and discards\).} \def\YELCCGMfigSSBCap{Trends in spawning stock biomass of Cape Cod\-Gulf of Maine Yellowtail flounder between 1985 and 2014 from the current  \(solid line\)  and previous \(dashed line\)  assessment and the corresponding  \$SSB\_{Threshold}\${} \(\$\dfrac{1}{2}\${} \$SSB\_{MSY}\${} \textit{proxy}{}\; horizontal dashed line\)  as well as  \$SSB\_{Target}\${} \(\$SSB\_{MSY}\${} \textit{proxy}{}\; horizontal dotted line\)   based on the 2015 assessment.  Biomass was adjusted for a retrospective pattern  and the adjustment is shown in red.   The 90\\percent bootstrap probability intervals are shown.} \def\YELCCGMfigFCap{Trends in the fully selected fishing mortality \(\$F\_{Full}\${}\)  of Cape Cod\-Gulf of Maine Yellowtail flounder between 1985 and 2014 from the current  \(solid line\)  and previous \(dashed line\)  assessment and the corresponding  \$F\_{Threshold}\${} \(\$F\_{MSY}\${} \textit{proxy}{}=0.279\; horizontal dashed line\).  \$F\_{Full}\${} was adjusted for a retrospective pattern  and the adjustment is shown in red  based on the 2015 assessment.  The 90\\percent bootstrap probability intervals are shown.} \def\YELCCGMfigRecrCap{Trends in Recruits \(age 1\)  \(000s\)  of Cape Cod\-Gulf of Maine Yellowtail flounder between 1985 and 2014 from the current \(solid line\)  and previous \(dashed line\)  assessment.  The 90\\percent bootstrap probability intervals are shown.} \def\YELCCGMfigSurvCap{Indices of biomass for the Cape Cod\-Gulf of Maine Yellowtail flounder between 1985 and 2015 for the Northeast Fisheries Science Center \(NEFSC\)  spring and fall bottom trawl surveys,  Massachusetts Department of Marine Fisheries \(MADMF\)  inshore state spring and fall bottom trawl surveys,and the Maine New Hampshire inshore state spring and fall state surveys  The 90\\percent bootstrap probability intervals are shown.} \def\YELCCGMPreAmb{This assessment of the Cape Cod\-Gulf of Maine Yellowtail flounder \(\textit{Limanda ferruginea}\)  stock is an operational update of the existing 2012 VPA assessment \(Legault et al., 2012\). The last benchmark for this stock was in 2008 \(Legault et al., 2008\). Based on the previous assessment the stock was overfished, and overfishing was ocurring. This assessment updates commercial fishery catch data, research survey indices of abundance, weights at age, and the analytical VPA assessment model and reference points through 2014. Additionally, stock projections have been updated through 2018} \def\YELCCGMSoS{ \textbf{State of Stock: }{}Based on this updated assessment, Cape Cod\-Gulf of Maine Yellowtail flounder \(\textit{Limanda ferruginea}\)  stock is overfished and overfishing is occurring \(Figures \ref{YELCCGMSSB\_plot1}\-\ref{YELCCGMF\_plot1}\){}.  Retrospective adjustments were made to the model results.  Spawning stock biomass \(SSB\)  in 2014 was estimated to be 857 \(mt\)  which is 16\\percent of the biomass target \(\$SSB\_{MSY}\${} \textit{proxy}{} = 5,259\;  Figure \ref{YELCCGMSSB\_plot1}{}\).  The 2014 fully selected fishing mortality was estimated to be 0.64 which is 229\\percent of the overfishing threshold proxy \(\$F\_{MSY}\${} \textit{proxy}{} = 0.279\;  Figure \ref{YELCCGMF\_plot1}{}\).} \def\YELCCGMProj{ \textbf{Projections: }{}Short term projections of biomass were derived by sampling from a cumulative  distribution function of recruitment estimates from ADAPT VPA. Recruitment estimates were hindcasted based on a simple linear regression between the NEFSC Fall survey abundance at age 1 and the VPA estimate at age 1.  The most recent two years \(2013 and 2014\)  were not included in the series of values due to high uncertainty in these estimates. This resulted in a total of 36 recruitment values: 8 from the hindcast predictions \(years 1977\-1984\)  and 28 from the VPA \(years 1985\-2012\). The annual fishery selectivity, maturity ogive, and mean weights at age used  in projection  are the most recent 5 year averages\;  retrospective adjustments were applied in the projections.} \def\YELCCGMSpecCmt{ \textbf{Special Comments: } \begin{itemize}{} \item{}What are the most important sources of uncertainty in this stock assessment?  Explain, and describe qualitatively how they affect the assessment results \(such as estimates of biomass, F, recruitment, and population projections\).  \linebreak{} \hspace\*{0.5cm} \textit{The largest source of uncertainty is the source of the retrospective pattern.This pattern has persisted for a number of years causing SSB estimates to decrease and F estimates to increaseas more years of data are added.}  \item{} Does this assessment model have a retrospective pattern? If so, is the pattern minor, or major? \(A major retrospective pattern occurs when the adjusted SSB or  \$F\_{Full}\${} lies outside of the approximate  joint confidence region for SSB and  \$F\_{Full}\${}\; see RhoDecisionTab.ref\). \linebreak{} \hspace\*{0.5cm} \textit{ The 7\-year Mohn\'s  \textrho{}, relative to SSB, was 0.68 in the 2012 assessment and was 0.98 in 2014. The 7\-year Mohn\'s  \textrho{}, relative to F, was \-0.19 in the 2012 assessment and was \-0.45 in 2014. There was a major retrospective pattern for this assessment because the  \textrho{} adjusted estimates of 2014 SSB \(\$SSB\_{\rho}\${}=857\)  and 2014 F \(\$F\_{\rho}\${}=0.64\)  were outside the approximate 90\\percent confidence regions around SSB \(1,375 \- 2,111\)  and F \(0.25 \- 0.52\).  A retrospective  adjustment was made for both the determination of stock status and for projections of catch in 2016. The retrospective adjustment changed the 2014 SSB from 1,695 to 857 and the 2014  \$F\_{Full}\${} from 0.355 to 0.64.}  \item{}Based on this stock assessment, are population projections well determined or uncertain? \linebreak{} \hspace\*{0.5cm} \textit{Population projections for Cape Cod\-Gulf of Maine Yellowtail flounder, are uncertain with projected biomass from the last assessmentabove the confidence bounds of the biomass estimated in the current assessment.}  \item{}Describe any changes that were made to the current stock assessment, beyond incorporating additional years of data  and the affect these changes had on the assessment and stock status. \linebreak{} \hspace\*{0.5cm} \textit{ No changes, other than the incorporation of new data were made to the Cape Cod\-Gulf of Maine Yellowtail flounder assessment for this update.}  \item{}If the stock status has changed a lot since the previous assessment, explain why this occurred.  \linebreak{} \hspace\*{0.5cm} \textit{The stock status has not changed since the previous assessment.}  \item{}Indicate what data or studies are currently lacking and which would be needed most to improve this stock assessment in the future.  \linebreak{} \hspace\*{0.5cm} \textit{Extensive studies have examined the causes of the retrospective patterns with no definitive conclusions other than a change in model does not resolve the issue.}  \item{}Are there other important issues? \linebreak{} \hspace\*{0.5cm} \textit{No. } \end{itemize}{}} \def\YELCCGMRefr{ \textbf{References: }{} \linebreak{}Legault, C,  L. Alade, S.Cadrin, J. King, and S. Sherman.  2008.  In.  Northeast Fisheries Science Center. 2008. Assessment of 19 Northeast Groundfish Stocks through 2007: Report of the 3$^{rd}$ Groundfish Assessment Review Meeting \(GARM III\), Northeast Fisheries Science Center, Woods Hole, Massachusetts, August 4\-8, 2008. US Dep Commer, NOAA Fisheries, Northeast Fish Sci Cent Ref Doc. 08\-15\; 884 p + xvii. http:\/\/www.nefsc.noaa.gov\/publications\/crd\/crd0815\/ \linebreak{} \linebreak{} Legault, C,  L. Alade, S.Emery, J. King, and S. Sherman.  2012.  In.  Northeast Fisheries Science Center. 2012. Assessment or Data Updates of 13 Northeast Groundfish Stocks through 2010. US Dept Commer, NOAA Fisheries, Northeast Fish Sci Cent Ref Doc. 12\-06.\; 789 p. http:\/\/nefsc.noaa.gov\/publications\/crd\/crd1206\/ \linebreak{} \linebreak{}} \def\YELCCGMDraft{} \def\YELCCGMSPPname{Cape Cod-Gulf of Maine Yellowtail flounder} \def\YELCCGMSPPnameT{Cape Cod-Gulf of Maine Yellowtail flounder} \def\YELCCGMRptYr{2015} \def\YELCCGMAuthor{Larry Alade} \def\YELCCGMReviewerComments{/home/dhennen/EIEIO/BigReport/YEL_CCGM/latex}  \def\FLWGMMyPathTab{/home/dhennen/EIEIO/BigReport/FLW_GM/tables} \def\FLWGMMyPathFig{/home/dhennen/EIEIO/BigReport/FLW_GM/figures} \def\FLWGMfigFishCap{Total catch of Gulf of Maine Winter Flounder between 2009 and 2014 by fleet \(commercial and recreational\)  and disposition \(landings and discards\). A 15\\percent mortality rate is assumed on recreational discards and a 50\\percent mortality rate on commercial discards.} \def\FLWGMfigSSBCap{Trends in 30+ cm area\-swept biomass of Gulf of Maine Winter Flounder between 2009 and 2014 from the current assessment based on the fall \(MENH, MDMF, NEFSC\)  surveys.  The approximate 90\\percent lognormal confidence intervals are shown.} \def\FLWGMfigFCap{Trends in the exploitation rates \(\$E\_{Full}\${}\)  of Gulf of Maine Winter Flounder between 2009 and 2014 from the current assessment and the corresponding  \$F\_{Threshold}\${} \(\$E\_{MSY}\${} \textit{proxy}{}=0.23\; horizontal dashed line\).  The approximate 90\\percent lognormal confidence intervals are shown.} \def\FLWGMfigRecrCap{} \def\FLWGMfigSurvCap{Indices of biomass for the Gulf of Maine Winter Flounder between 1978 and 2015 for the Northeast Fisheries Science Center \(NEFSC\), Massachusetts Division of Marine Fisheries \(MDMF\), and the Maine New Hampshire \(MENH\)  spring and fall bottom trawl surveys. NEFSC indices are calculated with gear and vessel conversion factors where appropriate.  The approximate 90\\percent lognormal confidence intervals are shown.} \def\FLWGMPreAmb{This assessment of the  Gulf of Maine Winter Flounder  \(\textit{Pseudopleuronectes americanus}\)   stock is an operational update of the  existing  2014  operational update area\-swept assessment \(NEFSC 2014\).  Based on the previous assessment the biomass status is unknown but overfishing was not occurring.  This assessment  updates commercial and recreational fishery catch data, research survey indices  of abundance, and the area\-swept estimates of 30+ cm biomass based on the fall NEFSC, MDMF, and MENH surveys.} \def\FLWGMSoS{ \textbf{State of Stock: }{}Based on this updated assessment, the Gulf of Maine Winter Flounder \(\textit{Pseudopleuronectes americanus}\)  stock biomass status is unknown and overfishing is not occurring \(Figures \ref{FLWGMSSB\_plot1}\-\ref{FLWGMF\_plot1}\){}. Retrospective adjustments were not made to the model results.  Biomass  \(30+ cm mt\)  in 2014 was estimated to be 4,655 mt \(Figure \ref{FLWGMSSB\_plot1}{}\). The 2014 30+ cm exploitation rate was estimated to be 0.06 which is 26\\percent of the overfishing exploitation threshold proxy \(\$E\_{MSY}\${} \textit{proxy}{} = 0.23\;  Figure \ref{FLWGMF\_plot1}{}\).} \def\FLWGMProj{ \textbf{Projections: }{}Projections are not possible with area\-swept based assessments. Catch advice was based on 75\\percent of  \$E\_{40\\percent}\${}\(75\\percent \$E\_{MSY}\${} \textit{proxy}{}\)  using the fall area\-swept estimate assuming q=0.6 on the wing spread. Updated 2014 fall 30+ cm area\-swept biomass \(4,655 mt\)  implies an OFL of 1,080 mt based on the  \$E\_{MSY}\${} \textit{proxy}{} and a catch of 810 mt for 75\\percent of the  \$E\_{MSY}\${} \textit{proxy}{}.} \def\FLWGMSpecCmt{ \textbf{Special Comments: } \begin{itemize}{} \item{}What are the most important sources of uncertainty in this stock assessment?  Explain, and describe qualitatively how they affect the assessment results \(such as estimates of biomass, F, recruitment, and population projections\).  \linebreak{} \hspace\*{0.5cm} \textit{The largest source of uncertainty with the direct estimates of stock biomass from survey area\-swept estimates originate from the assumption of survey gear catchability \(q\). Biomass and exploitation rate estimates are sensitive to the survey q assumption \(0.6 on wing spread\). The 2014 empirical benchmark assessement of Georges bank yellowtail flounder based the area\-swept q assumption on an average value taken from the literature for west coast flatfish \(0.37 on door spread\). The yellowtail q assumption corresponds to a value close to 1 on the wing spread which would result in a lower estimate of biomass \(2,995 mt\). Another major source of uncertainty with this method is that biomass based reference points cannot be determined and overfished status is unknown. }  \item{} Does this assessment model have a retrospective pattern? If so, is the pattern minor, or major? \(A major retrospective pattern occurs when the adjusted SSB or  \$F\_{Full}\${} lies outside of the approximate  joint confidence region for SSB and  \$F\_{Full}\${}\; see  Figure \ref{RhoDecision\_tab}{}\). \linebreak{} \hspace\*{0.5cm} \textit{ The model used to determine status of this stock does not allow estimation of a retrospective pattern.  An analytical stock assessment model does not exist for Gulf of Maine Winter Flounder.  An analytical model was no longer used for stock status determination at SARC 52 \(2011\)  due to concerns with a strong retrospective pattern.  Models have difficulty with the apparent lack of a relationship between a large decrease in the catch with little change in the indices and age and\/or size structure over time. }  \item{}Based on this stock assessment, are population projections well determined or uncertain? \linebreak{} \hspace\*{0.5cm} \textit{Population projections for Gulf of Maine Winter Flounder, do not exist for area\-swept assessments. Catch advice from area\-swept estimates tend to vary with interannual variability in the surveys.}  \item{}Describe any changes that were made to the current stock assessment, beyond incorporating additional years of data  and the affect these changes had on the assessment and stock status. \linebreak{} \hspace\*{0.5cm} \textit{ No changes, other than the incorporation of new data were made to the Gulf of Maine Winter Flounder assessment for this update. However, stabilizing the catch advice may be desired and could be obtained through the averaging of the area\-swept fall and spring survey estimates.}  \item{}If the stock status has changed a lot since the previous assessment, explain why this occurred.  \linebreak{} \hspace\*{0.5cm} \textit{The overfishing status of Gulf of Maine Winter Flounder has not changed. }  \item{}Indicate what data or studies are currently lacking and which would be needed most to improve this stock assessment in the future.  \linebreak{} \hspace\*{0.5cm} \textit{Direct area\-swept assessment could be improved with additional studies on survey gear efficiency.  Quantifying the degree of herding between the doors and escapement under the footrope and\/or above the headrope for each survey is needed since area\-swept biomass estimates and catch advice are sensitive to the assumed catchability.}  \item{}Are there other important issues? \linebreak{} \hspace\*{0.5cm} \textit{The general lack of a response in survey indices and age\/size structure is the primary source of concern with catches remaining far below the overfishing level. } \end{itemize}{}} \def\FLWGMRefr{ \textbf{References: }{} \linebreak{}Hendrickson L, Nitschke P, Linton B. 2015. 2014 Operational Stock Assessments for Georges Bank winter flounder, Gulf of Maine winter flounder, and pollock. US Dept Commer, Northeast Fish Sci Cent Ref Doc. 15\-01\; 228 p. Available from: National Marine Fisheries Service, 166 Water Street, Woods Hole, MA 02543\-1026, or online at http:\/\/nefsc.noaa.gov\/publications\/ \linebreak{} \linebreak{}Northeast Fisheries Science Center. 2011. 52$^{nd}$ Northeast Regional Stock AssessmentWorkshop \(52$^{nd}$ SAW\)  Assessment Report. US Dept Commer, Northeast Fish SciCent Ref Doc. 11\-17\; 962 p. Available from: National Marine Fisheries Service, 166 Water Street, Woods Hole, MA 02543\-1026, or online at http:\/\/www.nefsc.noaa.gov\/nefsc\/publications\/ \linebreak{} \linebreak{}} \def\FLWGMDraft{} \def\FLWGMSPPname{Gulf of Maine Winter Flounder} \def\FLWGMSPPnameT{Gulf of Maine Winter Flounder} \def\FLWGMRptYr{2015} \def\FLWGMAuthor{Paul Nitschke} \def\FLWGMReviewerComments{/home/dhennen/EIEIO/BigReport/FLW_GM/latex}  \def\FLWSNEMAMyPathTab{/home/dhennen/EIEIO/BigReport/FLW_SNEMA/tables} \def\FLWSNEMAMyPathFig{/home/dhennen/EIEIO/BigReport/FLW_SNEMA/figures} \def\FLWSNEMAfigFishCap{Total catch of Southern New England Mid\-Atlantic Winter Flounder between 1981 and 2014 by fleet \(commercial, recreational\)  and disposition \(landings and discards\).} \def\FLWSNEMAfigSSBCap{Trends in spawning stock biomass of Southern New England Mid\-Atlantic Winter Flounder between 1981 and 2014 from the current  \(solid line\)  and previous \(dashed line\)  assessment and the corresponding  \$SSB\_{Threshold}\${} \(\$\dfrac{1}{2}\${} \$SSB\_{MSY}\${} \textit{proxy}{}\; horizontal dashed line\)  as well as  \$SSB\_{Target}\${} \(\$SSB\_{MSY}\${} \textit{proxy}{}\; horizontal dotted line\)   based on the 2015 assessment. The approximate 90\\percent lognormal confidence intervals are shown.} \def\FLWSNEMAfigFCap{Trends in the fully selected fishing mortality \(\$F\_{Full}\${}\)  of Southern New England Mid\-Atlantic Winter Flounder between 1981 and 2014 from the current  \(solid line\)  and previous \(dashed line\)  assessment and the corresponding  \$F\_{Threshold}\${} \(\$F\_{MSY}\${}=0.325\; horizontal dashed line\)   based on the 2015 assessment. The approximate 90\\percent lognormal confidence intervals are shown.} \def\FLWSNEMAfigRecrCap{Trends in Recruits \(age 1\)  \(000s\)  of Southern New England Mid\-Atlantic Winter Flounder between 1981 and 2014 from the current \(solid line\)  and previous \(dashed line\)  assessment. The approximate 90\\percent lognormal confidence intervals are shown.} \def\FLWSNEMAfigSurvCap{Indices of biomass for the Southern New England Mid\-Atlantic Winter Flounder between 1963 and 2014 for the Northeast Fisheries Science Center \(NEFSC\)  spring and fall bottom trawl surveys, the MADMF spring survey, and the CT LISTS survey  The approximate 90\\percent lognormal confidence intervals are shown.} \def\FLWSNEMAPreAmb{This assessment of the Southern New England Mid\-Atlantic Winter Flounder \(\textit{Pseudopleuronectes americanus}\)  stock is an operational update of the existing 2011 benchmark ASAP assessment \(NEFSC 2011\). Based on the previous assessment the stock was overfished, but overfishing was not ocurring. This assessment updates commercial fishery catch data, recreational fishery catch data, and research survey indices of abundance, and the analytical ASAP assessment models and reference points through 2014. Additionally, stock projections have been updated through 2018} \def\FLWSNEMASoS{ \textbf{State of Stock: }{}Based on this updated assessment, the Southern New England Mid\-Atlantic Winter Flounder \(\textit{Pseudopleuronectes americanus}\)  stock is overfished but overfishing is not occurring \(Figures \ref{FLWSNEMASSB\_plot1}\-\ref{FLWSNEMAF\_plot1}\){}. Spawning stock biomass \(SSB\)  in 2014 was estimated to be 6,151 \(mt\)  which is 23\\percent of the biomass target \(26,928 mt\), and 23\\percent of the biomass threshold for an overfished stock \(\$SSB\_{Threshold}\${} = 13464 \(mt\)\;  Figure \ref{FLWSNEMASSB\_plot1}{}\).  The 2014 fully selected fishing mortality was estimated to be 0.16 which is 49\\percent of the overfishing threshold \(\$F\_{MSY}\${} = 0.325\;  Figure \ref{FLWSNEMAF\_plot1}{}\). Retrospective adjustments were not made to the model results. } \def\FLWSNEMAProj{ \textbf{Projections: }{}Short term projections of biomass were derived by sampling from a cumulative  distribution  function of recruitment estimates assuming a Beverton\-Holt stock recruitment relationship. The annual fishery selectivity, maturity ogive, and mean weights at age used  in projection  are the most recent 5 year averages\;  The model exhibited minor retrospective pattern in F and SSB so no retrospective adjustments were applied in the projections.} \def\FLWSNEMASpecCmt{ \textbf{Special Comments: } \begin{itemize}{} \item{}What are the most important sources of uncertainty in this stock assessment?  Explain, and describe qualitatively how they affect the assessment results \(such as estimates of biomass, F, recruitment, and population projections\).  \linebreak{} \hspace\*{0.5cm} \textit{A large source of uncertainty is the estimate of natural mortality based on longevity, which is not well studied in Southern New England Mid\-Atlantic Winter Flounder, and assumed constant over time.  Natural mortality affects the scale of the biomass and fishing mortality estimates.  Natural mortality was adjusted upwards from 0.2 to 0.3 during the last benchmark assessment assuming a max age of 16. However, there is still uncertainty in the true max age of the population and the resulting natural mortality estimate. Other sources of uncertainty include length distribution of the recreational discards.  The recreational discards, are a small component of the total catch, but the assessment suffers from very little length information used to characterize the recreational discards \(1 to 2 lengths in recent years\).}  \item{} Does this assessment model have a retrospective pattern? If so, is the pattern minor, or major? \(A major retrospective pattern occurs when the adjusted SSB or  \$F\_{Full}\${} lies outside of the approximate  joint confidence region for SSB and  \$F\_{Full}\${}\; see  Figure \ref{RhoDecision\_tab}{}\). \linebreak{} \hspace\*{0.5cm} \textit{ No retrospective adjustment of spawning stock biomass or fishing mortality in 2014 was required. }  \item{}Based on this stock assessment, are population projections well determined or uncertain? \linebreak{} \hspace\*{0.5cm} \textit{Population projections for Southern New England Mid\-Atlantic Winter Flounder are reasonably well determined. There is uncertainty in the estimates of M. In addition, while the retrospective pattern is considered minor \(within the 90\\percent CI of both F and SSB\)  the rho adjusted terminal value is very close to falling out of the bounds, becoming a major retrospective pattern. This would lead to retrospective adjustments being needed for the projections.}  \item{}Describe any changes that were made to the current stock assessment, beyond incorporating additional years of data  and the affect these changes had on the assessment and stock status. \linebreak{} \hspace\*{0.5cm} \textit{ No changes, other than the incorporation of new data were made to the Southern New England Mid\-Atlantic Winter Flounder assessment for this update.}  \item{}If the stock status has changed a lot since the previous assessment, explain why this occurred.  \linebreak{} \hspace\*{0.5cm} \textit{The stock status of Southern New England Mid\-Atlantic Winter Flounder has not changed since the previous benchmark in 2011.}  \item{}Indicate what data or studies are currently lacking and which would be needed most to improve this stock assessment in the future.  \linebreak{} \hspace\*{0.5cm} \textit{The Southern New England Mid\-Atlantic Winter Flounder assessment could be improved with additional studies on maximum age, as well additional information  of recreational discard lengths.  In addition, further investigation into the localized struture\/genetics of the stock is warranted. Also, a future shift to ASAP version 4 will provide the ability to model envirionmental factors that may influence both survey catchability and the modeled S\-R relationship}  \item{}Are there other important issues? \linebreak{} \hspace\*{0.5cm} \textit{None. } \end{itemize}{}} \def\FLWSNEMARefr{ \textbf{References: }{} \linebreak{}Smith, A. and S. Jones.  2008.  In.  Northeast Fisheries Science Center. 2008. Assessment of 19 Northeast Groundfish Stocks through 2007: Report of the 3$^{rd}$ Groundfish Assessment Review Meeting \(GARM III\), Northeast Fisheries Science Center, Woods Hole, Massachusetts, August 4\-8, 2008. US Dep Commer, NOAA Fisheries, Northeast Fish Sci Cent Ref Doc. 08\-15\; 884 p + xvii. http:\/\/www.nefsc.noaa.gov\/publications\/crd\/crd0815\/ \linebreak{} \linebreak{}Northeast Fisheries Science Center. 2011. 52$^{nd}$ Northeast Regional Stock AssessmentWorkshop \(52$^{nd}$ SAW\)  Assessment Report. US Dept Commer, Northeast Fish SciCent Ref Doc. 11\-17\; 962 p. Available from: National Marine Fisheries Service, 166Water Street, Woods Hole, MA 02543\-1026, or online at http:\/\/www.nefsc.noaa.gov\/nefsc\/publications\/ \linebreak{} \linebreak{}} \def\FLWSNEMADraft{} \def\FLWSNEMASPPname{Southern New England Mid-Atlantic Winter Flounder} \def\FLWSNEMASPPnameT{Southern New England Mid-Atlantic Winter Flounder} \def\FLWSNEMARptYr{2015} \def\FLWSNEMAAuthor{Anthony Wood} \def\FLWSNEMAReviewerComments{/home/dhennen/EIEIO/BigReport/FLW_SNEMA/latex}  \def\FLWGBMyPathTab{/home/dhennen/EIEIO/BigReport/FLW_GB/tables} \def\FLWGBMyPathFig{/home/dhennen/EIEIO/BigReport/FLW_GB/figures} \def\FLWGBfigFishCap{Total catches \(mt\)  of Georges Bank Winter Flounder between 1982 and 2015 by country and disposition \(landings and discards\).} \def\FLWGBfigSSBCap{Trends in spawning stock biomass \(mt\)  of Georges Bank Winter Flounder between 1982 and 2014 from the current  \(solid line\)  and previous \(dashed line\)  assessments and the corresponding  \$SSB\_{Threshold}\${} \(\$\dfrac{1}{2}\${} \$SSB\_{MSY}\${}\; horizontal dashed line\)  as well as  \$SSB\_{Target}\${} \(\$SSB\_{MSY}\${}\; horizontal dotted line\)   based on the 2015 assessment.  Biomass was adjusted for a retrospective pattern  and the adjustment is shown in red.  The approximate 90\\percent normal confidence intervals are shown.} \def\FLWGBfigFCap{Trends in fully selected fishing mortality \(\$F\_{Full}\${}\)  of Georges Bank Winter Flounder between 1982 and 2014 from the current  \(solid line\)  and previous \(dashed line\)  assessments and the corresponding  \$F\_{Threshold}\${} \(\$F\_{MSY}\${}=0.536\; horizontal dashed line\)  as well as \(\$F\_{Target}\${}= 75\\percent of FMSY\;  horizontal dotted line\). \$F\_{Full}\${} was adjusted for a retrospective pattern  and the adjustment is shown in red.  The approximate 90\\percent normal confidence intervals are also shown.} \def\FLWGBfigRecrCap{Trends in Recruits \(age 1\)  \(000s\)  of Georges Bank Winter Flounder between 1982 and 2014 from the current \(solid line\)  and previous \(dashed line\)  assessments. The approximate 90\\percent normal confidence intervals are shown.} \def\FLWGBfigSurvCap{Indices of biomass for the Georges Bank Winter Flounder for the Northeast Fisheries Science Center \(NEFSC\)  spring \(1968\-2015\)  and fall \(1963\-2014\)   bottom trawl surveys and the Canadian DFO spring survey \(1987\-2015\).  The approximate 90\\percent normal confidence intervals are shown.} \def\FLWGBPreAmb{This assessment of the Georges Bank Winter Flounder \(\textit{Pseudopleuronectes americanus}\)  stock is an operational update of the existing 2014 operational VPA assessment which included data for 1982\-2013 \(Hendrickson et al. 2015\). Based on the previous assessment the stock was not overfished and overfishing was not ocurring. This assessment updates commercial fishery catch data, research survey biomass indices, and the analytical VPA assessment model and reference points through 2014. Additionally, stock projections have been updated through 2018.} \def\FLWGBSoS{ \textbf{State of Stock: }{}Based on this updated assessment, the Georges Bank Winter Flounder \(\textit{Pseudopleuronectes americanus}\)  stock is overfished and overfishing is occurring \(Figures \ref{FLWGBSSB\_plot1}\-\ref{FLWGBF\_plot1}\){}. Retrospective adjustments were made to the model results.  Spawning stock biomass \(SSB\)  in 2014 was estimated to be 2,883 \(mt\)  which is 43\\percent of the biomass target for an overfished stock \(\$SSB\_{MSY}\${} = 6,700 with a threshold of 50\\percent of SSBMSY\;  Figure \ref{FLWGBSSB\_plot1}{}\).  The 2014 fully selected fishing mortality \(F\)  was estimated to be 0.778 which is 145\\percent of the overfishing threshold \(\$F\_{MSY}\${} = 0.536\;  Figure \ref{FLWGBF\_plot1}{}\). However, the 2014 point estimate of SSB and F, when adjusted for retrospective error \(83\\percent for SSB and \-51\\percent for F\), is outside the 90\\percent confidence interval of the unadjusted 2014 point estimate. Therefore, the 2014 F and SSB values used in the stock status determination were the retrospective\-adjusted values of 0.778 and 2,883 mt, respectively.} \def\FLWGBProj{ \textbf{Projections: }{}Short\-term projections of biomass were derived by sampling from a cumulative  distribution  function of recruitment estimates \(1982\-2013 YC\)  from the final run of the ADAPT VPA model. The annual fishery selectivity, maturity ogive, and mean weights\-at\-age used in the projection  are the most recent 5 year averages \(2010\-2014\). An SSB retrospective adjustment factor of 0.546 was applied in the projections.} \def\FLWGBSpecCmt{ \textbf{Special Comments: } \begin{itemize}{} \item{}What are the most important sources of uncertainty in this stock assessment?  Explain, and describe qualitatively how they affect the assessment results \(such as estimates of biomass, F, recruitment, and population projections\).  \linebreak{} \hspace\*{0.5cm} \textit{The largest source of uncertainty is the estimate of natural mortality based on longevity \(max. age = 20 for this stock\), which is not well studied in Georges Bank Winter Flounder, and assumed constant over time.  Natural mortality affects the scale of the biomass and fishing mortality estimates. Other sources of uncertainty include the underestimation of catches. Discards from the Canadian bottom trawl fleet were not provided by the CA DFO and the precision of the Canadian scallop dredge discard estimates, with only 1\-2 trips per month, are uncertain.The lack of age data for the Canadian spring survey catches requires the use of the US spring survey A\/L keys despite selectivity differences. In addition, there are no length or age composition data from the Canadian landings or discards GB winter flounder.}  \item{} Does this assessment model have a retrospective pattern? If so, is the pattern minor, or major? \(A major retrospective pattern occurs when the adjusted SSB or  \$F\_{Full}\${} lies outside of the approximate  joint confidence region for SSB and  \$F\_{Full}\${}\; see  Figure \ref{RhoDecision\_tab}{}\). \linebreak{} \hspace\*{0.5cm} \textit{ The 7\-year Mohn\'s  \textrho{}, relative to SSB, was 0.26 in the 2014 assessment and was 0.83 in 2014. The 7\-year Mohn\'s  \textrho{}, relative to F, was \-0.16 in the 2014 assessment and was \-0.51 in 2014. There was a major retrospective pattern for this assessment because the  \textrho{} adjusted estimates of 2014 SSB \(\$SSB\_{\rho}\${}=2,883\)  and 2014 F \(\$F\_{\rho}\${}=0.778\)  were outside the approximate 90\\percent confidence region around SSB \(3,783 \- 6,767\)  and F \(0.254 \- 0.504\).  A retrospective  adjustment was made for both the determination of stock status and for projections of catch in 2016. The retrospective adjustment changed the 2014 SSB from 5,275 to 2,883 and the 2014  \$F\_{Full}\${} from 0.379 to 0.778.}  \item{}Based on this stock assessment, are population projections well determined or uncertain? \linebreak{} \hspace\*{0.5cm} \textit{Population projections for Georges Bank Winter Flounder are reasonably well determined.}  \item{}Describe any changes that were made to the current stock assessment, beyond incorporating additional years of data  and the affect these changes had on the assessment and stock status. \linebreak{} \hspace\*{0.5cm} \textit{ The only change made to the Georges Bank Winter Flounder assessment, other than the incorporation of an additional  year of data, involved fishery selectivity.  During the 2014 assessment update, stock size estimates of age 1 and age 2 fish were not estimable  in the VPA during year t + 1 \(CVs near 1.0\). When age 2 stock size is not estimated in year t + 1,  the VPA model calculates the stock size of age 1 fish \(i.e., recruitment\)  in the terminal year by  using the age 1 partial recruitment \(PR\)  value to derive the F at age 1 in the terminal year. The  age 1 PR value used in the 2014 assessment update was 0.001. However, when this same age 1 PR value  was used in a VPA run for the current assessment update, the low PR value combined with the low age  1 catch in 2014 resulted in an unlikely high stock size estimate for age 1 recruitment in 2014 \(i.e.,  41,587,000 fish\)  when compared to survey observations of the same cohort \(i.e., age 1 in 2014 and age  2 in 2015\). In order to obtain a more realistic estimate of age 1 recruitment in 2014, I allowed the  VPA model to estimate age 2 stock size in 2015 \(i.e., and thereby avoided the use of an age 1 PR  value in the age 1 stock size calculation for 2014\)  and used the back\-calculated PR values from this  VPA run to derive a new PR\-at\-age vector which was used in the final 2015 VPA run. Similar to the  2014 assessment update, the final 2015 VPA run did not include the estimation of age 2 stock size  and the new PR\-at\-age vector was computed using the same methods as in the 2014 assessment.   Full selectivity occurs at age 4. For the 2015 assessment update, fishery selectivity for ages  1\-3 was changed from the 2014 assessment values of 0.001, 0.10 and 0.43, respectively, to 0.01,  0.08 and 0.55, respectively. Differences between estimates  of F, SSB and R values from the final  2015 VPA run, with the new PR vector, and a 2015 VPA run that utilized the PR vector from the 2014  assessment are shown in Table G30.}  \item{}If the stock status has changed a lot since the previous assessment, explain why this occurred.  \linebreak{} \hspace\*{0.5cm} \textit{The overfished and overfishing status of Georges Bank Winter Flounder has changed in the current assessment update due to a worsening of the retrospective error associated with fishing mortality and SSB.}  \item{}Indicate what data or studies are currently lacking and which would be needed most to improve this stock assessment in the future.  \linebreak{} \hspace\*{0.5cm} \textit{The Georges Bank Winter Flounder assessment could be improved with discard estimates from the Canadian bottom trawl fleet and age data from the Canadian spring bottom trawl surveys.}  \item{}Are there other important issues? \linebreak{} \hspace\*{0.5cm} \textit{None. } \end{itemize}{}} \def\FLWGBRefr{ \textbf{References: }{} \linebreak{} Hendrickson L, Nitschke P, Linton B. 2015. 2014 Operational Stock Assessments for Georges Bank winter flounder, Gulf of Maine winter flounder, and pollock. US Dept Commer, Northeast Fish Sci Cent Ref Doc. 15\-01\; 228 p. \linebreak{} \linebreak{}} \def\FLWGBDraft{} \def\FLWGBSPPname{Georges Bank Winter Flounder} \def\FLWGBSPPnameT{Georges Bank Winter Flounder} \def\FLWGBRptYr{2015} \def\FLWGBAuthor{Lisa Hendrickson} \def\FLWGBReviewerComments{/home/dhennen/EIEIO/BigReport/FLW_GB/latex}  \def\FLDGMGBMyPathTab{/home/dhennen/EIEIO/BigReport/FLD_GMGB/tables} \def\FLDGMGBMyPathFig{/home/dhennen/EIEIO/BigReport/FLD_GMGB/figures} \def\FLDGMGBfigFishCap{Total catch of northern windowpane flounder between 1975 and 2014 by disposition \(landings and discards\).} \def\FLDGMGBfigSSBCap{Trends in the biomass index \(a 3\-year moving average of the NEFSC fall bottom trawl survey index\)  of northern windowpane flounder between 1975 and 2014 from the current  assessment, and the corresponding  \$B\_{Threshold}\${} =  \$\dfrac{1}{2}\${} \$B\_{MSY}\${} \textit{proxy}{} = 0.777 kg\/tow \(horizontal dashed line\). } \def\FLDGMGBfigFCap{Trends in relative fishing mortality  of northern windowpane flounder between 1975 and 2014 from the current  assessment, and the corresponding  \$F\_{MSY}\${} \textit{proxy}{}=0.45 \(horizontal dashed line\). } \def\FLDGMGBfigRecrCap{} \def\FLDGMGBfigSurvCap{NEFSC fall bottom trawl survey indices in kg\/tow for northern windowpane flounder between 1975 and 2014  The approximate 90\\percent lognormal confidence intervals are shown.} \def\FLDGMGBPreAmb{This assessment of the northern windowpane flounder \(\textit{Scophthalmus aquosus}\)  stock is an operational update of the 2012 assessment which included updates through 2010 \(NEFSC 2012\). Based on the 2012 assessment the stock was overfished, and overfishing was ocurring. This assessment updates commercial fishery catch data, survey indices of abundance, AIM model results,  and reference points through 2014.} \def\FLDGMGBSoS{ \textbf{State of Stock: }{}Based on this updated assessment, the northern windowpane flounder \(\textit{Scophthalmus aquosus}\)  stock is overfished but overfishing is not occurring \(Figures \ref{FLDGMGBSSB\_plot1}\-\ref{FLDGMGBF\_plot1}\){}. Retrospective adjustments were not made to the model results. The mean NEFSC fall bottom trawl survey index from years 2012, 2013 and 2014 \(a 3\-year moving average is used as a biomass index\)  was 0.535 kg\/tow which is lower than the \$B\_{Threshold}\${} of 0.777 kg\/tow. The 2014 relative fishing mortality was estimated to be 0.393 kt per kg\/tow which is lower than the  \$F\_{MSY}\${} \textit{proxy}{} of 0.450 kt per kg\/tow.} \def\FLDGMGBProj{} \def\FLDGMGBSpecCmt{ \textbf{Special Comments: } \begin{itemize}{} \item{}What are the most important sources of uncertainty in this stock assessment?  Explain, and describe qualitatively how they affect the assessment results \(such as estimates of biomass, F, recruitment, and population projections\).  \linebreak{} \hspace\*{0.5cm} \textit{The main source of uncertainty in this assessment is the lack of windowpane discard estimates from Canadian fisheries to add to the catch component of model input. Discard estimates were from the U.S. only. There is overlap between the survey area and Canadian fishing grounds \(Van Eeckhaute et al. 2010\), which means catch from within the stock area was likely underestimated. }  \item{} Does this assessment model have a retrospective pattern? If so, is the pattern minor, or major? \(A major retrospective pattern occurs when the adjusted SSB or  \$F\_{Full}\${} lies outside of the approximate  joint confidence region for SSB and  \$F\_{Full}\${}\; see  Figure \ref{RhoDecision\_tab}{}\). \linebreak{} \hspace\*{0.5cm} \textit{ The model used to estimate status of this stock does not allow estimation of a retrospective pattern. }  \item{}Based on this stock assessment, are population projections well determined or uncertain? \linebreak{} \hspace\*{0.5cm} \textit{N\/A }  \item{}Describe any changes that were made to the current stock assessment, beyond incorporating additional years of data  and the affect these changes had on the assessment and stock status. \linebreak{} \hspace\*{0.5cm} \textit{No changes were made to the northern windowpane flounder assessment for this update  other than the incorporation of four years of new NEFSC fall bottom trawl survey data and  four years of new U.S. commercial landings and discard data \(2011 \- 2014\). }  \item{}If the stock status has changed a lot since the previous assessment, explain why this occurred.  \linebreak{} \hspace\*{0.5cm} \textit{The stock status of northern windowpane flounder changed from \'overfished and overfishing is occurring\' to \'overfished and overfishing is not occurring\' due to stable\-to\-decreasing catch since 2008, and an increasing trend in the survey index since 2010. }  \item{}Indicate what data or studies are currently lacking and which would be needed most to improve this stock assessment in the future.  \linebreak{} \hspace\*{0.5cm} \textit{The northern windowpane flounder assessment could be improved by estimating the Canadian windowpane removals and, although to a lesser degree, the \'general category\' scallop dredge fleet discards from within the stock area and using them as additional catch input to the AIM model.  While the model fit now is reasonable \(the relationship between ln\(relative F\)  and ln\(replacement ratio\), a measure of the relationship between catch and survey index values, has a p\-value of 0.079\)  there are probably removals unaccounted for in the model and the fit can likely be improved. }  \item{}Are there other important issues? \linebreak{} \hspace\*{0.5cm} \textit{None. } \end{itemize}{}} \def\FLDGMGBRefr{ \textbf{References: }{} \linebreak{} Most recent assessment update:  \linebreak{} Northeast Fisheries Science Center. 2012. Assessment or Data Updates of 13 Northeast Groundfish Stocks through 2010.  US Dept Commer, Northeast Fish Sci Cent Ref Doc. 12\-06\; 789 p. Available online at http:\/\/nefsc.noaa.gov\/publications\/  \linebreak{} \linebreak{} Most recent benchmark assessment:  \linebreak{} Northeast Fisheries Science Center. 2008. Assessment of 19 Northeast Groundfish Stocks through 2007:  Report of the 3$^{rd}$ Groundfish Assessment Review Meeting \(GARM III\), Northeast Fisheries Science Center,  Woods Hole, Massachusetts, August 4\-8, 2008. US Dep Commer, NOAA FIsheries, Northeast Fish Sci Cent Ref Doc. 08\-15\; 884 p + xvii.  \linebreak{} \linebreak{} Van Eeckhaute, L., Sameoto, J., and A. Glass. 2010. Discards of Atlantic cod, haddock and yellowtail flounder  from the 2009 Canadian scallop fishery on Georges Bank. TRAC Ref. Doc. 2010\/10. 7p.  \linebreak{} \linebreak{}} \def\FLDGMGBDraft{} \def\FLDGMGBSPPname{northern windowpane flounder} \def\FLDGMGBSPPnameT{Northern windowpane flounder} \def\FLDGMGBRptYr{2015} \def\FLDGMGBAuthor{Toni Chute} \def\FLDGMGBReviewerComments{/home/dhennen/EIEIO/BigReport/FLD_GMGB/latex}  \def\FLDSNEMAMyPathTab{/home/dhennen/EIEIO/BigReport/FLD_SNEMA/tables} \def\FLDSNEMAMyPathFig{/home/dhennen/EIEIO/BigReport/FLD_SNEMA/figures} \def\FLDSNEMAfigFishCap{Total catch of southern windowpane flounder between 1975 and 2014 by disposition \(landings and discards\).} \def\FLDSNEMAfigSSBCap{Trends in the biomass index \(a 3\-year moving average of the NEFSC fall bottom trawl survey index\)  of southern windowpane flounder between 1975 and 2014 from the current  assessment, and the corresponding  \$B\_{Threshold}\${} =  \$\dfrac{1}{2}\${} \$B\_{MSY}\${} \textit{proxy}{} = 0.123 kg\/tow\(horizontal dashed line\). } \def\FLDSNEMAfigFCap{Trends in relative fishing mortality  of southern windowpane flounder between 1975 and 2014 from the current  assessment, and the corresponding  \$F\_{MSY}\${} \textit{proxy}{}=2.027 \(horizontal dashed line\). } \def\FLDSNEMAfigRecrCap{} \def\FLDSNEMAfigSurvCap{NEFSC fall bottom trawl survey indices in kg\/tow for southern windowpane flounder between 1975 and 2014. The approximate 90\\percent lognormal confidence intervals are shown.} \def\FLDSNEMAPreAmb{This assessment of the southern windowpane flounder \(\textit{Scophthalmus aquosus}\)  stock is an operational update of the 2012 assessment which included updates through 2010 \(NEFSC 2012\). Based on the 2012 assessment the stock was not overfished, and overfishing was not ocurring. This assessment updates commercial fishery catch data, survey indices of abundance, AIM model results, and reference points through 2014. } \def\FLDSNEMASoS{ \textbf{State of Stock: }{}Based on this updated assessment, the southern windowpane flounder \(\textit{Scophthalmus aquosus}\)  stock is not overfished and overfishing is not occurring \(Figures \ref{FLDSNEMASSB\_plot1}\-\ref{FLDSNEMAF\_plot1}\){}. Retrospective adjustments were not made to the model results. The mean NEFSC fall bottom trawl survey index from years 2012, 2013, and 2014 \(a 3\-year moving average is used as a biomass index\)  was  0.413 \(kg\/tow\)  which is higher than the \$B\_{Threshold}\${}of 0.123 \(kg\/tow\). The 2014 relative fishing mortality was estimated to be  1.308 \(kt per kg\/tow\)  which is lower than the  \$F\_{MSY}\${} \textit{proxy}{} of 2.027 \(kt per kg\/tow\). } \def\FLDSNEMAProj{} \def\FLDSNEMASpecCmt{ \textbf{Special Comments: } \begin{itemize}{} \item{}What are the most important sources of uncertainty in this stock assessment?  Explain, and describe qualitatively how they affect the assessment results \(such as estimates of biomass, F, recruitment, and population projections\).  \linebreak{} \hspace\*{0.5cm} \textit{A source of uncertainty for this assessment is missing commercial discard estimates from the general category scallop dredge fleet that should be added to the catch time series for model input. }  \item{} Does this assessment model have a retrospective pattern? If so, is the pattern minor, or major? \(A major retrospective pattern occurs when the adjusted SSB or  \$F\_{Full}\${} lies outside of the approximate  joint confidence region for SSB and  \$F\_{Full}\${}\; see  Figure \ref{RhoDecision\_tab}{}\). \linebreak{} \hspace\*{0.5cm} \textit{ The model used to estimate status of this stock does not allow estimation of a retrospective pattern. }  \item{}Based on this stock assessment, are population projections well determined or uncertain? \linebreak{} \hspace\*{0.5cm} \textit{N\/A}  \item{}Describe any changes that were made to the current stock assessment, beyond incorporating additional years of data  and the affect these changes had on the assessment and stock status. \linebreak{} \hspace\*{0.5cm} \textit{ No changes were made to the southern windowpane flounder assessment for this update  other than the incorporation of four years of new NEFSC fall bottom trawl survey data and  four years of new U.S. commercial landings and discard data \(2011 \- 2014\). }  \item{}If the stock status has changed a lot since the previous assessment, explain why this occurred.  \linebreak{} \hspace\*{0.5cm} \textit{The stock status of southern windowpane flounder has not changed since the previous assessment. }  \item{}Indicate what data or studies are currently lacking and which would be needed most to improve this stock assessment in the future.  \linebreak{} \hspace\*{0.5cm} \textit{Estimates of discards from the general category scallop dredge fleet should be added to the catch time series for model input. However, the model fit is presently good with a randomization test indicating the correlation between ln\(relative F\)  and ln\(replacement ratio\), a measure of the relationship between catch and survey index values, is significant \(p = 0.002.\)  }  \item{}Are there other important issues? \linebreak{} \hspace\*{0.5cm} \textit{None. } \end{itemize}{}} \def\FLDSNEMARefr{ \textbf{References: }{} \linebreak{} Most recent assessment update:  \linebreak{} Northeast Fisheries Science Center. 2012. Assessment or Data Updates of 13 Northeast Groundfish Stocks through 2010.  US Dept Commer, Northeast Fish Sci Cent Ref Doc. 12\-06\; 789 p. Available online at http:\/\/nefsc.noaa.gov\/publications\/  \linebreak{} \linebreak{} Most recent benchmark assessment:  \linebreak{} Northeast Fisheries Science Center. 2008. Assessment of 19 Northeast Groundfish Stocks through 2007:  Report of the 3$^{rd}$ Groundfish Assessment Review Meeting \(GARM III\), Northeast Fisheries Science Center,  Woods Hole, MA, August 4\-8, 2008. US Dep Commer, NOAA Fisheries, Northeast Fish Sci Cent Ref Doc. 08\-15\; 884 p + xvii. \linebreak{} \linebreak{}} \def\FLDSNEMADraft{} \def\FLDSNEMASPPname{southern windowpane flounder} \def\FLDSNEMASPPnameT{Southern windowpane flounder} \def\FLDSNEMARptYr{2015} \def\FLDSNEMAAuthor{Toni Chute} \def\FLDSNEMAReviewerComments{/home/dhennen/EIEIO/BigReport/FLD_SNEMA/latex}  \def\PLAUNITMyPathTab{/home/dhennen/EIEIO/BigReport/PLA_UNIT/tables} \def\PLAUNITMyPathFig{/home/dhennen/EIEIO/BigReport/PLA_UNIT/figures} \def\PLAUNITfigFishCap{Total catch of Gulf of Maine\-Georges Bank American Plaice between 1980 and 2015 by fleet \(Gulf of Maine, Georges Bank, Southern New England, and Canadian\)  and disposition \(landings and discards\).} \def\PLAUNITfigSSBCap{Trends in spawning stock biomass of Gulf of Maine\-Georges Bank American Plaice between 1980 and 2015 from the current  \(solid line\)  and previous \(dashed line\)  assessment and the corresponding  \$SSB\_{Threshold}\${} \(\$\dfrac{1}{2}\${} \$SSB\_{MSY}\${} \textit{proxy}{}\; horizontal dashed line\)  as well as  \$SSB\_{Target}\${} \(\$SSB\_{MSY}\${} \textit{proxy}{}\; horizontal dotted line\)   based on the 2015 assessment.  Biomass was adjusted for a retrospective pattern  and the adjustment is shown in red.  The approximate 90\\percent normal confidence intervals are shown.} \def\PLAUNITfigFCap{Trends in the fully selected fishing mortality \(\$F\_{Full}\${}\)  of Gulf of Maine\-Georges Bank American Plaice between 1980 and 2015 from the current  \(solid line\)  and previous \(dashed line\)  assessment and the corresponding  \$F\_{Threshold}\${} \(\$F\_{MSY}\${} \textit{proxy}{}=0.196\; horizontal dashed line\).  \$F\_{Full}\${} was adjusted for a retrospective pattern  and the adjustment is shown in red,  based on the 2015 assessment. The approximate 90\\percent normal confidence intervals are shown.} \def\PLAUNITfigRecrCap{Trends in Recruits \(age 1\)  \(000s\)  of Gulf of Maine\-Georges Bank American Plaice between 1980 and 2015 from the current \(solid line\)  and previous \(dashed line\)  assessment.} \def\PLAUNITfigSurvCap{Indices of biomass for the Gulf of Maine\-Georges Bank American Plaice between 1963 and 2015 for the Northeast Fisheries Science Center \(NEFSC\)  and Massachusetts Division of Marine Fisheries \(MADMF\)  spring and autumn research bottom trawl surveys.  The approximate 90\\percent normal confidence intervals are shown.} \def\PLAUNITPreAmb{This assessment of the Gulf of Maine\-Georges Bank American Plaice \(\textit{Hippoglossoides platessoides}\)  stock is an operational update of the existing 2012 benchmark assessment \(O\'Brien et al. 2012\). Based on the previous assessment the stock was not overfished, and overfishing was not ocurring. This 2015 assessment updates commercial fishery catch data, research survey indices of abundance, the analytical VPA assessment model, and reference points through 2014. Additionally, stock projections have been updated through 2018.} \def\PLAUNITSoS{ \textbf{State of Stock: }{}Based on this updated assessment, the Gulf of Maine\-Georges Bank American Plaice \(\textit{Hippoglossoides platessoides}\)  stock is not overfished and overfishing is not occurring \(Figures \ref{PLAUNITSSB\_plot1}\-\ref{PLAUNITF\_plot1}\){}.  Retrospective adjustments were made to the model results.  Spawning stock biomass \(SSB\)  in 2014 was estimated to be 10,915 mt which is 83\\percent of the biomass target for this stock \(\$SSB\_{MSY}\${} \textit{proxy}{} = 13,107\;  Figure \ref{PLAUNITSSB\_plot1}{}\). The 2014 fully selected fishing mortality was estimated to be 0.118 which is 60\\percent of the overfishing threshold proxy \(\$F\_{MSY}\${} \textit{proxy}{} = 0.196\;  Figure \ref{PLAUNITF\_plot1}{}\).} \def\PLAUNITProj{ \textbf{Projections: }{}Short term projections of biomass were derived by sampling from an empirical cumulative  distribution  function of 34 recruitment estimates from VPA model results. The annual fishery selectivity, maturity ogive, and mean weights at age used in projections are the most recent 5 year averages\;  retrospective adjustments were applied in the projections.} \def\PLAUNITSpecCmt{ \textbf{Special Comments: } \begin{itemize}{} \item{}What are the most important sources of uncertainty in this stock assessment?  Explain, and describe qualitatively how they affect the assessment results \(such as estimates of biomass, F, recruitment, and population projections\).  \linebreak{} \hspace\*{0.5cm} \textit{A source of uncertainty in this assessment are the estimates of historical landings at age, prior to 1984, and the magnitude of  historical discards, prior to 1989. Both of these affect the scale of the biomass and fishing mortality estimates, and influence reference point estimations.}  \item{} Does this assessment model have a retrospective pattern? If so, is the pattern minor, or major? \(A major retrospective pattern occurs when the adjusted SSB or  \$F\_{Full}\${} lies outside of the approximate  joint confidence region for SSB and  \$F\_{Full}\${}\; see  Figure \ref{RhoDecision\_tab}{}\). \linebreak{} \hspace\*{0.5cm} \textit{ The 7\-year Mohn\'s  \textrho{}, relative to SSB, was 0.63 in the 2012 assessment and was 0.32 in 2014. The 7\-year Mohn\'s  \textrho{}, relative to F, was \-0.35 in the 2012 assessment and was 0.32 in 2014. There was a major retrospective pattern for this assessment because the  \textrho{} adjusted estimates of 2014 SSB \(\$SSB\_{\rho}\${}=10,915\)  and 2014 F \(\$F\_{\rho}\${}=0.118\)  were outside the approximate 90\\percent confidence regions around SSB \(12,742 \- 16,439\)  and F \(0.069 \- 0.093\).  A retrospective  adjustment was made for both the determination of stock status and for projections of catch in 2016. The retrospective adjustment changed the 2014 SSB from 14,543 to 10,915 and the 2014  \$F\_{Full}\${} from 0.08 to 0.118.}  \item{}Based on this stock assessment, are population projections well determined or uncertain? \linebreak{} \hspace\*{0.5cm} \textit{Population projections for Gulf of Maine\-Georges Bank American Plaice are reasonably well determined.}  \item{}Describe any changes that were made to the current stock assessment, beyond incorporating additional years of data  and the effect these changes had on the assessment and stock status. \linebreak{} \hspace\*{0.5cm} \textit{ No major changes, other than the addition of recent years of data, were made to the Gulf of Maine\-Georges Bank American Plaice assessment for this update. A new version of VPA was used \(V3.3.0\)  which gave very similar results to the 2012 VPA 3.1.0 run, with the same F and slightly lower SSB. The MADMF spring and autumn survey indices were re\-estimated for the time series, accounting for revised stratum areas. The revision occurred in 2007, but was overlooked in the 2012 assessment. A comparison of 2010 terminal year VPAs indicated minimal differences in 2010 SSB \(now slightly lower\)  and no change in F.}  \item{}If the stock status has changed a lot since the previous assessment, explain why this occurred.  \linebreak{} \hspace\*{0.5cm} \textit{As in recent assessments for Gulf of Maine\-Georges Bank American Plaice the stock status remains as not overfished and overfishing not occurring.}  \item{}Indicate what data or studies are currently lacking and which would be needed most to improve this stock assessment in the future.  \linebreak{} \hspace\*{0.5cm} \textit{The Gulf of Maine\-Georges Bank American Plaice assessment could be improved with updated studies on growth of Georges Bank and Gulf of Maine fish.}  \item{}Are there other important issues? \linebreak{} \hspace\*{0.5cm} \textit{A difference in growth between GM and GB fish has been documented, however, historical catch data information for GB may not be sufficient to conduct a separate assessment. Also, the growth difference may not persist in the most recent years. This could all be explored further in an benchmark review.} \end{itemize}{}} \def\PLAUNITRefr{ \textbf{References: }{} \linebreak{}O\'Brien, L. and J. Dayton \(2012\). E. Gulf of Maine \- Georges Bank American plaice Assessment for 2012 in Northeast Fisheries Science Center, 2012, Assessment or Data Updates of 13 Northeast Groundfish Stocks through 2010. US Dept Commer, Northeast Fish Sci Cent Ref Doc. 12\-06\; 789 p. http:\/\/www.nefsc.noaa.gov\/publications\/crd\/crd1206\/. \linebreak{} \linebreak{}} \def\PLAUNITDraft{} \def\PLAUNITSPPname{Gulf of Maine-Georges Bank American Plaice} \def\PLAUNITSPPnameT{Gulf of Maine-Georges Bank American Plaice} \def\PLAUNITRptYr{2015} \def\PLAUNITAuthor{Loretta O\'Brien} \def\PLAUNITReviewerComments{/home/dhennen/EIEIO/BigReport/PLA_UNIT/latex}  \def\WITUNITMyPathTab{/home/dhennen/EIEIO/BigReport/WIT_UNIT/tables} \def\WITUNITMyPathFig{/home/dhennen/EIEIO/BigReport/WIT_UNIT/figures} \def\WITUNITfigFishCap{Total catch of witch flounder between 1982 and 2014 by fleet \(commercial\)  and disposition \(landings and discards\).} \def\WITUNITfigSSBCap{Trends in spawning stock biomass \(mt\)  of witch flounder between 1982 and 2014 from the current  \(solid line\)  and previous \(dashed line\)  assessment and the corresponding  \$SSB\_{Threshold}\${} \(\$\dfrac{1}{2}\${} \$SSB\_{MSY}\${}\; horizontal dashed line\)  as well as  \$SSB\_{Target}\${} \$SSB\_{MSY}\${}\; horizontal dotted line\)   based on the current assessment. Red solid vertical line indicates rho adjusted SSB. Black solid vertical line indicates 90\\percent confidence interval for 2014.} \def\WITUNITfigFCap{Trends in the fully selected fishing mortality \(\$F\_{Full}\${}\)  of witch flounder between 1982 and 2014 from the current  \(solid line\)  and previous \(dashed line\)  assessment and the corresponding  \$F\_{Threshold}\${} \(\$F\_{MSY}\${}=0.279\; horizontal dashed line\)  based on the current assessment.  Red solid vertical line indicates rho adjusted  \$F\_{Full}\${}. Black solid vertical line indicates 90\\percent confidence interval for 2014.} \def\WITUNITfigRecrCap{Trends in Age 3  \(000s\)  of witch flounder between 1982 and 2014 from the current \(solid line\)  and previous \(dashed line\)  assessment.} \def\WITUNITfigSurvCap{Indices of biomass \(kg\/tow\)  for the witch flounder between 1963 and 2015 for the Northeast Fisheries Science Center \(NEFSC\)  spring and fall bottom trawl surveys.  The 90\\percent lognormal confidence intervals are shown.} \def\WITUNITPreAmb{This assessment of the witch flounder \(\textit{Glyptocephalus cynoglossus}\)  stock is an operational update of the 2012 assessment \(NEFSC 2012\)  and the 2008 benchmark assessment \(NEFSC 2008\). This assessment updates commercial fishery catch data, research survey indices, and the analytical assessment model through 2014. Additionally, stock projections have been updated through 2018. Reference points have been updated. } \def\WITUNITSoS{ \textbf{State of Stock: }{}witch flounder \(\textit{Glyptocephalus cynoglossus}\)  stock is overfished and overfishing is occurring \(Figures \ref{WITUNITSSB\_plot1}\-\ref{WITUNITF\_plot1}\){}. Retrospective adjustments were made to the model results.  Spawning stock biomass \(SSB\)  in 2014 was estimated to be 2,077 \(mt\)  which is 22\\percent of the  \$SSB\_{MSY}\${} proxy \(9,473\;  Figure \ref{WITUNITSSB\_plot1}{}\).  The 2014 fully selected fishing mortality was estimated to be 0.687 which is 246\\percent of the  \$F\_{MSY}\${} proxy \(0.279\;  Figure \ref{WITUNITF\_plot1}{}\). A retrospective adjustment to  \$F\_{Full}\${} and SSB in 2014 was required but did not lead to a change in status.  } \def\WITUNITProj{ \textbf{Projections: }{}Short term projection recruitment was sampled from a cumulative distribution function derived from ADAPT VPA \(with split time series between 1994 and 1995\)  estimated age 3 recruitment between 1982 and 2013.  Average 2010\-2014 partial recruitment, average 2010\-2014 mean weights, and maturation ogive representing 2011\-2015 maturity data were used.} \def\WITUNITSpecCmt{ \textbf{Special Comments: } \begin{itemize}{} \item{}What are the most important sources of uncertainty in this stock assessment?  Explain, and describe qualitatively how they affect the assessment results \(such as estimates of biomass, F, recruitment, and population projections\).  \linebreak{} \hspace\*{0.5cm} \textit{An important source of uncertainty is the retrospective pattern where fishing mortality is underestimated and spawning stock biomass and recruitment are overestimated. }  \item{} Does this assessment model have a retrospective pattern? If so, is the pattern minor, or major? \(A major retrospective pattern occurs when the adjusted SSB or  \$F\_{Full}\${} lies outside of the approximate  joint confidence region for SSB and  \$F\_{Full}\${}\).  \linebreak{} \hspace\*{0.5cm} \textit{ The 7\-year Mohn\'s  \textrho{}, relative to SSB, was 0.61 in the 2012 assessment and was 0.51 in 2014. The 7\-year Mohn\'s  \textrho{}, relative to F, was \-0.33 in the 2012 assessment and was \-0.38 in 2014. There was a major retrospective pattern for this assessment because the  \textrho{} adjusted estimates of 2014 SSB \(\$SSB\_{\rho}\${}=2,077\)  and 2014 F \(\$F\_{\rho}\${}=0.687\)  were outside the approximate 90\\percent confidence regions around SSB \(2,643 \- 3,864\)  and F \(0.321 \- 0.603\).  A retrospective  adjustment was made for both the determination of stock status and for projections of catch in 2016. The retrospective adjustment changed the 2014 SSB from 3,129 to 2,077 and the 2014  \$F\_{Full}\${} from 0.428 to 0.687.}  \item{}Based on this stock assessment, are population projections well determined or uncertain? \linebreak{} \hspace\*{0.5cm} \textit{Population projections for witch flounder appear to be optimistic\; the projected rho adjusted biomass from the last assessment  was above the upper confidence bounds of the projected rho adjusted biomass estimated in the current assessment. }  \item{}Describe any changes that were made to the current stock assessment, beyond incorporating additional years of data  and the effect these changes had on the assessment and stock status.  \linebreak{} \hspace\*{0.5cm} \textit{TOGA \(Type, Operation, Gear, Acquisition\)  values were used for haul criteria for NEFSC surveys for 2009 onward and minor changes in the use of observer data for discard estimates were made to the current witch flounder assessment. These changes had negligible effect on the assessment and stock status.  }  \item{}If the stock status has changed a lot since the previous assessment, explain why this occurred.  \linebreak{} \hspace\*{0.5cm} \textit{No change in stock status has occurred for witch flounder since the previous assessment. }  \item{}Indicate what data or studies are currently lacking and which would be needed most to improve this stock assessment in the future.  \linebreak{} \hspace\*{0.5cm} \textit{Extensive studies have examined the causes of retrospective patterns with no definitive conclusions other than a change in model does not resolve the issue. }  \item{}Are there other important comments? \linebreak{} \hspace\*{0.5cm} \textit{The VPA analysis was performed with survey time series split between 1994 and 1995. This time split corresponds to changes in the commercial reporting methods as well as other regulatory management changes.  } \end{itemize}{}} \def\WITUNITRefr{ \textbf{References: }{} \linebreak{}Northeast Fisheries Science Center. 2008. Assessment of 19 Northeast Groundfish Stocks through 2007: Report of the 3$^{rd}$ Groundfish Assessment Review Meeting \(GARM III\), Northeast Fisheries Science Center, Woods Hole, Massachusetts, August 4\-8, 2008. US Dep Commer, NOAA Fisheries, Northeast Fish Sci Cent Ref Doc. 08\-15\; 884 p + xvii. http:\/\/www.nefsc.noaa.gov\/publications\/crd\/crd0815\/ \linebreak{} \linebreak{}Northeast Fisheries Science Center. 2012. Assessment or Data Updates of 13 Northeast Groundfish Stocks through 2010.  US Dep Commer, NOAA Fisheries, Northeast Fish Sci Cent Ref Doc. 12\-06\; 789 p. http:\/\/www.nefsc.noaa.gov\/publications\/crd\/crd1206\/ \linebreak{} \linebreak{}} \def\WITUNITDraft{} \def\WITUNITSPPname{witch flounder} \def\WITUNITSPPnameT{Witch flounder} \def\WITUNITRptYr{2015} \def\WITUNITAuthor{Susan Wigley} \def\WITUNITReviewerComments{/home/dhennen/EIEIO/BigReport/WIT_UNIT/latex}  \def\HKWUNITMyPathTab{/home/dhennen/EIEIO/BigReport/HKW_UNIT/tables} \def\HKWUNITMyPathFig{/home/dhennen/EIEIO/BigReport/HKW_UNIT/figures} \def\HKWUNITfigFishCap{Total catch of white hake between 1963 and 2014 by fleet \(commercial, recreational, or Canadian\)  and disposition \(landings and discards\).} \def\HKWUNITfigSSBCap{Trends in spawning stock biomass of white hake between 1963 and 2014 from the current  \(solid line\)  and previous \(dashed line\)  assessment and the corresponding  \$SSB\_{Threshold}\${} \(\$\dfrac{1}{2}\${} \$SSB\_{MSY}\${} \textit{proxy}{}\; horizontal dashed line\)  as well as  \$SSB\_{Target}\${} \(\$SSB\_{MSY}\${} \textit{proxy}{}\; horizontal dotted line\)   based on the 2014 assessment.  The red dot indicates the rho\-adjusted SSB values that would have resulted had a retrospective  adjusment been made \(see Special Comments section\).  The approximate 90\\percent lognormal confidence intervals are shown.} \def\HKWUNITfigFCap{Trends in the fully selected fishing mortality \(\$F\_{Full}\${}\)  of white hake between 1963 and 2014 from the current  \(solid line\)  and previous \(dashed line\)  assessment and the corresponding  \$F\_{Threshold}\${} \(\$F\_{MSY}\${} \textit{proxy}{}=0.188\; horizontal dashed line\).  The red dot indicates the rho\-adjusted SSB values that would have resulted had a retrospective  adjusment been made \(see Special Comments section\).  The approximate 90\\percent lognormal confidence intervals are shown.} \def\HKWUNITfigRecrCap{Trends in Recruits \(age 1\)  \(000s\)  of white hake between 1963 and 2014 from the current \(solid line\)  and previous \(dashed line\)  assessment. The approximate 90\\percent lognormal confidence intervals are shown.} \def\HKWUNITfigSurvCap{Indices of biomass for the white hake between 1963 and 2015 for the Northeast Fisheries Science Center \(NEFSC\)  spring and fall bottom trawl surveys.  The approximate 90\\percent lognormal confidence intervals are shown.} \def\HKWUNITPreAmb{This assessment of the white hake \(\textit{Urophycis tenuis}\)  stock is an operational update of the existing 2013 benchmark ASAP assessment \(NEFSC 2013\). Based on the previous assessment the stock was not overfished, and overfishing was not ocurring. This assessment updates commercial fishery catch data, research survey indices of abundance, and the ASAP assessment models and reference points through 2014. Additionally, stock projections have been updated through 2018.} \def\HKWUNITSoS{ \textbf{State of Stock: }{}Based on this updated assessment, white hake \(\textit{Urophycis tenuis}\)  stock is not overfished and overfishing is not occurring \(Figures \ref{HKWUNITSSB\_plot1}\-\ref{HKWUNITF\_plot1}\){}. Retrospective adjustments were not made to the model results.  Spawning stock biomass \(SSB\)  in 2014 was estimated to be 28,553 \(mt\)  which is 88\\percent of the biomass threshold for an overfished stock \(\$SSB\_{MSY}\${} \textit{proxy}{} = 32,550\;  Figure \ref{HKWUNITSSB\_plot1}{}\).  The 2014 fully selected fishing mortality was estimated to be 0.076 which is 40\\percent of the overfishing threshold proxy \(\$F\_{MSY}\${} \textit{proxy}{} = 0.188\;  Figure \ref{HKWUNITF\_plot1}{}\).} \def\HKWUNITProj{ \textbf{Projections: }{}Short term projections of catch and SSB were derived by sampling from a cumulative  distribution  function of recruitment estimates from ASAP from 1995\-2012. The annual fishery selectivity, maturity ogive, and mean weights at age used in the projection  are the most recent 5 year averages. } \def\HKWUNITSpecCmt{ \textbf{Special Comments: } \begin{itemize}{} \item{}What are the most important sources of uncertainty in this stock assessment?  Explain, and describe qualitatively how they affect the assessment results \(such as estimates of biomass, F, recruitment, and population projections\).  \linebreak{} \hspace\*{0.5cm} \textit{1. Catch at age information is not well characterized due to possible mis\-identification of species in the commercial and sea sampling data, particularly in early years, low sampling of commercial landings in  some years, and sparse discard data particularly in early years.  \linebreak{} \hspace\*{0.5cm}2. Since the commercial catch is aged primarily with survey age\/length keys, there is considerable augmentation required, mainly for ages 5 and older. The numbers at age and mean weights at age in the catch for these ages may therefore not be well specified.  \linebreak{} \hspace\*{0.5cm}3. White hake may move seasonally into and out of the defined stock area.  \linebreak{} \hspace\*{0.5cm}4. There are no commercial catch at age data prior to 1989 and the catchability of older ages in the surveys is very low. This results in a large uncertainty in starting numbers at age.  \linebreak{} \hspace\*{0.5cm}5. Since 2003, dealers have been culling very large fish out of the large category. However, there was no market category to input into the landings until June 2014. The length compositions are distinct from large and have been identified since 2011. This may bias the age composition of the landings, particularly in 2014 when 2000 of the 5000 large samples were these extra\-large fish.  \linebreak{} \hspace\*{0.5cm}6. A pooled age\/length key is used for 1963\-1981, fall 2003 \(second half of commercial key\)  and 2014.Age data were not available for 2014 in time for this assessment. The same pooled key that was used for 1963\-1981 was used for 2014.}  \item{} Does this assessment model have a retrospective pattern? If so, is the pattern minor, or major? \(A major retrospective pattern occurs when the adjusted SSB or  \$F\_{Full}\${} lies outside of the approximate  joint confidence region for SSB and  \$F\_{Full}\${}\; see  Figure \ref{RhoDecision\_tab}{}\). \linebreak{} \hspace\*{0.5cm} \textit{ No retrospective adjustment of spawning stock biomass or fishing mortality in 2014 was required.  The pattern in this assessment is considered minor \(Mohn’s rho of 0.18 on SSB, Mohn’s rho of 0.12 on F\)  with the adjusted SSB within the 90 \\percent CI of the MCMC. However, the Mohn’s rho for Age 1 estimates is 0.54. This may have an impact on projections if this continues into the future.}  \item{}Based on this stock assessment, are population projections well determined or uncertain? \linebreak{} \hspace\*{0.5cm} \textit{Population projections for white hake, are not well determined and projected biomass from the last assessment  was outside the confidence bounds of the biomass estimated in the current assessment. }  \item{}Describe any changes that were made to the current stock assessment, beyond incorporating additional years of data  and the affect these changes had on the assessment and stock status. \linebreak{} \hspace\*{0.5cm} \textit{ The 2011 catch\-at\-length and age were re\-estimated for both landings and discards. For the  landings, two samples were adjusted for dorsal length to total length that had been missed in the previous assessment.}  \item{}If the stock status has changed a lot since the previous assessment, explain why this occurred.  \linebreak{} \hspace\*{0.5cm} \textit{While stock status of white hake has not changed, the stock has not rebuilt as the projections from the last assessment indicated. This is due to the retrospective in recruitment. The numbers for the 2005\-2009 year classes, which were included in the age 2\-6 starting numbers in the projections, were over\-estimated which led to over\-estimating SSB in 2014.}  \item{}Indicate what data or studies are currently lacking and which would be needed most to improve this stock assessment in the future.  \linebreak{} \hspace\*{0.5cm} \textit{ Age structures from the observer program are available and should be aged to augment  the survey keys. There is a also a new market category for heads and age structures could be  acquired from these is an otolith length\/total length relationship can be established. }  \item{}Are there other important issues? \linebreak{} \hspace\*{0.5cm} \textit{None. } \end{itemize}{}} \def\HKWUNITRefr{ \textbf{References: }{} \linebreak{} NEFSC. 2013. 56$^{th}$ Northeast Regional Stock Assessment Workshop \(56$^{th}$ SAW\)  Assessment  Report.US Dep Commer, NOAA Fisheries, Northeast Fish Sci Cent Ref Doc. 13\-10\; 868 p.  http:\/\/www.nefsc.noaa.gov\/publications\/crd\/crd1310\/  \linebreak{} \linebreak{}} \def\HKWUNITDraft{} \def\HKWUNITSPPname{white hake} \def\HKWUNITSPPnameT{White hake} \def\HKWUNITRptYr{2015} \def\HKWUNITAuthor{Katherine Sosebee} \def\HKWUNITReviewerComments{/home/dhennen/EIEIO/BigReport/HKW_UNIT/latex}  \def\OPTUNITMyPathTab{/home/dhennen/EIEIO/BigReport/OPT_UNIT/tables} \def\OPTUNITMyPathFig{/home/dhennen/EIEIO/BigReport/OPT_UNIT/figures} \def\OPTUNITfigFishCap{Total catch of ocean pout  between 1968 and 2014 by fleet \(US and Other\)  and disposition \(landings and discards\).} \def\OPTUNITfigSSBCap{Trends in biomass \(kg\/tow\)  of ocean pout  between 1968 and 2014 from the current  \(solid line\)  and previous \(dashed line\)  assessment and the corresponding  \$B\_{Threshold}\${} \(\$\dfrac{1}{2}\${} \$B\_{MSY}\${} \textit{proxy}{}\; horizontal dashed line\)  as well as  \$B\_{Target}\${} \(\$B\_{MSY}\${} \textit{proxy}{}\; horizontal dotted line\)   based on the current assessment. } \def\OPTUNITfigFCap{Trends in the exploitation rate of ocean pout between 1968 and 2014 from the current  \(solid line\)  and previous \(dashed line\)  assessment and the corresponding  \$F\_{Threshold}\${} \(\$F\_{MSY}\${} \textit{proxy}{}=0.76\; horizontal dashed line\)   based on the current assessment. } \def\OPTUNITfigRecrCap{} \def\OPTUNITfigSurvCap{Indices of biomass \(kg\/tow\)  for ocean pout  between 1968 and 2015 for the Northeast Fisheries Science Center \(NEFSC\)  spring survey.   The approximate 90\\percent lognormal confidence intervals are shown.} \def\OPTUNITPreAmb{This assessment of the ocean pout  \(\textit{Zoarces americanus}\)  stock is an operational update of the 2012 assessment \(NEFSC 2012\)  and the 2008 benchmark assessment \(NEFSC 2008\). Based on the 2012 assessment, the stock was overfished but overfishing was not ocurring. This assessment updates commercial fishery catch data, research survey indices and the exploitation ratios through 2014. There are no stock projections.} \def\OPTUNITSoS{ \textbf{State of Stock: }{}Based on the current assessment, the ocean pout  \(\textit{Zoarces americanus}\)  stock is overfished and overfishing is not occurring \(Figures \ref{OPTUNITSSB\_plot1}\-\ref{OPTUNITF\_plot1}\){}. Retrospective adjustments were not made to the model results. Biomass proxy \(B\)  in 2014 was estimated to be 0.29 \(kg\/tow\)  which is 6\\percent of the biomass target \(\$B\_{MSY}\${} \textit{proxy}{} = 4.94\;  Figure \ref{OPTUNITSSB\_plot1}{}\).  The 2014 fully selected fishing mortality was estimated to be 0.269 which is 35\\percent of the overfishing threshold proxy \(\$F\_{MSY}\${} \textit{proxy}{} = 0.76\;  Figure \ref{OPTUNITF\_plot1}{}\).} \def\OPTUNITProj{ \textbf{Projections: }{}The index\-based assessment approach does not support catch projections\; catch advice for ocean pout has been based on the target exploitation rate and the most recent centered 3\-year average biomass index from the NEFSC spring survey. } \def\OPTUNITSpecCmt{ \textbf{Special Comments: } \begin{itemize}{} \item{}What are the most important sources of uncertainty in this stock assessment?  Explain, and describe qualitatively how they affect the assessment results \(such as estimates of biomass, F, recruitment, and population projections\).  \linebreak{} \hspace\*{0.5cm} \textit{ An important source of uncertainty is the stock has not responded to low catch as expected. }  \item{}Does this assessment model have a retrospective pattern? If so, is the pattern minor or major?  \(A major retrospective pattern occurs when the adjusted SSB or  \$F\_{Full}\${} lies outside of the approximate  joint confidence region for SSB and  \$F\_{Full}\${}\; see  Figure \ref{RhoDecision\_tab}{}\). \linebreak{} \hspace\*{0.5cm} \textit{ The model used to estimate status of this stock does not allow estimation of a retrospective pattern. }  \item{}Based on this stock assessment, are population projections well determined or uncertain? \linebreak{} \hspace\*{0.5cm} \textit{ N\/A}  \item{}Describe any changes that were made to the current stock assessment, beyond incorporating additional years of data  and the effect these changes had in the assessment and stock status. \linebreak{} \hspace\*{0.5cm} \textit{TOGA \(Type, Operation, Gear, Acquisition\)  values were used for haul criteria for NEFSC surveys for 2009 onward and minor changes in the use of observer data for discard estimates were made to the current assessment. These changes had a negligible effect on the assessment and stock status.   Recreational landings were updated and found to be negligible \(time series average of recreational landings to total catch was less than 1\\percent\)  and therefore not included in this assessment.}  \item{}If the stock status has changed a lot since the previous assessment, explain why this occurred.  \linebreak{} \hspace\*{0.5cm} \textit{Ocean pout stock status has not changed since the previous assessment.}  \item{}Indicate what data or studies are currently lacking and which would be needed most to improve this stock assessment in the future.  \linebreak{} \hspace\*{0.5cm} \textit{The ocean pout assessment could be improved with studies that explore why this stock is not rebuilding as expected. }  \item{}Are there other important comments? \linebreak{} \hspace\*{0.5cm} \textit{Biological reference points are based on catch\; the estimated discards used in the catch are based on a mix of direct \(1989 onward\)  and indirect \(1988 and back\)  methods. The catch used to determine MSY is based on indirect methods. } \end{itemize}{}} \def\OPTUNITRefr{ \textbf{References: }{} \linebreak{}Northeast Fisheries Science Center. 2012. Assessment or Data Updates of 13 Northeast Groundfish Stocks through 2010.  US Dep Commer, NOAA Fisheries, Northeast Fish Sci Cent Ref Doc. 12\-06\; 789 p. http:\/\/www.nefsc.noaa.gov\/publications\/crd\/crd1206\/ \linebreak{} \linebreak{}Northeast Fisheries Science Center. 2008. Assessment of 19 Northeast Groundfish Stocks through 2007: Report of the 3$^{rd}$ Groundfish Assessment Review Meeting \(GARM III\), Northeast Fisheries Science Center, Woods Hole, Massachusetts, August 4\-8, 2008. US Dep Commer, NOAA Fisheries, Northeast Fish Sci Cent Ref Doc. 08\-15\; 884 p + xvii. http:\/\/www.nefsc.noaa.gov\/publications\/crd\/crd0815\/ \linebreak{} \linebreak{}} \def\OPTUNITDraft{} \def\OPTUNITSPPname{Ocean Pout} \def\OPTUNITSPPnameT{Ocean Pout} \def\OPTUNITRptYr{2015} \def\OPTUNITAuthor{Susan Wigley} \def\OPTUNITReviewerComments{/home/dhennen/EIEIO/BigReport/OPT_UNIT/latex}  \def\POKUNITMyPathTab{/home/dhennen/EIEIO/BigReport/POK_UNIT/tables} \def\POKUNITMyPathFig{/home/dhennen/EIEIO/BigReport/POK_UNIT/figures} \def\POKUNITfigFishCap{Total catch of pollock between 1970 and 2014 by fleet \(commercial, Canadian, distant water fleet, and recreational\)  and disposition \(landings and discards\).} \def\POKUNITfigSSBCap{Estimated trends in the spawning stock biomass of pollock between 1970 and 2014 from the current  \(solid line\)  and previous \(dashed line\)  assessment and the corresponding  \$SSB\_{Threshold}\${} \(0.5 \* \$SSB\_{MSY}\${} proxy\; horizontal dashed line\)  as well as  \$SSB\_{Target}\${} \(\$SSB\_{MSY}\${} proxy\; horizontal dotted line\)   based on the 2015 assessment models base \(A\)  and flat sel sensitivity \(B\). Biomass was adjusted for a retrospective pattern and the adjustment is shown in red. The approximate 90\\percent lognormal confidence intervals are shown.} \def\POKUNITfigFCap{Estimated trends in age 5 to 7 average F \(\$F\_{AVG}\${}\)  of pollock between 1970 and 2014 from the current  \(solid line\)  and previous \(dashed line\)  assessment and the corresponding  \$F\_{Threshold}\${} \(\$F\_{MSY}\${} proxy\; dashed line\)  based on the 2015 assessment models base \(A\)  and flat sel sensitivity \(B\).  \$F\_{AVG}\${} was adjusted for a retrospective pattern and the adjustment is shown in red. The approximate 90\\percent lognormal confidence intervals are shown.} \def\POKUNITfigRecrCap{Estimated trends in age 1 recruitment  \(000s\)  of pollock between 1970 and 2014 from the current \(solid line\)  and previous \(dashed line\)  assessment for the assessment models base \(A\)  and flat sel sensitivity \(B\).  The approximate 90\\percent lognormal confidence intervals are shown.} \def\POKUNITfigSurvCap{Indices of abundance for pollock from the Northeast Fisheries Science Center \(NEFSC\)  spring \(1970 to 2015\)  and fall \(1970 to 2014\)  bottom trawl surveys. The approximate 90\\percent lognormal confidence intervals are shown.} \def\POKUNITPreAmb{This assessment of the pollock \(\textit{Pollachius virens}\)  stock is an update of the existing 2014 operational assessment \(Hendrickson et al. 2015\). This assessment updates commercial and recreational fishery catch data, research survey indices of abundance, the ASAP analytical models, and biological reference points through 2014. Additionally, stock projections have been updated through 2018. In what follows, there are two population assessment models brought forward from the 2014 operational assessment, the base model \(dome\-shaped survey selectivity\)  , which is used to provide management advice, and the flat sel sensitivity model \(flat\-topped survey selectivity\), which is included for the sole purpose of demonstrating the sensitivity of assessment results to survey selectivity assumptions. The most recent benchmark assessment of the pollock stock was in 2010 as part of the 50$^{th}$ Stock Assessment Review Committee \(SARC 50\; NEFSC 2010\), which includes a full description of the model formulations.} \def\POKUNITSoS{ \textbf{State of Stock: }{} The pollock \(\textit{Pollachius virens}\)  stock is not overfished and overfishing is not occurring \(Figures \ref{POKUNITSSB\_plot1}\-\ref{POKUNITF\_plot1}\){}. Retrospective adjustments were made to the model results. Retrospective adjusted spawning stock biomass \(SSB\)  in 2014 was estimated to be 154,919 \(mt\)  under the base model and 32,040 \(mt\)  under the flat sel sensitivity model which is 147 and 58\\percent \(respectively\)  of the biomass target, an  \$SSB\_{MSY}\${} proxy of SSB at  \$F\_{40\\percent}\${} \(105,226 and 54,900  \(mt\)\;  Figure \ref{POKUNITSSB\_plot1}{}\). Retrospective adjusted 2014 age 5 to 7 average fishing mortality \(F\)   was estimated to be 0.07 under the base model and 0.233 under the flat sel sensitivity model which is 25 and 92\\percent \(respectively\)  of the overfishing threshold, an  \$F\_{MSY}\${} proxy of  \$F\_{40\\percent}\${} \(0.277 and 0.252\;  Figure \ref{POKUNITF\_plot1}{}\).} \def\POKUNITProj{ \textbf{Projections: }{}Short term projections of median total fishery yield and spawning stock biomass for pollock were conducted based on a harvest scenario of fishing at an  \$F\_{MSY}\${} proxy of  \$F\_{40\\percent}\${} between 2016 and 2018. Catch in 2015 has been estimated at 5,208 \(mt\). Recruitments were sampled from a cumulative distribution function derived from ASAP estimated age 1 recruitment between 1970 and 2012.  Recruitments in 2013 and 2014 were not included due to uncertainty in those estimates. The annual fishery selectivity, natural mortality, maturity ogive, and mean weights used  in projections are the most recent 5 year averages. Retrospective adjusted age 5 to 7 average F in 2014 fell outside the 90\\percent confidence intervals of the unadjusted 2014 value under the base model \(Figure \ref{POKUNITF\_plot1}{}\). Retrospective adjusted SSB and age 5 to 7 average F in 2014 fell outside the 90\\percent confidence intervals of the unadjusted 2014 values under the flat sel sensitivity model  \(Figures \ref{POKUNITSSB\_plot1}\-\ref{POKUNITF\_plot1}\){}. Therefore, retrospective adjustments were applied in the projections for the base model and the flat sel sensitivity model.} \def\POKUNITSpecCmt{ \textbf{Special Comments: } \begin{itemize}{} \item{}What are the most important sources of uncertainty in this stock assessment?  Explain, and describe qualitatively how they affect the assessment results \(such as estimates of biomass, F, recruitment, and population projections\).  \linebreak{} \hspace\*{0.5cm} \textit{The largest source of uncertainty in the pollock assessment is selectivity, as the base model with dome\-shaped survey and fishery selectivities implies the existence of a large cryptic biomass that neither current surveys nor the fishery can confirm. Assuming flat\-topped survey selectivities leads to lower estimates of SSB and higher estimates of F  \(Figures \ref{POKUNITSSB\_plot1}\-\ref{POKUNITF\_plot1}\){}. Stock status is insensitive to the shape of the survey selectivity patterns at older ages.}  \item{} Does this assessment model have a retrospective pattern? If so, is the pattern minor, or major? \(A major retrospective pattern occurs when the adjusted SSB or  \$F\_{AVG}\${} lies outside of the approximate  joint confidence region for SSB and  \$F\_{AVG}\${}\; see  Figure \ref{RhoDecision\_tab}{}\). \linebreak{} \hspace\*{0.5cm} \textit{ The 7\-year Mohn\'s  \textrho{}, relative to SSB, was 0.291 under the base model and 0.66 under the flat sel sensitivity model in the 2014 assessment and was 0.284 and 0.789, respectively, in 2014. The 7\-year Mohn\'s  \textrho{}, relative to F, was \-0.252 under the base model and \-0.359 under the flat sel sensitivity model in the 2014 assessment and was \-0.276 and \-0.43, respectively, in 2014. There was a major retrospective pattern for the base model because the  \textrho{} adjusted estimate of 2014 F \(\$F\_{\rho}\${}=0.07\)  was outside the approximate 90\\percent confidence regions around F \(0.035 \- 0.066\). There was a major retrospective pattern for the flat sel sensitivity model because the  \textrho{} adjusted estimates of 2014 SSB \(\$SSB\_{\rho}\${}=32,040\)  and 2014 F \(\$F\_{\rho}\${}=0.233\)  were outside the approximate 90\\percent confidence regions around SSB \(37,243 \- 77,410  \(mt\)\)  and F \(0.084 \- 0.182\). A retrospective adjustment was made for both the determination of stock status and for projections of catch in 2016. The base model retrospective adjustment changed the 2014 SSB from 198,847 to 154,919 and the 2014  \$F\_{AVG}\${} from 0.051 to 0.07. The flat sel sensitivity model retrospective adjustment changed the 2014 SSB from 57,327 to 32,040 and the 2014  \$F\_{AVG}\${} from 0.133 to 0.233.}  \item{}Based on this stock assessment, are population projections well determined or uncertain? \linebreak{} \hspace\*{0.5cm} \textit{Population projections for pollock, appear to be reasonably well determined for both the base model and the flat sel sensitivity model. }  \item{}Describe any changes that were made to the current stock assessment, beyond incorporating additional years of data  and the affect these changes had on the assessment and stock status. \linebreak{} \hspace\*{0.5cm} \textit{Only one major change was made to the pollock assessment as part of this update. Likelihood constants were excluded from likelihood calculations to avoid potential bias caused by one of the recruitment likelihood constants, which is the sum of the log\-scale predicted recruitments, and therefore not a constant. Inclusion of this likelihood constant allows the assessment model to minimize the negative log likelihood by estimating lower recruitments. Exclusion of the likelihood constants led to higher estimates of SSB  and lower estimates of F  \(Figures \ref{POKUNITSSB\_plot1}\-\ref{POKUNITF\_plot1}\){}.}  \item{}If the stock status has changed a lot since the previous assessment, explain why this occurred.  \linebreak{} \hspace\*{0.5cm} \textit{Stock status based on the base model has not changed since the previous assessment. Stock status based on the flat sel sensitivity model has changed from \'overfishing is occurring\' in the previous assessment to \'overfishing is not occurring\' in the current assessment. Though, the retrospective adjusted 2014 age 5 to 7 average fishing mortality  from the flat sel sensitivity model \(0.233\)  is close to the  \$F\_{MSY}\${} proxy \(0.252\). This change in status likely is due to a decline in predicted F from 2013 to 2014, as well as to the exclusion of the likelihood constants, which led to higher predicted stock productivity.}  \item{}Indicate what data or studies are currently lacking and which would be needed most to improve this stock assessment in the future.  \linebreak{} \hspace\*{0.5cm} \textit{The pollock assessment could be improved with additional studies on gear selectivity. These studies could cover topics such as physical selectivity \(e.g., multi\-mesh gillnet\), behavior \(e.g., swimming endurance, escape behavior\), geographic and vertical distribution by size and age, tag\-recovery at size and age, and evaluating information on length\-specific selectivity at older ages.}  \item{}Are there other important issues? \linebreak{} \hspace\*{0.5cm} \textit{As in the previous assessment, the pollock assessment models had difficulty converging on a solution in some of the retrospective peels. One possible explanation for this convergence issue is that the model may be overparameterized, because the commercial and recreational fleets are modeled separately in this assessment. The possibility of combining the two fleets into a single fleet should be explored during the next benchmark assessment.} \end{itemize}{}} \def\POKUNITRefr{ \textbf{References: }{} \linebreak{}Hendrickson L, Nitschke P, Linton B. 2015. 2014 Operational stock assessments for Georges Bank winter flounder, Gulf of Maine winter flounder, and pollock. US Dept Commer, Northeast Fish Sci Cent Ref Doc. 15\-01\; 228 p. Available from: NationalMarine Fisheries Service, 166 Water Street, Woods Hole, MA 02543\-1026, or online at http:\/\/www.nefsc.noaa.gov\/publications\/ \linebreak{} \linebreak{}Northeast Fisheries Science Center. 2010. 50$^{th}$ Northeast Regional Stock Assessment Workshop \(50$^{th}$ SAW\)  Assessment Report. US Dept Commer, Northeast Fish Sci Cent Ref Doc. 10\-17\; 844 p. Available from: National Marine Fisheries Service, 166 Water Street, Woods Hole, MA 02543\-1026, or online at http:\/\/www.nefsc.noaa.gov\/nefsc\/publications\/ } \def\POKUNITDraft{} \def\POKUNITSPPname{pollock} \def\POKUNITSPPnameT{Pollock} \def\POKUNITRptYr{2015} \def\POKUNITAuthor{Brian Linton} \def\POKUNITReviewerComments{/home/dhennen/EIEIO/BigReport/POK_UNIT/latex}  \def\REDUNITMyPathTab{/home/dhennen/EIEIO/BigReport/RED_UNIT/tables} \def\REDUNITMyPathFig{/home/dhennen/EIEIO/BigReport/RED_UNIT/figures} \def\REDUNITfigFishCap{Total catch of Acadian redfish between 1913 and 2014 by fleet \(commercial and other\)  and disposition \(landings and discards\).} \def\REDUNITfigSSBCap{Trends in spawning stock biomass of Acadian redfish between 1913 and 2014 from the current  \(solid line\)  and previous \(dashed line\)  assessment and the corresponding  \$SSB\_{Threshold}\${} \(0.5 \* \$SSB\_{MSY}\${} \textit{proxy}{}\; horizontal dashed line\)  as well as  \$SSB\_{Target}\${} \(\$SSB\_{MSY}\${} \textit{proxy}{}\; horizontal dotted line\)  based on the 2015 assessment. Biomass was adjusted for a retrospective pattern and the adjustment is shown in red. The approximate 90\\percent lognormal confidence intervals are shown.} \def\REDUNITfigFCap{Trends in the fully selected fishing mortality \(\$F\_{Full}\${}\)  of Acadian redfish between 1913 and 2014 from the current \(solid line\)  and previous \(dashed line\)  assessment and the corresponding  \$F\_{Threshold}\${} \(\$F\_{MSY}\${} \textit{proxy}{}=0.038\; horizontal dashed line\)  based on the 2015 assessment.  \$F\_{Full}\${} was adjusted for a retrospective pattern and the adjustment is shown in red. The approximate 90\\percent lognormal confidence intervals are shown.} \def\REDUNITfigRecrCap{Trends in Recruits \(age 1\)  \(000s\)  of Acadian redfish between 1913 and 2014 from the current \(solid line\)  and previous \(dashed line\)  assessment. The approximate 90\\percent lognormal confidence intervals are shown.} \def\REDUNITfigSurvCap{Indices of abundance for Acadian redfish from the Northeast Fisheries Science Center \(NEFSC\)  spring \(1963 to 2015\)  and fall \(1963 to 2014\)  bottom trawl surveys. The approximate 90\\percent lognormal confidence intervals are shown.} \def\REDUNITPreAmb{This assessment of the Acadian redfish \(\textit{Sebastes fasciatus}\)  stock is an update of the existing 2012 operational assessment \(NEFSC 2012\). This assessment updates commercial fishery catch data, research survey indices of abundance, the ASAP analytical model, and biological reference points through 2014. Additionally, stock projections have been updated through 2018. The most recent benchmark assessment of the Acadian redfish stock was in 2008 as part of the 3$^{rd}$ Groundfish Assessment Review Meeting \(GARM III\; NEFSC 2008\), which includes a full description of the model formulations.} \def\REDUNITSoS{ \textbf{State of Stock: }{}Based on this updated assessment, the Acadian redfish \(\textit{Sebastes fasciatus}\)  stock is not overfished and overfishing is not occurring \(Figures \ref{REDUNITSSB\_plot1}\-\ref{REDUNITF\_plot1}\){}. Retrospective adjustments were made to the model results. Retrospective adjusted spawning stock biomass \(SSB\)  in 2014 was estimated to be 330,004 \(mt\)  which is 117\\percent of the biomass target \(\$SSB\_{MSY}\${} \textit{proxy}{} of SSB at  \$F\_{50\\percent}\${} = 281,112\;  Figure \ref{REDUNITSSB\_plot1}{}\).  The retrospective adjusted 2014 fully selected fishing mortality \(F\)  was estimated to be 0.015 which is 39\\percent of the overfishing threshold \(\$F\_{MSY}\${} \textit{proxy}{} of  \$F\_{50\\percent}\${} = 0.038\;  Figure \ref{REDUNITF\_plot1}{}\).} \def\REDUNITProj{ \textbf{Projections: }{}Short term projections of median total fishery yield and spawning stock biomass for Acadian redfish were conducted based on a harvest scenario of fishing at the  \$F\_{MSY}\${} \textit{proxy}{} between 2016 and 2018. Catch in 2015 has been estimated at 5,204 \(mt\). Recruitments were sampled from a cumulative distribution function derived from ASAP estimated age 1 recruitment between 1969 and 2014. The annual fishery selectivity, natural mortality, maturity ogive, and mean weights used  in projections are the same as those used in the assessment model. Retrospective adjusted SSB and fully selected F in 2014 fell outside the 90\\percent confidence intervals of the unadjusted 2014 values. Therefore, retrospective adjustments were applied in the projections. } \def\REDUNITSpecCmt{ \textbf{Special Comments: } \begin{itemize}{} \item{}What are the most important sources of uncertainty in this stock assessment?  Explain, and describe qualitatively how they affect the assessment results \(such as estimates of biomass, F, recruitment, and population projections\).  \linebreak{} \hspace\*{0.5cm} \textit{The largest source of uncertainty in the Acadian redfish assessment is the lack of age data, particularly from the commercial fishery. Age measurements from landings halted after 1985, due to relatively low landings. Current landings have increased to levels seen in the mid\-1980s. If landings continue to increase, then age data from the fishery will become increasingly important. Dimorphic growth is another source of uncertainty in this assessment, with females growing faster than males. The use of female weights at age in the stock projections may lead to overestimation of stock productivity, as well as having an unknown effect on biological reference points.}  \item{} Does this assessment model have a retrospective pattern? If so, is the pattern minor, or major? \(A major retrospective pattern occurs when the adjusted SSB or  \$F\_{Full}\${} lies outside of the approximate  joint confidence region for SSB and  \$F\_{Full}\${}\; see  Figure \ref{RhoDecision\_tab}{}\). \linebreak{} \hspace\*{0.5cm} \textit{ The 7\-year Mohn\'s  \textrho{}, relative to SSB, was 0.036 in the 2012 assessment and was 0.256 in 2014. The 7\-year Mohn\'s  \textrho{}, relative to F, was \-0.035 in the 2012 assessment and was \-0.190 in 2014. There was a major retrospective pattern for this assessment because the  \textrho{} adjusted estimates of 2014 SSB \(\$SSB\_{\rho}\${}=330,004\)  and 2014 F \(\$F\_{\rho}\${}=0.015\)  were outside the approximate 90\\percent confidence regions around SSB \(368,906 \- 465,828\)  and F \(0.011 \- 0.014\).  A retrospective  adjustment was made for both the determination of stock status and for projections of catch in 2016. The retrospective adjustment changed the 2014 SSB from 414,544 to 330,004 and the 2014  \$F\_{Full}\${} from 0.012 to 0.015.}  \item{}Based on this stock assessment, are population projections well determined or uncertain? \linebreak{} \hspace\*{0.5cm} \textit{Population projections for Acadian redfish appear to be reasonably well determined. }  \item{}Describe any changes that were made to the current stock assessment, beyond incorporating additional years of data  and the affect these changes had on the assessment and stock status. \linebreak{} \hspace\*{0.5cm} \textit{Only one major change was made to the Acadian redfish assessment as part of this update. Likelihood constants were excluded from likelihood calculations to avoid potential bias caused by one of the recruitment likelihood constants, which is the sum of the log\-scale predicted recruitments, and therefore not a constant. Inclusion of this likelihood constant allows the assessment model to minimize the negative log likelihood by estimating lower recruitments. Exclusion of the likelihood constants led to slightly higher estimates of SSB in recent years. }  \item{}If the stock status has changed a lot since the previous assessment, explain why this occurred.  \linebreak{} \hspace\*{0.5cm} \textit{There has been no change in the stock status of Acadian redfish since the previous assessment.}  \item{}Indicate what data or studies are currently lacking and which would be needed most to improve this stock assessment in the future.  \linebreak{} \hspace\*{0.5cm} \textit{The Acadian redfish assessment could be improved by 1\)  including additional age data, particularly from the commercial fishery, and 2\)  investigating the sensitivity of biological reference points and stock projections to the mean weights at age. }  \item{}Are there other important issues? \linebreak{} \hspace\*{0.5cm} \textit{Northeast Fisheries Science Center \(NEFSC\)  fall bottom trawl index values for 2013 and 2014 are lower than in previous years \(Figure \ref{REDUNITSurv\_plot1}{}\), but the current assessment model continues to predict an increase in SSB for the last two years \(Figure \ref{REDUNITSSB\_plot1}{}\). If future index values remain low \(i.e., if the index is responding to a change in abundance, rather than interannual variability\), then the predicted trend in SSB may change abruptly in a future assessment. Such an abrupt change may lead to an increase in the retrospective pattern.} \end{itemize}{}} \def\REDUNITRefr{ \textbf{References: }{} \linebreak{}Northeast Fisheries Science Center. 2008. Assessment of 19 Northeast Groundfish Stocks through 2007: Report of the 3$^{rd}$ Groundfish Assessment Review Meeting \(GARM III\), Northeast Fisheries Science Center, Woods Hole, Massachusetts, August 4\-8, 2008. US Dept Commer, Northeast Fish Sci Cent Ref Doc. 08\-15\; 884 p + xvii. Available from: National Marine Fisheries Service, 166 Water Street, Woods Hole, MA 02543\-1026, or online at http:\/\/www.nefsc.noaa.gov\/nefsc\/publications\/ \linebreak{} \linebreak{}Northeast Fisheries Science Center. 2012. Assessment or Data Updates of 13 Northeast Groundfish Stocks through 2010. US Dept Commer, Northeast Fish Sci Cent Ref Doc. 12\-06\; 789 p. Available from: National Marine Fisheries Service, 166 Water Street, Woods Hole, MA 02543\-1026, or online at http:\/\/www.nefsc.noaa.gov\/nefsc\/publications\/} \def\REDUNITDraft{} \def\REDUNITSPPname{Acadian redfish} \def\REDUNITSPPnameT{Acadian redfish} \def\REDUNITRptYr{2015} \def\REDUNITAuthor{Brian Linton} \def\REDUNITReviewerComments{/home/dhennen/EIEIO/BigReport/RED_UNIT/latex}  \def\CATUNITMyPathTab{/home/dhennen/EIEIO/BigReport/CAT_UNIT/tables} \def\CATUNITMyPathFig{/home/dhennen/EIEIO/BigReport/CAT_UNIT/figures} \def\CATUNITfigFishCap{Total catch of Atlantic wolffish between 1968 and 2014 by fleet \(commercial and recreational\)  and disposition \(landings and discards\). Note that a no possession limit was put in place in May 2010.} \def\CATUNITfigSSBCap{Trends in spawning stock biomass of Atlantic wolffish between 1968 and 2014 from the current  \(solid line\)  and previous \(dashed line\)  assessment and the corresponding  \$SSB\_{Threshold}\${} \(\$\dfrac{1}{2}\${} \$SSB\_{MSY}\${} \textit{proxy}{}\; horizontal dashed line\)  as well as  \$SSB\_{Target}\${} \(\$SSB\_{MSY}\${} \textit{proxy}{}\; horizontal dotted line\)   based on the current assessment.} \def\CATUNITfigFCap{Trends in the fully selected fishing mortality \(\$F\_{Full}\${}\)  of Atlantic wolffish between 1968 and 2014 from the current  \(solid line\)  and previous \(dashed line\)  assessment and the corresponding  \$F\_{Threshold}\${} \(\$F\_{MSY}\${} \textit{proxy}{}=0.243\; horizontal dashed line\). } \def\CATUNITfigRecrCap{Trends in age 1 recruits of Atlantic wolffish between 1968 and 2014 from the current \(solid line\)  and previous \(dashed line\)  assessment.} \def\CATUNITfigSurvCap{Indices of biomass for the Atlantic wolffish between 1968 and 2015 for the Northeast Fisheries Science Center \(NEFSC\)  spring and fall bottom trawl surveys, and the Massachusetts Division of Marine Fisheries \(MADMF\)  spring bottom trawl survey. The approximate 90\\percent lognormal confidence intervals are shown. NEFSC indices for 2009\-2015 are calibrated using the ocean pout coefficient from Miller et al. \(2010\).} \def\CATUNITPreAmb{This assessment of the Atlantic wolffish \(\textit{Anarhichas lupus}\)  stock is an update of the existing 2012 operational assessment \(NEFSC 2012\). Based on the previous assessment the stock was overfished, but overfishing was not occurring. This assessment updates commercial fishery catch data, research survey indices of abundance, and the analytical assessment models and reference points through 2014.} \def\CATUNITSoS{ \textbf{State of Stock: }{}Based on this updated assessment, the Atlantic wolffish \(\textit{Anarhichas lupus}\)  stock is overfished and overfishing is not occurring \(Figures \ref{CATUNITSSB\_plot1}\-\ref{CATUNITF\_plot1}\){}. Retrospective adjustments were not made to the model results. Spawning stock biomass \(SSB\)  in 2014 was estimated to be 638 \(mt\)  which is 38\\percent of the biomass target \(\$SSB\_{MSY}\${} \textit{proxy}{} = 1,663\;  Figure \ref{CATUNITSSB\_plot1}{}\).  The 2014 fully selected fishing mortality was estimated to be 0.003 which is 1\\percent of the overfishing threshold proxy \(\$F\_{MSY}\${} \textit{proxy}{} = 0.243\;  Figure \ref{CATUNITF\_plot1}{}\).} \def\CATUNITProj{} \def\CATUNITSpecCmt{ \textbf{Special Comments: } \begin{itemize}{} \item{}What are the most important sources of uncertainty in this stock assessment?  Explain, and describe qualitatively how they affect the assessment results \(such as estimates of biomass, F, recruitment, and population projections\).  \linebreak{} \hspace\*{0.5cm} \textit{The primary sources of uncertainty are the use of the ocean pout calibration coefficient, and the change to a no possession limit in May 2010. The ocean pout calibration coefficient \(4.575\)  is one of the largest for any species \(Miller et al. 2010\), and results in lower biomass estimates. The change to a no possession limit places greater importance on discard mortality. Additionally, it is unclear whether the lack of a recruitment index since 2004 is due to an actual decrease in recruitment, or a change in catchability resulting from the increase in liner mesh size associated with the switch to the Bigelow. Other sources of uncertainty were identified in previous Atlantic wolffish assessments \(NDPSWG 2009, NEFSC 2012\): the surveys may have reached the limit of wolffish detectability due to the decline in abundance\; and the lack of commercial length information results in model estimation difficulties for fishery selectivity.}  \item{} Does this assessment model have a retrospective pattern? If so, is the pattern minor, or major? \(A major retrospective pattern occurs when the adjusted SSB or  \$F\_{Full}\${} lies outside of the approximate  joint confidence region for SSB and  \$F\_{Full}\${}\; see  Figure \ref{RhoDecision\_tab}{}\). \linebreak{} \hspace\*{0.5cm} \textit{This assessment has retrospective patterns with Mohn\'s rho = 0.83 for SSB and \-0.36 for F. Confidence intervals are not available because MCMC is not fully developed for the SCALE model. Thus, retrospective adjustments were not done for this assessment.}  \item{}Based on this stock assessment, are population projections well determined or uncertain? \linebreak{} \hspace\*{0.5cm} \textit{Population projections for Atlantic wolffish were not done. Due to the uncertainties in the assessment, the Northeast Data Poor Stocks Working Group \(NDPSWG 2009\)  concluded that stock projections would be unreliable and should not be conducted.}  \item{}Describe any changes that were made to the current stock assessment, beyond incorporating additional years of data  and the affect these changes had on the assessment and stock status. \linebreak{} \hspace\*{0.5cm} \textit{Commercial discards for the entire time series were revised assuming 8\\percent discard mortality based on a recent study by Grant and Hiscock \(2014\). A sensitivity run with the revised discard estimates was presented to the Peer Review Panel during the 2015 Operational Assessments. This became the accepted run. There was no change in stock status resulting from the adoption of the 8\\percent discard mortality run. \linebreak{} \hspace\*{0.5cm}Recreational landings for the entire time series were revised due to an updated grand mean, and the MRFSS\/MRIP calibration for 1981\-2003. This had a negligible effect on the assessment, and there was no change in stock status.}  \item{}If the stock status has changed a lot since the previous assessment, explain why this occurred.  \linebreak{} \hspace\*{0.5cm} \textit{Stock status has not changed since the previous assessment.}  \item{}Indicate what data or studies are currently lacking and which would be needed most to improve this stock assessment in the future.  \linebreak{} \hspace\*{0.5cm} \textit{The Atlantic wolffish maturity study in the Gulf of Maine is ongoing. Increased sample size since the previous assessment allowed the use of a revised knife edge maturity of 50 cm in this assessment. Continued histological sampling over the next several years should allow for the development of a definitive maturity ogive that can be used in the next assessment.}  \item{}Are there other important issues? \linebreak{} \hspace\*{0.5cm} \textit{Recruitment at the end of the time series increases toward the initial recruitment estimate \(Table 1\; Figure 3\)  because there is no information in the model to inform these estimates. There is no indication in the data that recruitment has increased recently.  \linebreak{} \hspace\*{0.5cm}Approximate 90\\percent lognormal confidence intervals are not shown in Figures 1\-3 because MCMC is not fully developed for the SCALE model.} \end{itemize}{}} \def\CATUNITRefr{ \textbf{References: }{} \linebreak{} \linebreak{}Grant SM, Hiscock W. 2014. Post\-capture survival of Atlantic wolffish  \(\textit{Anarhichas lupus}\)  captured by bottom otter trawl: Can live release programs contribute to the recovery of species at risk? Fish Res 151:169\-176 \linebreak{} \linebreak{}Miller TJ, Das C, Politis PJ, Miller AS, Lucey SM, Legault CM, Brown RW, Rago PJ. 2010. Estimation of Albatross IV to Henry B. Bigelow calibration factors. US Dep Commer, Northeast Fish Sci Cent Ref Doc. 10\-05\; 233 p. http:\/\/www.nefsc.noaa.gov\/publications\/crd\/crd1005\/ \linebreak{} \linebreak{}Northeast Fisheries Science Center \(NEFSC\). 2012. Assessment or Data Updates of 13 Northeast Groundfish Stocks through 2010. US Dep Commer, Northeast Fish Sci Cent Ref Doc. 12\-06\; 789 p. http:\/\/www.nefsc.noaa.gov\/publications\/crd\/crd1206\/ \linebreak{} \linebreak{}Northeast Data Poor Stocks Working Group \(NDPSWG\). 2009. The Northeast Data Poor Stocks Working Group Report, December 8\-12, 2008 Meeting. Part A. Skate species complex, deep sea red crab, Atlantic wolffish, scup, and black sea bass. US Dept Commer, Northeast Fish Sci Cent Ref Doc. 09\-02\; 496 p. http:\/\/www.nefsc.noaa.gov\/publications\/crd\/crd0902\/ \linebreak{} \linebreak{}} \def\CATUNITDraft{} \def\CATUNITSPPname{Atlantic wolffish} \def\CATUNITSPPnameT{Atlantic wolffish} \def\CATUNITRptYr{2015} \def\CATUNITAuthor{Charles Adams} \def\CATUNITReviewerComments{/home/dhennen/EIEIO/BigReport/CAT_UNIT/latex} 
\input{./preamble}


\begin{document}

\pagenumbering{roman}
\setcounter{page}{1}  %beginning page number

%--------Here are the Ref. Doc coverpages -------

\input{crd_front_page}
\input{crd_second_page}
\input{crd_third_page}

%-------------- Table of Contents ----------------
%\section*{Table of Contents}
\tableofcontents
\clearpage
\pagenumbering{arabic}

%\input{def2.tex}
%%%%%%%%%%%%%%%%%%%%%%%%%%%%%%%%%%%%%%%%%%%%%%%%%%%%%%%%%%%%%%%%%%%%%%%%%%%%%%%%%%%%%%%%%%%%%%%%%%%%%%%%%%%%
%Executive Summary
%\newcommand{\ExSumPath}{/net/home2/dhennen/testEIEIO/BigReport/ExSum}
\newcommand{\ExSumPath}{../ExSum}
\input{\ExSumPath/latex/ExecSumm}
%%%%%%%%%%%%%%%%%%%%%%%%%%%%%%%%%%%%%%%%%%%%%%%%%%%%%%%%%%%%%%%%%%%%%%%%%%%%%%%%%%%%%%%%%%%%%%%%%%%%%%%%%%%%
\input{BigCall5.tex}
%\input{BigCall.tex}
%\input{BigCall2.tex}
%\input{BigCall3.tex}
%\input{BigCall4.tex}
%%%%%%%%%%%%%%%%%%%%%%%%%%%%%%%%%%%%%%%%%%%%%%%%%%%%%%%%%%%%%%%%%%%%%%%%%%%%%%%%%%%%%%%%%%%%%%%%%%%%%%%%%%%%
%GOMCOD
\input{\CODGMReviewerComments/BigReportCODGM}
%%%%%%%%%%%%%%%%%%%%%%%%%%%%%%%%%%%%%%%%%%%%%%%%%%%%%%%%%%%%%%%%%%%%%%%%%%%%%%%%%%%%%%%%%%%%%%%%%%%%%%%%%%%%
%GBCOD
\input{\CODGBReviewerComments/BigReportCODGB}
%%%%%%%%%%%%%%%%%%%%%%%%%%%%%%%%%%%%%%%%%%%%%%%%%%%%%%%%%%%%%%%%%%%%%%%%%%%%%%%%%%%%%%%%%%%%%%%%%%%%%%%%%%%%
%GBHAD
\input{\HADGBReviewerComments/BigReportHADGB}
%%%%%%%%%%%%%%%%%%%%%%%%%%%%%%%%%%%%%%%%%%%%%%%%%%%%%%%%%%%%%%%%%%%%%%%%%%%%%%%%%%%%%%%%%%%%%%%%%%%%%%%%%%%%
%GOMHAD
\input{\HADGMReviewerComments/BigReportHADGM}
%%%%%%%%%%%%%%%%%%%%%%%%%%%%%%%%%%%%%%%%%%%%%%%%%%%%%%%%%%%%%%%%%%%%%%%%%%%%%%%%%%%%%%%%%%%%%%%%%%%%%%%%%%%%
%CCGMYEL
\input{\YELCCGMReviewerComments/BigReportYELCCGM}
%%%%%%%%%%%%%%%%%%%%%%%%%%%%%%%%%%%%%%%%%%%%%%%%%%%%%%%%%%%%%%%%%%%%%%%%%%%%%%%%%%%%%%%%%%%%%%%%%%%%%%%%%%%%
%SNEMAYEL
\input{\YELSNEMAReviewerComments/BigReportYELSNEMA}
%%%%%%%%%%%%%%%%%%%%%%%%%%%%%%%%%%%%%%%%%%%%%%%%%%%%%%%%%%%%%%%%%%%%%%%%%%%%%%%%%%%%%%%%%%%%%%%%%%%%%%%%%%%%
%FLWGB
\input{\FLWGBReviewerComments/BigReportFLWGB}
%%%%%%%%%%%%%%%%%%%%%%%%%%%%%%%%%%%%%%%%%%%%%%%%%%%%%%%%%%%%%%%%%%%%%%%%%%%%%%%%%%%%%%%%%%%%%%%%%%%%%%%%%%%%
%FLWSNEMA
\input{\FLWSNEMAReviewerComments/BigReportFLWSNEMA}
%%%%%%%%%%%%%%%%%%%%%%%%%%%%%%%%%%%%%%%%%%%%%%%%%%%%%%%%%%%%%%%%%%%%%%%%%%%%%%%%%%%%%%%%%%%%%%%%%%%%%%%%%%%%
%PLAUNIT
\input{\PLAUNITReviewerComments/BigReportPLAUNIT}
%%%%%%%%%%%%%%%%%%%%%%%%%%%%%%%%%%%%%%%%%%%%%%%%%%%%%%%%%%%%%%%%%%%%%%%%%%%%%%%%%%%%%%%%%%%%%%%%%%%%%%%%%%%%
%WITUNIT
\input{\WITUNITReviewerComments/BigReportWITUNIT}
%%%%%%%%%%%%%%%%%%%%%%%%%%%%%%%%%%%%%%%%%%%%%%%%%%%%%%%%%%%%%%%%%%%%%%%%%%%%%%%%%%%%%%%%%%%%%%%%%%%%%%%%%%%%
%REDUNIT
\input{\REDUNITReviewerComments/BigReportREDUNIT}
%%%%%%%%%%%%%%%%%%%%%%%%%%%%%%%%%%%%%%%%%%%%%%%%%%%%%%%%%%%%%%%%%%%%%%%%%%%%%%%%%%%%%%%%%%%%%%%%%%%%%%%%%%%%
%HKWUNIT
\input{\HKWUNITReviewerComments/BigReportHKWUNIT}
%%%%%%%%%%%%%%%%%%%%%%%%%%%%%%%%%%%%%%%%%%%%%%%%%%%%%%%%%%%%%%%%%%%%%%%%%%%%%%%%%%%%%%%%%%%%%%%%%%%%%%%%%%%%
%POKUNIT
\input{\POKUNITReviewerComments/BigReportPOKUNIT}
%%%%%%%%%%%%%%%%%%%%%%%%%%%%%%%%%%%%%%%%%%%%%%%%%%%%%%%%%%%%%%%%%%%%%%%%%%%%%%%%%%%%%%%%%%%%%%%%%%%%%%%%%%%%
%CATUNIT
\input{\CATUNITReviewerComments/BigReportCATUNIT}
%%%%%%%%%%%%%%%%%%%%%%%%%%%%%%%%%%%%%%%%%%%%%%%%%%%%%%%%%%%%%%%%%%%%%%%%%%%%%%%%%%%%%%%%%%%%%%%%%%%%%%%%%%%%
%HALUNIT
\input{\HALUNITReviewerComments/BigReportHALUNIT}
%%%%%%%%%%%%%%%%%%%%%%%%%%%%%%%%%%%%%%%%%%%%%%%%%%%%%%%%%%%%%%%%%%%%%%%%%%%%%%%%%%%%%%%%%%%%%%%%%%%%%%%%%%%%
%FLDGMGB
\input{\FLDGMGBReviewerComments/BigReportFLDGMGB}
%%%%%%%%%%%%%%%%%%%%%%%%%%%%%%%%%%%%%%%%%%%%%%%%%%%%%%%%%%%%%%%%%%%%%%%%%%%%%%%%%%%%%%%%%%%%%%%%%%%%%%%%%%%%
%FLDSNEMA
\input{\FLDSNEMAReviewerComments/BigReportFLDSNEMA}
%%%%%%%%%%%%%%%%%%%%%%%%%%%%%%%%%%%%%%%%%%%%%%%%%%%%%%%%%%%%%%%%%%%%%%%%%%%%%%%%%%%%%%%%%%%%%%%%%%%%%%%%%%%%
%OPTUNIT
\input{\OPTUNITReviewerComments/BigReportOPTUNIT}
%%%%%%%%%%%%%%%%%%%%%%%%%%%%%%%%%%%%%%%%%%%%%%%%%%%%%%%%%%%%%%%%%%%%%%%%%%%%%%%%%%%%%%%%%%%%%%%%%%%%%%%%%%%%
%FLWGM
\input{\FLWGMReviewerComments/BigReportFLWGM}
%%%%%%%%%%%%%%%%%%%%%%%%%%%%%%%%%%%%%%%%%%%%%%%%%%%%%%%%%%%%%%%%%%%%%%%%%%%%%%%%%%%%%%%%%%%%%%%%%%%%%%%%%%%%
%GBYEL
\input{\YELGBReviewerComments/BigReportYELGB}








\end{document}; ls -al;
%\def\FLWGMMyPathTab{/home/dhennen/EIEIO/BigReport/FLW_GM/tables} \def\FLWGMMyPathFig{/home/dhennen/EIEIO/BigReport/FLW_GM/figures} \def\FLWGMfigFishCap{Total catch of Gulf of Maine Winter Flounder between 2009 and 2014 by fleet \(commercial and recreational\)  and disposition \(landings and discards\). A 15\% mortality rate is assumed on recreational discards and a 50\% mortality rate on commercial discards.} \def\FLWGMfigSSBCap{Trends in 30+ cm area\-swept biomass of Gulf of Maine Winter Flounder between 2009 and 2014 from the current assessment based on the fall \(MENH, MDMF, NEFSC\)  surveys.  The approximate 90\% lognormal confidence intervals are shown.} \def\FLWGMfigFCap{Trends in the exploitation rates \(\$E\_{Full}\${}\)  of Gulf of Maine Winter Flounder between 2009 and 2014 from the current assessment and the corresponding  \$F\_{Threshold}\${} \(\$E\_{MSY}\${} \textit{proxy}{}=0.23\; horizontal dashed line\).  The approximate 90\% lognormal confidence intervals are shown.} \def\FLWGMfigRecrCap{} \def\FLWGMfigSurvCap{Indices of biomass for the Gulf of Maine Winter Flounder between 1978 and 2015 for the Northeast Fisheries Science Center \(NEFSC\), Massachusetts Division of Marine Fisheries \(MDMF\), and the Maine New Hampshire \(MENH\)  spring and fall bottom trawl surveys. NEFSC indices are calculated with gear and vessel conversion factors where appropriate.  The approximate 90\% lognormal confidence intervals are shown.} \def\FLWGMPreAmb{This assessment of the  Gulf of Maine Winter Flounder  \(\textit{Pseudopleuronectes americanus}\)   stock is an operational update of the  existing  2014  operational update area\-swept assessment \(NEFSC 2014\).  Based on the previous assessment the biomass status is unknown but overfishing was not occurring.  This assessment  updates commercial and recreational fishery catch data, research survey indices  of abundance, and the area\-swept estimates of 30+ cm biomass based on the fall NEFSC, MDMF, and MENH surveys.} \def\FLWGMSoS{ \textbf{State of Stock: }{}Based on this updated assessment, the Gulf of Maine Winter Flounder \(\textit{Pseudopleuronectes americanus}\)  stock biomass status is unknown and overfishing is not occurring \(Figures \ref{FLWGMSSB\_plot1}\-\ref{FLWGMF\_plot1}\){}. Retrospective adjustments were not made to the model results.  Biomass  \(30+ cm mt\)  in 2014 was estimated to be 4,655 mt \(Figure \ref{FLWGMSSB\_plot1}{}\). The 2014 30+ cm exploitation rate was estimated to be 0.06 which is 26\% of the overfishing exploitation threshold proxy \(\$E\_{MSY}\${} \textit{proxy}{} = 0.23\;  Figure \ref{FLWGMF\_plot1}{}\).} \def\FLWGMProj{ \textbf{Projections: }{}Projections are not possible with area\-swept based assessments. Catch advice was based on 75\% of  \$E\_{40\%}\${}\(75\% \$E\_{MSY}\${} \textit{proxy}{}\)  using the fall area\-swept estimate assuming q=0.6 on the wing spread. Updated 2014 fall 30+ cm area\-swept biomass \(4,655 mt\)  implies an OFL of 1,080 mt based on the  \$E\_{MSY}\${} \textit{proxy}{} and a catch of 810 mt for 75\% of the  \$E\_{MSY}\${} \textit{proxy}{}.} \def\FLWGMSpecCmt{ \textbf{Special Comments: } \begin{itemize}{} \item{}What are the most important sources of uncertainty in this stock assessment?  Explain, and describe qualitatively how they affect the assessment results \(such as estimates of biomass, F, recruitment, and population projections\).  \linebreak{} \hspace\*{0.5cm} \textit{The largest source of uncertainty with the direct estimates of stock biomass from survey area\-swept estimates originate from the assumption of survey gear catchability \(q\). Biomass and exploitation rate estimates are sensitive to the survey q assumption \(0.6 on wing spread\). The 2014 empirical benchmark assessement of Georges bank yellowtail flounder based the area\-swept q assumption on an average value taken from the literature for west coast flatfish \(0.37 on door spread\). The yellowtail q assumption corresponds to a value close to 1 on the wing spread which would result in a lower estimate of biomass \(2,995 mt\). Another major source of uncertainty with this method is that biomass based reference points cannot be determined and overfished status is unknown. }  \item{} Does this assessment model have a retrospective pattern? If so, is the pattern minor, or major? \(A major retrospective pattern occurs when the adjusted SSB or  \$F\_{Full}\${} lies outside of the approximate  joint confidence region for SSB and  \$F\_{Full}\${}\; see  Figure \ref{RhoDecision\_tab}{}\). \linebreak{} \hspace\*{0.5cm} \textit{ The model used to determine status of this stock does not allow estimation of a retrospective pattern.  An analytical stock assessment model does not exist for Gulf of Maine Winter Flounder.  An analytical model was no longer used for stock status determination at SARC 52 \(2011\)  due to concerns with a strong retrospective pattern.  Models have difficulty with the apparent lack of a relationship between a large decrease in the catch with little change in the indices and age and\/or size structure over time. }  \item{}Based on this stock assessment, are population projections well determined or uncertain? \linebreak{} \hspace\*{0.5cm} \textit{Population projections for Gulf of Maine Winter Flounder, do not exist for area\-swept assessments. Catch advice from area\-swept estimates tend to vary with interannual variability in the surveys.}  \item{}Describe any changes that were made to the current stock assessment, beyond incorporating additional years of data  and the affect these changes had on the assessment and stock status. \linebreak{} \hspace\*{0.5cm} \textit{ No changes, other than the incorporation of new data were made to the Gulf of Maine Winter Flounder assessment for this update. However, stabilizing the catch advice may be desired and could be obtained through the averaging of the area\-swept fall and spring survey estimates.}  \item{}If the stock status has changed a lot since the previous assessment, explain why this occurred.  \linebreak{} \hspace\*{0.5cm} \textit{The overfishing status of Gulf of Maine Winter Flounder has not changed. }  \item{}Indicate what data or studies are currently lacking and which would be needed most to improve this stock assessment in the future.  \linebreak{} \hspace\*{0.5cm} \textit{Direct area\-swept assessment could be improved with additional studies on survey gear efficiency.  Quantifying the degree of herding between the doors and escapement under the footrope and\/or above the headrope for each survey is needed since area\-swept biomass estimates and catch advice are sensitive to the assumed catchability.}  \item{}Are there other important issues? \linebreak{} \hspace\*{0.5cm} \textit{The general lack of a response in survey indices and age\/size structure is the primary source of concern with catches remaining far below the overfishing level. } \end{itemize}{}} \def\FLWGMRefr{ \textbf{References: }{} \linebreak{}Hendrickson L, Nitschke P, Linton B. 2015. 2014 Operational Stock Assessments for Georges Bank winter flounder, Gulf of Maine winter flounder, and pollock. US Dept Commer, Northeast Fish Sci Cent Ref Doc. 15\-01\; 228 p. Available from: National Marine Fisheries Service, 166 Water Street, Woods Hole, MA 02543\-1026, or online at http:\/\/nefsc.noaa.gov\/publications\/ \linebreak{} \linebreak{}Northeast Fisheries Science Center. 2011. 52$^{nd}$ Northeast Regional Stock AssessmentWorkshop \(52$^{nd}$ SAW\)  Assessment Report. US Dept Commer, Northeast Fish SciCent Ref Doc. 11\-17\; 962 p. Available from: National Marine Fisheries Service, 166 Water Street, Woods Hole, MA 02543\-1026, or online at http:\/\/www.nefsc.noaa.gov\/nefsc\/publications\/ \linebreak{} \linebreak{}} \def\FLWGMDraft{} \def\FLWGMSPPname{Gulf of Maine Winter Flounder} \def\FLWGMSPPnameT{Gulf of Maine Winter Flounder} \def\FLWGMRptYr{2015} \def\FLWGMAuthor{Paul Nitschke} \def\FLWGMReviewerComments{/home/dhennen/EIEIO/BigReport/FLW_GM/latex}  \def\FLWSNEMAMyPathTab{/home/dhennen/EIEIO/BigReport/FLW_SNEMA/tables} \def\FLWSNEMAMyPathFig{/home/dhennen/EIEIO/BigReport/FLW_SNEMA/figures} \def\FLWSNEMAfigFishCap{Total catch of Southern New England Mid\-Atlantic Winter Flounder between 1981 and 2014 by fleet \(commercial, recreational\)  and disposition \(landings and discards\).} \def\FLWSNEMAfigSSBCap{Trends in spawning stock biomass of Southern New England Mid\-Atlantic Winter Flounder between 1981 and 2014 from the current  \(solid line\)  and previous \(dashed line\)  assessment and the corresponding  \$SSB\_{Threshold}\${} \(\$\dfrac{1}{2}\${} \$SSB\_{MSY}\${} \textit{proxy}{}\; horizontal dashed line\)  as well as  \$SSB\_{Target}\${} \(\$SSB\_{MSY}\${} \textit{proxy}{}\; horizontal dotted line\)   based on the 2015 assessment. The approximate 90\% lognormal confidence intervals are shown.} \def\FLWSNEMAfigFCap{Trends in the fully selected fishing mortality \(\$F\_{Full}\${}\)  of Southern New England Mid\-Atlantic Winter Flounder between 1981 and 2014 from the current  \(solid line\)  and previous \(dashed line\)  assessment and the corresponding  \$F\_{Threshold}\${} \(\$F\_{MSY}\${}=0.325\; horizontal dashed line\)   based on the 2015 assessment. The approximate 90\% lognormal confidence intervals are shown.} \def\FLWSNEMAfigRecrCap{Trends in Recruits \(age 1\)  \(000s\)  of Southern New England Mid\-Atlantic Winter Flounder between 1981 and 2014 from the current \(solid line\)  and previous \(dashed line\)  assessment. The approximate 90\% lognormal confidence intervals are shown.} \def\FLWSNEMAfigSurvCap{Indices of biomass for the Southern New England Mid\-Atlantic Winter Flounder between 1963 and 2014 for the Northeast Fisheries Science Center \(NEFSC\)  spring and fall bottom trawl surveys, the MADMF spring survey, and the CT LISTS survey  The approximate 90\% lognormal confidence intervals are shown.} \def\FLWSNEMAPreAmb{This assessment of the Southern New England Mid\-Atlantic Winter Flounder \(\textit{Pseudopleuronectes americanus}\)  stock is an operational update of the existing 2011 benchmark ASAP assessment \(NEFSC 2011\). Based on the previous assessment the stock was overfished, but overfishing was not ocurring. This assessment updates commercial fishery catch data, recreational fishery catch data, and research survey indices of abundance, and the analytical ASAP assessment models and reference points through 2014. Additionally, stock projections have been updated through 2018} \def\FLWSNEMASoS{ \textbf{State of Stock: }{}Based on this updated assessment, the Southern New England Mid\-Atlantic Winter Flounder \(\textit{Pseudopleuronectes americanus}\)  stock is overfished but overfishing is not occurring \(Figures \ref{FLWSNEMASSB\_plot1}\-\ref{FLWSNEMAF\_plot1}\){}. Spawning stock biomass \(SSB\)  in 2014 was estimated to be 6,151 \(mt\)  which is 23\% of the biomass target \(26,928 mt\), and 23\% of the biomass threshold for an overfished stock \(\$SSB\_{Threshold}\${} = 13464 \(mt\)\;  Figure \ref{FLWSNEMASSB\_plot1}{}\).  The 2014 fully selected fishing mortality was estimated to be 0.16 which is 49\% of the overfishing threshold \(\$F\_{MSY}\${} = 0.325\;  Figure \ref{FLWSNEMAF\_plot1}{}\). Retrospective adjustments were not made to the model results. } \def\FLWSNEMAProj{ \textbf{Projections: }{}Short term projections of biomass were derived by sampling from a cumulative  distribution  function of recruitment estimates assuming a Beverton\-Holt stock recruitment relationship. The annual fishery selectivity, maturity ogive, and mean weights at age used  in projection  are the most recent 5 year averages\;  The model exhibited minor retrospective pattern in F and SSB so no retrospective adjustments were applied in the projections.} \def\FLWSNEMASpecCmt{ \textbf{Special Comments: } \begin{itemize}{} \item{}What are the most important sources of uncertainty in this stock assessment?  Explain, and describe qualitatively how they affect the assessment results \(such as estimates of biomass, F, recruitment, and population projections\).  \linebreak{} \hspace\*{0.5cm} \textit{A large source of uncertainty is the estimate of natural mortality based on longevity, which is not well studied in Southern New England Mid\-Atlantic Winter Flounder, and assumed constant over time.  Natural mortality affects the scale of the biomass and fishing mortality estimates.  Natural mortality was adjusted upwards from 0.2 to 0.3 during the last benchmark assessment assuming a max age of 16. However, there is still uncertainty in the true max age of the population and the resulting natural mortality estimate. Other sources of uncertainty include length distribution of the recreational discards.  The recreational discards, are a small component of the total catch, but the assessment suffers from very little length information used to characterize the recreational discards \(1 to 2 lengths in recent years\).}  \item{} Does this assessment model have a retrospective pattern? If so, is the pattern minor, or major? \(A major retrospective pattern occurs when the adjusted SSB or  \$F\_{Full}\${} lies outside of the approximate  joint confidence region for SSB and  \$F\_{Full}\${}\; see  Figure \ref{RhoDecision\_tab}{}\). \linebreak{} \hspace\*{0.5cm} \textit{ No retrospective adjustment of spawning stock biomass or fishing mortality in 2014 was required. }  \item{}Based on this stock assessment, are population projections well determined or uncertain? \linebreak{} \hspace\*{0.5cm} \textit{Population projections for Southern New England Mid\-Atlantic Winter Flounder are reasonably well determined. There is uncertainty in the estimates of M. In addition, while the retrospective pattern is considered minor \(within the 90\% CI of both F and SSB\)  the rho adjusted terminal value is very close to falling out of the bounds, becoming a major retrospective pattern. This would lead to retrospective adjustments being needed for the projections.}  \item{}Describe any changes that were made to the current stock assessment, beyond incorporating additional years of data  and the affect these changes had on the assessment and stock status. \linebreak{} \hspace\*{0.5cm} \textit{ No changes, other than the incorporation of new data were made to the Southern New England Mid\-Atlantic Winter Flounder assessment for this update.}  \item{}If the stock status has changed a lot since the previous assessment, explain why this occurred.  \linebreak{} \hspace\*{0.5cm} \textit{The stock status of Southern New England Mid\-Atlantic Winter Flounder has not changed since the previous benchmark in 2011.}  \item{}Indicate what data or studies are currently lacking and which would be needed most to improve this stock assessment in the future.  \linebreak{} \hspace\*{0.5cm} \textit{The Southern New England Mid\-Atlantic Winter Flounder assessment could be improved with additional studies on maximum age, as well additional information  of recreational discard lengths.  In addition, further investigation into the localized struture\/genetics of the stock is warranted. Also, a future shift to ASAP version 4 will provide the ability to model envirionmental factors that may influence both survey catchability and the modeled S\-R relationship}  \item{}Are there other important issues? \linebreak{} \hspace\*{0.5cm} \textit{None. } \end{itemize}{}} \def\FLWSNEMARefr{ \textbf{References: }{} \linebreak{}Smith, A. and S. Jones.  2008.  In.  Northeast Fisheries Science Center. 2008. Assessment of 19 Northeast Groundfish Stocks through 2007: Report of the 3$^{rd}$ Groundfish Assessment Review Meeting \(GARM III\), Northeast Fisheries Science Center, Woods Hole, Massachusetts, August 4\-8, 2008. US Dep Commer, NOAA Fisheries, Northeast Fish Sci Cent Ref Doc. 08\-15\; 884 p + xvii. http:\/\/www.nefsc.noaa.gov\/publications\/crd\/crd0815\/ \linebreak{} \linebreak{}Northeast Fisheries Science Center. 2011. 52$^{nd}$ Northeast Regional Stock AssessmentWorkshop \(52$^{nd}$ SAW\)  Assessment Report. US Dept Commer, Northeast Fish SciCent Ref Doc. 11\-17\; 962 p. Available from: National Marine Fisheries Service, 166Water Street, Woods Hole, MA 02543\-1026, or online at http:\/\/www.nefsc.noaa.gov\/nefsc\/publications\/ \linebreak{} \linebreak{}} \def\FLWSNEMADraft{} \def\FLWSNEMASPPname{Southern New England Mid-Atlantic Winter Flounder} \def\FLWSNEMASPPnameT{Southern New England Mid-Atlantic Winter Flounder} \def\FLWSNEMARptYr{2015} \def\FLWSNEMAAuthor{Anthony Wood} \def\FLWSNEMAReviewerComments{/home/dhennen/EIEIO/BigReport/FLW_SNEMA/latex}  \def\FLWGBMyPathTab{/home/dhennen/EIEIO/BigReport/FLW_GB/tables} \def\FLWGBMyPathFig{/home/dhennen/EIEIO/BigReport/FLW_GB/figures} \def\FLWGBfigFishCap{Total catches \(mt\)  of Georges Bank Winter Flounder between 1982 and 2015 by country and disposition \(landings and discards\).} \def\FLWGBfigSSBCap{Trends in spawning stock biomass \(mt\)  of Georges Bank Winter Flounder between 1982 and 2014 from the current  \(solid line\)  and previous \(dashed line\)  assessments and the corresponding  \$SSB\_{Threshold}\${} \(\$\dfrac{1}{2}\${} \$SSB\_{MSY}\${}\; horizontal dashed line\)  as well as  \$SSB\_{Target}\${} \(\$SSB\_{MSY}\${}\; horizontal dotted line\)   based on the 2015 assessment.  Biomass was adjusted for a retrospective pattern  and the adjustment is shown in red.  The approximate 90\% normal confidence intervals are shown.} \def\FLWGBfigFCap{Trends in fully selected fishing mortality \(\$F\_{Full}\${}\)  of Georges Bank Winter Flounder between 1982 and 2014 from the current  \(solid line\)  and previous \(dashed line\)  assessments and the corresponding  \$F\_{Threshold}\${} \(\$F\_{MSY}\${}=0.536\; horizontal dashed line\)  as well as \(\$F\_{Target}\${}= 75\% of FMSY\;  horizontal dotted line\). \$F\_{Full}\${} was adjusted for a retrospective pattern  and the adjustment is shown in red.  The approximate 90\% normal confidence intervals are also shown.} \def\FLWGBfigRecrCap{Trends in Recruits \(age 1\)  \(000s\)  of Georges Bank Winter Flounder between 1982 and 2014 from the current \(solid line\)  and previous \(dashed line\)  assessments. The approximate 90\% normal confidence intervals are shown.} \def\FLWGBfigSurvCap{Indices of biomass for the Georges Bank Winter Flounder for the Northeast Fisheries Science Center \(NEFSC\)  spring \(1968\-2015\)  and fall \(1963\-2014\)   bottom trawl surveys and the Canadian DFO spring survey \(1987\-2015\).  The approximate 90\% normal confidence intervals are shown.} \def\FLWGBPreAmb{This assessment of the Georges Bank Winter Flounder \(\textit{Pseudopleuronectes americanus}\)  stock is an operational update of the existing 2014 operational VPA assessment which included data for 1982\-2013 \(Hendrickson et al. 2015\). Based on the previous assessment the stock was not overfished and overfishing was not ocurring. This assessment updates commercial fishery catch data, research survey biomass indices, and the analytical VPA assessment model and reference points through 2014. Additionally, stock projections have been updated through 2018.} \def\FLWGBSoS{ \textbf{State of Stock: }{}Based on this updated assessment, the Georges Bank Winter Flounder \(\textit{Pseudopleuronectes americanus}\)  stock is overfished and overfishing is occurring \(Figures \ref{FLWGBSSB\_plot1}\-\ref{FLWGBF\_plot1}\){}. Retrospective adjustments were made to the model results.  Spawning stock biomass \(SSB\)  in 2014 was estimated to be 2,883 \(mt\)  which is 43\% of the biomass target for an overfished stock \(\$SSB\_{MSY}\${} = 6,700 with a threshold of 50\% of SSBMSY\;  Figure \ref{FLWGBSSB\_plot1}{}\).  The 2014 fully selected fishing mortality \(F\)  was estimated to be 0.778 which is 145\% of the overfishing threshold \(\$F\_{MSY}\${} = 0.536\;  Figure \ref{FLWGBF\_plot1}{}\). However, the 2014 point estimate of SSB and F, when adjusted for retrospective error \(83\% for SSB and \-51\% for F\), is outside the 90\% confidence interval of the unadjusted 2014 point estimate. Therefore, the 2014 F and SSB values used in the stock status determination were the retrospective\-adjusted values of 0.778 and 2,883 mt, respectively.} \def\FLWGBProj{ \textbf{Projections: }{}Short\-term projections of biomass were derived by sampling from a cumulative  distribution  function of recruitment estimates \(1982\-2013 YC\)  from the final run of the ADAPT VPA model. The annual fishery selectivity, maturity ogive, and mean weights\-at\-age used in the projection  are the most recent 5 year averages \(2010\-2014\). An SSB retrospective adjustment factor of 0.546 was applied in the projections.} \def\FLWGBSpecCmt{ \textbf{Special Comments: } \begin{itemize}{} \item{}What are the most important sources of uncertainty in this stock assessment?  Explain, and describe qualitatively how they affect the assessment results \(such as estimates of biomass, F, recruitment, and population projections\).  \linebreak{} \hspace\*{0.5cm} \textit{The largest source of uncertainty is the estimate of natural mortality based on longevity \(max. age = 20 for this stock\), which is not well studied in Georges Bank Winter Flounder, and assumed constant over time.  Natural mortality affects the scale of the biomass and fishing mortality estimates. Other sources of uncertainty include the underestimation of catches. Discards from the Canadian bottom trawl fleet were not provided by the CA DFO and the precision of the Canadian scallop dredge discard estimates, with only 1\-2 trips per month, are uncertain.The lack of age data for the Canadian spring survey catches requires the use of the US spring survey A\/L keys despite selectivity differences. In addition, there are no length or age composition data from the Canadian landings or discards GB winter flounder.}  \item{} Does this assessment model have a retrospective pattern? If so, is the pattern minor, or major? \(A major retrospective pattern occurs when the adjusted SSB or  \$F\_{Full}\${} lies outside of the approximate  joint confidence region for SSB and  \$F\_{Full}\${}\; see  Figure \ref{RhoDecision\_tab}{}\). \linebreak{} \hspace\*{0.5cm} \textit{ The 7\-year Mohn\'s  \textrho{}, relative to SSB, was 0.26 in the 2014 assessment and was 0.83 in 2014. The 7\-year Mohn\'s  \textrho{}, relative to F, was \-0.16 in the 2014 assessment and was \-0.51 in 2014. There was a major retrospective pattern for this assessment because the  \textrho{} adjusted estimates of 2014 SSB \(\$SSB\_{\rho}\${}=2,883\)  and 2014 F \(\$F\_{\rho}\${}=0.778\)  were outside the approximate 90\% confidence region around SSB \(3,783 \- 6,767\)  and F \(0.254 \- 0.504\).  A retrospective  adjustment was made for both the determination of stock status and for projections of catch in 2016. The retrospective adjustment changed the 2014 SSB from 5,275 to 2,883 and the 2014  \$F\_{Full}\${} from 0.379 to 0.778.}  \item{}Based on this stock assessment, are population projections well determined or uncertain? \linebreak{} \hspace\*{0.5cm} \textit{Population projections for Georges Bank Winter Flounder are reasonably well determined.}  \item{}Describe any changes that were made to the current stock assessment, beyond incorporating additional years of data  and the affect these changes had on the assessment and stock status. \linebreak{} \hspace\*{0.5cm} \textit{ The only change made to the Georges Bank Winter Flounder assessment, other than the incorporation of an additional  year of data, involved fishery selectivity.  During the 2014 assessment update, stock size estimates of age 1 and age 2 fish were not estimable  in the VPA during year t + 1 \(CVs near 1.0\). When age 2 stock size is not estimated in year t + 1,  the VPA model calculates the stock size of age 1 fish \(i.e., recruitment\)  in the terminal year by  using the age 1 partial recruitment \(PR\)  value to derive the F at age 1 in the terminal year. The  age 1 PR value used in the 2014 assessment update was 0.001. However, when this same age 1 PR value  was used in a VPA run for the current assessment update, the low PR value combined with the low age  1 catch in 2014 resulted in an unlikely high stock size estimate for age 1 recruitment in 2014 \(i.e.,  41,587,000 fish\)  when compared to survey observations of the same cohort \(i.e., age 1 in 2014 and age  2 in 2015\). In order to obtain a more realistic estimate of age 1 recruitment in 2014, I allowed the  VPA model to estimate age 2 stock size in 2015 \(i.e., and thereby avoided the use of an age 1 PR  value in the age 1 stock size calculation for 2014\)  and used the back\-calculated PR values from this  VPA run to derive a new PR\-at\-age vector which was used in the final 2015 VPA run. Similar to the  2014 assessment update, the final 2015 VPA run did not include the estimation of age 2 stock size  and the new PR\-at\-age vector was computed using the same methods as in the 2014 assessment.   Full selectivity occurs at age 4. For the 2015 assessment update, fishery selectivity for ages  1\-3 was changed from the 2014 assessment values of 0.001, 0.10 and 0.43, respectively, to 0.01,  0.08 and 0.55, respectively. Differences between estimates  of F, SSB and R values from the final  2015 VPA run, with the new PR vector, and a 2015 VPA run that utilized the PR vector from the 2014  assessment are shown in Table G30.}  \item{}If the stock status has changed a lot since the previous assessment, explain why this occurred.  \linebreak{} \hspace\*{0.5cm} \textit{The overfished and overfishing status of Georges Bank Winter Flounder has changed in the current assessment update due to a worsening of the retrospective error associated with fishing mortality and SSB.}  \item{}Indicate what data or studies are currently lacking and which would be needed most to improve this stock assessment in the future.  \linebreak{} \hspace\*{0.5cm} \textit{The Georges Bank Winter Flounder assessment could be improved with discard estimates from the Canadian bottom trawl fleet and age data from the Canadian spring bottom trawl surveys.}  \item{}Are there other important issues? \linebreak{} \hspace\*{0.5cm} \textit{None. } \end{itemize}{}} \def\FLWGBRefr{ \textbf{References: }{} \linebreak{} Hendrickson L, Nitschke P, Linton B. 2015. 2014 Operational Stock Assessments for Georges Bank winter flounder, Gulf of Maine winter flounder, and pollock. US Dept Commer, Northeast Fish Sci Cent Ref Doc. 15\-01\; 228 p. \linebreak{} \linebreak{}} \def\FLWGBDraft{} \def\FLWGBSPPname{Georges Bank Winter Flounder} \def\FLWGBSPPnameT{Georges Bank Winter Flounder} \def\FLWGBRptYr{2015} \def\FLWGBAuthor{Lisa Hendrickson} \def\FLWGBReviewerComments{/home/dhennen/EIEIO/BigReport/FLW_GB/latex}  \def\FLDGMGBMyPathTab{/home/dhennen/EIEIO/BigReport/FLD_GMGB/tables} \def\FLDGMGBMyPathFig{/home/dhennen/EIEIO/BigReport/FLD_GMGB/figures} \def\FLDGMGBfigFishCap{Total catch of northern windowpane flounder between 1975 and 2014 by disposition \(landings and discards\).} \def\FLDGMGBfigSSBCap{Trends in the biomass index \(a 3\-year moving average of the NEFSC fall bottom trawl survey index\)  of northern windowpane flounder between 1975 and 2014 from the current  assessment, and the corresponding  \$B\_{Threshold}\${} =  \$\dfrac{1}{2}\${} \$B\_{MSY}\${} \textit{proxy}{} = 0.777 kg\/tow \(horizontal dashed line\). } \def\FLDGMGBfigFCap{Trends in relative fishing mortality  of northern windowpane flounder between 1975 and 2014 from the current  assessment, and the corresponding  \$F\_{MSY}\${} \textit{proxy}{}=0.45 \(horizontal dashed line\). } \def\FLDGMGBfigRecrCap{} \def\FLDGMGBfigSurvCap{NEFSC fall bottom trawl survey indices in kg\/tow for northern windowpane flounder between 1975 and 2014  The approximate 90\% lognormal confidence intervals are shown.} \def\FLDGMGBPreAmb{This assessment of the northern windowpane flounder \(\textit{Scophthalmus aquosus}\)  stock is an operational update of the 2012 assessment which included updates through 2010 \(NEFSC 2012\). Based on the 2012 assessment the stock was overfished, and overfishing was ocurring. This assessment updates commercial fishery catch data, survey indices of abundance, AIM model results,  and reference points through 2014.} \def\FLDGMGBSoS{ \textbf{State of Stock: }{}Based on this updated assessment, the northern windowpane flounder \(\textit{Scophthalmus aquosus}\)  stock is overfished but overfishing is not occurring \(Figures \ref{FLDGMGBSSB\_plot1}\-\ref{FLDGMGBF\_plot1}\){}. Retrospective adjustments were not made to the model results. The mean NEFSC fall bottom trawl survey index from years 2012, 2013 and 2014 \(a 3\-year moving average is used as a biomass index\)  was 0.535 kg\/tow which is lower than the \$B\_{Threshold}\${} of 0.777 kg\/tow. The 2014 relative fishing mortality was estimated to be 0.393 kt per kg\/tow which is lower than the  \$F\_{MSY}\${} \textit{proxy}{} of 0.450 kt per kg\/tow.} \def\FLDGMGBProj{} \def\FLDGMGBSpecCmt{ \textbf{Special Comments: } \begin{itemize}{} \item{}What are the most important sources of uncertainty in this stock assessment?  Explain, and describe qualitatively how they affect the assessment results \(such as estimates of biomass, F, recruitment, and population projections\).  \linebreak{} \hspace\*{0.5cm} \textit{The main source of uncertainty in this assessment is the lack of windowpane discard estimates from Canadian fisheries to add to the catch component of model input. Discard estimates were from the U.S. only. There is overlap between the survey area and Canadian fishing grounds \(Van Eeckhaute et al. 2010\), which means catch from within the stock area was likely underestimated. }  \item{} Does this assessment model have a retrospective pattern? If so, is the pattern minor, or major? \(A major retrospective pattern occurs when the adjusted SSB or  \$F\_{Full}\${} lies outside of the approximate  joint confidence region for SSB and  \$F\_{Full}\${}\; see  Figure \ref{RhoDecision\_tab}{}\). \linebreak{} \hspace\*{0.5cm} \textit{ The model used to estimate status of this stock does not allow estimation of a retrospective pattern. }  \item{}Based on this stock assessment, are population projections well determined or uncertain? \linebreak{} \hspace\*{0.5cm} \textit{N\/A }  \item{}Describe any changes that were made to the current stock assessment, beyond incorporating additional years of data  and the affect these changes had on the assessment and stock status. \linebreak{} \hspace\*{0.5cm} \textit{No changes were made to the northern windowpane flounder assessment for this update  other than the incorporation of four years of new NEFSC fall bottom trawl survey data and  four years of new U.S. commercial landings and discard data \(2011 \- 2014\). }  \item{}If the stock status has changed a lot since the previous assessment, explain why this occurred.  \linebreak{} \hspace\*{0.5cm} \textit{The stock status of northern windowpane flounder changed from \'overfished and overfishing is occurring\' to \'overfished and overfishing is not occurring\' due to stable\-to\-decreasing catch since 2008, and an increasing trend in the survey index since 2010. }  \item{}Indicate what data or studies are currently lacking and which would be needed most to improve this stock assessment in the future.  \linebreak{} \hspace\*{0.5cm} \textit{The northern windowpane flounder assessment could be improved by estimating the Canadian windowpane removals and, although to a lesser degree, the \'general category\' scallop dredge fleet discards from within the stock area and using them as additional catch input to the AIM model.  While the model fit now is reasonable \(the relationship between ln\(relative F\)  and ln\(replacement ratio\), a measure of the relationship between catch and survey index values, has a p\-value of 0.079\)  there are probably removals unaccounted for in the model and the fit can likely be improved. }  \item{}Are there other important issues? \linebreak{} \hspace\*{0.5cm} \textit{None. } \end{itemize}{}} \def\FLDGMGBRefr{ \textbf{References: }{} \linebreak{} Most recent assessment update:  \linebreak{} Northeast Fisheries Science Center. 2012. Assessment or Data Updates of 13 Northeast Groundfish Stocks through 2010.  US Dept Commer, Northeast Fish Sci Cent Ref Doc. 12\-06\; 789 p. Available online at http:\/\/nefsc.noaa.gov\/publications\/  \linebreak{} \linebreak{} Most recent benchmark assessment:  \linebreak{} Northeast Fisheries Science Center. 2008. Assessment of 19 Northeast Groundfish Stocks through 2007:  Report of the 3$^{rd}$ Groundfish Assessment Review Meeting \(GARM III\), Northeast Fisheries Science Center,  Woods Hole, Massachusetts, August 4\-8, 2008. US Dep Commer, NOAA FIsheries, Northeast Fish Sci Cent Ref Doc. 08\-15\; 884 p + xvii.  \linebreak{} \linebreak{} Van Eeckhaute, L., Sameoto, J., and A. Glass. 2010. Discards of Atlantic cod, haddock and yellowtail flounder  from the 2009 Canadian scallop fishery on Georges Bank. TRAC Ref. Doc. 2010\/10. 7p.  \linebreak{} \linebreak{}} \def\FLDGMGBDraft{} \def\FLDGMGBSPPname{northern windowpane flounder} \def\FLDGMGBSPPnameT{Northern windowpane flounder} \def\FLDGMGBRptYr{2015} \def\FLDGMGBAuthor{Toni Chute} \def\FLDGMGBReviewerComments{/home/dhennen/EIEIO/BigReport/FLD_GMGB/latex}  \def\FLDSNEMAMyPathTab{/home/dhennen/EIEIO/BigReport/FLD_SNEMA/tables} \def\FLDSNEMAMyPathFig{/home/dhennen/EIEIO/BigReport/FLD_SNEMA/figures} \def\FLDSNEMAfigFishCap{Total catch of southern windowpane flounder between 1975 and 2014 by disposition \(landings and discards\).} \def\FLDSNEMAfigSSBCap{Trends in the biomass index \(a 3\-year moving average of the NEFSC fall bottom trawl survey index\)  of southern windowpane flounder between 1975 and 2014 from the current  assessment, and the corresponding  \$B\_{Threshold}\${} =  \$\dfrac{1}{2}\${} \$B\_{MSY}\${} \textit{proxy}{} = 0.123 kg\/tow\(horizontal dashed line\). } \def\FLDSNEMAfigFCap{Trends in relative fishing mortality  of southern windowpane flounder between 1975 and 2014 from the current  assessment, and the corresponding  \$F\_{MSY}\${} \textit{proxy}{}=2.027 \(horizontal dashed line\). } \def\FLDSNEMAfigRecrCap{} \def\FLDSNEMAfigSurvCap{NEFSC fall bottom trawl survey indices in kg\/tow for southern windowpane flounder between 1975 and 2014. The approximate 90\% lognormal confidence intervals are shown.} \def\FLDSNEMAPreAmb{This assessment of the southern windowpane flounder \(\textit{Scophthalmus aquosus}\)  stock is an operational update of the 2012 assessment which included updates through 2010 \(NEFSC 2012\). Based on the 2012 assessment the stock was not overfished, and overfishing was not ocurring. This assessment updates commercial fishery catch data, survey indices of abundance, AIM model results, and reference points through 2014. } \def\FLDSNEMASoS{ \textbf{State of Stock: }{}Based on this updated assessment, the southern windowpane flounder \(\textit{Scophthalmus aquosus}\)  stock is not overfished and overfishing is not occurring \(Figures \ref{FLDSNEMASSB\_plot1}\-\ref{FLDSNEMAF\_plot1}\){}. Retrospective adjustments were not made to the model results. The mean NEFSC fall bottom trawl survey index from years 2012, 2013, and 2014 \(a 3\-year moving average is used as a biomass index\)  was  0.413 \(kg\/tow\)  which is higher than the \$B\_{Threshold}\${}of 0.123 \(kg\/tow\). The 2014 relative fishing mortality was estimated to be  1.308 \(kt per kg\/tow\)  which is lower than the  \$F\_{MSY}\${} \textit{proxy}{} of 2.027 \(kt per kg\/tow\). } \def\FLDSNEMAProj{} \def\FLDSNEMASpecCmt{ \textbf{Special Comments: } \begin{itemize}{} \item{}What are the most important sources of uncertainty in this stock assessment?  Explain, and describe qualitatively how they affect the assessment results \(such as estimates of biomass, F, recruitment, and population projections\).  \linebreak{} \hspace\*{0.5cm} \textit{A source of uncertainty for this assessment is missing commercial discard estimates from the general category scallop dredge fleet that should be added to the catch time series for model input. }  \item{} Does this assessment model have a retrospective pattern? If so, is the pattern minor, or major? \(A major retrospective pattern occurs when the adjusted SSB or  \$F\_{Full}\${} lies outside of the approximate  joint confidence region for SSB and  \$F\_{Full}\${}\; see  Figure \ref{RhoDecision\_tab}{}\). \linebreak{} \hspace\*{0.5cm} \textit{ The model used to estimate status of this stock does not allow estimation of a retrospective pattern. }  \item{}Based on this stock assessment, are population projections well determined or uncertain? \linebreak{} \hspace\*{0.5cm} \textit{N\/A}  \item{}Describe any changes that were made to the current stock assessment, beyond incorporating additional years of data  and the affect these changes had on the assessment and stock status. \linebreak{} \hspace\*{0.5cm} \textit{ No changes were made to the southern windowpane flounder assessment for this update  other than the incorporation of four years of new NEFSC fall bottom trawl survey data and  four years of new U.S. commercial landings and discard data \(2011 \- 2014\). }  \item{}If the stock status has changed a lot since the previous assessment, explain why this occurred.  \linebreak{} \hspace\*{0.5cm} \textit{The stock status of southern windowpane flounder has not changed since the previous assessment. }  \item{}Indicate what data or studies are currently lacking and which would be needed most to improve this stock assessment in the future.  \linebreak{} \hspace\*{0.5cm} \textit{Estimates of discards from the general category scallop dredge fleet should be added to the catch time series for model input. However, the model fit is presently good with a randomization test indicating the correlation between ln\(relative F\)  and ln\(replacement ratio\), a measure of the relationship between catch and survey index values, is significant \(p = 0.002.\)  }  \item{}Are there other important issues? \linebreak{} \hspace\*{0.5cm} \textit{None. } \end{itemize}{}} \def\FLDSNEMARefr{ \textbf{References: }{} \linebreak{} Most recent assessment update:  \linebreak{} Northeast Fisheries Science Center. 2012. Assessment or Data Updates of 13 Northeast Groundfish Stocks through 2010.  US Dept Commer, Northeast Fish Sci Cent Ref Doc. 12\-06\; 789 p. Available online at http:\/\/nefsc.noaa.gov\/publications\/  \linebreak{} \linebreak{} Most recent benchmark assessment:  \linebreak{} Northeast Fisheries Science Center. 2008. Assessment of 19 Northeast Groundfish Stocks through 2007:  Report of the 3$^{rd}$ Groundfish Assessment Review Meeting \(GARM III\), Northeast Fisheries Science Center,  Woods Hole, MA, August 4\-8, 2008. US Dep Commer, NOAA Fisheries, Northeast Fish Sci Cent Ref Doc. 08\-15\; 884 p + xvii. \linebreak{} \linebreak{}} \def\FLDSNEMADraft{} \def\FLDSNEMASPPname{southern windowpane flounder} \def\FLDSNEMASPPnameT{Southern windowpane flounder} \def\FLDSNEMARptYr{2015} \def\FLDSNEMAAuthor{Toni Chute}
%\def\POKUNITMyPathTab{/home/dhennen/EIEIO/BigReport/POK_UNIT/tables} \def\POKUNITMyPathFig{/home/dhennen/EIEIO/BigReport/POK_UNIT/figures} \def\POKUNITfigFishCap{Total catch of pollock between 1970 and 2014 by fleet \(commercial, Canadian, distant water fleet, and recreational\)  and disposition \(landings and discards\).} \def\POKUNITfigSSBCap{Estimated trends in the spawning stock biomass of pollock between 1970 and 2014 from the current  \(solid line\)  and previous \(dashed line\)  assessment and the corresponding  \$SSB\_{Threshold}\${} \(0.5 \* \$SSB\_{MSY}\${} proxy\; horizontal dashed line\)  as well as  \$SSB\_{Target}\${} \(\$SSB\_{MSY}\${} proxy\; horizontal dotted line\)   based on the 2015 assessment models base \(A\)  and flat sel sensitivity \(B\). Biomass was adjusted for a retrospective pattern and the adjustment is shown in red. The approximate 90\% lognormal confidence intervals are shown.} \def\POKUNITfigFCap{Estimated trends in age 5 to 7 average F \(\$F\_{AVG}\${}\)  of pollock between 1970 and 2014 from the current  \(solid line\)  and previous \(dashed line\)  assessment and the corresponding  \$F\_{Threshold}\${} \(\$F\_{MSY}\${} proxy\; dashed line\)  based on the 2015 assessment models base \(A\)  and flat sel sensitivity \(B\).  \$F\_{AVG}\${} was adjusted for a retrospective pattern and the adjustment is shown in red. The approximate 90\% lognormal confidence intervals are shown.} \def\POKUNITfigRecrCap{Estimated trends in age 1 recruitment  \(000s\)  of pollock between 1970 and 2014 from the current \(solid line\)  and previous \(dashed line\)  assessment for the assessment models base \(A\)  and flat sel sensitivity \(B\).  The approximate 90\% lognormal confidence intervals are shown.} \def\POKUNITfigSurvCap{Indices of abundance for pollock from the Northeast Fisheries Science Center \(NEFSC\)  spring \(1970 to 2015\)  and fall \(1970 to 2014\)  bottom trawl surveys. The approximate 90\% lognormal confidence intervals are shown.} \def\POKUNITPreAmb{This assessment of the pollock \(\textit{Pollachius virens}\)  stock is an update of the existing 2014 operational assessment \(Hendrickson et al. 2015\). This assessment updates commercial and recreational fishery catch data, research survey indices of abundance, the ASAP analytical models, and biological reference points through 2014. Additionally, stock projections have been updated through 2018. In what follows, there are two population assessment models brought forward from the 2014 operational assessment, the base model \(dome\-shaped survey selectivity\)  , which is used to provide management advice, and the flat sel sensitivity model \(flat\-topped survey selectivity\), which is included for the sole purpose of demonstrating the sensitivity of assessment results to survey selectivity assumptions. The most recent benchmark assessment of the pollock stock was in 2010 as part of the 50$^{th}$ Stock Assessment Review Committee \(SARC 50\; NEFSC 2010\), which includes a full description of the model formulations.} \def\POKUNITSoS{ \textbf{State of Stock: }{} The pollock \(\textit{Pollachius virens}\)  stock is not overfished and overfishing is not occurring \(Figures \ref{POKUNITSSB\_plot1}\-\ref{POKUNITF\_plot1}\){}. Retrospective adjustments were made to the model results. Retrospective adjusted spawning stock biomass \(SSB\)  in 2014 was estimated to be 154,919 \(mt\)  under the base model and 32,040 \(mt\)  under the flat sel sensitivity model which is 147 and 58\% \(respectively\)  of the biomass target, an  \$SSB\_{MSY}\${} proxy of SSB at  \$F\_{40\%}\${} \(105,226 and 54,900  \(mt\)\;  Figure \ref{POKUNITSSB\_plot1}{}\). Retrospective adjusted 2014 age 5 to 7 average fishing mortality \(F\)   was estimated to be 0.07 under the base model and 0.233 under the flat sel sensitivity model which is 25 and 92\% \(respectively\)  of the overfishing threshold, an  \$F\_{MSY}\${} proxy of  \$F\_{40\%}\${} \(0.277 and 0.252\;  Figure \ref{POKUNITF\_plot1}{}\).} \def\POKUNITProj{ \textbf{Projections: }{}Short term projections of median total fishery yield and spawning stock biomass for pollock were conducted based on a harvest scenario of fishing at an  \$F\_{MSY}\${} proxy of  \$F\_{40\%}\${} between 2016 and 2018. Catch in 2015 has been estimated at 5,208 \(mt\). Recruitments were sampled from a cumulative distribution function derived from ASAP estimated age 1 recruitment between 1970 and 2012.  Recruitments in 2013 and 2014 were not included due to uncertainty in those estimates. The annual fishery selectivity, natural mortality, maturity ogive, and mean weights used  in projections are the most recent 5 year averages. Retrospective adjusted age 5 to 7 average F in 2014 fell outside the 90\% confidence intervals of the unadjusted 2014 value under the base model \(Figure \ref{POKUNITF\_plot1}{}\). Retrospective adjusted SSB and age 5 to 7 average F in 2014 fell outside the 90\% confidence intervals of the unadjusted 2014 values under the flat sel sensitivity model  \(Figures \ref{POKUNITSSB\_plot1}\-\ref{POKUNITF\_plot1}\){}. Therefore, retrospective adjustments were applied in the projections for the base model and the flat sel sensitivity model.} \def\POKUNITSpecCmt{ \textbf{Special Comments: } \begin{itemize}{} \item{}What are the most important sources of uncertainty in this stock assessment?  Explain, and describe qualitatively how they affect the assessment results \(such as estimates of biomass, F, recruitment, and population projections\).  \linebreak{} \hspace\*{0.5cm} \textit{The largest source of uncertainty in the pollock assessment is selectivity, as the base model with dome\-shaped survey and fishery selectivities implies the existence of a large cryptic biomass that neither current surveys nor the fishery can confirm. Assuming flat\-topped survey selectivities leads to lower estimates of SSB and higher estimates of F  \(Figures \ref{POKUNITSSB\_plot1}\-\ref{POKUNITF\_plot1}\){}. Stock status is insensitive to the shape of the survey selectivity patterns at older ages.}  \item{} Does this assessment model have a retrospective pattern? If so, is the pattern minor, or major? \(A major retrospective pattern occurs when the adjusted SSB or  \$F\_{AVG}\${} lies outside of the approximate  joint confidence region for SSB and  \$F\_{AVG}\${}\; see  Figure \ref{RhoDecision\_tab}{}\). \linebreak{} \hspace\*{0.5cm} \textit{ The 7\-year Mohn\'s  \textrho{}, relative to SSB, was 0.291 under the base model and 0.66 under the flat sel sensitivity model in the 2014 assessment and was 0.284 and 0.789, respectively, in 2014. The 7\-year Mohn\'s  \textrho{}, relative to F, was \-0.252 under the base model and \-0.359 under the flat sel sensitivity model in the 2014 assessment and was \-0.276 and \-0.43, respectively, in 2014. There was a major retrospective pattern for the base model because the  \textrho{} adjusted estimate of 2014 F \(\$F\_{\rho}\${}=0.07\)  was outside the approximate 90\% confidence regions around F \(0.035 \- 0.066\). There was a major retrospective pattern for the flat sel sensitivity model because the  \textrho{} adjusted estimates of 2014 SSB \(\$SSB\_{\rho}\${}=32,040\)  and 2014 F \(\$F\_{\rho}\${}=0.233\)  were outside the approximate 90\% confidence regions around SSB \(37,243 \- 77,410  \(mt\)\)  and F \(0.084 \- 0.182\). A retrospective adjustment was made for both the determination of stock status and for projections of catch in 2016. The base model retrospective adjustment changed the 2014 SSB from 198,847 to 154,919 and the 2014  \$F\_{AVG}\${} from 0.051 to 0.07. The flat sel sensitivity model retrospective adjustment changed the 2014 SSB from 57,327 to 32,040 and the 2014  \$F\_{AVG}\${} from 0.133 to 0.233.}  \item{}Based on this stock assessment, are population projections well determined or uncertain? \linebreak{} \hspace\*{0.5cm} \textit{Population projections for pollock, appear to be reasonably well determined for both the base model and the flat sel sensitivity model. }  \item{}Describe any changes that were made to the current stock assessment, beyond incorporating additional years of data  and the affect these changes had on the assessment and stock status. \linebreak{} \hspace\*{0.5cm} \textit{Only one major change was made to the pollock assessment as part of this update. Likelihood constants were excluded from likelihood calculations to avoid potential bias caused by one of the recruitment likelihood constants, which is the sum of the log\-scale predicted recruitments, and therefore not a constant. Inclusion of this likelihood constant allows the assessment model to minimize the negative log likelihood by estimating lower recruitments. Exclusion of the likelihood constants led to higher estimates of SSB  and lower estimates of F  \(Figures \ref{POKUNITSSB\_plot1}\-\ref{POKUNITF\_plot1}\){}.}  \item{}If the stock status has changed a lot since the previous assessment, explain why this occurred.  \linebreak{} \hspace\*{0.5cm} \textit{Stock status based on the base model has not changed since the previous assessment. Stock status based on the flat sel sensitivity model has changed from \'overfishing is occurring\' in the previous assessment to \'overfishing is not occurring\' in the current assessment. Though, the retrospective adjusted 2014 age 5 to 7 average fishing mortality  from the flat sel sensitivity model \(0.233\)  is close to the  \$F\_{MSY}\${} proxy \(0.252\). This change in status likely is due to a decline in predicted F from 2013 to 2014, as well as to the exclusion of the likelihood constants, which led to higher predicted stock productivity.}  \item{}Indicate what data or studies are currently lacking and which would be needed most to improve this stock assessment in the future.  \linebreak{} \hspace\*{0.5cm} \textit{The pollock assessment could be improved with additional studies on gear selectivity. These studies could cover topics such as physical selectivity \(e.g., multi\-mesh gillnet\), behavior \(e.g., swimming endurance, escape behavior\), geographic and vertical distribution by size and age, tag\-recovery at size and age, and evaluating information on length\-specific selectivity at older ages.}  \item{}Are there other important issues? \linebreak{} \hspace\*{0.5cm} \textit{As in the previous assessment, the pollock assessment models had difficulty converging on a solution in some of the retrospective peels. One possible explanation for this convergence issue is that the model may be overparameterized, because the commercial and recreational fleets are modeled separately in this assessment. The possibility of combining the two fleets into a single fleet should be explored during the next benchmark assessment.} \end{itemize}{}} \def\POKUNITRefr{ \textbf{References: }{} \linebreak{}Hendrickson L, Nitschke P, Linton B. 2015. 2014 Operational stock assessments for Georges Bank winter flounder, Gulf of Maine winter flounder, and pollock. US Dept Commer, Northeast Fish Sci Cent Ref Doc. 15\-01\; 228 p. Available from: NationalMarine Fisheries Service, 166 Water Street, Woods Hole, MA 02543\-1026, or online at http:\/\/www.nefsc.noaa.gov\/publications\/ \linebreak{} \linebreak{}Northeast Fisheries Science Center. 2010. 50$^{th}$ Northeast Regional Stock Assessment Workshop \(50$^{th}$ SAW\)  Assessment Report. US Dept Commer, Northeast Fish Sci Cent Ref Doc. 10\-17\; 844 p. Available from: National Marine Fisheries Service, 166 Water Street, Woods Hole, MA 02543\-1026, or online at http:\/\/www.nefsc.noaa.gov\/nefsc\/publications\/ } \def\POKUNITDraft{} \def\POKUNITSPPname{pollock} \def\POKUNITSPPnameT{Pollock} \def\POKUNITRptYr{2015} \def\POKUNITAuthor{Brian Linton} \def\POKUNITReviewerComments{/home/dhennen/EIEIO/BigReport/POK_UNIT/latex}  \def\REDUNITMyPathTab{/home/dhennen/EIEIO/BigReport/RED_UNIT/tables} \def\REDUNITMyPathFig{/home/dhennen/EIEIO/BigReport/RED_UNIT/figures} \def\REDUNITfigFishCap{Total catch of Acadian redfish between 1913 and 2014 by fleet \(commercial and other\)  and disposition \(landings and discards\).} \def\REDUNITfigSSBCap{Trends in spawning stock biomass of Acadian redfish between 1913 and 2014 from the current  \(solid line\)  and previous \(dashed line\)  assessment and the corresponding  \$SSB\_{Threshold}\${} \(0.5 \* \$SSB\_{MSY}\${} \textit{proxy}{}\; horizontal dashed line\)  as well as  \$SSB\_{Target}\${} \(\$SSB\_{MSY}\${} \textit{proxy}{}\; horizontal dotted line\)  based on the 2015 assessment. Biomass was adjusted for a retrospective pattern and the adjustment is shown in red. The approximate 90\% lognormal confidence intervals are shown.} \def\REDUNITfigFCap{Trends in the fully selected fishing mortality \(\$F\_{Full}\${}\)  of Acadian redfish between 1913 and 2014 from the current \(solid line\)  and previous \(dashed line\)  assessment and the corresponding  \$F\_{Threshold}\${} \(\$F\_{MSY}\${} \textit{proxy}{}=0.038\; horizontal dashed line\)  based on the 2015 assessment.  \$F\_{Full}\${} was adjusted for a retrospective pattern and the adjustment is shown in red. The approximate 90\% lognormal confidence intervals are shown.} \def\REDUNITfigRecrCap{Trends in Recruits \(age 1\)  \(000s\)  of Acadian redfish between 1913 and 2014 from the current \(solid line\)  and previous \(dashed line\)  assessment. The approximate 90\% lognormal confidence intervals are shown.} \def\REDUNITfigSurvCap{Indices of abundance for Acadian redfish from the Northeast Fisheries Science Center \(NEFSC\)  spring \(1963 to 2015\)  and fall \(1963 to 2014\)  bottom trawl surveys. The approximate 90\% lognormal confidence intervals are shown.} \def\REDUNITPreAmb{This assessment of the Acadian redfish \(\textit{Sebastes fasciatus}\)  stock is an update of the existing 2012 operational assessment \(NEFSC 2012\). This assessment updates commercial fishery catch data, research survey indices of abundance, the ASAP analytical model, and biological reference points through 2014. Additionally, stock projections have been updated through 2018. The most recent benchmark assessment of the Acadian redfish stock was in 2008 as part of the 3$^{rd}$ Groundfish Assessment Review Meeting \(GARM III\; NEFSC 2008\), which includes a full description of the model formulations.} \def\REDUNITSoS{ \textbf{State of Stock: }{}Based on this updated assessment, the Acadian redfish \(\textit{Sebastes fasciatus}\)  stock is not overfished and overfishing is not occurring \(Figures \ref{REDUNITSSB\_plot1}\-\ref{REDUNITF\_plot1}\){}. Retrospective adjustments were made to the model results. Retrospective adjusted spawning stock biomass \(SSB\)  in 2014 was estimated to be 330,004 \(mt\)  which is 117\% of the biomass target \(\$SSB\_{MSY}\${} \textit{proxy}{} of SSB at  \$F\_{50\%}\${} = 281,112\;  Figure \ref{REDUNITSSB\_plot1}{}\).  The retrospective adjusted 2014 fully selected fishing mortality \(F\)  was estimated to be 0.015 which is 39\% of the overfishing threshold \(\$F\_{MSY}\${} \textit{proxy}{} of  \$F\_{50\%}\${} = 0.038\;  Figure \ref{REDUNITF\_plot1}{}\).} \def\REDUNITProj{ \textbf{Projections: }{}Short term projections of median total fishery yield and spawning stock biomass for Acadian redfish were conducted based on a harvest scenario of fishing at the  \$F\_{MSY}\${} \textit{proxy}{} between 2016 and 2018. Catch in 2015 has been estimated at 5,204 \(mt\). Recruitments were sampled from a cumulative distribution function derived from ASAP estimated age 1 recruitment between 1969 and 2014. The annual fishery selectivity, natural mortality, maturity ogive, and mean weights used  in projections are the same as those used in the assessment model. Retrospective adjusted SSB and fully selected F in 2014 fell outside the 90\% confidence intervals of the unadjusted 2014 values. Therefore, retrospective adjustments were applied in the projections. } \def\REDUNITSpecCmt{ \textbf{Special Comments: } \begin{itemize}{} \item{}What are the most important sources of uncertainty in this stock assessment?  Explain, and describe qualitatively how they affect the assessment results \(such as estimates of biomass, F, recruitment, and population projections\).  \linebreak{} \hspace\*{0.5cm} \textit{The largest source of uncertainty in the Acadian redfish assessment is the lack of age data, particularly from the commercial fishery. Age measurements from landings halted after 1985, due to relatively low landings. Current landings have increased to levels seen in the mid\-1980s. If landings continue to increase, then age data from the fishery will become increasingly important. Dimorphic growth is another source of uncertainty in this assessment, with females growing faster than males. The use of female weights at age in the stock projections may lead to overestimation of stock productivity, as well as having an unknown effect on biological reference points.}  \item{} Does this assessment model have a retrospective pattern? If so, is the pattern minor, or major? \(A major retrospective pattern occurs when the adjusted SSB or  \$F\_{Full}\${} lies outside of the approximate  joint confidence region for SSB and  \$F\_{Full}\${}\; see  Figure \ref{RhoDecision\_tab}{}\). \linebreak{} \hspace\*{0.5cm} \textit{ The 7\-year Mohn\'s  \textrho{}, relative to SSB, was 0.036 in the 2012 assessment and was 0.256 in 2014. The 7\-year Mohn\'s  \textrho{}, relative to F, was \-0.035 in the 2012 assessment and was \-0.190 in 2014. There was a major retrospective pattern for this assessment because the  \textrho{} adjusted estimates of 2014 SSB \(\$SSB\_{\rho}\${}=330,004\)  and 2014 F \(\$F\_{\rho}\${}=0.015\)  were outside the approximate 90\% confidence regions around SSB \(368,906 \- 465,828\)  and F \(0.011 \- 0.014\).  A retrospective  adjustment was made for both the determination of stock status and for projections of catch in 2016. The retrospective adjustment changed the 2014 SSB from 414,544 to 330,004 and the 2014  \$F\_{Full}\${} from 0.012 to 0.015.}  \item{}Based on this stock assessment, are population projections well determined or uncertain? \linebreak{} \hspace\*{0.5cm} \textit{Population projections for Acadian redfish appear to be reasonably well determined. }  \item{}Describe any changes that were made to the current stock assessment, beyond incorporating additional years of data  and the affect these changes had on the assessment and stock status. \linebreak{} \hspace\*{0.5cm} \textit{Only one major change was made to the Acadian redfish assessment as part of this update. Likelihood constants were excluded from likelihood calculations to avoid potential bias caused by one of the recruitment likelihood constants, which is the sum of the log\-scale predicted recruitments, and therefore not a constant. Inclusion of this likelihood constant allows the assessment model to minimize the negative log likelihood by estimating lower recruitments. Exclusion of the likelihood constants led to slightly higher estimates of SSB in recent years. }  \item{}If the stock status has changed a lot since the previous assessment, explain why this occurred.  \linebreak{} \hspace\*{0.5cm} \textit{There has been no change in the stock status of Acadian redfish since the previous assessment.}  \item{}Indicate what data or studies are currently lacking and which would be needed most to improve this stock assessment in the future.  \linebreak{} \hspace\*{0.5cm} \textit{The Acadian redfish assessment could be improved by 1\)  including additional age data, particularly from the commercial fishery, and 2\)  investigating the sensitivity of biological reference points and stock projections to the mean weights at age. }  \item{}Are there other important issues? \linebreak{} \hspace\*{0.5cm} \textit{Northeast Fisheries Science Center \(NEFSC\)  fall bottom trawl index values for 2013 and 2014 are lower than in previous years \(Figure \ref{REDUNITSurv\_plot1}{}\), but the current assessment model continues to predict an increase in SSB for the last two years \(Figure \ref{REDUNITSSB\_plot1}{}\). If future index values remain low \(i.e., if the index is responding to a change in abundance, rather than interannual variability\), then the predicted trend in SSB may change abruptly in a future assessment. Such an abrupt change may lead to an increase in the retrospective pattern.} \end{itemize}{}} \def\REDUNITRefr{ \textbf{References: }{} \linebreak{}Northeast Fisheries Science Center. 2008. Assessment of 19 Northeast Groundfish Stocks through 2007: Report of the 3$^{rd}$ Groundfish Assessment Review Meeting \(GARM III\), Northeast Fisheries Science Center, Woods Hole, Massachusetts, August 4\-8, 2008. US Dept Commer, Northeast Fish Sci Cent Ref Doc. 08\-15\; 884 p + xvii. Available from: National Marine Fisheries Service, 166 Water Street, Woods Hole, MA 02543\-1026, or online at http:\/\/www.nefsc.noaa.gov\/nefsc\/publications\/ \linebreak{} \linebreak{}Northeast Fisheries Science Center. 2012. Assessment or Data Updates of 13 Northeast Groundfish Stocks through 2010. US Dept Commer, Northeast Fish Sci Cent Ref Doc. 12\-06\; 789 p. Available from: National Marine Fisheries Service, 166 Water Street, Woods Hole, MA 02543\-1026, or online at http:\/\/www.nefsc.noaa.gov\/nefsc\/publications\/} \def\REDUNITDraft{} \def\REDUNITSPPname{Acadian redfish} \def\REDUNITSPPnameT{Acadian redfish} \def\REDUNITRptYr{2015} \def\REDUNITAuthor{Brian Linton} \def\REDUNITReviewerComments{/home/dhennen/EIEIO/BigReport/RED_UNIT/latex}  \def\CATUNITMyPathTab{/home/dhennen/EIEIO/BigReport/CAT_UNIT/tables} \def\CATUNITMyPathFig{/home/dhennen/EIEIO/BigReport/CAT_UNIT/figures} \def\CATUNITfigFishCap{Total catch of Atlantic wolffish between 1968 and 2014 by fleet \(commercial and recreational\)  and disposition \(landings and discards\). Note that a no possession limit was put in place in May 2010.} \def\CATUNITfigSSBCap{Trends in spawning stock biomass of Atlantic wolffish between 1968 and 2014 from the current  \(solid line\)  and previous \(dashed line\)  assessment and the corresponding  \$SSB\_{Threshold}\${} \(\$\dfrac{1}{2}\${} \$SSB\_{MSY}\${} \textit{proxy}{}\; horizontal dashed line\)  as well as  \$SSB\_{Target}\${} \(\$SSB\_{MSY}\${} \textit{proxy}{}\; horizontal dotted line\)   based on the current assessment.} \def\CATUNITfigFCap{Trends in the fully selected fishing mortality \(\$F\_{Full}\${}\)  of Atlantic wolffish between 1968 and 2014 from the current  \(solid line\)  and previous \(dashed line\)  assessment and the corresponding  \$F\_{Threshold}\${} \(\$F\_{MSY}\${} \textit{proxy}{}=0.243\; horizontal dashed line\). } \def\CATUNITfigRecrCap{Trends in age 1 recruits of Atlantic wolffish between 1968 and 2014 from the current \(solid line\)  and previous \(dashed line\)  assessment.} \def\CATUNITfigSurvCap{Indices of biomass for the Atlantic wolffish between 1968 and 2015 for the Northeast Fisheries Science Center \(NEFSC\)  spring and fall bottom trawl surveys, and the Massachusetts Division of Marine Fisheries \(MADMF\)  spring bottom trawl survey. The approximate 90\% lognormal confidence intervals are shown. NEFSC indices for 2009\-2015 are calibrated using the ocean pout coefficient from Miller et al. \(2010\).} \def\CATUNITPreAmb{This assessment of the Atlantic wolffish \(\textit{Anarhichas lupus}\)  stock is an update of the existing 2012 operational assessment \(NEFSC 2012\). Based on the previous assessment the stock was overfished, but overfishing was not occurring. This assessment updates commercial fishery catch data, research survey indices of abundance, and the analytical assessment models and reference points through 2014.} \def\CATUNITSoS{ \textbf{State of Stock: }{}Based on this updated assessment, the Atlantic wolffish \(\textit{Anarhichas lupus}\)  stock is overfished and overfishing is not occurring \(Figures \ref{CATUNITSSB\_plot1}\-\ref{CATUNITF\_plot1}\){}. Retrospective adjustments were not made to the model results. Spawning stock biomass \(SSB\)  in 2014 was estimated to be 638 \(mt\)  which is 38\% of the biomass target \(\$SSB\_{MSY}\${} \textit{proxy}{} = 1,663\;  Figure \ref{CATUNITSSB\_plot1}{}\).  The 2014 fully selected fishing mortality was estimated to be 0.003 which is 1\% of the overfishing threshold proxy \(\$F\_{MSY}\${} \textit{proxy}{} = 0.243\;  Figure \ref{CATUNITF\_plot1}{}\).} \def\CATUNITProj{} \def\CATUNITSpecCmt{ \textbf{Special Comments: } \begin{itemize}{} \item{}What are the most important sources of uncertainty in this stock assessment?  Explain, and describe qualitatively how they affect the assessment results \(such as estimates of biomass, F, recruitment, and population projections\).  \linebreak{} \hspace\*{0.5cm} \textit{The primary sources of uncertainty are the use of the ocean pout calibration coefficient, and the change to a no possession limit in May 2010. The ocean pout calibration coefficient \(4.575\)  is one of the largest for any species \(Miller et al. 2010\), and results in lower biomass estimates. The change to a no possession limit places greater importance on discard mortality. Additionally, it is unclear whether the lack of a recruitment index since 2004 is due to an actual decrease in recruitment, or a change in catchability resulting from the increase in liner mesh size associated with the switch to the Bigelow. Other sources of uncertainty were identified in previous Atlantic wolffish assessments \(NDPSWG 2009, NEFSC 2012\): the surveys may have reached the limit of wolffish detectability due to the decline in abundance\; and the lack of commercial length information results in model estimation difficulties for fishery selectivity.}  \item{} Does this assessment model have a retrospective pattern? If so, is the pattern minor, or major? \(A major retrospective pattern occurs when the adjusted SSB or  \$F\_{Full}\${} lies outside of the approximate  joint confidence region for SSB and  \$F\_{Full}\${}\; see  Figure \ref{RhoDecision\_tab}{}\). \linebreak{} \hspace\*{0.5cm} \textit{This assessment has retrospective patterns with Mohn\'s rho = 0.83 for SSB and \-0.36 for F. Confidence intervals are not available because MCMC is not fully developed for the SCALE model. Thus, retrospective adjustments were not done for this assessment.}  \item{}Based on this stock assessment, are population projections well determined or uncertain? \linebreak{} \hspace\*{0.5cm} \textit{Population projections for Atlantic wolffish were not done. Due to the uncertainties in the assessment, the Northeast Data Poor Stocks Working Group \(NDPSWG 2009\)  concluded that stock projections would be unreliable and should not be conducted.}  \item{}Describe any changes that were made to the current stock assessment, beyond incorporating additional years of data  and the affect these changes had on the assessment and stock status. \linebreak{} \hspace\*{0.5cm} \textit{Commercial discards for the entire time series were revised assuming 8\% discard mortality based on a recent study by Grant and Hiscock \(2014\). A sensitivity run with the revised discard estimates was presented to the Peer Review Panel during the 2015 Operational Assessments. This became the accepted run. There was no change in stock status resulting from the adoption of the 8\% discard mortality run. \linebreak{} \hspace\*{0.5cm}Recreational landings for the entire time series were revised due to an updated grand mean, and the MRFSS\/MRIP calibration for 1981\-2003. This had a negligible effect on the assessment, and there was no change in stock status.}  \item{}If the stock status has changed a lot since the previous assessment, explain why this occurred.  \linebreak{} \hspace\*{0.5cm} \textit{Stock status has not changed since the previous assessment.}  \item{}Indicate what data or studies are currently lacking and which would be needed most to improve this stock assessment in the future.  \linebreak{} \hspace\*{0.5cm} \textit{The Atlantic wolffish maturity study in the Gulf of Maine is ongoing. Increased sample size since the previous assessment allowed the use of a revised knife edge maturity of 50 cm in this assessment. Continued histological sampling over the next several years should allow for the development of a definitive maturity ogive that can be used in the next assessment.}  \item{}Are there other important issues? \linebreak{} \hspace\*{0.5cm} \textit{Recruitment at the end of the time series increases toward the initial recruitment estimate \(Table 1\; Figure 3\)  because there is no information in the model to inform these estimates. There is no indication in the data that recruitment has increased recently.  \linebreak{} \hspace\*{0.5cm}Approximate 90\% lognormal confidence intervals are not shown in Figures 1\-3 because MCMC is not fully developed for the SCALE model.} \end{itemize}{}} \def\CATUNITRefr{ \textbf{References: }{} \linebreak{} \linebreak{}Grant SM, Hiscock W. 2014. Post\-capture survival of Atlantic wolffish  \(\textit{Anarhichas lupus}\)  captured by bottom otter trawl: Can live release programs contribute to the recovery of species at risk? Fish Res 151:169\-176 \linebreak{} \linebreak{}Miller TJ, Das C, Politis PJ, Miller AS, Lucey SM, Legault CM, Brown RW, Rago PJ. 2010. Estimation of Albatross IV to Henry B. Bigelow calibration factors. US Dep Commer, Northeast Fish Sci Cent Ref Doc. 10\-05\; 233 p. http:\/\/www.nefsc.noaa.gov\/publications\/crd\/crd1005\/ \linebreak{} \linebreak{}Northeast Fisheries Science Center \(NEFSC\). 2012. Assessment or Data Updates of 13 Northeast Groundfish Stocks through 2010. US Dep Commer, Northeast Fish Sci Cent Ref Doc. 12\-06\; 789 p. http:\/\/www.nefsc.noaa.gov\/publications\/crd\/crd1206\/ \linebreak{} \linebreak{}Northeast Data Poor Stocks Working Group \(NDPSWG\). 2009. The Northeast Data Poor Stocks Working Group Report, December 8\-12, 2008 Meeting. Part A. Skate species complex, deep sea red crab, Atlantic wolffish, scup, and black sea bass. US Dept Commer, Northeast Fish Sci Cent Ref Doc. 09\-02\; 496 p. http:\/\/www.nefsc.noaa.gov\/publications\/crd\/crd0902\/ \linebreak{} \linebreak{}} \def\CATUNITDraft{} \def\CATUNITSPPname{Atlantic wolffish} \def\CATUNITSPPnameT{Atlantic wolffish} \def\CATUNITRptYr{2015} \def\CATUNITAuthor{Charles Adams} \def\CATUNITReviewerComments{/home/dhennen/EIEIO/BigReport/CAT_UNIT/latex} 
\input{./preamble}


\begin{document}

\pagenumbering{roman}
\setcounter{page}{1}  %beginning page number

%--------Here are the Ref. Doc coverpages -------

\input{crd_front_page}
\input{crd_second_page}
\input{crd_third_page}

%-------------- Table of Contents ----------------
%\section*{Table of Contents}
\tableofcontents
\clearpage
\pagenumbering{arabic}

%\input{def2.tex}
%%%%%%%%%%%%%%%%%%%%%%%%%%%%%%%%%%%%%%%%%%%%%%%%%%%%%%%%%%%%%%%%%%%%%%%%%%%%%%%%%%%%%%%%%%%%%%%%%%%%%%%%%%%%
%Executive Summary
%\newcommand{\ExSumPath}{/net/home2/dhennen/testEIEIO/BigReport/ExSum}
\newcommand{\ExSumPath}{../ExSum}
\input{\ExSumPath/latex/ExecSumm}
%%%%%%%%%%%%%%%%%%%%%%%%%%%%%%%%%%%%%%%%%%%%%%%%%%%%%%%%%%%%%%%%%%%%%%%%%%%%%%%%%%%%%%%%%%%%%%%%%%%%%%%%%%%%
\input{BigCall5.tex}
%\input{BigCall.tex}
%\input{BigCall2.tex}
%\input{BigCall3.tex}
%\input{BigCall4.tex}
%%%%%%%%%%%%%%%%%%%%%%%%%%%%%%%%%%%%%%%%%%%%%%%%%%%%%%%%%%%%%%%%%%%%%%%%%%%%%%%%%%%%%%%%%%%%%%%%%%%%%%%%%%%%
%GOMCOD
\input{\CODGMReviewerComments/BigReportCODGM}
%%%%%%%%%%%%%%%%%%%%%%%%%%%%%%%%%%%%%%%%%%%%%%%%%%%%%%%%%%%%%%%%%%%%%%%%%%%%%%%%%%%%%%%%%%%%%%%%%%%%%%%%%%%%
%GBCOD
\input{\CODGBReviewerComments/BigReportCODGB}
%%%%%%%%%%%%%%%%%%%%%%%%%%%%%%%%%%%%%%%%%%%%%%%%%%%%%%%%%%%%%%%%%%%%%%%%%%%%%%%%%%%%%%%%%%%%%%%%%%%%%%%%%%%%
%GBHAD
\input{\HADGBReviewerComments/BigReportHADGB}
%%%%%%%%%%%%%%%%%%%%%%%%%%%%%%%%%%%%%%%%%%%%%%%%%%%%%%%%%%%%%%%%%%%%%%%%%%%%%%%%%%%%%%%%%%%%%%%%%%%%%%%%%%%%
%GOMHAD
\input{\HADGMReviewerComments/BigReportHADGM}
%%%%%%%%%%%%%%%%%%%%%%%%%%%%%%%%%%%%%%%%%%%%%%%%%%%%%%%%%%%%%%%%%%%%%%%%%%%%%%%%%%%%%%%%%%%%%%%%%%%%%%%%%%%%
%CCGMYEL
\input{\YELCCGMReviewerComments/BigReportYELCCGM}
%%%%%%%%%%%%%%%%%%%%%%%%%%%%%%%%%%%%%%%%%%%%%%%%%%%%%%%%%%%%%%%%%%%%%%%%%%%%%%%%%%%%%%%%%%%%%%%%%%%%%%%%%%%%
%SNEMAYEL
\input{\YELSNEMAReviewerComments/BigReportYELSNEMA}
%%%%%%%%%%%%%%%%%%%%%%%%%%%%%%%%%%%%%%%%%%%%%%%%%%%%%%%%%%%%%%%%%%%%%%%%%%%%%%%%%%%%%%%%%%%%%%%%%%%%%%%%%%%%
%FLWGB
\input{\FLWGBReviewerComments/BigReportFLWGB}
%%%%%%%%%%%%%%%%%%%%%%%%%%%%%%%%%%%%%%%%%%%%%%%%%%%%%%%%%%%%%%%%%%%%%%%%%%%%%%%%%%%%%%%%%%%%%%%%%%%%%%%%%%%%
%FLWSNEMA
\input{\FLWSNEMAReviewerComments/BigReportFLWSNEMA}
%%%%%%%%%%%%%%%%%%%%%%%%%%%%%%%%%%%%%%%%%%%%%%%%%%%%%%%%%%%%%%%%%%%%%%%%%%%%%%%%%%%%%%%%%%%%%%%%%%%%%%%%%%%%
%PLAUNIT
\input{\PLAUNITReviewerComments/BigReportPLAUNIT}
%%%%%%%%%%%%%%%%%%%%%%%%%%%%%%%%%%%%%%%%%%%%%%%%%%%%%%%%%%%%%%%%%%%%%%%%%%%%%%%%%%%%%%%%%%%%%%%%%%%%%%%%%%%%
%WITUNIT
\input{\WITUNITReviewerComments/BigReportWITUNIT}
%%%%%%%%%%%%%%%%%%%%%%%%%%%%%%%%%%%%%%%%%%%%%%%%%%%%%%%%%%%%%%%%%%%%%%%%%%%%%%%%%%%%%%%%%%%%%%%%%%%%%%%%%%%%
%REDUNIT
\input{\REDUNITReviewerComments/BigReportREDUNIT}
%%%%%%%%%%%%%%%%%%%%%%%%%%%%%%%%%%%%%%%%%%%%%%%%%%%%%%%%%%%%%%%%%%%%%%%%%%%%%%%%%%%%%%%%%%%%%%%%%%%%%%%%%%%%
%HKWUNIT
\input{\HKWUNITReviewerComments/BigReportHKWUNIT}
%%%%%%%%%%%%%%%%%%%%%%%%%%%%%%%%%%%%%%%%%%%%%%%%%%%%%%%%%%%%%%%%%%%%%%%%%%%%%%%%%%%%%%%%%%%%%%%%%%%%%%%%%%%%
%POKUNIT
\input{\POKUNITReviewerComments/BigReportPOKUNIT}
%%%%%%%%%%%%%%%%%%%%%%%%%%%%%%%%%%%%%%%%%%%%%%%%%%%%%%%%%%%%%%%%%%%%%%%%%%%%%%%%%%%%%%%%%%%%%%%%%%%%%%%%%%%%
%CATUNIT
\input{\CATUNITReviewerComments/BigReportCATUNIT}
%%%%%%%%%%%%%%%%%%%%%%%%%%%%%%%%%%%%%%%%%%%%%%%%%%%%%%%%%%%%%%%%%%%%%%%%%%%%%%%%%%%%%%%%%%%%%%%%%%%%%%%%%%%%
%HALUNIT
\input{\HALUNITReviewerComments/BigReportHALUNIT}
%%%%%%%%%%%%%%%%%%%%%%%%%%%%%%%%%%%%%%%%%%%%%%%%%%%%%%%%%%%%%%%%%%%%%%%%%%%%%%%%%%%%%%%%%%%%%%%%%%%%%%%%%%%%
%FLDGMGB
\input{\FLDGMGBReviewerComments/BigReportFLDGMGB}
%%%%%%%%%%%%%%%%%%%%%%%%%%%%%%%%%%%%%%%%%%%%%%%%%%%%%%%%%%%%%%%%%%%%%%%%%%%%%%%%%%%%%%%%%%%%%%%%%%%%%%%%%%%%
%FLDSNEMA
\input{\FLDSNEMAReviewerComments/BigReportFLDSNEMA}
%%%%%%%%%%%%%%%%%%%%%%%%%%%%%%%%%%%%%%%%%%%%%%%%%%%%%%%%%%%%%%%%%%%%%%%%%%%%%%%%%%%%%%%%%%%%%%%%%%%%%%%%%%%%
%OPTUNIT
\input{\OPTUNITReviewerComments/BigReportOPTUNIT}
%%%%%%%%%%%%%%%%%%%%%%%%%%%%%%%%%%%%%%%%%%%%%%%%%%%%%%%%%%%%%%%%%%%%%%%%%%%%%%%%%%%%%%%%%%%%%%%%%%%%%%%%%%%%
%FLWGM
\input{\FLWGMReviewerComments/BigReportFLWGM}
%%%%%%%%%%%%%%%%%%%%%%%%%%%%%%%%%%%%%%%%%%%%%%%%%%%%%%%%%%%%%%%%%%%%%%%%%%%%%%%%%%%%%%%%%%%%%%%%%%%%%%%%%%%%
%GBYEL
\input{\YELGBReviewerComments/BigReportYELGB}








\end{document}; ls -al;
%\def\YELCCGMReviewerComments{/home/dhennen/EIEIO/BigReport/YEL_CCGM/latex}   \def\FLDSNEMAReviewerComments{/home/dhennen/EIEIO/BigReport/FLD_SNEMA/latex}  \def\PLAUNITMyPathTab{/home/dhennen/EIEIO/BigReport/PLA_UNIT/tables} \def\PLAUNITMyPathFig{/home/dhennen/EIEIO/BigReport/PLA_UNIT/figures} \def\PLAUNITfigFishCap{Total catch of Gulf of Maine\-Georges Bank American Plaice between 1980 and 2015 by fleet \(Gulf of Maine, Georges Bank, Southern New England, and Canadian\)  and disposition \(landings and discards\).} \def\PLAUNITfigSSBCap{Trends in spawning stock biomass of Gulf of Maine\-Georges Bank American Plaice between 1980 and 2015 from the current  \(solid line\)  and previous \(dashed line\)  assessment and the corresponding  \$SSB\_{Threshold}\${} \(\$\dfrac{1}{2}\${} \$SSB\_{MSY}\${} \textit{proxy}{}\; horizontal dashed line\)  as well as  \$SSB\_{Target}\${} \(\$SSB\_{MSY}\${} \textit{proxy}{}\; horizontal dotted line\)   based on the 2015 assessment.  Biomass was adjusted for a retrospective pattern  and the adjustment is shown in red.  The approximate 90\% normal confidence intervals are shown.} \def\PLAUNITfigFCap{Trends in the fully selected fishing mortality \(\$F\_{Full}\${}\)  of Gulf of Maine\-Georges Bank American Plaice between 1980 and 2015 from the current  \(solid line\)  and previous \(dashed line\)  assessment and the corresponding  \$F\_{Threshold}\${} \(\$F\_{MSY}\${} \textit{proxy}{}=0.196\; horizontal dashed line\).  \$F\_{Full}\${} was adjusted for a retrospective pattern  and the adjustment is shown in red,  based on the 2015 assessment. The approximate 90\% normal confidence intervals are shown.} \def\PLAUNITfigRecrCap{Trends in Recruits \(age 1\)  \(000s\)  of Gulf of Maine\-Georges Bank American Plaice between 1980 and 2015 from the current \(solid line\)  and previous \(dashed line\)  assessment.} \def\PLAUNITfigSurvCap{Indices of biomass for the Gulf of Maine\-Georges Bank American Plaice between 1963 and 2015 for the Northeast Fisheries Science Center \(NEFSC\)  and Massachusetts Division of Marine Fisheries \(MADMF\)  spring and autumn research bottom trawl surveys.  The approximate 90\% normal confidence intervals are shown.} \def\PLAUNITPreAmb{This assessment of the Gulf of Maine\-Georges Bank American Plaice \(\textit{Hippoglossoides platessoides}\)  stock is an operational update of the existing 2012 benchmark assessment \(O\'Brien et al. 2012\). Based on the previous assessment the stock was not overfished, and overfishing was not ocurring. This 2015 assessment updates commercial fishery catch data, research survey indices of abundance, the analytical VPA assessment model, and reference points through 2014. Additionally, stock projections have been updated through 2018.} \def\PLAUNITSoS{ \textbf{State of Stock: }{}Based on this updated assessment, the Gulf of Maine\-Georges Bank American Plaice \(\textit{Hippoglossoides platessoides}\)  stock is not overfished and overfishing is not occurring \(Figures \ref{PLAUNITSSB\_plot1}\-\ref{PLAUNITF\_plot1}\){}.  Retrospective adjustments were made to the model results.  Spawning stock biomass \(SSB\)  in 2014 was estimated to be 10,915 mt which is 83\% of the biomass target for this stock \(\$SSB\_{MSY}\${} \textit{proxy}{} = 13,107\;  Figure \ref{PLAUNITSSB\_plot1}{}\). The 2014 fully selected fishing mortality was estimated to be 0.118 which is 60\% of the overfishing threshold proxy \(\$F\_{MSY}\${} \textit{proxy}{} = 0.196\;  Figure \ref{PLAUNITF\_plot1}{}\).} \def\PLAUNITProj{ \textbf{Projections: }{}Short term projections of biomass were derived by sampling from an empirical cumulative  distribution  function of 34 recruitment estimates from VPA model results. The annual fishery selectivity, maturity ogive, and mean weights at age used in projections are the most recent 5 year averages\;  retrospective adjustments were applied in the projections.} \def\PLAUNITSpecCmt{ \textbf{Special Comments: } \begin{itemize}{} \item{}What are the most important sources of uncertainty in this stock assessment?  Explain, and describe qualitatively how they affect the assessment results \(such as estimates of biomass, F, recruitment, and population projections\).  \linebreak{} \hspace\*{0.5cm} \textit{A source of uncertainty in this assessment are the estimates of historical landings at age, prior to 1984, and the magnitude of  historical discards, prior to 1989. Both of these affect the scale of the biomass and fishing mortality estimates, and influence reference point estimations.}  \item{} Does this assessment model have a retrospective pattern? If so, is the pattern minor, or major? \(A major retrospective pattern occurs when the adjusted SSB or  \$F\_{Full}\${} lies outside of the approximate  joint confidence region for SSB and  \$F\_{Full}\${}\; see  Figure \ref{RhoDecision\_tab}{}\). \linebreak{} \hspace\*{0.5cm} \textit{ The 7\-year Mohn\'s  \textrho{}, relative to SSB, was 0.63 in the 2012 assessment and was 0.32 in 2014. The 7\-year Mohn\'s  \textrho{}, relative to F, was \-0.35 in the 2012 assessment and was 0.32 in 2014. There was a major retrospective pattern for this assessment because the  \textrho{} adjusted estimates of 2014 SSB \(\$SSB\_{\rho}\${}=10,915\)  and 2014 F \(\$F\_{\rho}\${}=0.118\)  were outside the approximate 90\% confidence regions around SSB \(12,742 \- 16,439\)  and F \(0.069 \- 0.093\).  A retrospective  adjustment was made for both the determination of stock status and for projections of catch in 2016. The retrospective adjustment changed the 2014 SSB from 14,543 to 10,915 and the 2014  \$F\_{Full}\${} from 0.08 to 0.118.}  \item{}Based on this stock assessment, are population projections well determined or uncertain? \linebreak{} \hspace\*{0.5cm} \textit{Population projections for Gulf of Maine\-Georges Bank American Plaice are reasonably well determined.}  \item{}Describe any changes that were made to the current stock assessment, beyond incorporating additional years of data  and the effect these changes had on the assessment and stock status. \linebreak{} \hspace\*{0.5cm} \textit{ No major changes, other than the addition of recent years of data, were made to the Gulf of Maine\-Georges Bank American Plaice assessment for this update. A new version of VPA was used \(V3.3.0\)  which gave very similar results to the 2012 VPA 3.1.0 run, with the same F and slightly lower SSB. The MADMF spring and autumn survey indices were re\-estimated for the time series, accounting for revised stratum areas. The revision occurred in 2007, but was overlooked in the 2012 assessment. A comparison of 2010 terminal year VPAs indicated minimal differences in 2010 SSB \(now slightly lower\)  and no change in F.}  \item{}If the stock status has changed a lot since the previous assessment, explain why this occurred.  \linebreak{} \hspace\*{0.5cm} \textit{As in recent assessments for Gulf of Maine\-Georges Bank American Plaice the stock status remains as not overfished and overfishing not occurring.}  \item{}Indicate what data or studies are currently lacking and which would be needed most to improve this stock assessment in the future.  \linebreak{} \hspace\*{0.5cm} \textit{The Gulf of Maine\-Georges Bank American Plaice assessment could be improved with updated studies on growth of Georges Bank and Gulf of Maine fish.}  \item{}Are there other important issues? \linebreak{} \hspace\*{0.5cm} \textit{A difference in growth between GM and GB fish has been documented, however, historical catch data information for GB may not be sufficient to conduct a separate assessment. Also, the growth difference may not persist in the most recent years. This could all be explored further in an benchmark review.} \end{itemize}{}} \def\PLAUNITRefr{ \textbf{References: }{} \linebreak{}O\'Brien, L. and J. Dayton \(2012\). E. Gulf of Maine \- Georges Bank American plaice Assessment for 2012 in Northeast Fisheries Science Center, 2012, Assessment or Data Updates of 13 Northeast Groundfish Stocks through 2010. US Dept Commer, Northeast Fish Sci Cent Ref Doc. 12\-06\; 789 p. http:\/\/www.nefsc.noaa.gov\/publications\/crd\/crd1206\/. \linebreak{} \linebreak{}} \def\PLAUNITDraft{} \def\PLAUNITSPPname{Gulf of Maine-Georges Bank American Plaice} \def\PLAUNITSPPnameT{Gulf of Maine-Georges Bank American Plaice} \def\PLAUNITRptYr{2015} \def\PLAUNITAuthor{Loretta O\'Brien} \def\PLAUNITReviewerComments{/home/dhennen/EIEIO/BigReport/PLA_UNIT/latex}  \def\WITUNITMyPathTab{/home/dhennen/EIEIO/BigReport/WIT_UNIT/tables} \def\WITUNITMyPathFig{/home/dhennen/EIEIO/BigReport/WIT_UNIT/figures} \def\WITUNITfigFishCap{Total catch of witch flounder between 1982 and 2014 by fleet \(commercial\)  and disposition \(landings and discards\).} \def\WITUNITfigSSBCap{Trends in spawning stock biomass \(mt\)  of witch flounder between 1982 and 2014 from the current  \(solid line\)  and previous \(dashed line\)  assessment and the corresponding  \$SSB\_{Threshold}\${} \(\$\dfrac{1}{2}\${} \$SSB\_{MSY}\${}\; horizontal dashed line\)  as well as  \$SSB\_{Target}\${} \$SSB\_{MSY}\${}\; horizontal dotted line\)   based on the current assessment. Red solid vertical line indicates rho adjusted SSB. Black solid vertical line indicates 90\% confidence interval for 2014.} \def\WITUNITfigFCap{Trends in the fully selected fishing mortality \(\$F\_{Full}\${}\)  of witch flounder between 1982 and 2014 from the current  \(solid line\)  and previous \(dashed line\)  assessment and the corresponding  \$F\_{Threshold}\${} \(\$F\_{MSY}\${}=0.279\; horizontal dashed line\)  based on the current assessment.  Red solid vertical line indicates rho adjusted  \$F\_{Full}\${}. Black solid vertical line indicates 90\% confidence interval for 2014.} \def\WITUNITfigRecrCap{Trends in Age 3  \(000s\)  of witch flounder between 1982 and 2014 from the current \(solid line\)  and previous \(dashed line\)  assessment.} \def\WITUNITfigSurvCap{Indices of biomass \(kg\/tow\)  for the witch flounder between 1963 and 2015 for the Northeast Fisheries Science Center \(NEFSC\)  spring and fall bottom trawl surveys.  The 90\% lognormal confidence intervals are shown.} \def\WITUNITPreAmb{This assessment of the witch flounder \(\textit{Glyptocephalus cynoglossus}\)  stock is an operational update of the 2012 assessment \(NEFSC 2012\)  and the 2008 benchmark assessment \(NEFSC 2008\). This assessment updates commercial fishery catch data, research survey indices, and the analytical assessment model through 2014. Additionally, stock projections have been updated through 2018. Reference points have been updated. } \def\WITUNITSoS{ \textbf{State of Stock: }{}witch flounder \(\textit{Glyptocephalus cynoglossus}\)  stock is overfished and overfishing is occurring \(Figures \ref{WITUNITSSB\_plot1}\-\ref{WITUNITF\_plot1}\){}. Retrospective adjustments were made to the model results.  Spawning stock biomass \(SSB\)  in 2014 was estimated to be 2,077 \(mt\)  which is 22\% of the  \$SSB\_{MSY}\${} proxy \(9,473\;  Figure \ref{WITUNITSSB\_plot1}{}\).  The 2014 fully selected fishing mortality was estimated to be 0.687 which is 246\% of the  \$F\_{MSY}\${} proxy \(0.279\;  Figure \ref{WITUNITF\_plot1}{}\). A retrospective adjustment to  \$F\_{Full}\${} and SSB in 2014 was required but did not lead to a change in status.  } \def\WITUNITProj{ \textbf{Projections: }{}Short term projection recruitment was sampled from a cumulative distribution function derived from ADAPT VPA \(with split time series between 1994 and 1995\)  estimated age 3 recruitment between 1982 and 2013.  Average 2010\-2014 partial recruitment, average 2010\-2014 mean weights, and maturation ogive representing 2011\-2015 maturity data were used.} \def\WITUNITSpecCmt{ \textbf{Special Comments: } \begin{itemize}{} \item{}What are the most important sources of uncertainty in this stock assessment?  Explain, and describe qualitatively how they affect the assessment results \(such as estimates of biomass, F, recruitment, and population projections\).  \linebreak{} \hspace\*{0.5cm} \textit{An important source of uncertainty is the retrospective pattern where fishing mortality is underestimated and spawning stock biomass and recruitment are overestimated. }  \item{} Does this assessment model have a retrospective pattern? If so, is the pattern minor, or major? \(A major retrospective pattern occurs when the adjusted SSB or  \$F\_{Full}\${} lies outside of the approximate  joint confidence region for SSB and  \$F\_{Full}\${}\).  \linebreak{} \hspace\*{0.5cm} \textit{ The 7\-year Mohn\'s  \textrho{}, relative to SSB, was 0.61 in the 2012 assessment and was 0.51 in 2014. The 7\-year Mohn\'s  \textrho{}, relative to F, was \-0.33 in the 2012 assessment and was \-0.38 in 2014. There was a major retrospective pattern for this assessment because the  \textrho{} adjusted estimates of 2014 SSB \(\$SSB\_{\rho}\${}=2,077\)  and 2014 F \(\$F\_{\rho}\${}=0.687\)  were outside the approximate 90\% confidence regions around SSB \(2,643 \- 3,864\)  and F \(0.321 \- 0.603\).  A retrospective  adjustment was made for both the determination of stock status and for projections of catch in 2016. The retrospective adjustment changed the 2014 SSB from 3,129 to 2,077 and the 2014  \$F\_{Full}\${} from 0.428 to 0.687.}  \item{}Based on this stock assessment, are population projections well determined or uncertain? \linebreak{} \hspace\*{0.5cm} \textit{Population projections for witch flounder appear to be optimistic\; the projected rho adjusted biomass from the last assessment  was above the upper confidence bounds of the projected rho adjusted biomass estimated in the current assessment. }  \item{}Describe any changes that were made to the current stock assessment, beyond incorporating additional years of data  and the effect these changes had on the assessment and stock status.  \linebreak{} \hspace\*{0.5cm} \textit{TOGA \(Type, Operation, Gear, Acquisition\)  values were used for haul criteria for NEFSC surveys for 2009 onward and minor changes in the use of observer data for discard estimates were made to the current witch flounder assessment. These changes had negligible effect on the assessment and stock status.  }  \item{}If the stock status has changed a lot since the previous assessment, explain why this occurred.  \linebreak{} \hspace\*{0.5cm} \textit{No change in stock status has occurred for witch flounder since the previous assessment. }  \item{}Indicate what data or studies are currently lacking and which would be needed most to improve this stock assessment in the future.  \linebreak{} \hspace\*{0.5cm} \textit{Extensive studies have examined the causes of retrospective patterns with no definitive conclusions other than a change in model does not resolve the issue. }  \item{}Are there other important comments? \linebreak{} \hspace\*{0.5cm} \textit{The VPA analysis was performed with survey time series split between 1994 and 1995. This time split corresponds to changes in the commercial reporting methods as well as other regulatory management changes.  } \end{itemize}{}} \def\WITUNITRefr{ \textbf{References: }{} \linebreak{}Northeast Fisheries Science Center. 2008. Assessment of 19 Northeast Groundfish Stocks through 2007: Report of the 3$^{rd}$ Groundfish Assessment Review Meeting \(GARM III\), Northeast Fisheries Science Center, Woods Hole, Massachusetts, August 4\-8, 2008. US Dep Commer, NOAA Fisheries, Northeast Fish Sci Cent Ref Doc. 08\-15\; 884 p + xvii. http:\/\/www.nefsc.noaa.gov\/publications\/crd\/crd0815\/ \linebreak{} \linebreak{}Northeast Fisheries Science Center. 2012. Assessment or Data Updates of 13 Northeast Groundfish Stocks through 2010.  US Dep Commer, NOAA Fisheries, Northeast Fish Sci Cent Ref Doc. 12\-06\; 789 p. http:\/\/www.nefsc.noaa.gov\/publications\/crd\/crd1206\/ \linebreak{} \linebreak{}} \def\WITUNITDraft{} \def\WITUNITSPPname{witch flounder} \def\WITUNITSPPnameT{Witch flounder} \def\WITUNITRptYr{2015} \def\WITUNITAuthor{Susan Wigley} \def\WITUNITReviewerComments{/home/dhennen/EIEIO/BigReport/WIT_UNIT/latex}  \def\HKWUNITMyPathTab{/home/dhennen/EIEIO/BigReport/HKW_UNIT/tables} \def\HKWUNITMyPathFig{/home/dhennen/EIEIO/BigReport/HKW_UNIT/figures} \def\HKWUNITfigFishCap{Total catch of white hake between 1963 and 2014 by fleet \(commercial, recreational, or Canadian\)  and disposition \(landings and discards\).} \def\HKWUNITfigSSBCap{Trends in spawning stock biomass of white hake between 1963 and 2014 from the current  \(solid line\)  and previous \(dashed line\)  assessment and the corresponding  \$SSB\_{Threshold}\${} \(\$\dfrac{1}{2}\${} \$SSB\_{MSY}\${} \textit{proxy}{}\; horizontal dashed line\)  as well as  \$SSB\_{Target}\${} \(\$SSB\_{MSY}\${} \textit{proxy}{}\; horizontal dotted line\)   based on the 2014 assessment.  The red dot indicates the rho\-adjusted SSB values that would have resulted had a retrospective  adjusment been made \(see Special Comments section\).  The approximate 90\% lognormal confidence intervals are shown.} \def\HKWUNITfigFCap{Trends in the fully selected fishing mortality \(\$F\_{Full}\${}\)  of white hake between 1963 and 2014 from the current  \(solid line\)  and previous \(dashed line\)  assessment and the corresponding  \$F\_{Threshold}\${} \(\$F\_{MSY}\${} \textit{proxy}{}=0.188\; horizontal dashed line\).  The red dot indicates the rho\-adjusted SSB values that would have resulted had a retrospective  adjusment been made \(see Special Comments section\).  The approximate 90\% lognormal confidence intervals are shown.} \def\HKWUNITfigRecrCap{Trends in Recruits \(age 1\)  \(000s\)  of white hake between 1963 and 2014 from the current \(solid line\)  and previous \(dashed line\)  assessment. The approximate 90\% lognormal confidence intervals are shown.} \def\HKWUNITfigSurvCap{Indices of biomass for the white hake between 1963 and 2015 for the Northeast Fisheries Science Center \(NEFSC\)  spring and fall bottom trawl surveys.  The approximate 90\% lognormal confidence intervals are shown.} \def\HKWUNITPreAmb{This assessment of the white hake \(\textit{Urophycis tenuis}\)  stock is an operational update of the existing 2013 benchmark ASAP assessment \(NEFSC 2013\). Based on the previous assessment the stock was not overfished, and overfishing was not ocurring. This assessment updates commercial fishery catch data, research survey indices of abundance, and the ASAP assessment models and reference points through 2014. Additionally, stock projections have been updated through 2018.} \def\HKWUNITSoS{ \textbf{State of Stock: }{}Based on this updated assessment, white hake \(\textit{Urophycis tenuis}\)  stock is not overfished and overfishing is not occurring \(Figures \ref{HKWUNITSSB\_plot1}\-\ref{HKWUNITF\_plot1}\){}. Retrospective adjustments were not made to the model results.  Spawning stock biomass \(SSB\)  in 2014 was estimated to be 28,553 \(mt\)  which is 88\% of the biomass threshold for an overfished stock \(\$SSB\_{MSY}\${} \textit{proxy}{} = 32,550\;  Figure \ref{HKWUNITSSB\_plot1}{}\).  The 2014 fully selected fishing mortality was estimated to be 0.076 which is 40\% of the overfishing threshold proxy \(\$F\_{MSY}\${} \textit{proxy}{} = 0.188\;  Figure \ref{HKWUNITF\_plot1}{}\).} \def\HKWUNITProj{ \textbf{Projections: }{}Short term projections of catch and SSB were derived by sampling from a cumulative  distribution  function of recruitment estimates from ASAP from 1995\-2012. The annual fishery selectivity, maturity ogive, and mean weights at age used in the projection  are the most recent 5 year averages. } \def\HKWUNITSpecCmt{ \textbf{Special Comments: } \begin{itemize}{} \item{}What are the most important sources of uncertainty in this stock assessment?  Explain, and describe qualitatively how they affect the assessment results \(such as estimates of biomass, F, recruitment, and population projections\).  \linebreak{} \hspace\*{0.5cm} \textit{1. Catch at age information is not well characterized due to possible mis\-identification of species in the commercial and sea sampling data, particularly in early years, low sampling of commercial landings in  some years, and sparse discard data particularly in early years.  \linebreak{} \hspace\*{0.5cm}2. Since the commercial catch is aged primarily with survey age\/length keys, there is considerable augmentation required, mainly for ages 5 and older. The numbers at age and mean weights at age in the catch for these ages may therefore not be well specified.  \linebreak{} \hspace\*{0.5cm}3. White hake may move seasonally into and out of the defined stock area.  \linebreak{} \hspace\*{0.5cm}4. There are no commercial catch at age data prior to 1989 and the catchability of older ages in the surveys is very low. This results in a large uncertainty in starting numbers at age.  \linebreak{} \hspace\*{0.5cm}5. Since 2003, dealers have been culling very large fish out of the large category. However, there was no market category to input into the landings until June 2014. The length compositions are distinct from large and have been identified since 2011. This may bias the age composition of the landings, particularly in 2014 when 2000 of the 5000 large samples were these extra\-large fish.  \linebreak{} \hspace\*{0.5cm}6. A pooled age\/length key is used for 1963\-1981, fall 2003 \(second half of commercial key\)  and 2014.Age data were not available for 2014 in time for this assessment. The same pooled key that was used for 1963\-1981 was used for 2014.}  \item{} Does this assessment model have a retrospective pattern? If so, is the pattern minor, or major? \(A major retrospective pattern occurs when the adjusted SSB or  \$F\_{Full}\${} lies outside of the approximate  joint confidence region for SSB and  \$F\_{Full}\${}\; see  Figure \ref{RhoDecision\_tab}{}\). \linebreak{} \hspace\*{0.5cm} \textit{ No retrospective adjustment of spawning stock biomass or fishing mortality in 2014 was required.  The pattern in this assessment is considered minor \(Mohn’s rho of 0.18 on SSB, Mohn’s rho of 0.12 on F\)  with the adjusted SSB within the 90 \% CI of the MCMC. However, the Mohn’s rho for Age 1 estimates is 0.54. This may have an impact on projections if this continues into the future.}  \item{}Based on this stock assessment, are population projections well determined or uncertain? \linebreak{} \hspace\*{0.5cm} \textit{Population projections for white hake, are not well determined and projected biomass from the last assessment  was outside the confidence bounds of the biomass estimated in the current assessment. }  \item{}Describe any changes that were made to the current stock assessment, beyond incorporating additional years of data  and the affect these changes had on the assessment and stock status. \linebreak{} \hspace\*{0.5cm} \textit{ The 2011 catch\-at\-length and age were re\-estimated for both landings and discards. For the  landings, two samples were adjusted for dorsal length to total length that had been missed in the previous assessment.}  \item{}If the stock status has changed a lot since the previous assessment, explain why this occurred.  \linebreak{} \hspace\*{0.5cm} \textit{While stock status of white hake has not changed, the stock has not rebuilt as the projections from the last assessment indicated. This is due to the retrospective in recruitment. The numbers for the 2005\-2009 year classes, which were included in the age 2\-6 starting numbers in the projections, were over\-estimated which led to over\-estimating SSB in 2014.}  \item{}Indicate what data or studies are currently lacking and which would be needed most to improve this stock assessment in the future.  \linebreak{} \hspace\*{0.5cm} \textit{ Age structures from the observer program are available and should be aged to augment  the survey keys. There is a also a new market category for heads and age structures could be  acquired from these is an otolith length\/total length relationship can be established. }  \item{}Are there other important issues? \linebreak{} \hspace\*{0.5cm} \textit{None. } \end{itemize}{}} \def\HKWUNITRefr{ \textbf{References: }{} \linebreak{} NEFSC. 2013. 56$^{th}$ Northeast Regional Stock Assessment Workshop \(56$^{th}$ SAW\)  Assessment  Report.US Dep Commer, NOAA Fisheries, Northeast Fish Sci Cent Ref Doc. 13\-10\; 868 p.  http:\/\/www.nefsc.noaa.gov\/publications\/crd\/crd1310\/  \linebreak{} \linebreak{}} \def\HKWUNITDraft{} \def\HKWUNITSPPname{white hake} \def\HKWUNITSPPnameT{White hake} \def\HKWUNITRptYr{2015} \def\HKWUNITAuthor{Katherine Sosebee} \def\HKWUNITReviewerComments{/home/dhennen/EIEIO/BigReport/HKW_UNIT/latex}  \def\OPTUNITMyPathTab{/home/dhennen/EIEIO/BigReport/OPT_UNIT/tables} \def\OPTUNITMyPathFig{/home/dhennen/EIEIO/BigReport/OPT_UNIT/figures} \def\OPTUNITfigFishCap{Total catch of ocean pout  between 1968 and 2014 by fleet \(US and Other\)  and disposition \(landings and discards\).} \def\OPTUNITfigSSBCap{Trends in biomass \(kg\/tow\)  of ocean pout  between 1968 and 2014 from the current  \(solid line\)  and previous \(dashed line\)  assessment and the corresponding  \$B\_{Threshold}\${} \(\$\dfrac{1}{2}\${} \$B\_{MSY}\${} \textit{proxy}{}\; horizontal dashed line\)  as well as  \$B\_{Target}\${} \(\$B\_{MSY}\${} \textit{proxy}{}\; horizontal dotted line\)   based on the current assessment. } \def\OPTUNITfigFCap{Trends in the exploitation rate of ocean pout between 1968 and 2014 from the current  \(solid line\)  and previous \(dashed line\)  assessment and the corresponding  \$F\_{Threshold}\${} \(\$F\_{MSY}\${} \textit{proxy}{}=0.76\; horizontal dashed line\)   based on the current assessment. } \def\OPTUNITfigRecrCap{} \def\OPTUNITfigSurvCap{Indices of biomass \(kg\/tow\)  for ocean pout  between 1968 and 2015 for the Northeast Fisheries Science Center \(NEFSC\)  spring survey.   The approximate 90\% lognormal confidence intervals are shown.} \def\OPTUNITPreAmb{This assessment of the ocean pout  \(\textit{Zoarces americanus}\)  stock is an operational update of the 2012 assessment \(NEFSC 2012\)  and the 2008 benchmark assessment \(NEFSC 2008\). Based on the 2012 assessment, the stock was overfished but overfishing was not ocurring. This assessment updates commercial fishery catch data, research survey indices and the exploitation ratios through 2014. There are no stock projections.} \def\OPTUNITSoS{ \textbf{State of Stock: }{}Based on the current assessment, the ocean pout  \(\textit{Zoarces americanus}\)  stock is overfished and overfishing is not occurring \(Figures \ref{OPTUNITSSB\_plot1}\-\ref{OPTUNITF\_plot1}\){}. Retrospective adjustments were not made to the model results. Biomass proxy \(B\)  in 2014 was estimated to be 0.29 \(kg\/tow\)  which is 6\% of the biomass target \(\$B\_{MSY}\${} \textit{proxy}{} = 4.94\;  Figure \ref{OPTUNITSSB\_plot1}{}\).  The 2014 fully selected fishing mortality was estimated to be 0.269 which is 35\% of the overfishing threshold proxy \(\$F\_{MSY}\${} \textit{proxy}{} = 0.76\;  Figure \ref{OPTUNITF\_plot1}{}\).} \def\OPTUNITProj{ \textbf{Projections: }{}The index\-based assessment approach does not support catch projections\; catch advice for ocean pout has been based on the target exploitation rate and the most recent centered 3\-year average biomass index from the NEFSC spring survey. } \def\OPTUNITSpecCmt{ \textbf{Special Comments: } \begin{itemize}{} \item{}What are the most important sources of uncertainty in this stock assessment?  Explain, and describe qualitatively how they affect the assessment results \(such as estimates of biomass, F, recruitment, and population projections\).  \linebreak{} \hspace\*{0.5cm} \textit{ An important source of uncertainty is the stock has not responded to low catch as expected. }  \item{}Does this assessment model have a retrospective pattern? If so, is the pattern minor or major?  \(A major retrospective pattern occurs when the adjusted SSB or  \$F\_{Full}\${} lies outside of the approximate  joint confidence region for SSB and  \$F\_{Full}\${}\; see  Figure \ref{RhoDecision\_tab}{}\). \linebreak{} \hspace\*{0.5cm} \textit{ The model used to estimate status of this stock does not allow estimation of a retrospective pattern. }  \item{}Based on this stock assessment, are population projections well determined or uncertain? \linebreak{} \hspace\*{0.5cm} \textit{ N\/A}  \item{}Describe any changes that were made to the current stock assessment, beyond incorporating additional years of data  and the effect these changes had in the assessment and stock status. \linebreak{} \hspace\*{0.5cm} \textit{TOGA \(Type, Operation, Gear, Acquisition\)  values were used for haul criteria for NEFSC surveys for 2009 onward and minor changes in the use of observer data for discard estimates were made to the current assessment. These changes had a negligible effect on the assessment and stock status.   Recreational landings were updated and found to be negligible \(time series average of recreational landings to total catch was less than 1\%\)  and therefore not included in this assessment.}  \item{}If the stock status has changed a lot since the previous assessment, explain why this occurred.  \linebreak{} \hspace\*{0.5cm} \textit{Ocean pout stock status has not changed since the previous assessment.}  \item{}Indicate what data or studies are currently lacking and which would be needed most to improve this stock assessment in the future.  \linebreak{} \hspace\*{0.5cm} \textit{The ocean pout assessment could be improved with studies that explore why this stock is not rebuilding as expected. }  \item{}Are there other important comments? \linebreak{} \hspace\*{0.5cm} \textit{Biological reference points are based on catch\; the estimated discards used in the catch are based on a mix of direct \(1989 onward\)  and indirect \(1988 and back\)  methods. The catch used to determine MSY is based on indirect methods. } \end{itemize}{}} \def\OPTUNITRefr{ \textbf{References: }{} \linebreak{}Northeast Fisheries Science Center. 2012. Assessment or Data Updates of 13 Northeast Groundfish Stocks through 2010.  US Dep Commer, NOAA Fisheries, Northeast Fish Sci Cent Ref Doc. 12\-06\; 789 p. http:\/\/www.nefsc.noaa.gov\/publications\/crd\/crd1206\/ \linebreak{} \linebreak{}Northeast Fisheries Science Center. 2008. Assessment of 19 Northeast Groundfish Stocks through 2007: Report of the 3$^{rd}$ Groundfish Assessment Review Meeting \(GARM III\), Northeast Fisheries Science Center, Woods Hole, Massachusetts, August 4\-8, 2008. US Dep Commer, NOAA Fisheries, Northeast Fish Sci Cent Ref Doc. 08\-15\; 884 p + xvii. http:\/\/www.nefsc.noaa.gov\/publications\/crd\/crd0815\/ \linebreak{} \linebreak{}} \def\OPTUNITDraft{} \def\OPTUNITSPPname{Ocean Pout} \def\OPTUNITSPPnameT{Ocean Pout} \def\OPTUNITRptYr{2015} \def\OPTUNITAuthor{Susan Wigley} \def\OPTUNITReviewerComments{/home/dhennen/EIEIO/BigReport/OPT_UNIT/latex}  
%%%%%%%%%%%%%%%%%%%%%%%%%%%%%%%%%%%%%%%%%%%%%%%%%%%%%%%%%%%%%%%%%%%%%%%%%%%%%%%%%%%%%%%%%%%%%%%%%%%%%%%%%%%%
%GOMCOD
\input{\CODGMReviewerComments/BigReportCODGM}
%%%%%%%%%%%%%%%%%%%%%%%%%%%%%%%%%%%%%%%%%%%%%%%%%%%%%%%%%%%%%%%%%%%%%%%%%%%%%%%%%%%%%%%%%%%%%%%%%%%%%%%%%%%%
%GBCOD
\input{\CODGBReviewerComments/BigReportCODGB}
%%%%%%%%%%%%%%%%%%%%%%%%%%%%%%%%%%%%%%%%%%%%%%%%%%%%%%%%%%%%%%%%%%%%%%%%%%%%%%%%%%%%%%%%%%%%%%%%%%%%%%%%%%%%
%GBHAD
\input{\HADGBReviewerComments/BigReportHADGB}
%%%%%%%%%%%%%%%%%%%%%%%%%%%%%%%%%%%%%%%%%%%%%%%%%%%%%%%%%%%%%%%%%%%%%%%%%%%%%%%%%%%%%%%%%%%%%%%%%%%%%%%%%%%%
%GOMHAD
\input{\HADGMReviewerComments/BigReportHADGM}
%%%%%%%%%%%%%%%%%%%%%%%%%%%%%%%%%%%%%%%%%%%%%%%%%%%%%%%%%%%%%%%%%%%%%%%%%%%%%%%%%%%%%%%%%%%%%%%%%%%%%%%%%%%%
%CCGMYEL
\input{\YELCCGMReviewerComments/BigReportYELCCGM}
%%%%%%%%%%%%%%%%%%%%%%%%%%%%%%%%%%%%%%%%%%%%%%%%%%%%%%%%%%%%%%%%%%%%%%%%%%%%%%%%%%%%%%%%%%%%%%%%%%%%%%%%%%%%
%SNEMAYEL
\input{\YELSNEMAReviewerComments/BigReportYELSNEMA}
%%%%%%%%%%%%%%%%%%%%%%%%%%%%%%%%%%%%%%%%%%%%%%%%%%%%%%%%%%%%%%%%%%%%%%%%%%%%%%%%%%%%%%%%%%%%%%%%%%%%%%%%%%%%
%FLWGB
\input{\FLWGBReviewerComments/BigReportFLWGB}
%%%%%%%%%%%%%%%%%%%%%%%%%%%%%%%%%%%%%%%%%%%%%%%%%%%%%%%%%%%%%%%%%%%%%%%%%%%%%%%%%%%%%%%%%%%%%%%%%%%%%%%%%%%%
%FLWSNEMA
\input{\FLWSNEMAReviewerComments/BigReportFLWSNEMA}
%%%%%%%%%%%%%%%%%%%%%%%%%%%%%%%%%%%%%%%%%%%%%%%%%%%%%%%%%%%%%%%%%%%%%%%%%%%%%%%%%%%%%%%%%%%%%%%%%%%%%%%%%%%%
%PLAUNIT
\input{\PLAUNITReviewerComments/BigReportPLAUNIT}
%%%%%%%%%%%%%%%%%%%%%%%%%%%%%%%%%%%%%%%%%%%%%%%%%%%%%%%%%%%%%%%%%%%%%%%%%%%%%%%%%%%%%%%%%%%%%%%%%%%%%%%%%%%%
%WITUNIT
\input{\WITUNITReviewerComments/BigReportWITUNIT}
%%%%%%%%%%%%%%%%%%%%%%%%%%%%%%%%%%%%%%%%%%%%%%%%%%%%%%%%%%%%%%%%%%%%%%%%%%%%%%%%%%%%%%%%%%%%%%%%%%%%%%%%%%%%
%REDUNIT
\input{\REDUNITReviewerComments/BigReportREDUNIT}
%%%%%%%%%%%%%%%%%%%%%%%%%%%%%%%%%%%%%%%%%%%%%%%%%%%%%%%%%%%%%%%%%%%%%%%%%%%%%%%%%%%%%%%%%%%%%%%%%%%%%%%%%%%%
%HKWUNIT
\input{\HKWUNITReviewerComments/BigReportHKWUNIT}
%%%%%%%%%%%%%%%%%%%%%%%%%%%%%%%%%%%%%%%%%%%%%%%%%%%%%%%%%%%%%%%%%%%%%%%%%%%%%%%%%%%%%%%%%%%%%%%%%%%%%%%%%%%%
%POKUNIT
\input{\POKUNITReviewerComments/BigReportPOKUNIT}
%%%%%%%%%%%%%%%%%%%%%%%%%%%%%%%%%%%%%%%%%%%%%%%%%%%%%%%%%%%%%%%%%%%%%%%%%%%%%%%%%%%%%%%%%%%%%%%%%%%%%%%%%%%%
%CATUNIT
\input{\CATUNITReviewerComments/BigReportCATUNIT}
%%%%%%%%%%%%%%%%%%%%%%%%%%%%%%%%%%%%%%%%%%%%%%%%%%%%%%%%%%%%%%%%%%%%%%%%%%%%%%%%%%%%%%%%%%%%%%%%%%%%%%%%%%%%
%HALUNIT
\input{\HALUNITReviewerComments/BigReportHALUNIT}
%%%%%%%%%%%%%%%%%%%%%%%%%%%%%%%%%%%%%%%%%%%%%%%%%%%%%%%%%%%%%%%%%%%%%%%%%%%%%%%%%%%%%%%%%%%%%%%%%%%%%%%%%%%%
%FLDGMGB
\input{\FLDGMGBReviewerComments/BigReportFLDGMGB}
%%%%%%%%%%%%%%%%%%%%%%%%%%%%%%%%%%%%%%%%%%%%%%%%%%%%%%%%%%%%%%%%%%%%%%%%%%%%%%%%%%%%%%%%%%%%%%%%%%%%%%%%%%%%
%FLDSNEMA
\input{\FLDSNEMAReviewerComments/BigReportFLDSNEMA}
%%%%%%%%%%%%%%%%%%%%%%%%%%%%%%%%%%%%%%%%%%%%%%%%%%%%%%%%%%%%%%%%%%%%%%%%%%%%%%%%%%%%%%%%%%%%%%%%%%%%%%%%%%%%
%OPTUNIT
\input{\OPTUNITReviewerComments/BigReportOPTUNIT}
%%%%%%%%%%%%%%%%%%%%%%%%%%%%%%%%%%%%%%%%%%%%%%%%%%%%%%%%%%%%%%%%%%%%%%%%%%%%%%%%%%%%%%%%%%%%%%%%%%%%%%%%%%%%
%FLWGM
\input{\FLWGMReviewerComments/BigReportFLWGM}
%%%%%%%%%%%%%%%%%%%%%%%%%%%%%%%%%%%%%%%%%%%%%%%%%%%%%%%%%%%%%%%%%%%%%%%%%%%%%%%%%%%%%%%%%%%%%%%%%%%%%%%%%%%%
%GBYEL
\input{\YELGBReviewerComments/BigReportYELGB}








\end{document}; ls -al;