 \def\HKWUNITMyPathTab{/home/dhennen/EIEIO/BigReport/HKW_UNIT/tables} \def\HKWUNITMyPathFig{/home/dhennen/EIEIO/BigReport/HKW_UNIT/figures} \def\HKWUNITfigFishCap{Total catch of white hake between 1963 and 2014 by fleet (commercial, recreational, or Canadian)  and disposition (landings and discards).} \def\HKWUNITfigSSBCap{Trends in spawning stock biomass of white hake between 1963 and 2014 from the current  (solid line)  and previous (dashed line)  assessment and the corresponding  $SSB_{Threshold}${} ($\dfrac{1}{2}${} $SSB_{MSY}${} \textit{proxy}{}; horizontal dashed line)  as well as  $SSB_{Target}${} ($SSB_{MSY}${} \textit{proxy}{}; horizontal dotted line)   based on the 2014 assessment.  The red dot indicates the rho-adjusted SSB values that would have resulted had a retrospective  adjusment been made (see Special Comments section).  The approximate 90\percent{} lognormal confidence intervals are shown.} \def\HKWUNITfigFCap{Trends in the fully selected fishing mortality ($F_{Full}${})  of white hake between 1963 and 2014 from the current  (solid line)  and previous (dashed line)  assessment and the corresponding  $F_{Threshold}${} ($F_{MSY}${} \textit{proxy}{}=0.188; horizontal dashed line).  The red dot indicates the rho-adjusted SSB values that would have resulted had a retrospective  adjusment been made (see Special Comments section).  The approximate 90\percent{} lognormal confidence intervals are shown.} \def\HKWUNITfigRecrCap{Trends in Recruits (age 1)  (000s)  of white hake between 1963 and 2014 from the current (solid line)  and previous (dashed line)  assessment. The approximate 90\percent{} lognormal confidence intervals are shown.} \def\HKWUNITfigSurvCap{Indices of biomass for the white hake between 1963 and 2015 for the Northeast Fisheries Science Center (NEFSC)  spring and fall bottom trawl surveys.  The approximate 90\percent{} lognormal confidence intervals are shown.} \def\HKWUNITPreAmb{This assessment of the white hake (\textit{Urophycis tenuis})  stock is an operational assessment of the existing 2013 benchmark ASAP assessment (NEFSC 2013). Based on the previous assessment the stock was not overfished, and overfishing was not occurring. This assessment updates commercial fishery catch data, research survey indices of abundance, the ASAP assessment model and reference points through 2014. Additionally, stock projections have been updated through 2018.} \def\HKWUNITSoS{ \textbf{State of Stock: }{}Based on this updated assessment, white hake (\textit{Urophycis tenuis})  stock is not overfished and overfishing is not occurring (Figures \ref{HKWUNITSSB_plot1}-\ref{HKWUNITF_plot1}){}. Retrospective adjustments were not made to the model results.  Spawning stock biomass (SSB)  in 2014 was estimated to be 28,553 (mt)  which is 88\percent{} of the biomass target ($SSB_{MSY}${} \textit{proxy}{} = 32,550;  Figure \ref{HKWUNITSSB_plot1}{}).  The 2014 fully selected fishing mortality was estimated to be 0.076 which is 40\percent{} of the overfishing threshold proxy ($F_{MSY}${} \textit{proxy}{} = 0.188;  Figure \ref{HKWUNITF_plot1}{}).} \def\HKWUNITProj{ \textbf{Projections: }{}Short term projections of catch and SSB were derived by sampling from a cumulative  distribution  function of recruitment estimates from ASAP from 1995-2012. The annual fishery selectivity, maturity ogive, and mean weights at age used in the projection  are the most recent 5 year averages. } \def\HKWUNITSpecCmt{ \textbf{Special Comments: } \begin{itemize}{} \item{}What are the most important sources of uncertainty in this stock assessment?  Explain, and describe qualitatively how they affect the assessment results (such as estimates of biomass, F, recruitment, and population projections).  \linebreak{} \hspace*{0.5cm} \textit{1. Catch at age information is not well characterized due to possible mis-identification of species in the commercial and sea sampling data, particularly in early years, low sampling of commercial landings in  some years, and sparse discard data, particularly in early years.  \linebreak{} \hspace*{0.5cm}2. Since the commercial catch is aged primarily with survey age/length keys, there is considerable augmentation required, mainly for ages 5 and older. The numbers at age and mean weights at age in the catch for these ages may therefore not be well specified.  \linebreak{} \hspace*{0.5cm}3. White hake may move seasonally into and out of the defined stock area.  \linebreak{} \hspace*{0.5cm}4. There are no commercial catch at age data prior to 1989 and the catchability of older ages in the surveys is very low. This results in a large uncertainty in starting numbers at age.  \linebreak{} \hspace*{0.5cm}5. Since 2003, dealers have been culling very large fish out of the large market category. However, there was no market category to input into the landings until June 2014. The length compositions are distinct from fish categorized as large and have been identified since 2011. This may bias the age composition of the landings, particularly in 2014 when 2000 of the 5000 large samples were these extra-large fish.  \linebreak{} \hspace*{0.5cm}6. A pooled age/length key is used for 1963-1981, fall 2003 (second half of commercial key)  and 2014. Age data were not available for 2014 in time for this assessment. The same pooled key that was used for 1963-1981 was used for 2014.}  \item{} Does this assessment model have a retrospective pattern? If so, is the pattern minor, or major? (A major retrospective pattern occurs when the adjusted SSB or  $F_{Full}${} lies outside of the approximate  joint confidence region for SSB and  $F_{Full}${}; see  Table \ref{RhoDecision_tab}{}). \linebreak{} \hspace*{0.5cm} \textit{ No retrospective adjustment of spawning stock biomass or fishing mortality in 2014 was required.  The pattern in this assessment is considered minor (Mohn’s rho of 0.18 on SSB, Mohn’s rho of 0.12 on F)  with the adjusted SSB within the 90\percent{} CI of the MCMC. However, the Mohn’s rho for Age 1 estimates is 0.54. This may have an impact on projections if this continues into the future.}  \item{}Based on this stock assessment, are population projections well determined or uncertain? \linebreak{} \hspace*{0.5cm} \textit{Population projections for white hake are not well determined and projected biomass from the last assessment  was outside the confidence bounds of the biomass estimated in the current assessment. }  \item{}Describe any changes that were made to the current stock assessment, beyond incorporating additional years of data  and the effect these changes had on the assessment and stock status. \linebreak{} \hspace*{0.5cm} \textit{ The 2011 catch-at-length and age were re-estimated for both landings and discards. For the  landings, two samples were adjusted for dorsal length to total length that had been missed in the previous assessment.}  \item{}If the stock status has changed a lot since the previous assessment, explain why this occurred.  \linebreak{} \hspace*{0.5cm} \textit{While stock status of white hake has not changed, the stock has not rebuilt as the projections from the last assessment indicated. This is due to the retrospective pattern in recruitment. The numbers for the 2005-2009 year classes, which were included in the age 2-6 starting numbers in the projections, were over-estimated which led to over-estimating SSB in 2014.}  \item{}Indicate what data or studies are currently lacking and which would be needed most to improve this stock assessment in the future.  \linebreak{} \hspace*{0.5cm} \textit{ Age structures from the observer program are available and should be aged to augment  the survey keys. There is a also a new market category for heads, and age structures could be  acquired from these as an otolith length/total length relationship can be established. }  \item{}Are there other important issues? \linebreak{} \hspace*{0.5cm} \textit{None. } \end{itemize}{}} \def\HKWUNITRefr{ \textbf{References: }{} \linebreak{} NEFSC. 2013. 56$^{th}$ Northeast Regional Stock Assessment Workshop (56$^{th}$ SAW)  Assessment  Report. US Dep Commer, NOAA Fisheries, Northeast Fish Sci Cent Ref Doc. 13-10; 868 p.  \href{http://www.nefsc.noaa.gov/publications/crd/crd1310/}{CRD13-10}  \linebreak{} \linebreak{}} \def\HKWUNITDraft{} \def\HKWUNITSPPname{white hake} \def\HKWUNITSPPnameT{White hake} \def\HKWUNITRptYr{2015} \def\HKWUNITAuthor{Katherine Sosebee} \def\HKWUNITReviewerComments{/home/dhennen/EIEIO/BigReport/HKW_UNIT/latex}