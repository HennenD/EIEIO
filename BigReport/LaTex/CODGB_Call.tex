 \def\CODGBMyPathTab{/home/dhennen/EIEIO/BigReport/COD_GB/tables} \def\CODGBMyPathFig{/home/dhennen/EIEIO/BigReport/COD_GB/figures} \def\CODGBfigFishCap{Total catch of Georges Bank Atlantic cod between 1978 and 2014 by fleet (US commercial, US recreational, or Canadian)  and disposition (landings and discards).} \def\CODGBfigSSBCap{Trends in spawning stock biomass of Georges Bank Atlantic cod between 1978 and 2014 from the current  (solid line)  and previous (dashed line)  assessment and the corresponding  $SSB_{Threshold}${} ($\dfrac{1}{2}${} $SSB_{MSY}${} \textit{proxy}{}; horizontal dashed line)  as well as  $SSB_{Target}${} ($SSB_{MSY}${} \textit{proxy}{}; horizontal dotted line)   based on the 2015 assessment.  Biomass was adjusted for a retrospective pattern  and the adjustment is shown in red.  The approximate 90\percent{} lognormal confidence intervals are shown.} \def\CODGBfigFCap{Trends in the fully selected fishing mortality ($F_{Full}${})  of Georges Bank Atlantic cod between 1978 and 2014 from the current  (solid line)  and previous (dashed line)  assessment and the corresponding  $F_{Threshold}${} ($F_{MSY}${} \textit{proxy}{}=0.169; horizontal dashed line).  $F_{Full}${} was adjusted for a retrospective pattern  and the adjustment is shown in red,  based on the 2015 assessment. The approximate 90\percent{} lognormal confidence intervals are shown.} \def\CODGBfigRecrCap{Trends in Recruits (age 1)  (000s)  of Georges Bank Atlantic cod between 1978 and 2014 from the current (solid line)  and previous (dashed line)  assessment. The approximate 90\percent{} lognormal confidence intervals are shown.} \def\CODGBfigSurvCap{Indices of biomass for the Georges Bank Atlantic cod between 1963 and 2015 for the Northeast Fisheries Science Center (NEFSC)  spring and fall trawl surveys, and the DFO research bottom trawl surveys.  The approximate 90\percent{} lognormal confidence intervals are shown.} \def\CODGBPreAmb{This assessment of the Georges Bank Atlantic cod (\textit{Gadus morhua})  stock is an operational assessment of the existing 2012 benchmark assessment (NEFSC 2013). Based on the previous assessment the stock was overfished, and overfishing was occurring. This 2015 assessment updates commercial fishery catch data, research survey indices of abundance, the analytical ASAP assessment model, and reference points through 2014. Additionally, stock projections have been updated through 2018.} \def\CODGBSoS{ \textbf{State of Stock: }{}Based on this updated assessment, the Georges Bank Atlantic cod (\textit{Gadus morhua})  stock is overfished and overfishing is occurring (Figures \ref{CODGBSSB_plot1}-\ref{CODGBF_plot1}){}.  Retrospective adjustments were made to the model results.  Spawning stock biomass (SSB)  in 2014 was estimated to be 1,804 (mt), which is 1\percent{} of the biomass target for this stock ($SSB_{MSY}${} \textit{proxy}{} = 201,152;  Figure \ref{CODGBSSB_plot1}{}).  The 2014 fully selected fishing mortality was estimated to be 1.68, which is 994\percent{} of the overfishing threshold proxy ($F_{MSY}${} \textit{proxy}{} = 0.169;  Figure \ref{CODGBF_plot1}{}).} \def\CODGBProj{ \textbf{Projections: }{}Short term projections of biomass were derived by sampling from a two-stage cumulative  distribution  function of recruitment estimates from ASAP model results, using a 50,000 mt cutpoint. The annual fishery selectivity, maturity ogive, and mean weights at age used in projections are the most recent 5 year averages;  retrospective adjustments were applied in the projections.} \def\CODGBSpecCmt{ \textbf{Special Comments: } \begin{itemize}{} \item{}What are the most important sources of uncertainty in this stock assessment?  Explain, and describe qualitatively how they affect the assessment results (such as estimates of biomass, F, recruitment, and population projections).  \linebreak{} \hspace*{0.5cm} \textit{The major source of uncertainty is presumably the estimate of catch or of natural mortality, considering the magnitude of the retrospective bias. These both affect the scale of the biomass, fishing mortality estimates, and the reference point estimates. The catch estimates do not include all discards (e.g., lobster gear)  and includes uncertain estimates of recreational landings and discards, and of some commercial discards (e.g., small mesh). Natural mortality (M)  of Georges Bank Atlantic cod is not well understood and is assumed constant over time in the model. Other sources of uncertainty include possible changes in growth parameters in recent years and how this affects fecundity, the viability of eggs/sperm, and the success rate of hatching - all influencing recruitment survival and year class strength.}  \item{} Does this assessment model have a retrospective pattern? If so, is the pattern minor, or major? (A major retrospective pattern occurs when the adjusted SSB or  $F_{Full}${} lies outside of the approximate  joint confidence region for SSB and  $F_{Full}${}; see  Table \ref{RhoDecision_tab}{}). \linebreak{} \hspace*{0.5cm} \textit{ The 7-year Mohn's  \textrho{}, relative to SSB, was 0.68 in the 2012 assessment and was 2.43 in 2014. The 7-year Mohn's  \textrho{}, relative to F, was -0.46 in the 2012 assessment and was -0.72 in 2014. There was a major retrospective pattern for this assessment because the  \textrho{} adjusted estimates of 2014 SSB ($SSB_{\rho}${}=1,804)  and 2014 F ($F_{\rho}${}=1.68)  were outside the approximate 90\percent{} confidence region around SSB (3,922 - 10,596)  and F (0.251 - 0.815).  A retrospective  adjustment was made for both the determination of stock status and for projections of catch in 2016. The retrospective adjustment changed the 2014 SSB from 6,180 to 1,804 and the 2014  $F_{Full}${} from 0.463 to 1.68.}  \item{}Based on this stock assessment, are population projections well determined or uncertain? \linebreak{} \hspace*{0.5cm} \textit{Population projections for Georges Bank Atlantic cod are uncertain and likely optimistic. The projections are based on a biomass cutpoint of 50,000 mt, which has not been produced since 1992. The average recruitment since 1992 has been 4.9 million age 1 fish, whereas during the last 10 years, average recruitment has been about 2.7 million age 1 fish. A sensistivity projection using the most recent 10 years of recruitment was conducted and results presented in the \href{http://www.nefsc.noaa.gov/saw/sasi/sasi_report_options.php}{SASINF}{} database. }  \item{}Describe any changes that were made to the current stock assessment, beyond incorporating additional years of data  and the effect these changes had on the assessment and stock status. \linebreak{} \hspace*{0.5cm} \textit{ No major changes, other than the addition of recent years of data, were made to the Georges Bank Atlantic cod assessment for this update. However, recreational catch and commercial discard estimates were revised slightly due to minor changes in the databases, and the application of length frequencies (annual instead of half year)  in one instance.}  \item{}If the stock status has changed a lot since the previous assessment, explain why this occurred.  \linebreak{} \hspace*{0.5cm} \textit{As in recent assessments for Georges Bank Atlantic cod the stock remains in an overfishing and overfished status.}  \item{}Indicate what data or studies are currently lacking and which would be needed most to improve this stock assessment in the future.  \linebreak{} \hspace*{0.5cm} \textit{The Georges Bank Atlantic cod assessment could be improved with additional studies on natural mortality, growth, and fecundity. Additionally, more precise estimates of recreational landings and discards, sampling of fish caught by individual recreational anglers, and incorporation of discards in the lobster fishery would decrease uncertainty in the discard esimates.}  \item{}Are there other important issues? \linebreak{} \hspace*{0.5cm} \textit{The differences in model assumptions of natural mortality between the SARC GB cod assessment and the TRAC EGB cod assessment is problematic for the recovery of the entire GB cod stock. Model results of the TRAC VPA M=0.8 model are used to determine quota for the EGB management unit, so by default, proportionally more cod are being removed from eastern GB than what the GB cod ASAP model would predict.} \end{itemize}{}} \def\CODGBRefr{ \textbf{References: }{} \linebreak{} Northeast Fisheries Science Center. 2013. 55$^{th}$ Northeast Regional Stock Assessment  Workshop (55$^{th}$ SAW)  Assessment Summary Report. Northeast Fisheries Science Center  Reference Document 13-11; 43 p.  \href{http://nefsc.noaa.gov/publications/crd/crd1311/}{CRD13-11} \linebreak{} \linebreak{}} \def\CODGBDraft{} \def\CODGBSPPname{Georges Bank Atlantic cod} \def\CODGBSPPnameT{Georges Bank Atlantic cod} \def\CODGBRptYr{2015} \def\CODGBAuthor{Loretta O'Brien} \def\CODGBReviewerComments{/home/dhennen/EIEIO/BigReport/COD_GB/latex}