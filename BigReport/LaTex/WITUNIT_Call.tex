 \def\WITUNITMyPathTab{/home/dhennen/EIEIO/BigReport/WIT_UNIT/tables} \def\WITUNITMyPathFig{/home/dhennen/EIEIO/BigReport/WIT_UNIT/figures} \def\WITUNITfigFishCap{Total catch of witch flounder between 1982 and 2014 by fleet (commercial)  and disposition (landings and discards).} \def\WITUNITfigSSBCap{Trends in spawning stock biomass (mt)  of witch flounder between 1982 and 2014 from the current  (solid line)  and previous (dashed line)  assessment and the corresponding  $SSB_{Threshold}${} ($\dfrac{1}{2}${} $SSB_{MSY}${}; horizontal dashed line)  as well as  $SSB_{Target}${} $SSB_{MSY}${}; horizontal dotted line)   based on the current assessment. Red solid vertical line indicates rho adjusted SSB. Black solid vertical line indicates 90\percent{} confidence interval for 2014.} \def\WITUNITfigFCap{Trends in the fully selected fishing mortality ($F_{Full}${})  of witch flounder between 1982 and 2014 from the current  (solid line)  and previous (dashed line)  assessment and the corresponding  $F_{Threshold}${} ($F_{MSY}${}=0.279; horizontal dashed line)  based on the current assessment.  Red solid vertical line indicates rho adjusted  $F_{Full}${}. Black solid vertical line indicates 90\percent{} confidence interval for 2014.} \def\WITUNITfigRecrCap{Trends in Age 3  (000s)  of witch flounder between 1982 and 2014 from the current (solid line)  and previous (dashed line)  assessment.} \def\WITUNITfigSurvCap{Indices of biomass (kg/tow)  for the witch flounder between 1963 and 2015 for the Northeast Fisheries Science Center (NEFSC)  spring and fall bottom trawl surveys.  The 90\percent{} lognormal confidence intervals are shown.} \def\WITUNITPreAmb{This assessment of the witch flounder (\textit{Glyptocephalus cynoglossus})  stock is an operational assessment of the 2012 assessment (NEFSC 2012)  and the 2008 benchmark assessment (NEFSC 2008). This assessment updates commercial fishery catch data, research survey indices, and the analytical assessment model through 2014. Additionally, stock projections have been updated through 2018. Reference points have been updated. } \def\WITUNITSoS{ \textbf{State of Stock: }{}witch flounder (\textit{Glyptocephalus cynoglossus})  stock is overfished and overfishing is occurring (Figures \ref{WITUNITSSB_plot1}-\ref{WITUNITF_plot1}){}. Retrospective adjustments were made to the model results.  Spawning stock biomass (SSB)  in 2014 was estimated to be 2,077 (mt)  which is 22\percent{} of the  $SSB_{MSY}${} proxy (9,473;  Figure \ref{WITUNITSSB_plot1}{}).  The 2014 fully selected fishing mortality was estimated to be 0.687 which is 246\percent{} of the  $F_{MSY}${} proxy (0.279;  Figure \ref{WITUNITF_plot1}{}). A retrospective adjustment to  $F_{Full}${} and SSB in 2014 was required but did not lead to a change in status.  } \def\WITUNITProj{ \textbf{Projections: }{}Short term projection recruitment was sampled from a cumulative distribution function derived from ADAPT VPA (with split time series between 1994 and 1995)  estimated age 3 recruitment between 1982 and 2013.  Average 2010-2014 partial recruitment, average 2010-2014 mean weights, and maturation ogive representing 2011-2015 maturity data were used.} \def\WITUNITSpecCmt{ \textbf{Special Comments: } \begin{itemize}{} \item{}What are the most important sources of uncertainty in this stock assessment?  Explain, and describe qualitatively how they affect the assessment results (such as estimates of biomass, F, recruitment, and population projections).  \linebreak{} \hspace*{0.5cm} \textit{An important source of uncertainty is the retrospective pattern where fishing mortality is underestimated and spawning stock biomass and recruitment are overestimated. }  \item{} Does this assessment model have a retrospective pattern? If so, is the pattern minor, or major? (A major retrospective pattern occurs when the adjusted SSB or  $F_{Full}${} lies outside of the approximate  joint confidence region for SSB and  $F_{Full}${}).  \linebreak{} \hspace*{0.5cm} \textit{ The 7-year Mohn's  \textrho{}, relative to SSB, was 0.61 in the 2012 assessment and was 0.51 in 2014. The 7-year Mohn's  \textrho{}, relative to F, was -0.33 in the 2012 assessment and was -0.38 in 2014. There was a major retrospective pattern for this assessment because the  \textrho{} adjusted estimates of 2014 SSB ($SSB_{\rho}${}=2,077)  and 2014 F ($F_{\rho}${}=0.687)  were outside the approximate 90\percent{} confidence region around SSB (2,643 - 3,864)  and F (0.321 - 0.603).  A retrospective  adjustment was made for both the determination of stock status and for projections of catch in 2016. The retrospective adjustment changed the 2014 SSB from 3,129 to 2,077 and the 2014  $F_{Full}${} from 0.428 to 0.687.}  \item{}Based on this stock assessment, are population projections well determined or uncertain? \linebreak{} \hspace*{0.5cm} \textit{Population projections for witch flounder appear to be optimistic; the projected rho adjusted biomass from the last assessment  was above the upper confidence bounds of the projected rho adjusted biomass estimated in the current assessment. }  \item{}Describe any changes that were made to the current stock assessment, beyond incorporating additional years of data  and the effect these changes had on the assessment and stock status.  \linebreak{} \hspace*{0.5cm} \textit{TOGA (Type, Operation, Gear, Acquisition)  values were used for haul criteria for NEFSC surveys for 2009 onward and minor changes in the use of observer data for discard estimates were made to the current witch flounder assessment. These changes had a negligible effect on the assessment and stock status.  }  \item{}If the stock status has changed a lot since the previous assessment, explain why this occurred.  \linebreak{} \hspace*{0.5cm} \textit{No change in stock status has occurred for witch flounder since the previous assessment. }  \item{}Indicate what data or studies are currently lacking and which would be needed most to improve this stock assessment in the future.  \linebreak{} \hspace*{0.5cm} \textit{Extensive studies have examined the causes of retrospective patterns with no definitive conclusions other than a change in model does not resolve the issue. }  \item{}Are there other important comments? \linebreak{} \hspace*{0.5cm} \textit{The VPA analysis was performed with survey time series split between 1994 and 1995. This time split corresponds to changes in the commercial reporting methods as well as other regulatory management changes.  } \end{itemize}{}} \def\WITUNITRefr{ \textbf{References: }{} \linebreak{}Northeast Fisheries Science Center. 2008. Assessment of 19 Northeast Groundfish Stocks through 2007: Report of the 3$^{rd}$ Groundfish Assessment Review Meeting (GARM III), Northeast Fisheries Science Center, Woods Hole, Massachusetts, August 4-8, 2008. US Dep Commer, NOAA Fisheries, Northeast Fish Sci Cent Ref Doc. 08-15; 884 p + xvii. \href{http://www.nefsc.noaa.gov/publications/crd/crd0815/}{CRD08-15} \linebreak{} \linebreak{}Northeast Fisheries Science Center. 2012. Assessment or Data Updates of 13 Northeast Groundfish Stocks through 2010.  US Dep Commer, NOAA Fisheries, Northeast Fish Sci Cent Ref Doc. 12-06; 789 p. \href{http://www.nefsc.noaa.gov/publications/crd/crd1206/}{CRD12-06} \linebreak{} \linebreak{}} \def\WITUNITDraft{} \def\WITUNITSPPname{witch flounder} \def\WITUNITSPPnameT{Witch flounder} \def\WITUNITRptYr{2015} \def\WITUNITAuthor{Susan Wigley} \def\WITUNITReviewerComments{/home/dhennen/EIEIO/BigReport/WIT_UNIT/latex}