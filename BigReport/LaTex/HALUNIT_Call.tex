 \def\HALUNITMyPathTab{/home/dhennen/EIEIO/BigReport/HAL_UNIT/tables} \def\HALUNITMyPathFig{/home/dhennen/EIEIO/BigReport/HAL_UNIT/figures} \def\HALUNITfigFishCap{Total catch of Atlantic halibut between 1963 and 2014 by disposition (landings and discards).} \def\HALUNITfigSSBCap{Estimated trends in the biomass of Atlantic halibut between 1963 and 2014 from the current  (solid line)  and previous (dashed line)  assessment and the corresponding  $B_{Threshold}${}= $\dfrac{1}{2}${} $B_{MSY}${} \textit{proxy}{}(horizontal dashed line)  as well as  $B_{Target}${} ($B_{MSY}${} \textit{proxy}{}; horizontal dotted line)   based on the 2015 assessment.} \def\HALUNITfigFCap{Estimated trends in the fully selected fishing mortality ($F_{Full}${})  of Atlantic halibut between 1963 and 2014 from the current  (solid line)  and previous (dashed line)  assessment and the corresponding  $F_{Threshold}${} (0.073; horizontal dashed line)  as well as  $F_{Target}${} (0.8 * $F_{MSY}${} \textit{proxy}{}; dotted line)   based on the 2015 assessment. } \def\HALUNITfigRecrCap{} \def\HALUNITfigSurvCap{Indices of biomass for the Atlantic halibut between 1963 and 2014 for the Northeast Fisheries Science Center (NEFSC)  fall bottom trawl survey.  The 90\percent{} lognormal confidence intervals are shown.} \def\HALUNITPreAmb{This assessment of the Atlantic halibut (\textit{Hippoglossus hippoglossus})  stock is an operational assessment of the existing benchmark assessment (NEFSC 2010)  and the 2012 operational assessment (NEFSC 2012). This assessment updates commercial fishery catch data, research survey indices of abundance, and the replacement yield assessment model through 2014. Additionally, stock projections have been updated through 2018. Reference points have not been updated. } \def\HALUNITSoS{ \textbf{State of Stock: }{}Based on this updated assessment, Atlantic halibut (\textit{Hippoglossus hippoglossus})  stock status is unknown (Figures \ref{HALUNITSSB_plot1}-\ref{HALUNITF_plot1}){}. Retrospective adjustments were not made to the model results.  Biomass (SSB)  in 2014 was estimated to be 96,464 (mt)  which is 199\percent{} of the biomass target ($SSB_{MSY}${} \textit{proxy}{} = 48,509;  Figure \ref{HALUNITSSB_plot1}{}).  The 2014 fully selected fishing mortality was estimated to be 0.001 which is 1\percent{} of the overfishing threshold proxy ($F_{MSY}${} \textit{proxy}{} = 0.073;  Figure \ref{HALUNITF_plot1}{}).} \def\HALUNITProj{ \textbf{Projections: }{} Short term projections were based on a constant F =  $F_{MSY}${} \textit{proxy}{} = 0.073.  Projections use the assessment model (replacement yield)  and maintain all other model assumptions.} \def\HALUNITSpecCmt{ \textbf{Special Comments: } \begin{itemize}{} \item{}What are the most important sources of uncertainty in this stock assessment?  Explain, and describe qualitatively how they affect the assessment results (such as estimates of biomass, F, recruitment, and population projections).  \linebreak{} \hspace*{0.5cm} \textit{The assessment model used for Atlantic halibut is highly uncertain.  It estimates one parameter, the initial biomass, and  proceeds deterministically from 1800 to 2014.  The model is highly sensitive to the initial biomass.  The model is  tuned to the survey index, which is inefficient for Atlantic halibut, catches very few animals and is therefore noisy.   The RYM model assumes no immigration or emmigration and that the population both began, and tends to, equilibrium.   These assumptions are unlikely to be true for Atlantic halibut. The model estimates a biomass that is approximately equal  to unfished biomass, which is not credible. Catch has been very low for at least 100 years relative  to the landings reported early in the time series, despite a strong market and high value  relative to other groundfish.  The low catch throughout the century implies that the Atlantic halibut stock is very likely  depleted relative to its unfished condition and is therefore likely to be overfished, even if its current biomass is  unknown.}  \item{} Does this assessment model have a retrospective pattern? If so, is the pattern minor, or major? (A major retrospective pattern occurs when the adjusted SSB or  $F_{Full}${} lies outside of the approximate  joint confidence region for SSB and  $F_{Full}${}; see  Table \ref{RhoDecision_tab}{}). \linebreak{} \hspace*{0.5cm} \textit{ The model used to determine the status of this stock does not allow estimation of a retrospective pattern. }  \item{}Based on this stock assessment, are population projections well determined or uncertain? \linebreak{} \hspace*{0.5cm} \textit{Population projections for Atlantic halibut are uncertain because biomass cannot be reasonably determined using  the current assessment model.}  \item{}Describe any changes that were made to the current stock assessment, beyond incorporating additional years of data  and the effect these changes had on the assessment and stock status. \linebreak{} \hspace*{0.5cm} \textit{ The catch data were slightly altered due to the exclusion of catch made in international waters and the  re-estimation of average discard ratio after 1998 (due to the incorporation of more years of data).}  \item{}If the stock status has changed a lot since the previous assessment, explain why this occurred.  \linebreak{} \hspace*{0.5cm} \textit{The overfishing and overfished status of Atlantic halibut cannot be determined using the current assessment.  This  occurred because diagnostics showed the model was unreliable.  }  \item{}Indicate what data or studies are currently lacking and which would be needed most to improve this stock assessment in the future.  \linebreak{} \hspace*{0.5cm} \textit{The Atlantic halibut assessment could be improved with additional studies on stock structure, additional age and length data,  a more precise and accurate survey, and an investigation of alternate assessment models.}  \item{}Are there other important issues? \linebreak{} \hspace*{0.5cm} \textit{Atlantic halibut are clearly depleted relative to their unfished state.  Catches have been far below historical landings  for more than 100 years, despite a lack of regulation before 1999 and a strong commercial market.  The current  assessment model implies that Atlantic halibut is near or above its unfished biomass and could support removals  commensurate with MSY.  The current assessment should probably not be used to inform management decisions.} \end{itemize}{}} \def\HALUNITRefr{ \textbf{References: }{} \linebreak{} Northeast Fisheries Science Center. 2012. Assessment or Data Updates of 13 Northeast Groundfish Stocks  through 2010. US Dept Commer, Northeast Fish Sci Cent Ref Doc. 12-06; 789 p. Available from: National  Marine Fisheries Service, 166 Water Street, Woods Hole, MA 02543-1026. \href{http://www.nefsc.noaa.gov/publications/crd/crd1206/}{CRD12-06} \linebreak{} \linebreak{}Col, L.A., Legault, C.M. 2009. The 2008 Assessment of Atlantic halibut in the Gulf of Maine Georges Bank region.  US Dept Commer, Northeast Fish Sci Cent Ref Doc. 09-08; 39 p. Available from: National Marine Fisheries Service, 166 Water Street, Woods Hole, MA 02543-1026. \href{http://www.nefsc.noaa.gov/publications/crd/crd0908/}{CRD09-08} } \def\HALUNITDraft{} \def\HALUNITSPPname{Atlantic halibut} \def\HALUNITSPPnameT{Atlantic halibut} \def\HALUNITRptYr{2015} \def\HALUNITAuthor{Daniel Hennen} \def\HALUNITReviewerComments{/home/dhennen/EIEIO/BigReport/HAL_UNIT/latex}