\def\CODGMMyPathTab{/net/home2/dhennen/testEIEIO/BigReport/COD_GM/tables} \de
f\CODGMMyPathFig{/net/home2/dhennen/testEIEIO/BigReport/COD_GM/figures} \def\CO
DGMfigFishCap{Total catch of Gulf of Maine Atlantic cod between 1982 and 2014 b
y fleet (commercial and recreational) and disposition (landings and discards).}
 \def\CODGMfigSSBCap{Estimated trends in the spawning stock biomass (SSB) of Gu
lf of Maine Atlantic cod between 1982 and 2014 from the current (solid line) an
d previous (dashed line) assessment and the corresponding $SSB_{Threshold}${} (
$\dfrac{1}{2}${} $SSB_{MSY}${}; horizontal dashed line) as well as $SSB_{Target
}${} $SSB_{MSY}${}; horizontal dotted line) based on the 2015 M=0.2 (A) and M-r
amp (B) assessment models. The 90\% lognormal confidence intervals are shown. T
he red dot indicates the rho-adjusted SSB values that would have resulted had a
 retrospective adjusment been made to either model (see Special Comments sectio
n).} \def\CODGMfigFCap{Estimated trends in the fully selected fishing mortality
 (F) of Gulf of Maine Atlantic cod between 1982 and 2014 from the current (soli
d line) and previous (dashed line) assessment and the corresponding $F_{Thresho
ld}${} (0.185 (M=0.2), 0.187 (M-ramp); dashed line) based on the 2015 M=0.2 (A)
 and M-ramp (B) assessment models. The 90\% lognormal confidence intervals are 
shown. The red dot indicates the rho-adjusted F values that would have resulted
 had a retrospective adjusment been made to either model (see Special Comments 
section).} \def\CODGMfigRecrCap{Estimated trends in age-1 recruitment (000s) of
 Gulf of Maine Atlantic cod between 1982 and 2014 from the current (solid line)
 and previous (dashed line) M=0.2 (A) and M-ramp (B) assessment models. The 90\
% lognormal confidence intervals are shown.} \def\CODGMfigSurvCap{Indices of bi
omass for the Gulf of Maine Atlantic cod between 1963 and 2015 for the Northeas
t Fisheries Science Center (NEFSC) spring and fall bottom trawl surveys and Mas
sachusetts Division of Marine Fisheries (MADMF) spring bottom trawl survey. The
 90\% lognormal confidence intervals are shown.} \def\CODGMPreAmb{This assessme
nt of the Gulf of Maine Atlantic cod (\textit{Gadus morhua}) stock is an update
 of the existing 2014 assessment (Palmer 2014). This assessment updates commerc
ial and recreational fishery catch data, research survey indices of abundance, 
and the analytical ASAPassessment models through 2014. Additionally, stock proj
ections have been updated through 2018. In what follows, there are two populati
on assessment models brought forward from the most recent benchmark assessment 
(2012), the M=0.2 (natural mortality = 0.2) and the M-ramp (M ramps from 0.2 to
 0.4) assessment models (see NEFSC 2013 for a full description of the model for
mulations).} \def\CODGMSoS{ \textbf{State of Stock: }{}Based on this updated as
sessment, the Gulf of Maine Atlantic cod (\textit{Gadus morhua}) stock is overf
ished and overfishing is occurring (Figures \ref{CODGMSSB_plot1}-\ref{CODGMF_pl
ot1}){}. Retrospective adjustments were not made to the model results (see Spec
ial Comments section of this report). Spawning stock biomass (SSB) in 2014 was 
estimated to be 2,225 (mt) under the M=0.2 model and 2,536 (mt) under the M-ram
p model scenario (Table \ref{CODGMCatch_Status_Table}{}) which is 6 and 4\% (re
spectively) of the biomass target, $SSB_{MSY}${} \textit{proxy}{} (40,187 (mt) 
and 59,045 (mt); Figure \ref{CODGMSSB_plot1}{}). The 2014 fully selected fishin
g mortality was estimated to be 0.956 and 0.932 which is 517 and 498\% of the $
F_{MSY}${} \textit{proxy}{}($F_{40\%}${}; 0.185 and 0.187; Figure \ref{CODGMF_p
lot1}{}).} \def\CODGMProj{ \textbf{Projections: }{} Short term projections of m
edian total fishery yield and spawning stock biomass for Gulf of Maine Atlantic
 cod were conducted based on a harvest scenario of fishing at the FMSY proxy be
tween 2016 and 2018. Catch in 2015 was estimated at 279 mt. Recruitment was sam
pled from a cumulative distribution function derived from ASAP estimated age-1 
recruitment between 1982 and 2012. The projection recruitment model declines li
nearly to zero when SSB is below 6.3 kmt under the M=0.2 model and 7.9 kmt unde
r the M-ramp model. The 2015 age-1 recruitment was estimated from the geometric
 mean of the 2010-2014 ASAP recruitment estimates. No retrospective adjustments
 were applied in the projections as the retrospective patterns are similar to t
he 2014 update for which no retrospective adjustments were made; however, the 2
015 assessment review panel recommended that that M=0.2 projections with retros
pective adjustments be brought forward to the SSC for consideration in the eval
uation of uncertainty when setting catch advice (provided in the Supplemental I
nformation Report, \href{http://www.nefsc.noaa.gov/saw/sasi/sasi_report_options
.php}{SASINF}{}). Assumed weights are based on an average of the most recent th
ree years. For the M-ramp model, projections are shown under two assumptions of
 short-term natural mortality: M=0.2 and M=0.4.} \def\CODGMSpecCmt{ \textbf{Spe
cial Comments: } \begin{itemize}{} \item{}What are the most important sources o
f uncertainty in this stock assessment? Explain, and describe qualitatively how
 they affect the assessment results (such as estimates of biomass, F, recruitme
nt, and population projections). \linebreak{} \hspace*{0.5cm} \textit{The large
st source of uncertainty is the estimate of natural mortality. Past investigati
ons into changes in natural mortality over time have been inconclusive (NEFSC 2
013). Different assumptions about natural mortality affect the scale of the bio
mass, recruitment, and fishing mortality estimates. Other areas of uncertainty 
include the retrospective error in the M=0.2 model, residual patterns in the mo
del fits to some of the survey series (e.g., aggregate MADMF spring survey) and
 stock structure.} \item{}Does this assessment model have a retrospective patte
rn? If so, is the pattern minor, or major? (A major retrospective pattern occur
s when the adjusted SSB or $F_{Full}${} lie outside of the approximate joint co
nfidence region for SSB and $F_{Full}${}). \linebreak{} \hspace*{0.5cm} \textit
{The M=0.2 model has a major retrospective pattern (7-year Mohn's rho SSB=0.54,
 F=-0.31) and the M-ramp model has a minor retrospective pattern (7-year Mohn's
 rho SSB=0.20, F=-0.08). The 7-year Mohn's rho values from the current assessme
nt are similar to those from the 2014 assessment (M=0.2: SSB=0.53, F=-0.33; M-r
amp: SSB=0.17, F=-0.05) where the M=0.2 model had a major retrospective pattern
 and the M-ramp model had a minor pattern. No retrospective adjustment have bee
n to the terminal model results or in the base catch projections following the 
recommendations of the SARC 55 and 2014 assessment review panels. The 2015 asse
ssment review panel supported this decision noting that the most recent retrosp
ective 'peel' suggested that an adjustment using the 7-year average may not be 
appropriate. However, the 2015 review panel highlighted the retrospective error
 in the M=0.2 model as a source of uncertainty - it should be noted that the re
trospective error of the most recent peel is larger for the M-ramp model. Shoul
d the retrospective patterns continue then the models may have overestimated sp
awning stock size and underestimated fishing mortality.} \item{}Based on this s
tock assessment, are population projections well determined or uncertain? \line
break{} \hspace*{0.5cm} \textit{Population projections for Gulf of Maine Atlant
ic cod are reasonably well determined and projected boimass from the last asses
sment was within the confidence bounds of the biomass estimated in the current 
assessment. } \item{}Describe any changes that were made to the current stock a
ssessment, beyond incorporating additional years of data and the affect these c
hanges had on the assessment and stock status. \linebreak{} \hspace*{0.5cm} \te
xtit{ This update included several minor changes to model input data including:
 (1) re-estimation of recreational catch from 2004-2014 to account for recent u
pdates to the MRIP data; (2) a revised assumption on recreational discard morta
lity from 30\% to 15\% following a Capizzano et al. 2015 study (unpublished); a
nd (3) re-estimation of 2009-2014 NEFSC spring and fall survey time series usin
g the TOGA station acceptance criterion. Additionally, the ASAP assessment mode
l was run with the likelihood constants option turned off. All of these changes
 had minimal impacts on model results - summaries of the impacts of these chang
es are provided in the Supplemental Information Report (\href{http://www.nefsc.
noaa.gov/saw/sasi/sasi_report_options.php}{SASINF}{}).} \item{}If the stock sta
tus has changed a lot since the previous assessment, explain why this occurred.
 \linebreak{} \hspace*{0.5cm} \textit{There has been no change in stock status 
since the 2014 udpate assessment.} \item{}Indicate what data or studies are cur
rently lacking and which would be needed most to improve this stock assessment 
in the future. \linebreak{} \hspace*{0.5cm} \textit{The Gulf of Maine Atlantic 
cod assessment could be improved with additional studies on natural mortality a
nd stock structure. Additionally, future assessments should consider possible c
hanges in recent fishery selectivity patterns and exlore alternative methods fo
r estimating recruitment. Potential causes of low stock productivity (i.e., low
 recruitment) should also be investigated.} \item{}Are there other important is
sues? \linebreak{} \hspace*{0.5cm} \textit{ When setting catch advice careful a
ttention should be given to the retrospective error present in both models, par
ticularly given the poor performance of previous stock projections. Additionall
y, it is unclear as to which level of natural mortality (M=0.2 or 0.4) to assum
e for the short-term projections under the M-ramp model.} \end{itemize}{}} \def
\CODGMRefr{ \textbf{References: }{} \linebreak{}Northeast Fisheries Science Cen
ter. 2013. 55$^{th}$ Northeast Regional Stock Assessment Workshop (55$^{th}$ SA
W) Assessment Summary Report. US Dept Commer, Northeast Fish Sci Cent Ref Doc. 
13-01; 41 p. Available from: National Marine Fisheries Service, 166 Water Stree
t, Woods Hole, MA 02543-1026 \linebreak{} \linebreak{}Palmer MC. 2014. 2014 Ass
essment update report of the Gulf of Maine Atlantic cod stock. US Dept Commer, 
Northeast Fish Sci Cent Ref Doc. 14-14; 119 p. Available from: National Marine 
Fisheries Service,166 Water Street, Woods Hole, MA 02543-1026 } \def\CODGMDraft
{} \def\CODGMSPPname{Gulf of Maine Atlantic cod} \def\CODGMSPPnameT{Gulf of Mai
ne Atlantic cod} \def\CODGMRptYr{2015} \def\CODGMAuthor{Michael Palmer} \def\CO
DGMReviewerComments{/net/home2/dhennen/testEIEIO/BigReport/COD_GM/latex} \def\C
ODGBMyPathTab{/net/home2/dhennen/testEIEIO/BigReport/COD_GB/tables} \def\CODGBM
yPathFig{/net/home2/dhennen/testEIEIO/BigReport/COD_GB/figures} \def\CODGBfigFi
shCap{Total catch of Georges Bank Atlantic Cod between 1978 and 2014 by fleet (
US commercial, US recreational, or Canadian) and disposition (landings and disc
ards).} \def\CODGBfigSSBCap{Trends in spawning stock biomass of Georges Bank At
lantic Cod between 1978 and 2014 from the current (solid line) and previous (da
shed line) assessment and the corresponding $SSB_{Threshold}${} ($\dfrac{1}{2}$
{} $SSB_{MSY}${} \textit{proxy}{}; horizontal dashed line) as well as $SSB_{Tar
get}${} ($SSB_{MSY}${} \textit{proxy}{}; horizontal dotted line) based on the 2
015 assessment. Biomass was adjusted for a retrospective pattern and the adjust
ment is shown in red. The approximate 90\% lognormal confidence intervals are s
hown.} \def\CODGBfigFCap{Trends in the fully selected fishing mortality ($F_{Fu
ll}${}) of Georges Bank Atlantic Cod between 1978 and 2014 from the current (so
lid line) and previous (dashed line) assessment and the corresponding $F_{Thres
hold}${} ($F_{MSY}${} \textit{proxy}{}=0.169; horizontal dashed line). $F_{Full
}${} was adjusted for a retrospective pattern and the adjustment is shown in re
d, based on the 2015 assessment. The approximate 90\% lognormal confidence inte
rvals are shown.} \def\CODGBfigRecrCap{Trends in Recruits (age 1) (000s) of Geo
rges Bank Atlantic Cod between 1978 and 2014 from the current (solid line) and 
previous (dashed line) assessment. The approximate 90\% lognormal confidence in
tervals are shown.} \def\CODGBfigSurvCap{Indices of biomass for the Georges Ban
k Atlantic Cod between 1963 and 2015 for the Northeast Fisheries Science Center
 (NEFSC) spring and fall, and the DFO research bottom trawl surveys. The approx
imate 90\% lognormal confidence intervals are shown.} \def\CODGBPreAmb{This ass
essment of the Georges Bank Atlantic Cod (\textit{Gadus morhua}) stock is an op
erational update of the existing 2012 benchmark assessment (NEFSC 2013). Based 
on the previous assessment the stock was overfished, and overfishing was ocurri
ng. This 2015 assessment updates commercial fishery catch data, research survey
 indices of abundance, the analytical ASAP assessment model, and reference poin
ts through 2014. Additionally, stock projections have been updated through 2018
.} \def\CODGBSoS{ \textbf{State of Stock: }{}Based on this updated assessment, 
the Georges Bank Atlantic Cod (\textit{Gadus morhua}) stock is overfished and o
verfishing is occurring (Figures \ref{CODGBSSB_plot1}-\ref{CODGBF_plot1}){}. Re
trospective adjustments were made to the model results. Spawning stock biomass 
(SSB) in 2014 was estimated to be 1,804 (mt) which is 1\% of the biomass target
 for this stock ($SSB_{MSY}${} \textit{proxy}{} = 201,152; Figure \ref{CODGBSSB
_plot1}{}). The 2014 fully selected fishing mortality was estimated to be 1.68 
which is 994\% of the overfishing threshold proxy ($F_{MSY}${} \textit{proxy}{}
 = 0.169; Figure \ref{CODGBF_plot1}{}).} \def\CODGBProj{ \textbf{Projections: }
{}Short term projections of biomass were derived by sampling from a two-stage c
umulative distribution function of recruitment estimates from ASAP model result
s, using a 50,000 mt cutpoint. The annual fishery selectivity, maturity ogive, 
and mean weights at age used in projections are the most recent 5 year averages
; retrospective adjustments were applied in the projections.} \def\CODGBSpecCmt
{ \textbf{Special Comments: } \begin{itemize}{} \item{}What are the most import
ant sources of uncertainty in this stock assessment? Explain, and describe qual
itatively how they affect the assessment results (such as estimates of biomass,
 F, recruitment, and population projections). \linebreak{} \hspace*{0.5cm} \tex
tit{The major source of uncertainty is presumbaly the estimate of catch or of n
atural mortality, considering the magnitude of the retrospective bias. These bo
th affect the scale of the biomass, fishing mortality estimates, and the refere
nce point estimates. The catch estimates do not include all discards (e.g.,lobs
ter gear) and includes uncertain estimates of recreational landings and discard
s, and of some commercial discards (e.g., small mesh). Natural mortality (M) of
 Georges Bank Atlantic Cod is not well understood and is assumed constant over 
time in the model. Other sources of uncertainty include possible changes in gro
wth parameters in recent years and how this affects fecundity, the viability of
 eggs/sperm, and the success rate of hatching - all influencing recruitment sur
vival and year class strength.} \item{} Does this assessment model have a retro
spective pattern? If so, is the pattern minor, or major? (A major retrospective
 pattern occurs when the adjusted SSB or $F_{Full}${} lies outside of the appro
ximate joint confidence region for SSB and $F_{Full}${}; see Figure \ref{RhoDec
ision_tab}{}). \linebreak{} \hspace*{0.5cm} \textit{ The 7-year Mohn's \textrho
{}, relative to SSB, was 0.68 in the 2012 assessment and was 2.43 in 2014. The 
7-year Mohn's \textrho{}, relative to F, was -0.46 in the 2012 assessment and w
as -0.72 in 2014. There was a major retrospective pattern for this assessment b
ecause the \textrho{} adjusted estimates of 2014 SSB ($SSB_{\rho}${}=1,804) and
 2014 F ($F_{\rho}${}=1.68) were outside the approximate 90\% confidence region
s around SSB (3,922 - 10,596) and F (0.251 - 0.815). A retrospective adjustment
 was made for both the determination of stock status and for projections of cat
ch in 2016. The retrospective adjustment changed the 2014 SSB from 6,180 to 1,8
04 and the 2014 $F_{Full}${} from 0.463 to 1.68.} \item{}Based on this stock as
sessment, are population projections well determined or uncertain? \linebreak{}
 \hspace*{0.5cm} \textit{Population projections for Georges Bank Atlantic Cod a
re uncertain and likely optimistic. The projections are based on a biomass cutp
oint of 50,000 mt, which has not been produced since 1992. The average recruitm
ent since 1992 has been 4.9 million age 1 fish, whereas during the last 10 year
s, average recruitment has been about 2.7 million age 1 fish. A sensistivity pr
ojection using the most recent 10 years of recruitment was conducted and result
s presented in the SASINF database. } \item{}Describe any changes that were mad
e to the current stock assessment, beyond incorporating additional years of dat
a and the effect these changes had on the assessment and stock status. \linebre
ak{} \hspace*{0.5cm} \textit{ No major changes, other than the addition of rece
nt years of data, were made to the Georges Bank Atlantic Cod assessment for thi
s update. However, recreational catch and commercial discard estimates were rev
ised slightly due to minor changes in the databases, and the application of len
gth frequencies (annual instead of half year) in one instance.} \item{}If the s
tock status has changed a lot since the previous assessment, explain why this o
ccurred. \linebreak{} \hspace*{0.5cm} \textit{As in recent assessments for Geor
ges Bank Atlantic Cod the stock remains in an overfishing and overfished status
.} \item{}Indicate what data or studies are currently lacking and which would b
e needed most to improve this stock assessment in the future. \linebreak{} \hsp
ace*{0.5cm} \textit{The Georges Bank Atlantic Cod assessment could be improved 
with additional studies on natural mortality, growth, and fecundity. Additional
ly, more precise estimates of recreational landings and discards, sampling of f
ish caught by individual recreational anglers, and incorporation of discards in
 the lobster fishery would decrease uncertainty in the discard esimates.} \item
{}Are there other important issues? \linebreak{} \hspace*{0.5cm} \textit{The di
fferences in model assumptions of natural mortality between the SARC GB cod and
 the TRAC eGB cod assessment is problematic for the recovery of the entire GB c
od stock. Model results of the TRAC VPA M=0.8 model are used to determine quota
 for the eGB management unit, so by default, proportionally more cod are being 
removed from eastern GB than what the GB cod ASAP model would predict.} \end{it
emize}{}} \def\CODGBRefr{ \textbf{References: }{} \linebreak{}Northeast Fisheri
es Science Center. 2013. 55$^{th}$ Northeast Regional Stock AssessmentWorkshop 
(55$^{th}$ SAW) Assessment Summary Report. Northeast Fisheries Science CenterRe
ference Document 13-01:43. \linebreak{} \linebreak{}} \def\CODGBDraft{} \def\CO
DGBSPPname{Georges Bank Atlantic Cod} \def\CODGBSPPnameT{Georges Bank Atlantic 
Cod} \def\CODGBRptYr{2015} \def\CODGBAuthor{Loretta O'Brien} \def\CODGBReviewer
Comments{/net/home2/dhennen/testEIEIO/BigReport/COD_GB/latex} \def\HADGBMyPathT
ab{/net/home2/dhennen/testEIEIO/BigReport/HAD_GB/tables} \def\HADGBMyPathFig{/n
et/home2/dhennen/testEIEIO/BigReport/HAD_GB/figures} \def\HADGBfigFishCap{Total
 catch of Georges Bank haddock between 1931 and 2014 by fleet (US Commercial, C
anadian, or foreign fleet) and disposition (landings and discards).} \def\HADGB
figSSBCap{Trends in spawning stock biomass of Georges Bank haddock between 1931
 and 2014 from the current (solid line) and previous (dashed line) assessment a
nd the corresponding $SSB_{Threshold}${} ($\dfrac{1}{2}${} $SSB_{MSY}${} \texti
t{proxy}{}; horizontal dashed line) as well as $SSB_{Target}${} ($SSB_{MSY}${} 
\textit{proxy}{}; horizontal dotted line) based on the 2015 assessment. Biomass
 was adjusted for a retrospective pattern and the adjustment is shown in red. T
he 90\% bootstrap probability intervals are shown.} \def\HADGBfigFCap{Trends in
 the fully selected fishing mortality ($F_{Full}${}) of Georges Bank haddock be
tween 1931 and 2014 from the current (solid line) and previous (dashed line) as
sessment and the corresponding $F_{Threshold}${} ($F_{MSY}${} \textit{proxy}{}=
0.39; horizontal dashed line) based on the 2015 assessment. $F_{Full}${} was ad
justed for a retrospective pattern and the adjustment is shown in red. The 90\%
 bootstrap probability intervals are shown.} \def\HADGBfigRecrCap{Trends in Rec
ruits (age 1) (000s) of Georges Bank haddock between 1931 and 2014 from the cur
rent (solid line) and previous (dashed line) assessment. The 90\% bootstrap pro
bability intervals are shown.} \def\HADGBfigSurvCap{Indices of biomass (Mean kg
/tow) for the Georges Bank haddock stock between 1963 and 2015 for the Northeas
t Fisheries Science Center (NEFSC) spring and fall bottom trawl surveys and the
 DFO winter bottom trawl survey. The approximate 90\% lognormal confidence inte
rvals are shown.} \def\HADGBPreAmb{This assessment of the Georges Bank haddock 
(\textit{Melanogrammus aeglefinus}) stock is an operational update of the exist
ing 2012 update VPA assessment (Brooks et al., 2012). The last benchmark for th
is stock was in 2008 (Brooks et al., 2008). Based on the previous assessment in
 2012, the stock was not overfished, and overfishing was not ocurring. This ass
essment updates commercial fishery catch data, research survey indices of abund
ance, weights and maturity at age, and the analytical VPA assessment model and 
reference points through 2014. Additionally, stock projections have been update
d through 2018.} \def\HADGBSoS{ \textbf{State of Stock: }{}Based on this update
d assessment, the Georges Bank haddock (\textit{Melanogrammus aeglefinus}) stoc
k is not overfished and overfishing is not occurring (Figures \ref{HADGBSSB_plo
t1}-\ref{HADGBF_plot1}){}. Spawning stock biomass (SSB) in 2014 was estimated t
o be 225,080 mt. There was a retrospective pattern, and advice should be based 
on correcting for this bias. The rho-adjusted estimate of SSB is 150,053 mt, wh
ich is 208\% of the biomass target ($SSB_{MSY}${} \textit{proxy}{} = 108,300 mt
; Figure \ref{HADGBSSB_plot1}{}). The 2014 fishing mortality (average for ages 
5-7) was estimated to be 0.159. The rho-adjusted value is 0.241 which is 41\% o
f the overfishing threshold proxy ($F_{MSY}${} \textit{proxy}{} = 0.39; Figure 
\ref{HADGBF_plot1}{}).} \def\HADGBProj{ \textbf{Projections: }{}Short term proj
ections of biomass were derived by sampling from a cumulative distribution func
tion of recruitment estimates from ADAPT VPA (corresponding to SSB$>$75,000 mt 
and dropping the extremely large 1963, 2003, and 2010 year classes, as well as 
the two final year class estimates for 2013 and 2014). The annual fishery selec
tivity, maturity ogive, and mean weights at age used in projection are the most
 recent 5 year averages; retrospective adjustments were applied to the starting
 numbers at age (2015) in the projections.} \def\HADGBSpecCmt{ \textbf{Special 
Comments: } \begin{itemize}{} \item{}What are the most important sources of unc
ertainty in this stock assessment? Explain, and describe qualitatively how they
 affect the assessment results (such as estimates of biomass, F, recruitment, a
nd population projections). \linebreak{} \hspace*{0.5cm} \textit{The largest so
urce of uncertainty is the estimate of 2013 recruitment, which accounts for a s
ubstantial portion of catch and SSB in projections. The rho adjusted projection
s reduce all starting numbers at age by 33\% (i.e., all 2015 numbers at age are
 multiplied by 0.667). Two other exceptionally large year classes were observed
 in 2003 and 2010. The 2003 year class is only 28\% of its initial model estima
te, while the 2010 year class is estimated to be 63\% of it's initial estimate.
 Given only 5 years of data are available to estimate the 2010 year class, it i
s possible that there may be further revisions to the magnitude with more years
 of data. It remains uncertain if the scalar applied to all age classes (based 
on Mohn's rho for SSB) is sufficient to account for future revisions to the 201
3 year class estimate. In addition, the median recruitment in the projections (
the proxy for recruitment at MSY) is 53.4 million, which is greater than 7 of t
he last 10 recruitments even though SSB is above the SSBMSY proxy (Table 1). Wh
ile projections of catch and SSB in the near-term are mostly driven by the 2013
 year class, it is worth noting the magnitude of median projected recruitment r
elative to recent recruitment observations.} \item{}Does this assessment model 
have a retrospective pattern? If so, is the pattern minor, or major? \linebreak
{} \hspace*{0.5cm} \textit{This assessment has a moderate retrospective pattern
, with a Mohn's rho of 0.5 for SSB and -0.34 for F (average F on ages 5 to 7). 
The panel agreed that the rho-adjustment should be applied prior to making proj
ections.} \item{}Based on this stock assessment, are population projections wel
l determined or uncertain? \linebreak{} \hspace*{0.5cm} \textit{As noted in (1)
 above, population projections for Georges Bank haddock are uncertain due to un
certainty about the size of the 2013 year class. Two sensitivity projections we
re conducted. The first sensitivity used biological parameters and fishery sele
ctivity values from the 2010 year class for the 2013 year class. In addition to
 this, a second sensitivity projection was made that doubled the rho-adjustment
 on the 2013 year class (age 2 at the start of 2015) by multiplying it by 0.33.
} \item{}Describe any changes that were made to the current stock assessment, b
eyond incorporating additional years of data and the affect these changes had o
n the assessment and stock status. \linebreak{} \hspace*{0.5cm} \textit{ No cha
nges, other than the incorporation of new data were made to the Georges Bank ha
ddock assessment for this update. However, the criterion for determining accept
able tows on NEFSC surveys used the TOGA protocol rather than the SHG protocol 
(TOGA=132x).} \item{}If the stock status has changed a lot since the previous a
ssessment, explain why this occurred. \linebreak{} \hspace*{0.5cm} \textit{The 
stock status of Georges Bank haddock has not changed.} \item{}Indicate what dat
a or studies are currently lacking and which would be needed most to improve th
is stock assessment in the future. \linebreak{} \hspace*{0.5cm} \textit{Project
ion advice and reference points for Georges Bank haddock are strongly dependent
 on recruitment. A decade ago, extremely large year classes were considered ano
malies (e.g., 1963 and 2003). However, since 2003, there have been two more ext
remely large (2010 and 2013) and one very large (2012) year classes. Future wor
k could focus on recruitment forecasting and providing robust catch advice.} \i
tem{}Are there other important issues? \linebreak{} \hspace*{0.5cm} \textit{The
 Georges Bank haddock assessment has recently developed a moderate retrospectiv
e pattern. This stock assessment has historically performed very consistently. 
This should continue to be monitored. Density-dependent responses in growth sho
uld also continue to be monitored. The switch from SHG to TOGA was ruled out as
 the cause of the retrospective pattern.} \end{itemize}{}} \def\HADGBRefr{ \tex
tbf{References: }{} \linebreak{}Brooks, E.N, M.L. Traver, S.J. Sutherland, L. V
an Eeckhaute, and L. Col. 2008. In. Northeast Fisheries Science Center. 2008. A
ssessment of 19 Northeast Groundfish Stocks through 2007: Report of the 3$^{rd}
$ Groundfish Assessment Review Meeting (GARM III), Northeast Fisheries Science 
Center, Woods Hole, Massachusetts, August 4-8, 2008. US Dep Commer, NOAA Fisher
ies, Northeast Fish Sci Cent Ref Doc. 08-15; 884 p + xvii. http://www.nefsc.noa
a.gov/publications/crd/crd0815/ \linebreak{} \linebreak{}Brooks, E.N, S.J. Suth
erland, L. Van Eeckhaute, and M. Palmer. 2012. In. Northeast Fisheries Science 
Center. 2012. Assessment or Data Updates of 13 Northeast Groundfish Stocks thro
ugh 2010. US Dept Commer, NOAA Fisheries, Northeast Fish Sci Cent Ref Doc. 12-0
6.; 789 p. http://nefsc.noaa.gov/publications/crd/crd1206/ \linebreak{} \linebr
eak{}} \def\HADGBDraft{} \def\HADGBSPPname{Georges Bank haddock} \def\HADGBSPPn
ameT{Georges Bank haddock} \def\HADGBRptYr{2015} \def\HADGBAuthor{Liz Brooks} \
def\HADGBReviewerComments{/net/home2/dhennen/testEIEIO/BigReport/HAD_GB/latex} 
\def\HADGMMyPathTab{/net/home2/dhennen/testEIEIO/BigReport/HAD_GM/tables} \def\
HADGMMyPathFig{/net/home2/dhennen/testEIEIO/BigReport/HAD_GM/figures} \def\HADG
MfigFishCap{Total catch of Gulf of Maine haddock between 1977 and 2014 by fleet
 (commercial, recreational, or foreign) and disposition (landings and discards)
.} \def\HADGMfigSSBCap{Trends in spawning stock biomass (SSB) of Gulf of Maine 
haddock between 1977 and 2014 from the current (solid line) and previous (dashe
d line) assessment and the corresponding $SSB_{Threshold}${} ($\dfrac{1}{2}${} 
$SSB_{MSY}${} \textit{proxy}{}; horizontal dashed line) as well as $SSB_{Target
}${} ($SSB_{MSY}${} \textit{proxy}{}; horizontal dotted line) based on the 2015
 assessment. The approximate 90\% lognormal confidence intervals are shown. The
 red dot indicates the rho-adjusted SSB values that would have resulted had a r
etrospective adjusment been made to either model (see Special Comments section)
.} \def\HADGMfigFCap{Trends in the fully selected fishing mortality (F) of Gulf
 of Maine haddock between 1977 and 2014 from the current (solid line) and previ
ous (dashed line) assessment and the corresponding $F_{Threshold}${} ($F_{MSY}$
{} \textit{proxy}{}=0.468; horizontal dashed line) from the 2015 assessment mod
el. The approximate 90\% lognormal confidence intervals are shown. The red dot 
indicates the rho-adjusted F values that would have resulted had a retrospectiv
e adjusment been made to either model (see Special Comments section).} \def\HAD
GMfigRecrCap{Trends in Recruits (age 1) (000s) of Gulf of Maine haddock between
 1977 and 2014 from the current (solid line) and previous (dashed line) assessm
ent. The approximate 90\% lognormal confidence intervals are shown.} \def\HADGM
figSurvCap{Indices of biomass for the Gulf of Maine haddock between 1963 and 20
15 for the Northeast Fisheries Science Center (NEFSC) spring and fall bottom tr
awl surveys. The approximate 90\% lognormal confidence intervals are shown.} \d
ef\HADGMPreAmb{This assessment of the Gulf of Maine haddock (\textit{Melanogram
mus aeglefinus}) stock is an operational update of the existing 2014 benchmark 
assessment (NEFSC 2014). Based on the previous assessment, the stock was not ov
erfished, and overfishing was not ocurring. This assessment updates commercial 
and recreational fishery catch data, research survey indices of abundance, and 
the analytical ASAP assessment model and reference points through 2014. Additio
nally, stock projections have been updated through 2018} \def\HADGMSoS{ \textbf
{State of Stock: }{}Based on this updated assessment, the Gulf of Maine haddock
 (\textit{Melanogrammus aeglefinus}) stock is not overfished and overfishing is
 not occurring (Figures \ref{HADGMSSB_plot1}-\ref{HADGMF_plot1}){}. Retrospecti
ve adjustments were not made to the model results (see Special Comments section
 of this report). Spawning stock biomass (SSB) in 2014 was estimated to be 10,3
25 (mt) which is 223\% of the biomass target ($SSB_{MSY}${} \textit{proxy}{} = 
4,623; Figure \ref{HADGMSSB_plot1}{}). The 2014 fully selected fishing mortalit
y was estimated to be 0.257 which is 55\% of the overfishing threshold proxy ($
F_{MSY}${} \textit{proxy}{} = $F_{40\%}${} = 0.468; Figure \ref{HADGMF_plot1}{}
).} \def\HADGMProj{ \textbf{Projections: }{}Short term projections of median to
tal fishery yield and spawning stock biomass for Gulf of Maine haddock were con
ducted based on a harvest scenario of fishing at the $F_{MSY}${} \textit{proxy}
{} between 2016 and 2018. Catch in 2015 has been estimated at 885 mt. Recruitme
nt was sampled from a cumulative distribution function of model estimated age-1
 recruitment from 1977-2012. The age-1 estimate in 2015 was generated from the 
geometric mean of the 1977-2014 recruitment series. The annual fishery selectiv
ity, maturity ogive, and mean weights at age used in the projections were estim
ated from the most recent 5 year averages; retrospective adjustments were not a
pplied in the projections. Given the uncertainty in the size of the 2012 and 20
13 year classes and the model's tendency to overestimate large terminal year cl
asses, the 2015 assessment review panel recommended that a sensitivity projecti
on scenario which constrains terminal recruitment ('Constrain terminal R') be b
rought forward to the New England Fishery Management Council's Scientific and S
tatistical Committee (NEFMC SSC) for consideration when setting catch advice; t
hese sensitivity projections are provided in the Supplemental Information Repor
t (\href{http://www.nefsc.noaa.gov/saw/sasi/sasi_report_options.php}{SASINF}{})
.} \def\HADGMSpecCmt{ \textbf{Special Comments: } \begin{itemize}{} \item{}What
 are the most important sources of uncertainty in this stock assessment? Explai
n, and describe qualitatively how they affect the assessment results (such as e
stimates of biomass, F, recruitment, and population projections). \linebreak{} 
\hspace*{0.5cm} \textit{ The largest source of uncertainty in the assessment is
 the estimated size of the 2012 and 2013 year classes. Based on the estimated s
electivity patterns, these year classes are projected to be 30\% selected to th
e fishery in 2016 and 2017 respectively. However, recent changes to the commerc
ial and recreational minimum retention size may result in these year classes re
cruiting to the fishery sooner than projected. The abundance and growth of the 
2012 and 2013 year classes should be monitored and frequent model updates would
 be expected to improve the estimates of year class size and validate projectio
n assumptions.} \item{}Does this assessment model have a retrospective pattern?
 If so, is the pattern minor, or major? (A major retrospective pattern occurs w
hen the adjusted SSB or $F_{Full}${} lie outside of the approximate joint confi
dence region for SSB and $F_{Full}${}). \linebreak{} \hspace*{0.5cm} \textit{Th
is assessment does not exhibit a retrospective pattern and therefore no retrosp
ective adjustments were made to the terminal model results or the short-term ca
tch projections. The 7-year Mohn's rho values on SSB (-0.04) and F (0.03) are s
mall and there were no consistent patterns in the directionality of the retrosp
ective 'peels' (see the Supplemental Information Report, \href{http://www.nefsc
.noaa.gov/saw/sasi/sasi_report_options.php}{SASINF}{}).} \item{}Based on this s
tock assessment, are population projections well determined or uncertain? \line
break{} \hspace*{0.5cm} \textit{Population projections for Gulf of Maine haddoc
k, are reasonably well determined. The projected boimass from the last assessme
nt is below the confidence bounds of the biomass estimated in the current asses
sment; however, this is primarily due to the positive rescaling of the populati
on size that occured from turning the ASAP model likelihood constants option of
f (see next Special Comment).} \item{}Describe any changes that were made to th
e current stock assessment, beyond incorporating additional years of data and t
he affect these changes had on the assessment and stock status. \linebreak{} \h
space*{0.5cm} \textit{ Recreational catch estimates from 2004-2014 were re-esti
mated as part of this update to account for updates to the MRIP data. Additiona
lly, the ASAP model was revised by turning the likelihood constants off; sensit
ivity runs on SAW/SARC 59 model suggest minor positive rescaling of recruitment
 and SSB, negative rescaling of F (sensitivity results are provided in the Supp
lemental Information Report, \href{http://www.nefsc.noaa.gov/saw/sasi/sasi_repo
rt_options.php}{SASINF}{}).} \item{}If the stock status has changed a lot since
 the previous assessment, explain why this occurred. \linebreak{} \hspace*{0.5c
m} \textit{There has been no change in stock status since the previous SAW/SARC
 59 assessment (2014).} \item{}Indicate what data or studies are currently lack
ing and which would be needed most to improve this stock assessment in the futu
re. \linebreak{} \hspace*{0.5cm} \textit{Currently the assessment assumes 50\% 
survival of haddock discarded in the recreational fishery - directed field rese
arch would improve this estimate. Additionally, a better understanding of recru
itment processes may help to improve recruitment forecasting.} \item{}Are there
 other important issues? \linebreak{} \hspace*{0.5cm} \textit{None.} \end{itemi
ze}{}} \def\HADGMRefr{ \textbf{References: }{} \linebreak{}Northeast Fisheries 
Science Center. 2014. 59$^{th}$ Northeast Regional Stock Assessment Workshop (5
9$^{th}$ SAW) Assessment Report. US Dept Commer, Northeast Fish Sci Cent Ref Do
c. 14-09; 782 p. Available from: National Marine Fisheries Service, 166 Water S
treet, Woods Hole, MA 02543-1026 \linebreak{} \linebreak{}} \def\HADGMDraft{} \
def\HADGMSPPname{Gulf of Maine haddock} \def\HADGMSPPnameT{Gulf of Maine haddoc
k} \def\HADGMRptYr{2015} \def\HADGMAuthor{Michael Palmer} \def\HADGMReviewerCom
ments{/net/home2/dhennen/testEIEIO/BigReport/HAD_GM/latex} \def\YELGBMyPathTab{
/net/home2/dhennen/testEIEIO/BigReport/YEL_GB/tables} \def\YELGBMyPathFig{/net/
home2/dhennen/testEIEIO/BigReport/YEL_GB/figures} \def\YELGBfigFishCap{Total ca
tch of Georges Bank Yellowtail Flounder between 1935 and 2014 by fleet (US, Can
adian, or Other) and disposition (landings or discards).} \def\YELGBfigSSBCap{T
rends in average survey biomass (mt) of Georges Bank Yellowtail Flounder betwee
n 2010 and 2015 from the current assessment.} \def\YELGBfigFCap{Trends in the e
xploitation rate (catch/average survey biomass) of Georges Bank Yellowtail Flou
nder between 2010 and 2014 from the current assessment.} \def\YELGBfigRecrCap{}
 \def\YELGBfigSurvCap{Indices of biomass for the Georges Bank Yellowtail Flound
er between 1963 and 2015 for the Canadian DFO and Northeast Fisheries Science C
enter (NEFSC) spring and fall bottom trawl surveys. The approximate 90\% lognor
mal confidence intervals are shown.} \def\YELGBPreAmb{This assessment of the Ge
orges Bank Yellowtail Flounder (\textit{Limanda ferruginea}) stock was reviewed
 during the July 2015 TRAC meeting (Legault et al. 2015). It is an operational 
update of the existing 2014 update assessment (Legault et al. 2014). Based on t
he previous assessment the stock status was unknown, but stock condition was po
or. This assessment updates commercial fishery catch data through 2014 (Table \
ref{YELGBCatch_Status_Table}{}, Figure \ref{YELGBFish_plot1}{}), and updates re
search survey indices of abundance and the empirical approach assessment throug
h 2015 (Figure \ref{YELGBSurv_plot1}{}). No stock projections can be computed u
sing the empirical approach.} \def\YELGBSoS{ \textbf{State of Stock: }{}Based o
n this updated assessment, Georges Bank Yellowtail Flounder (\textit{Limanda fe
rruginea}) stock status is unknown due to a lack of biological reference points
 associated with the empirical approach, but stock condition is poor. Retrospec
tive adjustments were not made to the model results. The average survey biomass
 in 2015 (the arithmetic average of the 2015 DFO, 2015 NEFSC spring, and 2014 N
EFSC fall surveys) was estimated to be 2,240 (mt) (Figure \ref{YELGBSSB_plot1}{
}). The 2014 exploitation rate (2014 catch divided by 2014 average survey bioma
ss) was estimated to be 0.071 (Figure \ref{YELGBF_plot1}{}).} \def\YELGBProj{ \
textbf{Projections: }{}Short term projections cannot be computed using the empi
rical approach. Application of an exploitation rate of 2\% to 16\% to the 2015 
average survey biomass (2,240 mt) results in catch advice for 2016 of 45 mt to 
359 mt.} \def\YELGBSpecCmt{ \textbf{Special Comments: } \begin{itemize}{} \item
{}What are the most important sources of uncertainty in this stock assessment? 
Explain, and describe qualitatively how they affect the assessment results (suc
h as estimates of biomass, F, recruitment, and population projections). \linebr
eak{} \hspace*{0.5cm} \textit{The largest source of uncertainty is the estimate
 of survey catchability, which currently relies on literature values for other 
species in other regions of the world using different gear. The survey catchabi
lity affects the expansion of the stratified mean catch per tow for each survey
 and is inversely related to the catch advice. Other sources of uncertainty inc
lude the appropriate exploitation rate to apply to this stock, which has seen c
ontinued decrease in survey biomass despite low exploitation rates. } \item{} D
oes this assessment model have a retrospective pattern? If so, is the pattern m
inor, or major? (A major retrospective pattern occurs when the adjusted SSB or 
$F_{Full}${} lies outside of the approximate joint confidence region for SSB an
d $F_{Full}${}; see RhoDecisionTab.ref). \linebreak{} \hspace*{0.5cm} \textit{ 
The model used to estimate status of this stock does not allow estimation of a 
retrospective pattern. } \item{}Based on this stock assessment, are population 
projections well determined or uncertain? \linebreak{} \hspace*{0.5cm} \textit{
Population projections for Georges Bank Yellowtail Flounder are not computed. C
atch advice is derived from applying an exploitation rate to the current estima
te of survey biomass. } \item{}Describe any changes that were made to the curre
nt stock assessment, beyond incorporating additional years of data and the affe
ct these changes had on the assessment and stock status. \linebreak{} \hspace*{
0.5cm} \textit{The 2014 NMFS spring survey value was changed from 2,684 mt to 2
,763 mt due to using preliminary data during the 2014 TRAC meeting. However, th
is has no impact on the 2015 stock status or 2016 catch advice in this update a
ssessment.} \item{}If the stock status has changed a lot since the previous ass
essment, explain why this occurred. \linebreak{} \hspace*{0.5cm} \textit{The st
ock status of Georges Bank Yellowtail Flounder remains unknown and stock condit
ion continues to be poor.} \item{}Indicate what data or studies are currently l
acking and which would be needed most to improve this stock assessment in the f
uture. \linebreak{} \hspace*{0.5cm} \textit{The Georges Bank Yellowtail Flounde
r assessment could be improved with studies on NMFS and DFO survey catchability
 for flatfish.} \item{}Are there other important issues? \linebreak{} \hspace*{
0.5cm} \textit{None. } \end{itemize}{}} \def\YELGBRefr{ \textbf{References: }{}
 \linebreak{}Legault, C.M., L. Alade, W.E. Gross, and H.H. Stone. 2014. Stock A
ssessment of Georges Bank Yellowtail Flounder for 2014. TRAC Ref. Doc. 2014/01.
 214 p. \linebreak{}Legault, C.M., L. Alade, D. Busawon, and H.H. Stone. 2015. 
Stock Assessment of Georges Bank Yellowtail Flounder for 2015. TRAC Ref. Doc. 2
015/01. 66 p. \linebreak{}} \def\YELGBDraft{} \def\YELGBSPPname{Georges Bank Ye
llowtail Flounder} \def\YELGBSPPnameT{Georges Bank Yellowtail Flounder} \def\YE
LGBRptYr{2015} \def\YELGBAuthor{Chris Legault} \def\YELGBReviewerComments{/net/
home2/dhennen/testEIEIO/BigReport/YEL_GB/latex} \def\YELSNEMAMyPathTab{/net/hom
e2/dhennen/testEIEIO/BigReport/YEL_SNEMA/tables} \def\YELSNEMAMyPathFig{/net/ho
me2/dhennen/testEIEIO/BigReport/YEL_SNEMA/figures} \def\YELSNEMAfigFishCap{Tota
l catch of Southern New England-Mid Atlantic Yellowtail flounder between 1973 a
nd 2014 by fleet (US domestic and foreign catch) and disposition (landings and 
discards).} \def\YELSNEMAfigSSBCap{Trends in spawning stock biomass of Southern
 New England-Mid Atlantic Yellowtail flounder between 1973 and 2014 from the cu
rrent (solid line) and previous (dashed line) assessment and the corresponding 
$SSB_{Threshold}${} ($\dfrac{1}{2}${} $SSB_{MSY}${} \textit{proxy}{}; horizonta
l dashed line) as well as $SSB_{Target}${} ($SSB_{MSY}${} \textit{proxy}{}; hor
izontal dotted line) based on the 2015 assessment. Biomass was adjusted for a r
etrospective pattern and the adjustment is shown in red. The approximate 90\% l
ognormal confidence intervals are shown.} \def\YELSNEMAfigFCap{Trends in the fu
lly selected fishing mortality ($F_{Full}${}) of Southern New England-Mid Atlan
tic Yellowtail flounder between 1973 and 2014 from the current (solid line) and
 previous (dashed line) assessment and the corresponding $F_{Threshold}${} ($F_
{MSY}${} \textit{proxy}{}=0.35; horizontal dashed line). $F_{Full}${} was adjus
ted for a retrospective pattern and the adjustment is shown in red based on the
 2015 assessment. The approximate 90\% lognormal confidence intervals are shown
.} \def\YELSNEMAfigRecrCap{Trends in Recruits (age 1) (000s) of Southern New En
gland-Mid Atlantic Yellowtail flounder between 1973 and 2014 from the current (
solid line) and previous (dashed line) assessment. The approximate 90\% lognorm
al confidence intervals are shown.} \def\YELSNEMAfigSurvCap{Indices of biomass 
for the Southern New England-Mid Atlantic Yellowtail flounder between 1973 and 
2015 for the Northeast Fisheries Science Center (NEFSC) spring, fall and winter
 bottom trawl surveys. The approximate 90\% lognormal confidence intervals are 
shown.Note: Larval index was also used in this assessment and is available in t
he supplemental documentation} \def\YELSNEMAPreAmb{This assessment of the South
ern New England-Mid Atlantic Yellowtail flounder (\textit{Limanda ferruginea}) 
stock is an operational update of the existing 2012 benchmark ASAP assessment (
NEFSC 2012). Based on the previous assessment the stock was not overfished, and
 overfishing was not ocurring. This assessment updates commercial fishery catch
 data, research survey indices of abundance, weights at age and the analytical 
ASAP assessment model and reference points through 2014. Additionally, stock pr
ojections have been updated through 2018} \def\YELSNEMASoS{ \textbf{State of St
ock: }{}Based on this updated assessment, Southern New England-Mid Atlantic Yel
lowtail flounder (\textit{Limanda ferruginea}) stock is overfished and overfish
ing is occurring (Figures \ref{YELSNEMASSB_plot1}-\ref{YELSNEMAF_plot1}){}. Ret
rospective adjustments were not made to the model results. Spawning stock bioma
ss (SSB) in 2014 was estimated to be 502 (mt) which is 26\% of the biomass targ
et ($SSB_{MSY}${} \textit{proxy}{} = 1,959; Figure \ref{YELSNEMASSB_plot1}{}). 
The 2014 fully selected fishing mortality was estimated to be 1.64 which is 469
\% of the overfishing threshold proxy ($F_{MSY}${} \textit{proxy}{} = 0.35; Fig
ure \ref{YELSNEMAF_plot1}{}).} \def\YELSNEMAProj{ \textbf{Projections: }{}Short
 term projections of biomass were derived by sampling from a cumulative distrib
ution function of recruitment estimates from ASAP. Following the previous and a
ccepted benchmark formulation, recruitment was based on the more recent estimat
es of the model time series (i.e. corresponding to year classes 1990 through 20
13) to reflect the low recent pattern in recruitment. The annual fishery select
ivity, maturity ogive, and mean weights at age used in projection are the most 
recent 5 year averages; retrospective adjustments were not applied in the proje
ctions.} \def\YELSNEMASpecCmt{ \textbf{Special Comments: } \begin{itemize}{} \i
tem{}What are the most important sources of uncertainty in this stock assessmen
t? Explain, and describe qualitatively how they affect the assessment results (
such as estimates of biomass, F, recruitment, and population projections). \lin
ebreak{} \hspace*{0.5cm} \textit{The largest source of uncertainty is the emerg
ence of the retrospective in this updated assessment. This retrospective bias h
as resulted in the reduction SSB estimates and F estimates to increase with add
itional years of data Further, the basis for recruitment assumption for stock s
tatus determination and population forecast (i.e. the inclusion of historical r
ecruitment values versus contemporary basis of recruitment) is another source o
f uncertainty. Although recent estmated recruitment likely reflect the realisti
c conditions for the stock, the basis for recruitment selection is not clearly 
understood.} \item{} Does this assessment model have a retrospective pattern? I
f so, is the pattern minor, or major? (A major retrospective pattern occurs whe
n the adjusted SSB or $F_{Full}${} lies outside of the approximate joint confid
ence region for SSB and $F_{Full}${}; see RhoDecisionTab.ref). \linebreak{} \hs
pace*{0.5cm} \textit{ The 7-year Mohn's \textrho{}, relative to SSB, was 0.14 i
n the 2012 assessment and was 1.06 in 2014. The 7-year Mohn's \textrho{}, relat
ive to F, was -0.16 in the 2012 assessment and was -0.53 in 2014. There was a m
ajor retrospective pattern for this assessment because the \textrho{} adjusted 
estimates of 2014 SSB ($SSB_{\rho}${}=502) and 2014 F ($F_{\rho}${}=1.64) were 
outside the approximate 90\% confidence regions around SSB (355 - 739) and F (1
.053 - 2.348). However, a retrospective adjustment was not made for both the de
termination of stock status and for projections of catch because of the large p
roportion of unfeasible projections (assumed 2015 catch required a fishing mort
ality rate greater than 5). This implies the retrospective adjustment was too l
arge or the assumed 2015 catch was too high. The review panel decided to use th
e unadjusted projections as an upper bound for OFL with the strong suggestion t
hat the OFL estimates were too high (meaning the ABC buffer should be larger th
an normal).} \item{}Based on this stock assessment, are population projections 
well determined or uncertain? \linebreak{} \hspace*{0.5cm} \textit{Population p
rojections are uncertain with projected biomass from the last assessment above 
the confidence bounds of the biomass estimate in the current assessment. Furthe
r, the short-term projections which accounted for retropective adjustment in th
e starting numbers-at-age were unrelaible due to the low percentage of feasible
 solutions (33\%) encountered durring the simulation. The feasibility problem i
n the projections were due to the assumed 2015 projected cacth exceeding the po
pulation biomass in several of the iteration caused by the retrospective adjust
ment. Evaluation of the the estimated January-1 2015 biomass from the few feasb
ile projections indicated that the assumed 2015 catch was approximately 98\% of
 the stock biomass. This suggests that the assumed 2015 catch is not sustainabl
e given the low starting abundance in the forecast. Alternatively, the retro un
adjusted projections performed well, but it is likely to result in an overly op
timistic projection of the fishery yield and population biomass.} \item{}Descri
be any changes that were made to the current stock assessment, beyond incorpora
ting additional years of data and the affect these changes had on the assessmen
t and stock status. \linebreak{} \hspace*{0.5cm} \textit{ There were no major c
hanges to the current stock assessment formulation. However, the criterion for 
determining acceptable tows on the NEFSC surveys were revised for years the Big
elow year (i.e. 2009-2011) and carried foreward to ensure consistency between t
he assessment and deck operations. The influence of the revised protocol on the
 survey indices was inconsequential.} \item{}If the stock status has changed a 
lot since the previous assessment, explain why this occurred. \linebreak{} \hsp
ace*{0.5cm} \textit{The overfishing and biomass stock status have changed since
 the previous assessment due to increased catches relative to the stock biomass
 and the very low recruitment of young fish, contributing very little to the ad
ult biomass.} \item{}Indicate what data or studies are currently lacking and wh
ich would be needed most to improve this stock assessment in the future. \lineb
reak{} \hspace*{0.5cm} \textit{The emergence of retrospective bias in this asse
ssment is not clearly understood and may result from a variety of sources. Futu
re studiesshould further investigate the source of this retrospective pattern t
o help improve the underlying diagnostics of the model for providing catch advi
ce for this stock. Recruitment for Southern New England-Mid Atlantic yellowtail
 flounder continues to be weak and it is likely that the stock is in a new prod
uctivity regime. Should this pattern of poor recruitment continue into the futu
re, the ability of the stock to recover will be impeded. Therefore, future stud
ies should build on current knowledge to further understand the underlying ecol
ogical mechanisms of poor recruitment in the stock as it may relate to the phys
ical environment.} \item{}Are there other important issues? \linebreak{} \hspac
e*{0.5cm} \textit{None. } \end{itemize}{}} \def\YELSNEMARefr{ \textbf{Reference
s: }{} \linebreak{} Alade, L, C. Legault, S.Cadrin. 2008. In. Northeast Fisheri
es Science Center. 2008. Assessment of 19 Northeast Groundfish Stocks through 2
007: Report of the 3$^{rd}$ Groundfish Assessment Review Meeting (GARM III), No
rtheast Fisheries Science Center, Woods Hole, Massachusetts, August 4-8, 2008. 
US Dep Commer, NOAA Fisheries, Northeast Fish Sci Cent Ref Doc. 08-15; 884 p + 
xvii. http://www.nefsc.noaa.gov/publications/crd/crd0815/ \linebreak{} \linebre
ak{} Northeast Fisheries Science Center. 2012. 54$^{th}$ Northeast Regional Sto
ck Assessment Workshop (54$^{th}$ SAW) Assessment Report. US Dept Commer, NOAA 
Fisheries, Northeast Fish Sci Cent Ref Doc. 12-18.; 600 p. http://nefsc.noaa.go
v/publications/crd/crd1218/ \linebreak{} \linebreak{}} \def\YELSNEMADraft{} \de
f\YELSNEMASPPname{Southern New England-Mid Atlantic Yellowtail flounder} \def\Y
ELSNEMASPPnameT{Southern New England-Mid Atlantic Yellowtail flounder} \def\YEL
SNEMARptYr{2015} \def\YELSNEMAAuthor{Larry Alade} \def\YELSNEMAReviewerComments
{/net/home2/dhennen/testEIEIO/BigReport/YEL_SNEMA/latex} \def\YELCCGMMyPathTab{
/net/home2/dhennen/testEIEIO/BigReport/YEL_CCGM/tables} \def\YELCCGMMyPathFig{/
net/home2/dhennen/testEIEIO/BigReport/YEL_CCGM/figures} \def\YELCCGMfigFishCap{
Total catch of Cape Cod-Gulf of Maine Yellowtail flounder between 1985 and 2014
 by disposition (landings and discards).} \def\YELCCGMfigSSBCap{Trends in spawn
ing stock biomass of Cape Cod-Gulf of Maine Yellowtail flounder between 1985 an
d 2014 from the current (solid line) and previous (dashed line) assessment and 
the corresponding $SSB_{Threshold}${} ($\dfrac{1}{2}${} $SSB_{MSY}${} \textit{p
roxy}{}; horizontal dashed line) as well as $SSB_{Target}${} ($SSB_{MSY}${} \te
xtit{proxy}{}; horizontal dotted line) based on the 2015 assessment. Biomass wa
s adjusted for a retrospective pattern and the adjustment is shown in red. The 
90\% bootstrap probability intervals are shown.} \def\YELCCGMfigFCap{Trends in 
the fully selected fishing mortality ($F_{Full}${}) of Cape Cod-Gulf of Maine Y
ellowtail flounder between 1985 and 2014 from the current (solid line) and prev
ious (dashed line) assessment and the corresponding $F_{Threshold}${} ($F_{MSY}
${} \textit{proxy}{}=0.279; horizontal dashed line). $F_{Full}${} was adjusted 
for a retrospective pattern and the adjustment is shown in red based on the 201
5 assessment. The 90\% bootstrap probability intervals are shown.} \def\YELCCGM
figRecrCap{Trends in Recruits (age 1) (000s) of Cape Cod-Gulf of Maine Yellowta
il flounder between 1985 and 2014 from the current (solid line) and previous (d
ashed line) assessment. The 90\% bootstrap probability intervals are shown.} \d
ef\YELCCGMfigSurvCap{Indices of biomass for the Cape Cod-Gulf of Maine Yellowta
il flounder between 1985 and 2015 for the Northeast Fisheries Science Center (N
EFSC) spring and fall bottom trawl surveys, Massachusetts Department of Marine 
Fisheries (MADMF) inshore state spring and fall bottom trawl surveys,and the Ma
ine New Hampshire inshore state spring and fall state surveys The 90\% bootstra
p probability intervals are shown.} \def\YELCCGMPreAmb{This assessment of the C
ape Cod-Gulf of Maine Yellowtail flounder (\textit{Limanda ferruginea}) stock i
s an operational update of the existing 2012 VPA assessment (Legault et al., 20
12). The last benchmark for this stock was in 2008 (Legault et al., 2008). Base
d on the previous assessment the stock was overfished, and overfishing was ocur
ring. This assessment updates commercial fishery catch data, research survey in
dices of abundance, weights at age, and the analytical VPA assessment model and
 reference points through 2014. Additionally, stock projections have been updat
ed through 2018} \def\YELCCGMSoS{ \textbf{State of Stock: }{}Based on this upda
ted assessment, Cape Cod-Gulf of Maine Yellowtail flounder (\textit{Limanda fer
ruginea}) stock is overfished and overfishing is occurring (Figures \ref{YELCCG
MSSB_plot1}-\ref{YELCCGMF_plot1}){}. Retrospective adjustments were made to the
 model results. Spawning stock biomass (SSB) in 2014 was estimated to be 857 (m
t) which is 16\% of the biomass target ($SSB_{MSY}${} \textit{proxy}{} = 5,259;
 Figure \ref{YELCCGMSSB_plot1}{}). The 2014 fully selected fishing mortality wa
s estimated to be 0.64 which is 229\% of the overfishing threshold proxy ($F_{M
SY}${} \textit{proxy}{} = 0.279; Figure \ref{YELCCGMF_plot1}{}).} \def\YELCCGMP
roj{ \textbf{Projections: }{}Short term projections of biomass were derived by 
sampling from a cumulative distribution function of recruitment estimates from 
ADAPT VPA. Recruitment estimates were hindcasted based on a simple linear regre
ssion between the NEFSC Fall survey abundance at age 1 and the VPA estimate at 
age 1. The most recent two years (2013 and 2014) were not included in the serie
s of values due to high uncertainty in these estimates. This resulted in a tota
l of 36 recruitment values: 8 from the hindcast predictions (years 1977-1984) a
nd 28 from the VPA (years 1985-2012). The annual fishery selectivity, maturity 
ogive, and mean weights at age used in projection are the most recent 5 year av
erages; retrospective adjustments were applied in the projections.} \def\YELCCG
MSpecCmt{ \textbf{Special Comments: } \begin{itemize}{} \item{}What are the mos
t important sources of uncertainty in this stock assessment? Explain, and descr
ibe qualitatively how they affect the assessment results (such as estimates of 
biomass, F, recruitment, and population projections). \linebreak{} \hspace*{0.5
cm} \textit{The largest source of uncertainty is the source of the retrospectiv
e pattern.This pattern has persisted for a number of years causing SSB estimate
s to decrease and F estimates to increaseas more years of data are added.} \ite
m{} Does this assessment model have a retrospective pattern? If so, is the patt
ern minor, or major? (A major retrospective pattern occurs when the adjusted SS
B or $F_{Full}${} lies outside of the approximate joint confidence region for S
SB and $F_{Full}${}; see RhoDecisionTab.ref). \linebreak{} \hspace*{0.5cm} \tex
tit{ The 7-year Mohn's \textrho{}, relative to SSB, was 0.68 in the 2012 assess
ment and was 0.98 in 2014. The 7-year Mohn's \textrho{}, relative to F, was -0.
19 in the 2012 assessment and was -0.45 in 2014. There was a major retrospectiv
e pattern for this assessment because the \textrho{} adjusted estimates of 2014
 SSB ($SSB_{\rho}${}=857) and 2014 F ($F_{\rho}${}=0.64) were outside the appro
ximate 90\% confidence regions around SSB (1,375 - 2,111) and F (0.25 - 0.52). 
A retrospective adjustment was made for both the determination of stock status 
and for projections of catch in 2016. The retrospective adjustment changed the 
2014 SSB from 1,695 to 857 and the 2014 $F_{Full}${} from 0.355 to 0.64.} \item
{}Based on this stock assessment, are population projections well determined or
 uncertain? \linebreak{} \hspace*{0.5cm} \textit{Population projections for Cap
e Cod-Gulf of Maine Yellowtail flounder, are uncertain with projected biomass f
rom the last assessmentabove the confidence bounds of the biomass estimated in 
the current assessment.} \item{}Describe any changes that were made to the curr
ent stock assessment, beyond incorporating additional years of data and the aff
ect these changes had on the assessment and stock status. \linebreak{} \hspace*
{0.5cm} \textit{ No changes, other than the incorporation of new data were made
 to the Cape Cod-Gulf of Maine Yellowtail flounder assessment for this update.}
 \item{}If the stock status has changed a lot since the previous assessment, ex
plain why this occurred. \linebreak{} \hspace*{0.5cm} \textit{The stock status 
has not changed since the previous assessment.} \item{}Indicate what data or st
udies are currently lacking and which would be needed most to improve this stoc
k assessment in the future. \linebreak{} \hspace*{0.5cm} \textit{Extensive stud
ies have examined the causes of the retrospective patterns with no definitive c
onclusions other than a change in model does not resolve the issue.} \item{}Are
 there other important issues? \linebreak{} \hspace*{0.5cm} \textit{No. } \end{
itemize}{}} \def\YELCCGMRefr{ \textbf{References: }{} \linebreak{}Legault, C, L
. Alade, S.Cadrin, J. King, and S. Sherman. 2008. In. Northeast Fisheries Scien
ce Center. 2008. Assessment of 19 Northeast Groundfish Stocks through 2007: Rep
ort of the 3$^{rd}$ Groundfish Assessment Review Meeting (GARM III), Northeast 
Fisheries Science Center, Woods Hole, Massachusetts, August 4-8, 2008. US Dep C
ommer, NOAA Fisheries, Northeast Fish Sci Cent Ref Doc. 08-15; 884 p + xvii. ht
tp://www.nefsc.noaa.gov/publications/crd/crd0815/ \linebreak{} \linebreak{} Leg
ault, C, L. Alade, S.Emery, J. King, and S. Sherman. 2012. In. Northeast Fisher
ies Science Center. 2012. Assessment or Data Updates of 13 Northeast Groundfish
 Stocks through 2010. US Dept Commer, NOAA Fisheries, Northeast Fish Sci Cent R
ef Doc. 12-06.; 789 p. http://nefsc.noaa.gov/publications/crd/crd1206/ \linebre
ak{} \linebreak{}} \def\YELCCGMDraft{} \def\YELCCGMSPPname{Cape Cod-Gulf of Mai
ne Yellowtail flounder} \def\YELCCGMSPPnameT{Cape Cod-Gulf of Maine Yellowtail 
flounder} \def\YELCCGMRptYr{2015} \def\YELCCGMAuthor{Larry Alade} \def\YELCCGMR
eviewerComments{/net/home2/dhennen/testEIEIO/BigReport/YEL_CCGM/latex} \def\FLW
GMMyPathTab{/net/home2/dhennen/testEIEIO/BigReport/FLW_GM/tables} \def\FLWGMMyP
athFig{/net/home2/dhennen/testEIEIO/BigReport/FLW_GM/figures} \def\FLWGMfigFish
Cap{Total catch of Gulf of Maine Winter Flounder between 2009 and 2014 by fleet
 (commercial and recreational) and disposition (landings and discards). A 15\% 
mortality rate is assumed on recreational discards and a 50\% mortality rate on
 commercial discards.} \def\FLWGMfigSSBCap{Trends in 30+ cm area-swept biomass 
of Gulf of Maine Winter Flounder between 2009 and 2014 from the current assessm
ent based on the fall (MENH, MDMF, NEFSC) surveys. The approximate 90\% lognorm
al confidence intervals are shown.} \def\FLWGMfigFCap{Trends in the exploitatio
n rates ($E_{Full}${}) of Gulf of Maine Winter Flounder between 2009 and 2014 f
rom the current assessment and the corresponding $F_{Threshold}${} ($E_{MSY}${}
 \textit{proxy}{}=0.23; horizontal dashed line). The approximate 90\% lognormal
 confidence intervals are shown.} \def\FLWGMfigRecrCap{} \def\FLWGMfigSurvCap{I
ndices of biomass for the Gulf of Maine Winter Flounder between 1978 and 2015 f
or the Northeast Fisheries Science Center (NEFSC), Massachusetts Division of Ma
rine Fisheries (MDMF), and the Maine New Hampshire (MENH) spring and fall botto
m trawl surveys. NEFSC indices are calculated with gear and vessel conversion f
actors where appropriate. The approximate 90\% lognormal confidence intervals a
re shown.} \def\FLWGMPreAmb{This assessment of the Gulf of Maine Winter Flounde
r (\textit{Pseudopleuronectes americanus}) stock is an operational update of th
e existing 2014 operational update area-swept assessment (NEFSC 2014). Based on
 the previous assessment the biomass status is unknown but overfishing was not 
occurring. This assessment updates commercial and recreational fishery catch da
ta, research survey indices of abundance, and the area-swept estimates of 30+ c
m biomass based on the fall NEFSC, MDMF, and MENH surveys.} \def\FLWGMSoS{ \tex
tbf{State of Stock: }{}Based on this updated assessment, the Gulf of Maine Wint
er Flounder (\textit{Pseudopleuronectes americanus}) stock biomass status is un
known and overfishing is not occurring (Figures \ref{FLWGMSSB_plot1}-\ref{FLWGM
F_plot1}){}. Retrospective adjustments were not made to the model results. Biom
ass (30+ cm mt) in 2014 was estimated to be 4,655 mt (Figure \ref{FLWGMSSB_plot
1}{}). The 2014 30+ cm exploitation rate was estimated to be 0.06 which is 26\%
 of the overfishing exploitation threshold proxy ($E_{MSY}${} \textit{proxy}{} 
= 0.23; Figure \ref{FLWGMF_plot1}{}).} \def\FLWGMProj{ \textbf{Projections: }{}
Projections are not possible with area-swept based assessments. Catch advice wa
s based on 75\% of $E_{40\%}${}(75\% $E_{MSY}${} \textit{proxy}{}) using the fa
ll area-swept estimate assuming q=0.6 on the wing spread. Updated 2014 fall 30+
 cm area-swept biomass (4,655 mt) implies an OFL of 1,080 mt based on the $E_{M
SY}${} \textit{proxy}{} and a catch of 810 mt for 75\% of the $E_{MSY}${} \text
it{proxy}{}.} \def\FLWGMSpecCmt{ \textbf{Special Comments: } \begin{itemize}{} 
\item{}What are the most important sources of uncertainty in this stock assessm
ent? Explain, and describe qualitatively how they affect the assessment results
 (such as estimates of biomass, F, recruitment, and population projections). \l
inebreak{} \hspace*{0.5cm} \textit{The largest source of uncertainty with the d
irect estimates of stock biomass from survey area-swept estimates originate fro
m the assumption of survey gear catchability (q). Biomass and exploitation rate
 estimates are sensitive to the survey q assumption (0.6 on wing spread). The 2
014 empirical benchmark assessement of Georges bank yellowtail flounder based t
he area-swept q assumption on an average value taken from the literature for we
st coast flatfish (0.37 on door spread). The yellowtail q assumption correspond
s to a value close to 1 on the wing spread which would result in a lower estima
te of biomass (2,995 mt). Another major source of uncertainty with this method 
is that biomass based reference points cannot be determined and overfished stat
us is unknown. } \item{} Does this assessment model have a retrospective patter
n? If so, is the pattern minor, or major? (A major retrospective pattern occurs
 when the adjusted SSB or $F_{Full}${} lies outside of the approximate joint co
nfidence region for SSB and $F_{Full}${}; see Figure \ref{RhoDecision_tab}{}). 
\linebreak{} \hspace*{0.5cm} \textit{ The model used to determine status of thi
s stock does not allow estimation of a retrospective pattern. An analytical sto
ck assessment model does not exist for Gulf of Maine Winter Flounder. An analyt
ical model was no longer used for stock status determination at SARC 52 (2011) 
due to concerns with a strong retrospective pattern. Models have difficulty wit
h the apparent lack of a relationship between a large decrease in the catch wit
h little change in the indices and age and/or size structure over time. } \item
{}Based on this stock assessment, are population projections well determined or
 uncertain? \linebreak{} \hspace*{0.5cm} \textit{Population projections for Gul
f of Maine Winter Flounder, do not exist for area-swept assessments. Catch advi
ce from area-swept estimates tend to vary with interannual variability in the s
urveys.} \item{}Describe any changes that were made to the current stock assess
ment, beyond incorporating additional years of data and the affect these change
s had on the assessment and stock status. \linebreak{} \hspace*{0.5cm} \textit{
 No changes, other than the incorporation of new data were made to the Gulf of 
Maine Winter Flounder assessment for this update. However, stabilizing the catc
h advice may be desired and could be obtained through the averaging of the area
-swept fall and spring survey estimates.} \item{}If the stock status has change
d a lot since the previous assessment, explain why this occurred. \linebreak{} 
\hspace*{0.5cm} \textit{The overfishing status of Gulf of Maine Winter Flounder
 has not changed. } \item{}Indicate what data or studies are currently lacking 
and which would be needed most to improve this stock assessment in the future. 
\linebreak{} \hspace*{0.5cm} \textit{Direct area-swept assessment could be impr
oved with additional studies on survey gear efficiency. Quantifying the degree 
of herding between the doors and escapement under the footrope and/or above the
 headrope for each survey is needed since area-swept biomass estimates and catc
h advice are sensitive to the assumed catchability.} \item{}Are there other imp
ortant issues? \linebreak{} \hspace*{0.5cm} \textit{The general lack of a respo
nse in survey indices and age/size structure is the primary source of concern w
ith catches remaining far below the overfishing level. } \end{itemize}{}} \def\
FLWGMRefr{ \textbf{References: }{} \linebreak{}Hendrickson L, Nitschke P, Linto
n B. 2015. 2014 Operational Stock Assessments for Georges Bank winter flounder,
 Gulf of Maine winter flounder, and pollock. US Dept Commer, Northeast Fish Sci
 Cent Ref Doc. 15-01; 228 p. Available from: National Marine Fisheries Service,
 166 Water Street, Woods Hole, MA 02543-1026, or online at http://nefsc.noaa.go
v/publications/ \linebreak{} \linebreak{}Northeast Fisheries Science Center. 20
11. 52$^{nd}$ Northeast Regional Stock AssessmentWorkshop (52$^{nd}$ SAW) Asses
sment Report. US Dept Commer, Northeast Fish SciCent Ref Doc. 11-17; 962 p. Ava
ilable from: National Marine Fisheries Service, 166 Water Street, Woods Hole, M
A 02543-1026, or online at http://www.nefsc.noaa.gov/nefsc/publications/ \lineb
reak{} \linebreak{}} \def\FLWGMDraft{} \def\FLWGMSPPname{Gulf of Maine Winter F
lounder} \def\FLWGMSPPnameT{Gulf of Maine Winter Flounder} \def\FLWGMRptYr{2015
} \def\FLWGMAuthor{Paul Nitschke} \def\FLWGMReviewerComments{/net/home2/dhennen
/testEIEIO/BigReport/FLW_GM/latex} \def\FLWSNEMAMyPathTab{/net/home2/dhennen/te
stEIEIO/BigReport/FLW_SNEMA/tables} \def\FLWSNEMAMyPathFig{/net/home2/dhennen/t
estEIEIO/BigReport/FLW_SNEMA/figures} \def\FLWSNEMAfigFishCap{Total catch of So
uthern New England Mid-Atlantic Winter Flounder between 1981 and 2014 by fleet 
(commercial, recreational) and disposition (landings and discards).} \def\FLWSN
EMAfigSSBCap{Trends in spawning stock biomass of Southern New England Mid-Atlan
tic Winter Flounder between 1981 and 2014 from the current (solid line) and pre
vious (dashed line) assessment and the corresponding $SSB_{Threshold}${} ($\dfr
ac{1}{2}${} $SSB_{MSY}${} \textit{proxy}{}; horizontal dashed line) as well as 
$SSB_{Target}${} ($SSB_{MSY}${} \textit{proxy}{}; horizontal dotted line) based
 on the 2015 assessment. The approximate 90\% lognormal confidence intervals ar
e shown.} \def\FLWSNEMAfigFCap{Trends in the fully selected fishing mortality (
$F_{Full}${}) of Southern New England Mid-Atlantic Winter Flounder between 1981
 and 2014 from the current (solid line) and previous (dashed line) assessment a
nd the corresponding $F_{Threshold}${} ($F_{MSY}${}=0.325; horizontal dashed li
ne) based on the 2015 assessment. The approximate 90\% lognormal confidence int
ervals are shown.} \def\FLWSNEMAfigRecrCap{Trends in Recruits (age 1) (000s) of
 Southern New England Mid-Atlantic Winter Flounder between 1981 and 2014 from t
he current (solid line) and previous (dashed line) assessment. The approximate 
90\% lognormal confidence intervals are shown.} \def\FLWSNEMAfigSurvCap{Indices
 of biomass for the Southern New England Mid-Atlantic Winter Flounder between 1
963 and 2014 for the Northeast Fisheries Science Center (NEFSC) spring and fall
 bottom trawl surveys, the MADMF spring survey, and the CT LISTS survey The app
roximate 90\% lognormal confidence intervals are shown.} \def\FLWSNEMAPreAmb{Th
is assessment of the Southern New England Mid-Atlantic Winter Flounder (\textit
{Pseudopleuronectes americanus}) stock is an operational update of the existing
 2011 benchmark ASAP assessment (NEFSC 2011). Based on the previous assessment 
the stock was overfished, but overfishing was not ocurring. This assessment upd
ates commercial fishery catch data, recreational fishery catch data, and resear
ch survey indices of abundance, and the analytical ASAP assessment models and r
eference points through 2014. Additionally, stock projections have been updated
 through 2018} \def\FLWSNEMASoS{ \textbf{State of Stock: }{}Based on this updat
ed assessment, the Southern New England Mid-Atlantic Winter Flounder (\textit{P
seudopleuronectes americanus}) stock is overfished but overfishing is not occur
ring (Figures \ref{FLWSNEMASSB_plot1}-\ref{FLWSNEMAF_plot1}){}. Spawning stock 
biomass (SSB) in 2014 was estimated to be 6,151 (mt) which is 23\% of the bioma
ss target (26,928 mt), and 23\% of the biomass threshold for an overfished stoc
k ($SSB_{Threshold}${} = 13464 (mt); Figure \ref{FLWSNEMASSB_plot1}{}). The 201
4 fully selected fishing mortality was estimated to be 0.16 which is 49\% of th
e overfishing threshold ($F_{MSY}${} = 0.325; Figure \ref{FLWSNEMAF_plot1}{}). 
Retrospective adjustments were not made to the model results. } \def\FLWSNEMAPr
oj{ \textbf{Projections: }{}Short term projections of biomass were derived by s
ampling from a cumulative distribution function of recruitment estimates assumi
ng a Beverton-Holt stock recruitment relationship. The annual fishery selectivi
ty, maturity ogive, and mean weights at age used in projection are the most rec
ent 5 year averages; The model exhibited minor retrospective pattern in F and S
SB so no retrospective adjustments were applied in the projections.} \def\FLWSN
EMASpecCmt{ \textbf{Special Comments: } \begin{itemize}{} \item{}What are the m
ost important sources of uncertainty in this stock assessment? Explain, and des
cribe qualitatively how they affect the assessment results (such as estimates o
f biomass, F, recruitment, and population projections). \linebreak{} \hspace*{0
.5cm} \textit{A large source of uncertainty is the estimate of natural mortalit
y based on longevity, which is not well studied in Southern New England Mid-Atl
antic Winter Flounder, and assumed constant over time. Natural mortality affect
s the scale of the biomass and fishing mortality estimates. Natural mortality w
as adjusted upwards from 0.2 to 0.3 during the last benchmark assessment assumi
ng a max age of 16. However, there is still uncertainty in the true max age of 
the population and the resulting natural mortality estimate. Other sources of u
ncertainty include length distribution of the recreational discards. The recrea
tional discards, are a small component of the total catch, but the assessment s
uffers from very little length information used to characterize the recreationa
l discards (1 to 2 lengths in recent years).} \item{} Does this assessment mode
l have a retrospective pattern? If so, is the pattern minor, or major? (A major
 retrospective pattern occurs when the adjusted SSB or $F_{Full}${} lies outsid
e of the approximate joint confidence region for SSB and $F_{Full}${}; see Figu
re \ref{RhoDecision_tab}{}). \linebreak{} \hspace*{0.5cm} \textit{ No retrospec
tive adjustment of spawning stock biomass or fishing mortality in 2014 was requ
ired. } \item{}Based on this stock assessment, are population projections well 
determined or uncertain? \linebreak{} \hspace*{0.5cm} \textit{Population projec
tions for Southern New England Mid-Atlantic Winter Flounder are reasonably well
 determined. There is uncertainty in the estimates of M. In addition, while the
 retrospective pattern is considered minor (within the 90\% CI of both F and SS
B) the rho adjusted terminal value is very close to falling out of the bounds, 
becoming a major retrospective pattern. This would lead to retrospective adjust
ments being needed for the projections.} \item{}Describe any changes that were 
made to the current stock assessment, beyond incorporating additional years of 
data and the affect these changes had on the assessment and stock status. \line
break{} \hspace*{0.5cm} \textit{ No changes, other than the incorporation of ne
w data were made to the Southern New England Mid-Atlantic Winter Flounder asses
sment for this update.} \item{}If the stock status has changed a lot since the 
previous assessment, explain why this occurred. \linebreak{} \hspace*{0.5cm} \t
extit{The stock status of Southern New England Mid-Atlantic Winter Flounder has
 not changed since the previous benchmark in 2011.} \item{}Indicate what data o
r studies are currently lacking and which would be needed most to improve this 
stock assessment in the future. \linebreak{} \hspace*{0.5cm} \textit{The Southe
rn New England Mid-Atlantic Winter Flounder assessment could be improved with a
dditional studies on maximum age, as well additional information of recreationa
l discard lengths. In addition, further investigation into the localized strutu
re/genetics of the stock is warranted. Also, a future shift to ASAP version 4 w
ill provide the ability to model envirionmental factors that may influence both
 survey catchability and the modeled S-R relationship} \item{}Are there other i
mportant issues? \linebreak{} \hspace*{0.5cm} \textit{None. } \end{itemize}{}} 
\def\FLWSNEMARefr{ \textbf{References: }{} \linebreak{}Smith, A. and S. Jones. 
2008. In. Northeast Fisheries Science Center. 2008. Assessment of 19 Northeast 
Groundfish Stocks through 2007: Report of the 3$^{rd}$ Groundfish Assessment Re
view Meeting (GARM III), Northeast Fisheries Science Center, Woods Hole, Massac
husetts, August 4-8, 2008. US Dep Commer, NOAA Fisheries, Northeast Fish Sci Ce
nt Ref Doc. 08-15; 884 p + xvii. http://www.nefsc.noaa.gov/publications/crd/crd
0815/ \linebreak{} \linebreak{}Northeast Fisheries Science Center. 2011. 52$^{n
d}$ Northeast Regional Stock AssessmentWorkshop (52$^{nd}$ SAW) Assessment Repo
rt. US Dept Commer, Northeast Fish SciCent Ref Doc. 11-17; 962 p. Available fro
m: National Marine Fisheries Service, 166Water Street, Woods Hole, MA 02543-102
6, or online at http://www.nefsc.noaa.gov/nefsc/publications/ \linebreak{} \lin
ebreak{}} \def\FLWSNEMADraft{} \def\FLWSNEMASPPname{Southern New England Mid-At
lantic Winter Flounder} \def\FLWSNEMASPPnameT{Southern New England Mid-Atlantic
 Winter Flounder} \def\FLWSNEMARptYr{2015} \def\FLWSNEMAAuthor{Anthony Wood} \d
ef\FLWSNEMAReviewerComments{/net/home2/dhennen/testEIEIO/BigReport/FLW_SNEMA/la
tex} \def\FLWGBMyPathTab{/net/home2/dhennen/testEIEIO/BigReport/FLW_GB/tables} 
\def\FLWGBMyPathFig{/net/home2/dhennen/testEIEIO/BigReport/FLW_GB/figures} \def
\FLWGBfigFishCap{Total catches (mt) of Georges Bank Winter Flounder between 198
2 and 2015 by country and disposition (landings and discards).} \def\FLWGBfigSS
BCap{Trends in spawning stock biomass (mt) of Georges Bank Winter Flounder betw
een 1982 and 2014 from the current (solid line) and previous (dashed line) asse
ssments and the corresponding $SSB_{Threshold}${} ($\dfrac{1}{2}${} $SSB_{MSY}$
{}; horizontal dashed line) as well as $SSB_{Target}${} ($SSB_{MSY}${}; horizon
tal dotted line) based on the 2015 assessment. Biomass was adjusted for a retro
spective pattern and the adjustment is shown in red. The approximate 90\% norma
l confidence intervals are shown.} \def\FLWGBfigFCap{Trends in fully selected f
ishing mortality ($F_{Full}${}) of Georges Bank Winter Flounder between 1982 an
d 2014 from the current (solid line) and previous (dashed line) assessments and
 the corresponding $F_{Threshold}${} ($F_{MSY}${}=0.536; horizontal dashed line
) as well as ($F_{Target}${}= 75\% of FMSY; horizontal dotted line). $F_{Full}$
{} was adjusted for a retrospective pattern and the adjustment is shown in red.
 The approximate 90\% normal confidence intervals are also shown.} \def\FLWGBfi
gRecrCap{Trends in Recruits (age 1) (000s) of Georges Bank Winter Flounder betw
een 1982 and 2014 from the current (solid line) and previous (dashed line) asse
ssments. The approximate 90\% normal confidence intervals are shown.} \def\FLWG
BfigSurvCap{Indices of biomass for the Georges Bank Winter Flounder for the Nor
theast Fisheries Science Center (NEFSC) spring (1968-2015) and fall (1963-2014)
 bottom trawl surveys and the Canadian DFO spring survey (1987-2015). The appro
ximate 90\% normal confidence intervals are shown.} \def\FLWGBPreAmb{This asses
sment of the Georges Bank Winter Flounder (\textit{Pseudopleuronectes americanu
s}) stock is an operational update of the existing 2014 operational VPA assessm
ent which included data for 1982-2013 (Hendrickson et al. 2015). Based on the p
revious assessment the stock was not overfished and overfishing was not ocurrin
g. This assessment updates commercial fishery catch data, research survey bioma
ss indices, and the analytical VPA assessment model and reference points throug
h 2014. Additionally, stock projections have been updated through 2018.} \def\F
LWGBSoS{ \textbf{State of Stock: }{}Based on this updated assessment, the Georg
es Bank Winter Flounder (\textit{Pseudopleuronectes americanus}) stock is overf
ished and overfishing is occurring (Figures \ref{FLWGBSSB_plot1}-\ref{FLWGBF_pl
ot1}){}. Retrospective adjustments were made to the model results. Spawning sto
ck biomass (SSB) in 2014 was estimated to be 2,883 (mt) which is 43\% of the bi
omass target for an overfished stock ($SSB_{MSY}${} = 6,700 with a threshold of
 50\% of SSBMSY; Figure \ref{FLWGBSSB_plot1}{}). The 2014 fully selected fishin
g mortality (F) was estimated to be 0.778 which is 145\% of the overfishing thr
eshold ($F_{MSY}${} = 0.536; Figure \ref{FLWGBF_plot1}{}). However, the 2014 po
int estimate of SSB and F, when adjusted for retrospective error (83\% for SSB 
and -51\% for F), is outside the 90\% confidence interval of the unadjusted 201
4 point estimate. Therefore, the 2014 F and SSB values used in the stock status
 determination were the retrospective-adjusted values of 0.778 and 2,883 mt, re
spectively.} \def\FLWGBProj{ \textbf{Projections: }{}Short-term projections of 
biomass were derived by sampling from a cumulative distribution function of rec
ruitment estimates (1982-2013 YC) from the final run of the ADAPT VPA model. Th
e annual fishery selectivity, maturity ogive, and mean weights-at-age used in t
he projection are the most recent 5 year averages (2010-2014). An SSB retrospec
tive adjustment factor of 0.546 was applied in the projections.} \def\FLWGBSpec
Cmt{ \textbf{Special Comments: } \begin{itemize}{} \item{}What are the most imp
ortant sources of uncertainty in this stock assessment? Explain, and describe q
ualitatively how they affect the assessment results (such as estimates of bioma
ss, F, recruitment, and population projections). \linebreak{} \hspace*{0.5cm} \
textit{The largest source of uncertainty is the estimate of natural mortality b
ased on longevity (max. age = 20 for this stock), which is not well studied in 
Georges Bank Winter Flounder, and assumed constant over time. Natural mortality
 affects the scale of the biomass and fishing mortality estimates. Other source
s of uncertainty include the underestimation of catches. Discards from the Cana
dian bottom trawl fleet were not provided by the CA DFO and the precision of th
e Canadian scallop dredge discard estimates, with only 1-2 trips per month, are
 uncertain.The lack of age data for the Canadian spring survey catches requires
 the use of the US spring survey A/L keys despite selectivity differences. In a
ddition, there are no length or age composition data from the Canadian landings
 or discards GB winter flounder.} \item{} Does this assessment model have a ret
rospective pattern? If so, is the pattern minor, or major? (A major retrospecti
ve pattern occurs when the adjusted SSB or $F_{Full}${} lies outside of the app
roximate joint confidence region for SSB and $F_{Full}${}; see Figure \ref{RhoD
ecision_tab}{}). \linebreak{} \hspace*{0.5cm} \textit{ The 7-year Mohn's \textr
ho{}, relative to SSB, was 0.26 in the 2014 assessment and was 0.83 in 2014. Th
e 7-year Mohn's \textrho{}, relative to F, was -0.16 in the 2014 assessment and
 was -0.51 in 2014. There was a major retrospective pattern for this assessment
 because the \textrho{} adjusted estimates of 2014 SSB ($SSB_{\rho}${}=2,883) a
nd 2014 F ($F_{\rho}${}=0.778) were outside the approximate 90\% confidence reg
ion around SSB (3,783 - 6,767) and F (0.254 - 0.504). A retrospective adjustmen
t was made for both the determination of stock status and for projections of ca
tch in 2016. The retrospective adjustment changed the 2014 SSB from 5,275 to 2,
883 and the 2014 $F_{Full}${} from 0.379 to 0.778.} \item{}Based on this stock 
assessment, are population projections well determined or uncertain? \linebreak
{} \hspace*{0.5cm} \textit{Population projections for Georges Bank Winter Floun
der are reasonably well determined.} \item{}Describe any changes that were made
 to the current stock assessment, beyond incorporating additional years of data
 and the affect these changes had on the assessment and stock status. \linebrea
k{} \hspace*{0.5cm} \textit{ The only change made to the Georges Bank Winter Fl
ounder assessment, other than the incorporation of an additional year of data, 
involved fishery selectivity. During the 2014 assessment update, stock size est
imates of age 1 and age 2 fish were not estimable in the VPA during year t + 1 
(CVs near 1.0). When age 2 stock size is not estimated in year t + 1, the VPA m
odel calculates the stock size of age 1 fish (i.e., recruitment) in the termina
l year by using the age 1 partial recruitment (PR) value to derive the F at age
 1 in the terminal year. The age 1 PR value used in the 2014 assessment update 
was 0.001. However, when this same age 1 PR value was used in a VPA run for the
 current assessment update, the low PR value combined with the low age 1 catch 
in 2014 resulted in an unlikely high stock size estimate for age 1 recruitment 
in 2014 (i.e., 41,587,000 fish) when compared to survey observations of the sam
e cohort (i.e., age 1 in 2014 and age 2 in 2015). In order to obtain a more rea
listic estimate of age 1 recruitment in 2014, I allowed the VPA model to estima
te age 2 stock size in 2015 (i.e., and thereby avoided the use of an age 1 PR v
alue in the age 1 stock size calculation for 2014) and used the back-calculated
 PR values from this VPA run to derive a new PR-at-age vector which was used in
 the final 2015 VPA run. Similar to the 2014 assessment update, the final 2015 
VPA run did not include the estimation of age 2 stock size and the new PR-at-ag
e vector was computed using the same methods as in the 2014 assessment. Full se
lectivity occurs at age 4. For the 2015 assessment update, fishery selectivity 
for ages 1-3 was changed from the 2014 assessment values of 0.001, 0.10 and 0.4
3, respectively, to 0.01, 0.08 and 0.55, respectively. Differences between esti
mates of F, SSB and R values from the final 2015 VPA run, with the new PR vecto
r, and a 2015 VPA run that utilized the PR vector from the 2014 assessment are 
shown in Table G30.} \item{}If the stock status has changed a lot since the pre
vious assessment, explain why this occurred. \linebreak{} \hspace*{0.5cm} \text
it{The overfished and overfishing status of Georges Bank Winter Flounder has ch
anged in the current assessment update due to a worsening of the retrospective 
error associated with fishing mortality and SSB.} \item{}Indicate what data or 
studies are currently lacking and which would be needed most to improve this st
ock assessment in the future. \linebreak{} \hspace*{0.5cm} \textit{The Georges 
Bank Winter Flounder assessment could be improved with discard estimates from t
he Canadian bottom trawl fleet and age data from the Canadian spring bottom tra
wl surveys.} \item{}Are there other important issues? \linebreak{} \hspace*{0.5
cm} \textit{None. } \end{itemize}{}} \def\FLWGBRefr{ \textbf{References: }{} \l
inebreak{} Hendrickson L, Nitschke P, Linton B. 2015. 2014 Operational Stock As
sessments for Georges Bank winter flounder, Gulf of Maine winter flounder, and 
pollock. US Dept Commer, Northeast Fish Sci Cent Ref Doc. 15-01; 228 p. \linebr
eak{} \linebreak{}} \def\FLWGBDraft{} \def\FLWGBSPPname{Georges Bank Winter Flo
under} \def\FLWGBSPPnameT{Georges Bank Winter Flounder} \def\FLWGBRptYr{2015} \
def\FLWGBAuthor{Lisa Hendrickson} \def\FLWGBReviewerComments{/net/home2/dhennen
/testEIEIO/BigReport/FLW_GB/latex} \def\FLDGMGBMyPathTab{/net/home2/dhennen/tes
tEIEIO/BigReport/FLD_GMGB/tables} \def\FLDGMGBMyPathFig{/net/home2/dhennen/test
EIEIO/BigReport/FLD_GMGB/figures} \def\FLDGMGBfigFishCap{Total catch of norther
n windowpane flounder between 1975 and 2014 by disposition (landings and discar
ds).} \def\FLDGMGBfigSSBCap{Trends in the biomass index (a 3-year moving averag
e of the NEFSC fall bottom trawl survey index) of northern windowpane flounder 
between 1975 and 2014 from the current assessment, and the corresponding $B_{Th
reshold}${} = $\dfrac{1}{2}${} $B_{MSY}${} \textit{proxy}{} = 0.777 kg/tow (hor
izontal dashed line). } \def\FLDGMGBfigFCap{Trends in relative fishing mortalit
y of northern windowpane flounder between 1975 and 2014 from the current assess
ment, and the corresponding $F_{MSY}${} \textit{proxy}{}=0.45 (horizontal dashe
d line). } \def\FLDGMGBfigRecrCap{} \def\FLDGMGBfigSurvCap{NEFSC fall bottom tr
awl survey indices in kg/tow for northern windowpane flounder between 1975 and 
2014 The approximate 90\% lognormal confidence intervals are shown.} \def\FLDGM
GBPreAmb{This assessment of the northern windowpane flounder (\textit{Scophthal
mus aquosus}) stock is an operational update of the 2012 assessment which inclu
ded updates through 2010 (NEFSC 2012). Based on the 2012 assessment the stock w
as overfished, and overfishing was ocurring. This assessment updates commercial
 fishery catch data, survey indices of abundance, AIM model results, and refere
nce points through 2014.} \def\FLDGMGBSoS{ \textbf{State of Stock: }{}Based on 
this updated assessment, the northern windowpane flounder (\textit{Scophthalmus
 aquosus}) stock is overfished but overfishing is not occurring (Figures \ref{F
LDGMGBSSB_plot1}-\ref{FLDGMGBF_plot1}){}. Retrospective adjustments were not ma
de to the model results. The mean NEFSC fall bottom trawl survey index from yea
rs 2012, 2013 and 2014 (a 3-year moving average is used as a biomass index) was
 0.535 kg/tow which is lower than the $B_{Threshold}${} of 0.777 kg/tow. The 20
14 relative fishing mortality was estimated to be 0.393 kt per kg/tow which is 
lower than the $F_{MSY}${} \textit{proxy}{} of 0.450 kt per kg/tow.} \def\FLDGM
GBProj{} \def\FLDGMGBSpecCmt{ \textbf{Special Comments: } \begin{itemize}{} \it
em{}What are the most important sources of uncertainty in this stock assessment
? Explain, and describe qualitatively how they affect the assessment results (s
uch as estimates of biomass, F, recruitment, and population projections). \line
break{} \hspace*{0.5cm} \textit{The main source of uncertainty in this assessme
nt is the lack of windowpane discard estimates from Canadian fisheries to add t
o the catch component of model input. Discard estimates were from the U.S. only
. There is overlap between the survey area and Canadian fishing grounds (Van Ee
ckhaute et al. 2010), which means catch from within the stock area was likely u
nderestimated. } \item{} Does this assessment model have a retrospective patter
n? If so, is the pattern minor, or major? (A major retrospective pattern occurs
 when the adjusted SSB or $F_{Full}${} lies outside of the approximate joint co
nfidence region for SSB and $F_{Full}${}; see Figure \ref{RhoDecision_tab}{}). 
\linebreak{} \hspace*{0.5cm} \textit{ The model used to estimate status of this
 stock does not allow estimation of a retrospective pattern. } \item{}Based on 
this stock assessment, are population projections well determined or uncertain?
 \linebreak{} \hspace*{0.5cm} \textit{N/A } \item{}Describe any changes that we
re made to the current stock assessment, beyond incorporating additional years 
of data and the affect these changes had on the assessment and stock status. \l
inebreak{} \hspace*{0.5cm} \textit{No changes were made to the northern windowp
ane flounder assessment for this update other than the incorporation of four ye
ars of new NEFSC fall bottom trawl survey data and four years of new U.S. comme
rcial landings and discard data (2011 - 2014). } \item{}If the stock status has
 changed a lot since the previous assessment, explain why this occurred. \lineb
reak{} \hspace*{0.5cm} \textit{The stock status of northern windowpane flounder
 changed from 'overfished and overfishing is occurring' to 'overfished and over
fishing is not occurring' due to stable-to-decreasing catch since 2008, and an 
increasing trend in the survey index since 2010. } \item{}Indicate what data or
 studies are currently lacking and which would be needed most to improve this s
tock assessment in the future. \linebreak{} \hspace*{0.5cm} \textit{The norther
n windowpane flounder assessment could be improved by estimating the Canadian w
indowpane removals and, although to a lesser degree, the 'general category' sca
llop dredge fleet discards from within the stock area and using them as additio
nal catch input to the AIM model. While the model fit now is reasonable (the re
lationship between ln(relative F) and ln(replacement ratio), a measure of the r
elationship between catch and survey index values, has a p-value of 0.079) ther
e are probably removals unaccounted for in the model and the fit can likely be 
improved. } \item{}Are there other important issues? \linebreak{} \hspace*{0.5c
m} \textit{None. } \end{itemize}{}} \def\FLDGMGBRefr{ \textbf{References: }{} \
linebreak{} Most recent assessment update: \linebreak{} Northeast Fisheries Sci
ence Center. 2012. Assessment or Data Updates of 13 Northeast Groundfish Stocks
 through 2010. US Dept Commer, Northeast Fish Sci Cent Ref Doc. 12-06; 789 p. A
vailable online at http://nefsc.noaa.gov/publications/ \linebreak{} \linebreak{
} Most recent benchmark assessment: \linebreak{} Northeast Fisheries Science Ce
nter. 2008. Assessment of 19 Northeast Groundfish Stocks through 2007: Report o
f the 3$^{rd}$ Groundfish Assessment Review Meeting (GARM III), Northeast Fishe
ries Science Center, Woods Hole, Massachusetts, August 4-8, 2008. US Dep Commer
, NOAA FIsheries, Northeast Fish Sci Cent Ref Doc. 08-15; 884 p + xvii. \linebr
eak{} \linebreak{} Van Eeckhaute, L., Sameoto, J., and A. Glass. 2010. Discards
 of Atlantic cod, haddock and yellowtail flounder from the 2009 Canadian scallo
p fishery on Georges Bank. TRAC Ref. Doc. 2010/10. 7p. \linebreak{} \linebreak{
}} \def\FLDGMGBDraft{} \def\FLDGMGBSPPname{northern windowpane flounder} \def\F
LDGMGBSPPnameT{Northern windowpane flounder} \def\FLDGMGBRptYr{2015} \def\FLDGM
GBAuthor{Toni Chute} \def\FLDGMGBReviewerComments{/net/home2/dhennen/testEIEIO/
BigReport/FLD_GMGB/latex} \def\FLDSNEMAMyPathTab{/net/home2/dhennen/testEIEIO/B
igReport/FLD_SNEMA/tables} \def\FLDSNEMAMyPathFig{/net/home2/dhennen/testEIEIO/
BigReport/FLD_SNEMA/figures} \def\FLDSNEMAfigFishCap{Total catch of southern wi
ndowpane flounder between 1975 and 2014 by disposition (landings and discards).
} \def\FLDSNEMAfigSSBCap{Trends in the biomass index (a 3-year moving average o
f the NEFSC fall bottom trawl survey index) of southern windowpane flounder bet
ween 1975 and 2014 from the current assessment, and the corresponding $B_{Thres
hold}${} = $\dfrac{1}{2}${} $B_{MSY}${} \textit{proxy}{} = 0.123 kg/tow(horizon
tal dashed line). } \def\FLDSNEMAfigFCap{Trends in relative fishing mortality o
f southern windowpane flounder between 1975 and 2014 from the current assessmen
t, and the corresponding $F_{MSY}${} \textit{proxy}{}=2.027 (horizontal dashed 
line). } \def\FLDSNEMAfigRecrCap{} \def\FLDSNEMAfigSurvCap{NEFSC fall bottom tr
awl survey indices in kg/tow for southern windowpane flounder between 1975 and 
2014. The approximate 90\% lognormal confidence intervals are shown.} \def\FLDS
NEMAPreAmb{This assessment of the southern windowpane flounder (\textit{Scophth
almus aquosus}) stock is an operational update of the 2012 assessment which inc
luded updates through 2010 (NEFSC 2012). Based on the 2012 assessment the stock
 was not overfished, and overfishing was not ocurring. This assessment updates 
commercial fishery catch data, survey indices of abundance, AIM model results, 
and reference points through 2014. } \def\FLDSNEMASoS{ \textbf{State of Stock: 
}{}Based on this updated assessment, the southern windowpane flounder (\textit{
Scophthalmus aquosus}) stock is not overfished and overfishing is not occurring
 (Figures \ref{FLDSNEMASSB_plot1}-\ref{FLDSNEMAF_plot1}){}. Retrospective adjus
tments were not made to the model results. The mean NEFSC fall bottom trawl sur
vey index from years 2012, 2013, and 2014 (a 3-year moving average is used as a
 biomass index) was 0.413 (kg/tow) which is higher than the $B_{Threshold}${}of
 0.123 (kg/tow). The 2014 relative fishing mortality was estimated to be 1.308 
(kt per kg/tow) which is lower than the $F_{MSY}${} \textit{proxy}{} of 2.027 (
kt per kg/tow). } \def\FLDSNEMAProj{} \def\FLDSNEMASpecCmt{ \textbf{Special Com
ments: } \begin{itemize}{} \item{}What are the most important sources of uncert
ainty in this stock assessment? Explain, and describe qualitatively how they af
fect the assessment results (such as estimates of biomass, F, recruitment, and 
population projections). \linebreak{} \hspace*{0.5cm} \textit{A source of uncer
tainty for this assessment is missing commercial discard estimates from the gen
eral category scallop dredge fleet that should be added to the catch time serie
s for model input. } \item{} Does this assessment model have a retrospective pa
ttern? If so, is the pattern minor, or major? (A major retrospective pattern oc
curs when the adjusted SSB or $F_{Full}${} lies outside of the approximate join
t confidence region for SSB and $F_{Full}${}; see Figure \ref{RhoDecision_tab}{
}). \linebreak{} \hspace*{0.5cm} \textit{ The model used to estimate status of 
this stock does not allow estimation of a retrospective pattern. } \item{}Based
 on this stock assessment, are population projections well determined or uncert
ain? \linebreak{} \hspace*{0.5cm} \textit{N/A} \item{}Describe any changes that
 were made to the current stock assessment, beyond incorporating additional yea
rs of data and the affect these changes had on the assessment and stock status.
 \linebreak{} \hspace*{0.5cm} \textit{ No changes were made to the southern win
dowpane flounder assessment for this update other than the incorporation of fou
r years of new NEFSC fall bottom trawl survey data and four years of new U.S. c
ommercial landings and discard data (2011 - 2014). } \item{}If the stock status
 has changed a lot since the previous assessment, explain why this occurred. \l
inebreak{} \hspace*{0.5cm} \textit{The stock status of southern windowpane flou
nder has not changed since the previous assessment. } \item{}Indicate what data
 or studies are currently lacking and which would be needed most to improve thi
s stock assessment in the future. \linebreak{} \hspace*{0.5cm} \textit{Estimate
s of discards from the general category scallop dredge fleet should be added to
 the catch time series for model input. However, the model fit is presently goo
d with a randomization test indicating the correlation between ln(relative F) a
nd ln(replacement ratio), a measure of the relationship between catch and surve
y index values, is significant (p = 0.002.) } \item{}Are there other important 
issues? \linebreak{} \hspace*{0.5cm} \textit{None. } \end{itemize}{}} \def\FLDS
NEMARefr{ \textbf{References: }{} \linebreak{} Most recent assessment update: \
linebreak{} Northeast Fisheries Science Center. 2012. Assessment or Data Update
s of 13 Northeast Groundfish Stocks through 2010. US Dept Commer, Northeast Fis
h Sci Cent Ref Doc. 12-06; 789 p. Available online at http://nefsc.noaa.gov/pub
lications/ \linebreak{} \linebreak{} Most recent benchmark assessment: \linebre
ak{} Northeast Fisheries Science Center. 2008. Assessment of 19 Northeast Groun
dfish Stocks through 2007: Report of the 3$^{rd}$ Groundfish Assessment Review 
Meeting (GARM III), Northeast Fisheries Science Center, Woods Hole, MA, August 
4-8, 2008. US Dep Commer, NOAA Fisheries, Northeast Fish Sci Cent Ref Doc. 08-1
5; 884 p + xvii. \linebreak{} \linebreak{}} \def\FLDSNEMADraft{} \def\FLDSNEMAS
PPname{southern windowpane flounder} \def\FLDSNEMASPPnameT{Southern windowpane 
flounder} \def\FLDSNEMARptYr{2015} \def\FLDSNEMAAuthor{Toni Chute} \def\FLDSNEM
AReviewerComments{/net/home2/dhennen/testEIEIO/BigReport/FLD_SNEMA/latex} \def\
PLAUNITMyPathTab{/net/home2/dhennen/testEIEIO/BigReport/PLA_UNIT/tables} \def\P
LAUNITMyPathFig{/net/home2/dhennen/testEIEIO/BigReport/PLA_UNIT/figures} \def\P
LAUNITfigFishCap{Total catch of Gulf of Maine-Georges Bank American Plaice betw
een 1980 and 2015 by fleet (Gulf of Maine, Georges Bank, Southern New England, 
and Canadian) and disposition (landings and discards).} \def\PLAUNITfigSSBCap{T
rends in spawning stock biomass of Gulf of Maine-Georges Bank American Plaice b
etween 1980 and 2015 from the current (solid line) and previous (dashed line) a
ssessment and the corresponding $SSB_{Threshold}${} ($\dfrac{1}{2}${} $SSB_{MSY
}${} \textit{proxy}{}; horizontal dashed line) as well as $SSB_{Target}${} ($SS
B_{MSY}${} \textit{proxy}{}; horizontal dotted line) based on the 2015 assessme
nt. Biomass was adjusted for a retrospective pattern and the adjustment is show
n in red. The approximate 90\% normal confidence intervals are shown.} \def\PLA
UNITfigFCap{Trends in the fully selected fishing mortality ($F_{Full}${}) of Gu
lf of Maine-Georges Bank American Plaice between 1980 and 2015 from the current
 (solid line) and previous (dashed line) assessment and the corresponding $F_{T
hreshold}${} ($F_{MSY}${} \textit{proxy}{}=0.196; horizontal dashed line). $F_{
Full}${} was adjusted for a retrospective pattern and the adjustment is shown i
n red, based on the 2015 assessment. The approximate 90\% normal confidence int
ervals are shown.} \def\PLAUNITfigRecrCap{Trends in Recruits (age 1) (000s) of 
Gulf of Maine-Georges Bank American Plaice between 1980 and 2015 from the curre
nt (solid line) and previous (dashed line) assessment.} \def\PLAUNITfigSurvCap{
Indices of biomass for the Gulf of Maine-Georges Bank American Plaice between 1
963 and 2015 for the Northeast Fisheries Science Center (NEFSC) and Massachuset
ts Division of Marine Fisheries (MADMF) spring and autumn research bottom trawl
 surveys. The approximate 90\% normal confidence intervals are shown.} \def\PLA
UNITPreAmb{This assessment of the Gulf of Maine-Georges Bank American Plaice (\
textit{Hippoglossoides platessoides}) stock is an operational update of the exi
sting 2012 benchmark assessment (O'Brien et al. 2012). Based on the previous as
sessment the stock was not overfished, and overfishing was not ocurring. This 2
015 assessment updates commercial fishery catch data, research survey indices o
f abundance, the analytical VPA assessment model, and reference points through 
2014. Additionally, stock projections have been updated through 2018.} \def\PLA
UNITSoS{ \textbf{State of Stock: }{}Based on this updated assessment, the Gulf 
of Maine-Georges Bank American Plaice (\textit{Hippoglossoides platessoides}) s
tock is not overfished and overfishing is not occurring (Figures \ref{PLAUNITSS
B_plot1}-\ref{PLAUNITF_plot1}){}. Retrospective adjustments were made to the mo
del results. Spawning stock biomass (SSB) in 2014 was estimated to be 10,915 mt
 which is 83\% of the biomass target for this stock ($SSB_{MSY}${} \textit{prox
y}{} = 13,107; Figure \ref{PLAUNITSSB_plot1}{}). The 2014 fully selected fishin
g mortality was estimated to be 0.118 which is 60\% of the overfishing threshol
d proxy ($F_{MSY}${} \textit{proxy}{} = 0.196; Figure \ref{PLAUNITF_plot1}{}).}
 \def\PLAUNITProj{ \textbf{Projections: }{}Short term projections of biomass we
re derived by sampling from an empirical cumulative distribution function of 34
 recruitment estimates from VPA model results. The annual fishery selectivity, 
maturity ogive, and mean weights at age used in projections are the most recent
 5 year averages; retrospective adjustments were applied in the projections.} \
def\PLAUNITSpecCmt{ \textbf{Special Comments: } \begin{itemize}{} \item{}What a
re the most important sources of uncertainty in this stock assessment? Explain,
 and describe qualitatively how they affect the assessment results (such as est
imates of biomass, F, recruitment, and population projections). \linebreak{} \h
space*{0.5cm} \textit{A source of uncertainty in this assessment are the estima
tes of historical landings at age, prior to 1984, and the magnitude of historic
al discards, prior to 1989. Both of these affect the scale of the biomass and f
ishing mortality estimates, and influence reference point estimations.} \item{}
 Does this assessment model have a retrospective pattern? If so, is the pattern
 minor, or major? (A major retrospective pattern occurs when the adjusted SSB o
r $F_{Full}${} lies outside of the approximate joint confidence region for SSB 
and $F_{Full}${}; see Figure \ref{RhoDecision_tab}{}). \linebreak{} \hspace*{0.
5cm} \textit{ The 7-year Mohn's \textrho{}, relative to SSB, was 0.63 in the 20
12 assessment and was 0.32 in 2014. The 7-year Mohn's \textrho{}, relative to F
, was -0.35 in the 2012 assessment and was 0.32 in 2014. There was a major retr
ospective pattern for this assessment because the \textrho{} adjusted estimates
 of 2014 SSB ($SSB_{\rho}${}=10,915) and 2014 F ($F_{\rho}${}=0.118) were outsi
de the approximate 90\% confidence regions around SSB (12,742 - 16,439) and F (
0.069 - 0.093). A retrospective adjustment was made for both the determination 
of stock status and for projections of catch in 2016. The retrospective adjustm
ent changed the 2014 SSB from 14,543 to 10,915 and the 2014 $F_{Full}${} from 0
.08 to 0.118.} \item{}Based on this stock assessment, are population projection
s well determined or uncertain? \linebreak{} \hspace*{0.5cm} \textit{Population
 projections for Gulf of Maine-Georges Bank American Plaice are reasonably well
 determined.} \item{}Describe any changes that were made to the current stock a
ssessment, beyond incorporating additional years of data and the effect these c
hanges had on the assessment and stock status. \linebreak{} \hspace*{0.5cm} \te
xtit{ No major changes, other than the addition of recent years of data, were m
ade to the Gulf of Maine-Georges Bank American Plaice assessment for this updat
e. A new version of VPA was used (V3.3.0) which gave very similar results to th
e 2012 VPA 3.1.0 run, with the same F and slightly lower SSB. The MADMF spring 
and autumn survey indices were re-estimated for the time series, accounting for
 revised stratum areas. The revision occurred in 2007, but was overlooked in th
e 2012 assessment. A comparison of 2010 terminal year VPAs indicated minimal di
fferences in 2010 SSB (now slightly lower) and no change in F.} \item{}If the s
tock status has changed a lot since the previous assessment, explain why this o
ccurred. \linebreak{} \hspace*{0.5cm} \textit{As in recent assessments for Gulf
 of Maine-Georges Bank American Plaice the stock status remains as not overfish
ed and overfishing not occurring.} \item{}Indicate what data or studies are cur
rently lacking and which would be needed most to improve this stock assessment 
in the future. \linebreak{} \hspace*{0.5cm} \textit{The Gulf of Maine-Georges B
ank American Plaice assessment could be improved with updated studies on growth
 of Georges Bank and Gulf of Maine fish.} \item{}Are there other important issu
es? \linebreak{} \hspace*{0.5cm} \textit{A difference in growth between GM and 
GB fish has been documented, however, historical catch data information for GB 
may not be sufficient to conduct a separate assessment. Also, the growth differ
ence may not persist in the most recent years. This could all be explored furth
er in an benchmark review.} \end{itemize}{}} \def\PLAUNITRefr{ \textbf{Referenc
es: }{} \linebreak{}O'Brien, L. and J. Dayton (2012). E. Gulf of Maine - George
s Bank American plaice Assessment for 2012 in Northeast Fisheries Science Cente
r, 2012, Assessment or Data Updates of 13 Northeast Groundfish Stocks through 2
010. US Dept Commer, Northeast Fish Sci Cent Ref Doc. 12-06; 789 p. http://www.
nefsc.noaa.gov/publications/crd/crd1206/. \linebreak{} \linebreak{}} \def\PLAUN
ITDraft{} \def\PLAUNITSPPname{Gulf of Maine-Georges Bank American Plaice} \def\
PLAUNITSPPnameT{Gulf of Maine-Georges Bank American Plaice} \def\PLAUNITRptYr{2
015} \def\PLAUNITAuthor{Loretta O'Brien} \def\PLAUNITReviewerComments{/net/home
2/dhennen/testEIEIO/BigReport/PLA_UNIT/latex} \def\WITUNITMyPathTab{/net/home2/
dhennen/testEIEIO/BigReport/WIT_UNIT/tables} \def\WITUNITMyPathFig{/net/home2/d
hennen/testEIEIO/BigReport/WIT_UNIT/figures} \def\WITUNITfigFishCap{Total catch
 of witch flounder between 1982 and 2014 by fleet (commercial) and disposition 
(landings and discards).} \def\WITUNITfigSSBCap{Trends in spawning stock biomas
s (mt) of witch flounder between 1982 and 2014 from the current (solid line) an
d previous (dashed line) assessment and the corresponding $SSB_{Threshold}${} (
$\dfrac{1}{2}${} $SSB_{MSY}${}; horizontal dashed line) as well as $SSB_{Target
}${} $SSB_{MSY}${}; horizontal dotted line) based on the current assessment. Re
d solid vertical line indicates rho adjusted SSB. Black solid vertical line ind
icates 90\% confidence interval for 2014.} \def\WITUNITfigFCap{Trends in the fu
lly selected fishing mortality ($F_{Full}${}) of witch flounder between 1982 an
d 2014 from the current (solid line) and previous (dashed line) assessment and 
the corresponding $F_{Threshold}${} ($F_{MSY}${}=0.279; horizontal dashed line)
 based on the current assessment. Red solid vertical line indicates rho adjuste
d $F_{Full}${}. Black solid vertical line indicates 90\% confidence interval fo
r 2014.} \def\WITUNITfigRecrCap{Trends in Age 3 (000s) of witch flounder betwee
n 1982 and 2014 from the current (solid line) and previous (dashed line) assess
ment.} \def\WITUNITfigSurvCap{Indices of biomass (kg/tow) for the witch flounde
r between 1963 and 2015 for the Northeast Fisheries Science Center (NEFSC) spri
ng and fall bottom trawl surveys. The 90\% lognormal confidence intervals are s
hown.} \def\WITUNITPreAmb{This assessment of the witch flounder (\textit{Glypto
cephalus cynoglossus}) stock is an operational update of the 2012 assessment (N
EFSC 2012) and the 2008 benchmark assessment (NEFSC 2008). This assessment upda
tes commercial fishery catch data, research survey indices, and the analytical 
assessment model through 2014. Additionally, stock projections have been update
d through 2018. Reference points have been updated. } \def\WITUNITSoS{ \textbf{
State of Stock: }{}witch flounder (\textit{Glyptocephalus cynoglossus}) stock i
s overfished and overfishing is occurring (Figures \ref{WITUNITSSB_plot1}-\ref{
WITUNITF_plot1}){}. Retrospective adjustments were made to the model results. S
pawning stock biomass (SSB) in 2014 was estimated to be 2,077 (mt) which is 22\
% of the $SSB_{MSY}${} proxy (9,473; Figure \ref{WITUNITSSB_plot1}{}). The 2014
 fully selected fishing mortality was estimated to be 0.687 which is 246\% of t
he $F_{MSY}${} proxy (0.279; Figure \ref{WITUNITF_plot1}{}). A retrospective ad
justment to $F_{Full}${} and SSB in 2014 was required but did not lead to a cha
nge in status. } \def\WITUNITProj{ \textbf{Projections: }{}Short term projectio
n recruitment was sampled from a cumulative distribution function derived from 
ADAPT VPA (with split time series between 1994 and 1995) estimated age 3 recrui
tment between 1982 and 2013. Average 2010-2014 partial recruitment, average 201
0-2014 mean weights, and maturation ogive representing 2011-2015 maturity data 
were used.} \def\WITUNITSpecCmt{ \textbf{Special Comments: } \begin{itemize}{} 
\item{}What are the most important sources of uncertainty in this stock assessm
ent? Explain, and describe qualitatively how they affect the assessment results
 (such as estimates of biomass, F, recruitment, and population projections). \l
inebreak{} \hspace*{0.5cm} \textit{An important source of uncertainty is the re
trospective pattern where fishing mortality is underestimated and spawning stoc
k biomass and recruitment are overestimated. } \item{} Does this assessment mod
el have a retrospective pattern? If so, is the pattern minor, or major? (A majo
r retrospective pattern occurs when the adjusted SSB or $F_{Full}${} lies outsi
de of the approximate joint confidence region for SSB and $F_{Full}${}). \lineb
reak{} \hspace*{0.5cm} \textit{ The 7-year Mohn's \textrho{}, relative to SSB, 
was 0.61 in the 2012 assessment and was 0.51 in 2014. The 7-year Mohn's \textrh
o{}, relative to F, was -0.33 in the 2012 assessment and was -0.38 in 2014. The
re was a major retrospective pattern for this assessment because the \textrho{}
 adjusted estimates of 2014 SSB ($SSB_{\rho}${}=2,077) and 2014 F ($F_{\rho}${}
=0.687) were outside the approximate 90\% confidence regions around SSB (2,643 
- 3,864) and F (0.321 - 0.603). A retrospective adjustment was made for both th
e determination of stock status and for projections of catch in 2016. The retro
spective adjustment changed the 2014 SSB from 3,129 to 2,077 and the 2014 $F_{F
ull}${} from 0.428 to 0.687.} \item{}Based on this stock assessment, are popula
tion projections well determined or uncertain? \linebreak{} \hspace*{0.5cm} \te
xtit{Population projections for witch flounder appear to be optimistic; the pro
jected rho adjusted biomass from the last assessment was above the upper confid
ence bounds of the projected rho adjusted biomass estimated in the current asse
ssment. } \item{}Describe any changes that were made to the current stock asses
sment, beyond incorporating additional years of data and the effect these chang
es had on the assessment and stock status. \linebreak{} \hspace*{0.5cm} \textit
{TOGA (Type, Operation, Gear, Acquisition) values were used for haul criteria f
or NEFSC surveys for 2009 onward and minor changes in the use of observer data 
for discard estimates were made to the current witch flounder assessment. These
 changes had negligible effect on the assessment and stock status. } \item{}If 
the stock status has changed a lot since the previous assessment, explain why t
his occurred. \linebreak{} \hspace*{0.5cm} \textit{No change in stock status ha
s occurred for witch flounder since the previous assessment. } \item{}Indicate 
what data or studies are currently lacking and which would be needed most to im
prove this stock assessment in the future. \linebreak{} \hspace*{0.5cm} \textit
{Extensive studies have examined the causes of retrospective patterns with no d
efinitive conclusions other than a change in model does not resolve the issue. 
} \item{}Are there other important comments? \linebreak{} \hspace*{0.5cm} \text
it{The VPA analysis was performed with survey time series split between 1994 an
d 1995. This time split corresponds to changes in the commercial reporting meth
ods as well as other regulatory management changes. } \end{itemize}{}} \def\WIT
UNITRefr{ \textbf{References: }{} \linebreak{}Northeast Fisheries Science Cente
r. 2008. Assessment of 19 Northeast Groundfish Stocks through 2007: Report of t
he 3$^{rd}$ Groundfish Assessment Review Meeting (GARM III), Northeast Fisherie
s Science Center, Woods Hole, Massachusetts, August 4-8, 2008. US Dep Commer, N
OAA Fisheries, Northeast Fish Sci Cent Ref Doc. 08-15; 884 p + xvii. http://www
.nefsc.noaa.gov/publications/crd/crd0815/ \linebreak{} \linebreak{}Northeast Fi
sheries Science Center. 2012. Assessment or Data Updates of 13 Northeast Ground
fish Stocks through 2010. US Dep Commer, NOAA Fisheries, Northeast Fish Sci Cen
t Ref Doc. 12-06; 789 p. http://www.nefsc.noaa.gov/publications/crd/crd1206/ \l
inebreak{} \linebreak{}} \def\WITUNITDraft{} \def\WITUNITSPPname{witch flounder
} \def\WITUNITSPPnameT{Witch flounder} \def\WITUNITRptYr{2015} \def\WITUNITAuth
or{Susan Wigley} \def\WITUNITReviewerComments{/net/home2/dhennen/testEIEIO/BigR
eport/WIT_UNIT/latex} \def\HKWUNITMyPathTab{/net/home2/dhennen/testEIEIO/BigRep
ort/HKW_UNIT/tables} \def\HKWUNITMyPathFig{/net/home2/dhennen/testEIEIO/BigRepo
rt/HKW_UNIT/figures} \def\HKWUNITfigFishCap{Total catch of white hake between 1
963 and 2014 by fleet (commercial, recreational, or Canadian) and disposition (
landings and discards).} \def\HKWUNITfigSSBCap{Trends in spawning stock biomass
 of white hake between 1963 and 2014 from the current (solid line) and previous
 (dashed line) assessment and the corresponding $SSB_{Threshold}${} ($\dfrac{1}
{2}${} $SSB_{MSY}${} \textit{proxy}{}; horizontal dashed line) as well as $SSB_
{Target}${} ($SSB_{MSY}${} \textit{proxy}{}; horizontal dotted line) based on t
he 2014 assessment. The red dot indicates the rho-adjusted SSB values that woul
d have resulted had a retrospective adjusment been made (see Special Comments s
ection). The approximate 90\% lognormal confidence intervals are shown.} \def\H
KWUNITfigFCap{Trends in the fully selected fishing mortality ($F_{Full}${}) of 
white hake between 1963 and 2014 from the current (solid line) and previous (da
shed line) assessment and the corresponding $F_{Threshold}${} ($F_{MSY}${} \tex
tit{proxy}{}=0.188; horizontal dashed line). The red dot indicates the rho-adju
sted SSB values that would have resulted had a retrospective adjusment been mad
e (see Special Comments section). The approximate 90\% lognormal confidence int
ervals are shown.} \def\HKWUNITfigRecrCap{Trends in Recruits (age 1) (000s) of 
white hake between 1963 and 2014 from the current (solid line) and previous (da
shed line) assessment. The approximate 90\% lognormal confidence intervals are 
shown.} \def\HKWUNITfigSurvCap{Indices of biomass for the white hake between 19
63 and 2015 for the Northeast Fisheries Science Center (NEFSC) spring and fall 
bottom trawl surveys. The approximate 90\% lognormal confidence intervals are s
hown.} \def\HKWUNITPreAmb{This assessment of the white hake (\textit{Urophycis 
tenuis}) stock is an operational update of the existing 2013 benchmark ASAP ass
essment (NEFSC 2013). Based on the previous assessment the stock was not overfi
shed, and overfishing was not ocurring. This assessment updates commercial fish
ery catch data, research survey indices of abundance, and the ASAP assessment m
odels and reference points through 2014. Additionally, stock projections have b
een updated through 2018.} \def\HKWUNITSoS{ \textbf{State of Stock: }{}Based on
 this updated assessment, white hake (\textit{Urophycis tenuis}) stock is not o
verfished and overfishing is not occurring (Figures \ref{HKWUNITSSB_plot1}-\ref
{HKWUNITF_plot1}){}. Retrospective adjustments were not made to the model resul
ts. Spawning stock biomass (SSB) in 2014 was estimated to be 28,553 (mt) which 
is 88\% of the biomass threshold for an overfished stock ($SSB_{MSY}${} \textit
{proxy}{} = 32,550; Figure \ref{HKWUNITSSB_plot1}{}). The 2014 fully selected f
ishing mortality was estimated to be 0.076 which is 40\% of the overfishing thr
eshold proxy ($F_{MSY}${} \textit{proxy}{} = 0.188; Figure \ref{HKWUNITF_plot1}
{}).} \def\HKWUNITProj{ \textbf{Projections: }{}Short term projections of catch
 and SSB were derived by sampling from a cumulative distribution function of re
cruitment estimates from ASAP from 1995-2012. The annual fishery selectivity, m
aturity ogive, and mean weights at age used in the projection are the most rece
nt 5 year averages. } \def\HKWUNITSpecCmt{ \textbf{Special Comments: } \begin{i
temize}{} \item{}What are the most important sources of uncertainty in this sto
ck assessment? Explain, and describe qualitatively how they affect the assessme
nt results (such as estimates of biomass, F, recruitment, and population projec
tions). \linebreak{} \hspace*{0.5cm} \textit{1. Catch at age information is not
 well characterized due to possible mis-identification of species in the commer
cial and sea sampling data, particularly in early years, low sampling of commer
cial landings in some years, and sparse discard data particularly in early year
s. \linebreak{} \hspace*{0.5cm}2. Since the commercial catch is aged primarily 
with survey age/length keys, there is considerable augmentation required, mainl
y for ages 5 and older. The numbers at age and mean weights at age in the catch
 for these ages may therefore not be well specified. \linebreak{} \hspace*{0.5c
m}3. White hake may move seasonally into and out of the defined stock area. \li
nebreak{} \hspace*{0.5cm}4. There are no commercial catch at age data prior to 
1989 and the catchability of older ages in the surveys is very low. This result
s in a large uncertainty in starting numbers at age. \linebreak{} \hspace*{0.5c
m}5. Since 2003, dealers have been culling very large fish out of the large cat
egory. However, there was no market category to input into the landings until J
une 2014. The length compositions are distinct from large and have been identif
ied since 2011. This may bias the age composition of the landings, particularly
 in 2014 when 2000 of the 5000 large samples were these extra-large fish. \line
break{} \hspace*{0.5cm}6. A pooled age/length key is used for 1963-1981, fall 2
003 (second half of commercial key) and 2014.Age data were not available for 20
14 in time for this assessment. The same pooled key that was used for 1963-1981
 was used for 2014.} \item{} Does this assessment model have a retrospective pa
ttern? If so, is the pattern minor, or major? (A major retrospective pattern oc
curs when the adjusted SSB or $F_{Full}${} lies outside of the approximate join
t confidence region for SSB and $F_{Full}${}; see Figure \ref{RhoDecision_tab}{
}). \linebreak{} \hspace*{0.5cm} \textit{ No retrospective adjustment of spawni
ng stock biomass or fishing mortality in 2014 was required. The pattern in this
 assessment is considered minor (Mohn’s rho of 0.18 on SSB, Mohn’s rho of 0
.12 on F) with the adjusted SSB within the 90 \% CI of the MCMC. However, the M
ohn’s rho for Age 1 estimates is 0.54. This may have an impact on projections
 if this continues into the future.} \item{}Based on this stock assessment, are
 population projections well determined or uncertain? \linebreak{} \hspace*{0.5
cm} \textit{Population projections for white hake, are not well determined and 
projected biomass from the last assessment was outside the confidence bounds of
 the biomass estimated in the current assessment. } \item{}Describe any changes
 that were made to the current stock assessment, beyond incorporating additiona
l years of data and the affect these changes had on the assessment and stock st
atus. \linebreak{} \hspace*{0.5cm} \textit{ The 2011 catch-at-length and age we
re re-estimated for both landings and discards. For the landings, two samples w
ere adjusted for dorsal length to total length that had been missed in the prev
ious assessment.} \item{}If the stock status has changed a lot since the previo
us assessment, explain why this occurred. \linebreak{} \hspace*{0.5cm} \textit{
While stock status of white hake has not changed, the stock has not rebuilt as 
the projections from the last assessment indicated. This is due to the retrospe
ctive in recruitment. The numbers for the 2005-2009 year classes, which were in
cluded in the age 2-6 starting numbers in the projections, were over-estimated 
which led to over-estimating SSB in 2014.} \item{}Indicate what data or studies
 are currently lacking and which would be needed most to improve this stock ass
essment in the future. \linebreak{} \hspace*{0.5cm} \textit{ Age structures fro
m the observer program are available and should be aged to augment the survey k
eys. There is a also a new market category for heads and age structures could b
e acquired from these is an otolith length/total length relationship can be est
ablished. } \item{}Are there other important issues? \linebreak{} \hspace*{0.5c
m} \textit{None. } \end{itemize}{}} \def\HKWUNITRefr{ \textbf{References: }{} \
linebreak{} NEFSC. 2013. 56$^{th}$ Northeast Regional Stock Assessment Workshop
 (56$^{th}$ SAW) Assessment Report.US Dep Commer, NOAA Fisheries, Northeast Fis
h Sci Cent Ref Doc. 13-10; 868 p. http://www.nefsc.noaa.gov/publications/crd/cr
d1310/ \linebreak{} \linebreak{}} \def\HKWUNITDraft{} \def\HKWUNITSPPname{white
 hake} \def\HKWUNITSPPnameT{White hake} \def\HKWUNITRptYr{2015} \def\HKWUNITAut
hor{Katherine Sosebee} \def\HKWUNITReviewerComments{/net/home2/dhennen/testEIEI
O/BigReport/HKW_UNIT/latex}
